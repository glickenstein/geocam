
\documentclass{book}
%%%%%%%%%%%%%%%%%%%%%%%%%%%%%%%%%%%%%%%%%%%%%%%%%%%%%%%%%%%%%%%%%%%%%%%%%%%%%%%%%%%%%%%%%%%%%%%%%%%%%%%%%%%%%%%%%%%%%%%%%%%%%%%%%%%%%%%%%%%%%%%%%%%%%%%%%%%%%%%%%%%%%%%%%%%%%%%%%%%%%%%%%%%%%%%%%%%%%%%%%%%%%%%%%%%%%%%%%%%%%%%%%%%%%%%%%%%%%%%%%%%%%%%%%%%%
\usepackage{makeidx}
\usepackage{amssymb}
\usepackage{eurosym}
\usepackage{amsfonts}
\usepackage{amsmath}

\setcounter{MaxMatrixCols}{10}
%TCIDATA{OutputFilter=LATEX.DLL}
%TCIDATA{Version=5.00.0.2606}
%TCIDATA{<META NAME="SaveForMode" CONTENT="1">}
%TCIDATA{BibliographyScheme=Manual}
%TCIDATA{Created=Monday, July 27, 2009 13:07:40}
%TCIDATA{LastRevised=Monday, March 08, 2010 11:48:10}
%TCIDATA{<META NAME="GraphicsSave" CONTENT="32">}
%TCIDATA{<META NAME="DocumentShell" CONTENT="Standard LaTeX\Standard LaTeX Book">}
%TCIDATA{CSTFile=40 LaTeX Book.cst}

\newtheorem{theorem}{Theorem}
\newtheorem{acknowledgement}[theorem]{Acknowledgement}
\newtheorem{algorithm}[theorem]{Algorithm}
\newtheorem{axiom}[theorem]{Axiom}
\newtheorem{case}[theorem]{Case}
\newtheorem{claim}[theorem]{Claim}
\newtheorem{conclusion}[theorem]{Conclusion}
\newtheorem{condition}[theorem]{Condition}
\newtheorem{conjecture}[theorem]{Conjecture}
\newtheorem{corollary}[theorem]{Corollary}
\newtheorem{criterion}[theorem]{Criterion}
\newtheorem{definition}[theorem]{Definition}
\newtheorem{example}[theorem]{Example}
\newtheorem{exercise}[theorem]{Exercise}
\newtheorem{lemma}[theorem]{Lemma}
\newtheorem{notation}[theorem]{Notation}
\newtheorem{problem}[theorem]{Problem}
\newtheorem{proposition}[theorem]{Proposition}
\newtheorem{remark}[theorem]{Remark}
\newtheorem{solution}[theorem]{Solution}
\newtheorem{summary}[theorem]{Summary}
\newenvironment{proof}[1][Proof]{\noindent\textbf{#1.} }{\ \rule{0.5em}{0.5em}}
\input{tcilatex}

\begin{document}

\frontmatter
\title{GEOCAM\\
Geometric Evolutions on Computational Abstract Manifolds}
\author{Daniel Champion, David Glickenstein, Yuliya Gorlina, \and Alex
Henniges, Kurtis Norwood, Jeff Taft, \and Joseph Thomas, Andrea Young}
\date{Summer 2009}
\maketitle
\tableofcontents

\chapter*{Preface}

\markboth{PREFACE}{PREFACE}This is is the preface. It is an unnumbered
chapter. The \verb|markboth| TeX field at the beginning of this paragraph
sets the correct page heading for the Preface portion of the document. The
preface does not appear in the table of contents.

\mainmatter

\part{Introductory Information}

%TCIMACRO{\QSubDoc{Include introduction}{\input{introduction.tex}}}%
%BeginExpansion
\input{introduction.tex}%
%EndExpansion

%TCIMACRO{\QSubDoc{Include cplusplus}{%TCIDATA{Version=5.00.0.2606}
%TCIDATA{LaTeXparent=0,0,geocam.tex}
                      

\chapter{The C++ Language}

%TCIMACRO{\QSubDoc{Include Tutorial}{\input{Cplusplus/Tutorial.tex}}}%
%BeginExpansion
\input{Cplusplus/Tutorial.tex}%
%EndExpansion

%TCIMACRO{%
%\QSubDoc{Include Building_and_modifying_a_memoized_geometry}{%TCIDATA{Version=5.00.0.2606}
%TCIDATA{LaTeXparent=0,0,cplusplus.tex}
                      

%%%%% BEGINNING OF DOCUMENT BODY %%%%%
% html: Beginning of file: `clean.html'
% DOCTYPE HTML PUBLIC "-//W3C//DTD HTML 4.01//EN"
%  This is a (PRE) block.  Make sure it's left aligned or your toc title will be off. 

\section*{Working with a ``memoized-pipeline'' data structure (WORK IN
PROGRESS)}

\label{f0}

\subsection*{Key Words}

geometry, memoized-pipeline, extending, modifying, data structure, geoquant,
quantities, singleton, observer, observable

\subsection*{Authors}

\begin{itemize}
\item Alex Henniges

\item Joseph Thomas
\end{itemize}

\subsection*{Introduction}

The memoized-pipeline is a data structure we developed for investigating
geometries defined on triangulations. It is particularly suited to the
situation in which we need to specify the values of some geometric
quantities (independent variables) and then need to rapidly calculate the
values of some other quantities (the dependent variables). Basically, we
achieve this speedup by trading space for time. Usually, the definitions of
the dependent variables have many intermediate values in common. By saving
these values the first time we compute them, and then reusing them later, we
can avoid a lot of useless recalculation. This strategy of saving calculated
values, which can be found in most algorithms textbooks, is called
``memoization.''

In implementing various geometries, we have already developed code and
techniques for making memoization an automatic part of encoding a geometry.
In this tutorial, we describe how to take advantage of this existing code.

\subsection*{Implementation Details}

The underlying implementation of the pipeline is designed to solve two
problems in a fairly user-friendly way:

\begin{enumerate}
\item We would like to be able to identify geometric quantities with
positions on the triangulation. For example, we can speak of the dihedral
angle associated with a particular edge on a tetrahedron. We would like to
be able to write code in the same way.

\item We would like memoization to be nearly automatic. In other words, when
writing a particular quantity, the programmer shouldn't have to think much
about what happens to memoize that quantity's value.
\end{enumerate}

Taking the programmer's perspective, we can view quantities as being
specified by 3 pieces of information:

\begin{enumerate}
\item A position on the triangulation.

\item A definition of the other quantities (if any) needed to calculate the
value of the current quantity, and where those quantities can be found on
the triangulation.

\item A formula for calculating a quantity's value, given the values of the
other quantities it depends on.
\end{enumerate}

Usually, specifying just these 3 pieces of information is enough to create a
new type of quantity. To help speed the development of quantities, we have
developed a Ruby script, \texttt{makeQuantity.rb}, that generates much of
the source code. This can be invoked at the command line as follows: {\small 
}
\begin{verbatim}
{\small > ruby makeQuantity.rb [quantity]
}
\end{verbatim}

This produces two files, \texttt{\mbox{$[$}quantity\mbox{$]$}.h} and \texttt{%
\mbox{$[$}quantity\mbox{$]$}.cpp}.

\subsection{The ``anatomy'' of \texttt{quantity.h}}

In \texttt{C++}, header files serve several purposes. Among other uses, a
header file can:

\begin{itemize}
\item Specify dependencies on other parts of the project.

\item Define an interface for other parts of your project to use. This
includes:

\begin{itemize}
\item Definitions for new data-types (like classes).

\item Definitions for procedure calls (what arguments a procedure takes, and
what it returns).
\end{itemize}
\end{itemize}

By default, \texttt{makeQuantity.rb} gives you the following header file to
use (here, we chose \texttt{quantity/QUANTITY} as the quantity name, in
practice, this is filled out by the script). {\small }
\begin{verbatim}
{\small #ifndef QUANTITY_H_
}
{\small #define QUANTITY_H_
}
 
{\small #include "geoquant.h"
}
 
{\small /******************REGION 1*******************
}
{\small  * This is where you load the headers of the *
}
{\small  * quantities you require.                   *
}
{\small  *********************************************/
}
 
{\small class quantity : public virtual GeoQuant {
}
{\small protected:
}
{\small   quantity( SIMPLICES );
}
{\small   void recalculate();
}
{\small   /****************REGION 2*********************
}
{\small    * The quantity references you need go here. *
}
{\small    *********************************************/
}
 
{\small public:
}
{\small   ~quantity();
}
{\small   static quantity* At( SIMPLICES );
}
{\small   static void CleanUp();
}
{\small };
}
{\small #endif /* QUANTITY_H_ */
}
\end{verbatim}

The two important areas of the header are labeled \texttt{REGION 1} and 
\texttt{REGION 2}. In region 1, you specify the header files for the
quantities and utilities you use in the rest of your quantity. These \texttt{%
\#include} statements can be thought of as providing definitions for the
data and procedures you want to use in building your quantity. In region 2,
you specify the data associated with a given instance of the quantity;
typically this amounts to several references to other quantities, or a data
structure that manages references to other quantities. Lastly, you will need
to modify the region tagged \texttt{SIMPLICES} so that it reflects a
collection of simplices that describe your quantity's position on the
triangulation.

\subsection{The ``anatomy'' of \texttt{quantity.cpp}}

In general, a \texttt{.cpp} file provides the internal implementation to
support the operations described in the corresponding header file. Editing
this file will be a little more complicated. By default, \texttt{%
makeQuantity.rb} will produce the following \texttt{.cpp} file: {\small }
\begin{verbatim}
{\small #include "quantity.h"
}
 
{\small #include <map>
}
{\small #include <new>
}
{\small using namespace std;
}
 
{\small 
#define map<TriPosition, quantity*, TriPositionCompare> quantityIndex 
}
{\small static quantityIndex* Index = NULL;
}
 
{\small quantity::quantity( SIMPLICES ){
}
{\small   /* REGION 1 */
}
{\small }
}
 
{\small quantity::~quantity(){
}
{\small   /* REGION 2 */
}
{\small }
}
 
{\small void quantity::recalculate(){
}
{\small   /* REGION 3 */
}
{\small }
}
 
{\small quantity* quantity::At( SIMPLICES ){
}
{\small   TriPosition T( NUMSIMPLICES, SIMPLICES );
}
{\small   if( Index == NULL ) Index = new quantityIndex();
}
{\small   quantityIndex::iterator iter = Index->find( T );
}
 
{\small   if( iter == Index->end() ){
}
{\small     quantity* val = new quantity( SIMPLICES );
}
{\small     Index->insert( make_pair( T, val ) );
}
{\small     return val;
}
{\small   } else {
}
{\small     return iter->second;
}
{\small   }
}
{\small }
}
 
{\small void quantity::CleanUp(){
}
{\small   if( Index == NULL ) return;
}
{\small   quantityIndex::iterator iter;
}
{\small   for(iter = Index->begin(); iter != Index->end(); iter++)
}
{\small     delete iter->second;
}
{\small   delete Index;
}
{\small }
}
\end{verbatim}

There are a few smaller areas to fill out, but in general defining the
quantity requires the following three definitions:

\begin{itemize}
\item Region 1 specifies how to obtain references on the quantities your
quantity depends on. Typically, this will involve using the input simplex
information and some utilities for inspecting the triangulation to look up
the quantities needed for later calculations.

\item Region 2 specifies how to release any data structures built up using
dynamic memory. In many cases, this field will be left blank.

\item Region 3 specifies how to calculate the value of an instance of the
quantity. Typically, this will occur in two steps:

\begin{enumerate}
\item Using the quantity references obtained in region 1, we acquire the
current values of the quantities used in the calculation.

\item Using a formula and the values found in step 1, we calculate the value
of the current quantity.
\end{enumerate}
\end{itemize}

\subsection*{An Extended Example}

Perhaps the easiest way to understand the system is by examining a few
working quantities. In this example, we consider the code written to
represent the ``Dual Area Segment\"{} quantity discussed in \"{}Discrete
conformal variations and scalar curvature on piecewise flat two and three
dimensional manifolds''\footnote{%
See URL http://arxiv.org/abs/0906.1560} (in the paper, this quantity is also
called A$_{ij,kl}$).

PICTURES AND A PROSE DESCRIPTION GO HERE

{\small }
\begin{verbatim}
{\small #ifndef DUALAREASEGMENT_H_
}
{\small #define DUALAREASEGMENT_H_
}
 
{\small #include "geoquant.h"
}
{\small #include "triposition.h"
}
 
{\small #include "edge_height.h"
}
{\small #include "face_height.h"
}
 
{\small class DualAreaSegment : public virtual GeoQuant {
}
{\small private:
}
{\small   EdgeHeight* hij_k;
}
{\small   EdgeHeight* hij_l;
}
{\small   FaceHeight* hijk_l;
}
{\small   FaceHeight* hijl_k;
}
 
{\small protected:
}
{\small   DualAreaSegment( Edge& e, Tetra& t );
}
{\small   void recalculate();
}
 
{\small public:
}
{\small   ~DualAreaSegment();
}
{\small   static DualAreaSegment* At( Edge& e, Tetra& t );
}
{\small   static void CleanUp();
}
{\small   static void Record( char* filename );
}
{\small };
}
 
{\small #endif /* DUALAREASEGMENT_H_ */
}
\end{verbatim}

{\small }
\begin{verbatim}
{\small #include "dualareasegment.h"
}
{\small #include "math/miscmath.h"
}
 
{\small #include <stdio.h>
}
 
{\small 
typedef map<TriPosition, DualAreaSegment*, TriPositionCompare> DualAreaSegmentIndex;
}
{\small static DualAreaSegmentIndex* Index = NULL;
}
 
{\small DualAreaSegment::DualAreaSegment( Edge& e, Tetra& t ){
}
{\small 
  StdTetra st = labelTetra( t, e );  // We use the topological tools in miscmath to
}
{\small 
                                     // label the tetrahedron with respect to edge e.
}
 
{\small   Face& fa123 = Triangulation::faceTable[ st.f123 ];
}
{\small   Face& fa124 = Triangulation::faceTable[ st.f124 ];
}
 
{\small 
  hij_k = EdgeHeight::At( e, fa123 );  // Here we use the calculated topological values
}
{\small 
  hij_l = EdgeHeight::At( e, fa124 );  // to look up the edge and face heights required
}
{\small 
  hijk_l = FaceHeight::At( fa123, t ); // to calculate the dual area.
}
{\small   hijl_k = FaceHeight::At( fa124, t );
}
 
{\small 
  hij_k->addDependent(this);  // Here we notify the quantities we reference
}
{\small 
  hij_l->addDependent(this);  // that we wish to observe them (this is the            
}
{\small 
  hijk_l->addDependent(this); // where an important part of our observer-
}
{\small   hijl_k->addDependent(this); // observable design is implemented).
}
{\small }
}
 
{\small 
DualAreaSegment::~DualAreaSegment(){} // We didn't allocate any memory to store
}
{\small 
                                      // this quantity's data, so the destructor 
}
{\small                                       // can be left blank.
}
 
{\small void DualAreaSegment::recalculate(){
}
{\small 
  double Hij_k = hij_k->getValue();   // Step 1: We use the quantity references
}
{\small 
  double Hijk_l = hijk_l->getValue(); // to obtain correct values for the referenced
}
{\small   double Hij_l = hij_l->getValue();   // quantities.
}
{\small   double Hijl_k = hijl_k->getValue();
}
 
{\small 
  // Step 2: We use a formula to calculate the value of the dual-area segment.
}
{\small   value = 0.5*(Hij_k * Hijk_l + Hij_l * Hijl_k); 
}
{\small }
}
 
{\small DualAreaSegment* DualAreaSegment::At( Edge& e, Tetra& t ){
}
{\small   TriPosition T( 2, e.getSerialNumber(), t.getSerialNumber() );
}
{\small   if( Index == NULL ) Index = new DualAreaSegmentIndex();
}
{\small   DualAreaSegmentIndex::iterator iter = Index->find( T );
}
 
{\small   if( iter == Index->end() ){
}
{\small     DualAreaSegment* val = new DualAreaSegment( e, t );
}
{\small     Index->insert( make_pair( T, val ) );
}
{\small     return val;
}
{\small   } else {
}
{\small     return iter->second;
}
{\small   }
}
{\small }
}
 
{\small void DualAreaSegment::CleanUp(){
}
{\small   if( Index == NULL ) return;
}
{\small   DualAreaSegmentIndex::iterator iter;
}
{\small   for(iter = Index->begin(); iter != Index->end(); iter++)
}
{\small     delete iter->second;
}
{\small   delete Index;
}
{\small }
}
 
\end{verbatim}

\subsection*{Common Mistakes}

A panoply of debugging hints should go here.

\subsection*{Fancier Tricks}

Techniques for less common quantities go here.

\subsection*{Limitations, Areas to Improve}

Discussion of the current topological assumptions our code makes.

% html: End of file: `clean.html'

%%%%% END OF DOCUMENT BODY %%%%%
% In the future, we might want to put some additional data here, such
% as when the documentation was converted from wiki to TeX.
%
}}%
%BeginExpansion
%TCIDATA{Version=5.00.0.2606}
%TCIDATA{LaTeXparent=0,0,cplusplus.tex}
                      

%%%%% BEGINNING OF DOCUMENT BODY %%%%%
% html: Beginning of file: `clean.html'
% DOCTYPE HTML PUBLIC "-//W3C//DTD HTML 4.01//EN"
%  This is a (PRE) block.  Make sure it's left aligned or your toc title will be off. 

\section*{Working with a ``memoized-pipeline'' data structure (WORK IN
PROGRESS)}

\label{f0}

\subsection*{Key Words}

geometry, memoized-pipeline, extending, modifying, data structure, geoquant,
quantities, singleton, observer, observable

\subsection*{Authors}

\begin{itemize}
\item Alex Henniges

\item Joseph Thomas
\end{itemize}

\subsection*{Introduction}

The memoized-pipeline is a data structure we developed for investigating
geometries defined on triangulations. It is particularly suited to the
situation in which we need to specify the values of some geometric
quantities (independent variables) and then need to rapidly calculate the
values of some other quantities (the dependent variables). Basically, we
achieve this speedup by trading space for time. Usually, the definitions of
the dependent variables have many intermediate values in common. By saving
these values the first time we compute them, and then reusing them later, we
can avoid a lot of useless recalculation. This strategy of saving calculated
values, which can be found in most algorithms textbooks, is called
``memoization.''

In implementing various geometries, we have already developed code and
techniques for making memoization an automatic part of encoding a geometry.
In this tutorial, we describe how to take advantage of this existing code.

\subsection*{Implementation Details}

The underlying implementation of the pipeline is designed to solve two
problems in a fairly user-friendly way:

\begin{enumerate}
\item We would like to be able to identify geometric quantities with
positions on the triangulation. For example, we can speak of the dihedral
angle associated with a particular edge on a tetrahedron. We would like to
be able to write code in the same way.

\item We would like memoization to be nearly automatic. In other words, when
writing a particular quantity, the programmer shouldn't have to think much
about what happens to memoize that quantity's value.
\end{enumerate}

Taking the programmer's perspective, we can view quantities as being
specified by 3 pieces of information:

\begin{enumerate}
\item A position on the triangulation.

\item A definition of the other quantities (if any) needed to calculate the
value of the current quantity, and where those quantities can be found on
the triangulation.

\item A formula for calculating a quantity's value, given the values of the
other quantities it depends on.
\end{enumerate}

Usually, specifying just these 3 pieces of information is enough to create a
new type of quantity. To help speed the development of quantities, we have
developed a Ruby script, \texttt{makeQuantity.rb}, that generates much of
the source code. This can be invoked at the command line as follows: {\small 
}
\begin{verbatim}
{\small > ruby makeQuantity.rb [quantity]
}
\end{verbatim}

This produces two files, \texttt{\mbox{$[$}quantity\mbox{$]$}.h} and \texttt{%
\mbox{$[$}quantity\mbox{$]$}.cpp}.

\subsection{The ``anatomy'' of \texttt{quantity.h}}

In \texttt{C++}, header files serve several purposes. Among other uses, a
header file can:

\begin{itemize}
\item Specify dependencies on other parts of the project.

\item Define an interface for other parts of your project to use. This
includes:

\begin{itemize}
\item Definitions for new data-types (like classes).

\item Definitions for procedure calls (what arguments a procedure takes, and
what it returns).
\end{itemize}
\end{itemize}

By default, \texttt{makeQuantity.rb} gives you the following header file to
use (here, we chose \texttt{quantity/QUANTITY} as the quantity name, in
practice, this is filled out by the script). {\small }
\begin{verbatim}
{\small #ifndef QUANTITY_H_
}
{\small #define QUANTITY_H_
}
 
{\small #include "geoquant.h"
}
 
{\small /******************REGION 1*******************
}
{\small  * This is where you load the headers of the *
}
{\small  * quantities you require.                   *
}
{\small  *********************************************/
}
 
{\small class quantity : public virtual GeoQuant {
}
{\small protected:
}
{\small   quantity( SIMPLICES );
}
{\small   void recalculate();
}
{\small   /****************REGION 2*********************
}
{\small    * The quantity references you need go here. *
}
{\small    *********************************************/
}
 
{\small public:
}
{\small   ~quantity();
}
{\small   static quantity* At( SIMPLICES );
}
{\small   static void CleanUp();
}
{\small };
}
{\small #endif /* QUANTITY_H_ */
}
\end{verbatim}

The two important areas of the header are labeled \texttt{REGION 1} and 
\texttt{REGION 2}. In region 1, you specify the header files for the
quantities and utilities you use in the rest of your quantity. These \texttt{%
\#include} statements can be thought of as providing definitions for the
data and procedures you want to use in building your quantity. In region 2,
you specify the data associated with a given instance of the quantity;
typically this amounts to several references to other quantities, or a data
structure that manages references to other quantities. Lastly, you will need
to modify the region tagged \texttt{SIMPLICES} so that it reflects a
collection of simplices that describe your quantity's position on the
triangulation.

\subsection{The ``anatomy'' of \texttt{quantity.cpp}}

In general, a \texttt{.cpp} file provides the internal implementation to
support the operations described in the corresponding header file. Editing
this file will be a little more complicated. By default, \texttt{%
makeQuantity.rb} will produce the following \texttt{.cpp} file: {\small }
\begin{verbatim}
{\small #include "quantity.h"
}
 
{\small #include <map>
}
{\small #include <new>
}
{\small using namespace std;
}
 
{\small 
#define map<TriPosition, quantity*, TriPositionCompare> quantityIndex 
}
{\small static quantityIndex* Index = NULL;
}
 
{\small quantity::quantity( SIMPLICES ){
}
{\small   /* REGION 1 */
}
{\small }
}
 
{\small quantity::~quantity(){
}
{\small   /* REGION 2 */
}
{\small }
}
 
{\small void quantity::recalculate(){
}
{\small   /* REGION 3 */
}
{\small }
}
 
{\small quantity* quantity::At( SIMPLICES ){
}
{\small   TriPosition T( NUMSIMPLICES, SIMPLICES );
}
{\small   if( Index == NULL ) Index = new quantityIndex();
}
{\small   quantityIndex::iterator iter = Index->find( T );
}
 
{\small   if( iter == Index->end() ){
}
{\small     quantity* val = new quantity( SIMPLICES );
}
{\small     Index->insert( make_pair( T, val ) );
}
{\small     return val;
}
{\small   } else {
}
{\small     return iter->second;
}
{\small   }
}
{\small }
}
 
{\small void quantity::CleanUp(){
}
{\small   if( Index == NULL ) return;
}
{\small   quantityIndex::iterator iter;
}
{\small   for(iter = Index->begin(); iter != Index->end(); iter++)
}
{\small     delete iter->second;
}
{\small   delete Index;
}
{\small }
}
\end{verbatim}

There are a few smaller areas to fill out, but in general defining the
quantity requires the following three definitions:

\begin{itemize}
\item Region 1 specifies how to obtain references on the quantities your
quantity depends on. Typically, this will involve using the input simplex
information and some utilities for inspecting the triangulation to look up
the quantities needed for later calculations.

\item Region 2 specifies how to release any data structures built up using
dynamic memory. In many cases, this field will be left blank.

\item Region 3 specifies how to calculate the value of an instance of the
quantity. Typically, this will occur in two steps:

\begin{enumerate}
\item Using the quantity references obtained in region 1, we acquire the
current values of the quantities used in the calculation.

\item Using a formula and the values found in step 1, we calculate the value
of the current quantity.
\end{enumerate}
\end{itemize}

\subsection*{An Extended Example}

Perhaps the easiest way to understand the system is by examining a few
working quantities. In this example, we consider the code written to
represent the ``Dual Area Segment\"{} quantity discussed in \"{}Discrete
conformal variations and scalar curvature on piecewise flat two and three
dimensional manifolds''\footnote{%
See URL http://arxiv.org/abs/0906.1560} (in the paper, this quantity is also
called A$_{ij,kl}$).

PICTURES AND A PROSE DESCRIPTION GO HERE

{\small }
\begin{verbatim}
{\small #ifndef DUALAREASEGMENT_H_
}
{\small #define DUALAREASEGMENT_H_
}
 
{\small #include "geoquant.h"
}
{\small #include "triposition.h"
}
 
{\small #include "edge_height.h"
}
{\small #include "face_height.h"
}
 
{\small class DualAreaSegment : public virtual GeoQuant {
}
{\small private:
}
{\small   EdgeHeight* hij_k;
}
{\small   EdgeHeight* hij_l;
}
{\small   FaceHeight* hijk_l;
}
{\small   FaceHeight* hijl_k;
}
 
{\small protected:
}
{\small   DualAreaSegment( Edge& e, Tetra& t );
}
{\small   void recalculate();
}
 
{\small public:
}
{\small   ~DualAreaSegment();
}
{\small   static DualAreaSegment* At( Edge& e, Tetra& t );
}
{\small   static void CleanUp();
}
{\small   static void Record( char* filename );
}
{\small };
}
 
{\small #endif /* DUALAREASEGMENT_H_ */
}
\end{verbatim}

{\small }
\begin{verbatim}
{\small #include "dualareasegment.h"
}
{\small #include "math/miscmath.h"
}
 
{\small #include <stdio.h>
}
 
{\small 
typedef map<TriPosition, DualAreaSegment*, TriPositionCompare> DualAreaSegmentIndex;
}
{\small static DualAreaSegmentIndex* Index = NULL;
}
 
{\small DualAreaSegment::DualAreaSegment( Edge& e, Tetra& t ){
}
{\small 
  StdTetra st = labelTetra( t, e );  // We use the topological tools in miscmath to
}
{\small 
                                     // label the tetrahedron with respect to edge e.
}
 
{\small   Face& fa123 = Triangulation::faceTable[ st.f123 ];
}
{\small   Face& fa124 = Triangulation::faceTable[ st.f124 ];
}
 
{\small 
  hij_k = EdgeHeight::At( e, fa123 );  // Here we use the calculated topological values
}
{\small 
  hij_l = EdgeHeight::At( e, fa124 );  // to look up the edge and face heights required
}
{\small 
  hijk_l = FaceHeight::At( fa123, t ); // to calculate the dual area.
}
{\small   hijl_k = FaceHeight::At( fa124, t );
}
 
{\small 
  hij_k->addDependent(this);  // Here we notify the quantities we reference
}
{\small 
  hij_l->addDependent(this);  // that we wish to observe them (this is the            
}
{\small 
  hijk_l->addDependent(this); // where an important part of our observer-
}
{\small   hijl_k->addDependent(this); // observable design is implemented).
}
{\small }
}
 
{\small 
DualAreaSegment::~DualAreaSegment(){} // We didn't allocate any memory to store
}
{\small 
                                      // this quantity's data, so the destructor 
}
{\small                                       // can be left blank.
}
 
{\small void DualAreaSegment::recalculate(){
}
{\small 
  double Hij_k = hij_k->getValue();   // Step 1: We use the quantity references
}
{\small 
  double Hijk_l = hijk_l->getValue(); // to obtain correct values for the referenced
}
{\small   double Hij_l = hij_l->getValue();   // quantities.
}
{\small   double Hijl_k = hijl_k->getValue();
}
 
{\small 
  // Step 2: We use a formula to calculate the value of the dual-area segment.
}
{\small   value = 0.5*(Hij_k * Hijk_l + Hij_l * Hijl_k); 
}
{\small }
}
 
{\small DualAreaSegment* DualAreaSegment::At( Edge& e, Tetra& t ){
}
{\small   TriPosition T( 2, e.getSerialNumber(), t.getSerialNumber() );
}
{\small   if( Index == NULL ) Index = new DualAreaSegmentIndex();
}
{\small   DualAreaSegmentIndex::iterator iter = Index->find( T );
}
 
{\small   if( iter == Index->end() ){
}
{\small     DualAreaSegment* val = new DualAreaSegment( e, t );
}
{\small     Index->insert( make_pair( T, val ) );
}
{\small     return val;
}
{\small   } else {
}
{\small     return iter->second;
}
{\small   }
}
{\small }
}
 
{\small void DualAreaSegment::CleanUp(){
}
{\small   if( Index == NULL ) return;
}
{\small   DualAreaSegmentIndex::iterator iter;
}
{\small   for(iter = Index->begin(); iter != Index->end(); iter++)
}
{\small     delete iter->second;
}
{\small   delete Index;
}
{\small }
}
 
\end{verbatim}

\subsection*{Common Mistakes}

A panoply of debugging hints should go here.

\subsection*{Fancier Tricks}

Techniques for less common quantities go here.

\subsection*{Limitations, Areas to Improve}

Discussion of the current topological assumptions our code makes.

% html: End of file: `clean.html'

%%%%% END OF DOCUMENT BODY %%%%%
% In the future, we might want to put some additional data here, such
% as when the documentation was converted from wiki to TeX.
%
%
%EndExpansion

\bigskip

\bigskip
}}%
%BeginExpansion
%TCIDATA{Version=5.00.0.2606}
%TCIDATA{LaTeXparent=0,0,geocam.tex}
                      

\chapter{The C++ Language}

%TCIMACRO{\QSubDoc{Include Tutorial}{\input{Cplusplus/Tutorial.tex}}}%
%BeginExpansion
\input{Cplusplus/Tutorial.tex}%
%EndExpansion

%TCIMACRO{%
%\QSubDoc{Include Building_and_modifying_a_memoized_geometry}{%TCIDATA{Version=5.00.0.2606}
%TCIDATA{LaTeXparent=0,0,cplusplus.tex}
                      

%%%%% BEGINNING OF DOCUMENT BODY %%%%%
% html: Beginning of file: `clean.html'
% DOCTYPE HTML PUBLIC "-//W3C//DTD HTML 4.01//EN"
%  This is a (PRE) block.  Make sure it's left aligned or your toc title will be off. 

\section*{Working with a ``memoized-pipeline'' data structure (WORK IN
PROGRESS)}

\label{f0}

\subsection*{Key Words}

geometry, memoized-pipeline, extending, modifying, data structure, geoquant,
quantities, singleton, observer, observable

\subsection*{Authors}

\begin{itemize}
\item Alex Henniges

\item Joseph Thomas
\end{itemize}

\subsection*{Introduction}

The memoized-pipeline is a data structure we developed for investigating
geometries defined on triangulations. It is particularly suited to the
situation in which we need to specify the values of some geometric
quantities (independent variables) and then need to rapidly calculate the
values of some other quantities (the dependent variables). Basically, we
achieve this speedup by trading space for time. Usually, the definitions of
the dependent variables have many intermediate values in common. By saving
these values the first time we compute them, and then reusing them later, we
can avoid a lot of useless recalculation. This strategy of saving calculated
values, which can be found in most algorithms textbooks, is called
``memoization.''

In implementing various geometries, we have already developed code and
techniques for making memoization an automatic part of encoding a geometry.
In this tutorial, we describe how to take advantage of this existing code.

\subsection*{Implementation Details}

The underlying implementation of the pipeline is designed to solve two
problems in a fairly user-friendly way:

\begin{enumerate}
\item We would like to be able to identify geometric quantities with
positions on the triangulation. For example, we can speak of the dihedral
angle associated with a particular edge on a tetrahedron. We would like to
be able to write code in the same way.

\item We would like memoization to be nearly automatic. In other words, when
writing a particular quantity, the programmer shouldn't have to think much
about what happens to memoize that quantity's value.
\end{enumerate}

Taking the programmer's perspective, we can view quantities as being
specified by 3 pieces of information:

\begin{enumerate}
\item A position on the triangulation.

\item A definition of the other quantities (if any) needed to calculate the
value of the current quantity, and where those quantities can be found on
the triangulation.

\item A formula for calculating a quantity's value, given the values of the
other quantities it depends on.
\end{enumerate}

Usually, specifying just these 3 pieces of information is enough to create a
new type of quantity. To help speed the development of quantities, we have
developed a Ruby script, \texttt{makeQuantity.rb}, that generates much of
the source code. This can be invoked at the command line as follows: {\small 
}
\begin{verbatim}
{\small > ruby makeQuantity.rb [quantity]
}
\end{verbatim}

This produces two files, \texttt{\mbox{$[$}quantity\mbox{$]$}.h} and \texttt{%
\mbox{$[$}quantity\mbox{$]$}.cpp}.

\subsection{The ``anatomy'' of \texttt{quantity.h}}

In \texttt{C++}, header files serve several purposes. Among other uses, a
header file can:

\begin{itemize}
\item Specify dependencies on other parts of the project.

\item Define an interface for other parts of your project to use. This
includes:

\begin{itemize}
\item Definitions for new data-types (like classes).

\item Definitions for procedure calls (what arguments a procedure takes, and
what it returns).
\end{itemize}
\end{itemize}

By default, \texttt{makeQuantity.rb} gives you the following header file to
use (here, we chose \texttt{quantity/QUANTITY} as the quantity name, in
practice, this is filled out by the script). {\small }
\begin{verbatim}
{\small #ifndef QUANTITY_H_
}
{\small #define QUANTITY_H_
}
 
{\small #include "geoquant.h"
}
 
{\small /******************REGION 1*******************
}
{\small  * This is where you load the headers of the *
}
{\small  * quantities you require.                   *
}
{\small  *********************************************/
}
 
{\small class quantity : public virtual GeoQuant {
}
{\small protected:
}
{\small   quantity( SIMPLICES );
}
{\small   void recalculate();
}
{\small   /****************REGION 2*********************
}
{\small    * The quantity references you need go here. *
}
{\small    *********************************************/
}
 
{\small public:
}
{\small   ~quantity();
}
{\small   static quantity* At( SIMPLICES );
}
{\small   static void CleanUp();
}
{\small };
}
{\small #endif /* QUANTITY_H_ */
}
\end{verbatim}

The two important areas of the header are labeled \texttt{REGION 1} and 
\texttt{REGION 2}. In region 1, you specify the header files for the
quantities and utilities you use in the rest of your quantity. These \texttt{%
\#include} statements can be thought of as providing definitions for the
data and procedures you want to use in building your quantity. In region 2,
you specify the data associated with a given instance of the quantity;
typically this amounts to several references to other quantities, or a data
structure that manages references to other quantities. Lastly, you will need
to modify the region tagged \texttt{SIMPLICES} so that it reflects a
collection of simplices that describe your quantity's position on the
triangulation.

\subsection{The ``anatomy'' of \texttt{quantity.cpp}}

In general, a \texttt{.cpp} file provides the internal implementation to
support the operations described in the corresponding header file. Editing
this file will be a little more complicated. By default, \texttt{%
makeQuantity.rb} will produce the following \texttt{.cpp} file: {\small }
\begin{verbatim}
{\small #include "quantity.h"
}
 
{\small #include <map>
}
{\small #include <new>
}
{\small using namespace std;
}
 
{\small 
#define map<TriPosition, quantity*, TriPositionCompare> quantityIndex 
}
{\small static quantityIndex* Index = NULL;
}
 
{\small quantity::quantity( SIMPLICES ){
}
{\small   /* REGION 1 */
}
{\small }
}
 
{\small quantity::~quantity(){
}
{\small   /* REGION 2 */
}
{\small }
}
 
{\small void quantity::recalculate(){
}
{\small   /* REGION 3 */
}
{\small }
}
 
{\small quantity* quantity::At( SIMPLICES ){
}
{\small   TriPosition T( NUMSIMPLICES, SIMPLICES );
}
{\small   if( Index == NULL ) Index = new quantityIndex();
}
{\small   quantityIndex::iterator iter = Index->find( T );
}
 
{\small   if( iter == Index->end() ){
}
{\small     quantity* val = new quantity( SIMPLICES );
}
{\small     Index->insert( make_pair( T, val ) );
}
{\small     return val;
}
{\small   } else {
}
{\small     return iter->second;
}
{\small   }
}
{\small }
}
 
{\small void quantity::CleanUp(){
}
{\small   if( Index == NULL ) return;
}
{\small   quantityIndex::iterator iter;
}
{\small   for(iter = Index->begin(); iter != Index->end(); iter++)
}
{\small     delete iter->second;
}
{\small   delete Index;
}
{\small }
}
\end{verbatim}

There are a few smaller areas to fill out, but in general defining the
quantity requires the following three definitions:

\begin{itemize}
\item Region 1 specifies how to obtain references on the quantities your
quantity depends on. Typically, this will involve using the input simplex
information and some utilities for inspecting the triangulation to look up
the quantities needed for later calculations.

\item Region 2 specifies how to release any data structures built up using
dynamic memory. In many cases, this field will be left blank.

\item Region 3 specifies how to calculate the value of an instance of the
quantity. Typically, this will occur in two steps:

\begin{enumerate}
\item Using the quantity references obtained in region 1, we acquire the
current values of the quantities used in the calculation.

\item Using a formula and the values found in step 1, we calculate the value
of the current quantity.
\end{enumerate}
\end{itemize}

\subsection*{An Extended Example}

Perhaps the easiest way to understand the system is by examining a few
working quantities. In this example, we consider the code written to
represent the ``Dual Area Segment\"{} quantity discussed in \"{}Discrete
conformal variations and scalar curvature on piecewise flat two and three
dimensional manifolds''\footnote{%
See URL http://arxiv.org/abs/0906.1560} (in the paper, this quantity is also
called A$_{ij,kl}$).

PICTURES AND A PROSE DESCRIPTION GO HERE

{\small }
\begin{verbatim}
{\small #ifndef DUALAREASEGMENT_H_
}
{\small #define DUALAREASEGMENT_H_
}
 
{\small #include "geoquant.h"
}
{\small #include "triposition.h"
}
 
{\small #include "edge_height.h"
}
{\small #include "face_height.h"
}
 
{\small class DualAreaSegment : public virtual GeoQuant {
}
{\small private:
}
{\small   EdgeHeight* hij_k;
}
{\small   EdgeHeight* hij_l;
}
{\small   FaceHeight* hijk_l;
}
{\small   FaceHeight* hijl_k;
}
 
{\small protected:
}
{\small   DualAreaSegment( Edge& e, Tetra& t );
}
{\small   void recalculate();
}
 
{\small public:
}
{\small   ~DualAreaSegment();
}
{\small   static DualAreaSegment* At( Edge& e, Tetra& t );
}
{\small   static void CleanUp();
}
{\small   static void Record( char* filename );
}
{\small };
}
 
{\small #endif /* DUALAREASEGMENT_H_ */
}
\end{verbatim}

{\small }
\begin{verbatim}
{\small #include "dualareasegment.h"
}
{\small #include "math/miscmath.h"
}
 
{\small #include <stdio.h>
}
 
{\small 
typedef map<TriPosition, DualAreaSegment*, TriPositionCompare> DualAreaSegmentIndex;
}
{\small static DualAreaSegmentIndex* Index = NULL;
}
 
{\small DualAreaSegment::DualAreaSegment( Edge& e, Tetra& t ){
}
{\small 
  StdTetra st = labelTetra( t, e );  // We use the topological tools in miscmath to
}
{\small 
                                     // label the tetrahedron with respect to edge e.
}
 
{\small   Face& fa123 = Triangulation::faceTable[ st.f123 ];
}
{\small   Face& fa124 = Triangulation::faceTable[ st.f124 ];
}
 
{\small 
  hij_k = EdgeHeight::At( e, fa123 );  // Here we use the calculated topological values
}
{\small 
  hij_l = EdgeHeight::At( e, fa124 );  // to look up the edge and face heights required
}
{\small 
  hijk_l = FaceHeight::At( fa123, t ); // to calculate the dual area.
}
{\small   hijl_k = FaceHeight::At( fa124, t );
}
 
{\small 
  hij_k->addDependent(this);  // Here we notify the quantities we reference
}
{\small 
  hij_l->addDependent(this);  // that we wish to observe them (this is the            
}
{\small 
  hijk_l->addDependent(this); // where an important part of our observer-
}
{\small   hijl_k->addDependent(this); // observable design is implemented).
}
{\small }
}
 
{\small 
DualAreaSegment::~DualAreaSegment(){} // We didn't allocate any memory to store
}
{\small 
                                      // this quantity's data, so the destructor 
}
{\small                                       // can be left blank.
}
 
{\small void DualAreaSegment::recalculate(){
}
{\small 
  double Hij_k = hij_k->getValue();   // Step 1: We use the quantity references
}
{\small 
  double Hijk_l = hijk_l->getValue(); // to obtain correct values for the referenced
}
{\small   double Hij_l = hij_l->getValue();   // quantities.
}
{\small   double Hijl_k = hijl_k->getValue();
}
 
{\small 
  // Step 2: We use a formula to calculate the value of the dual-area segment.
}
{\small   value = 0.5*(Hij_k * Hijk_l + Hij_l * Hijl_k); 
}
{\small }
}
 
{\small DualAreaSegment* DualAreaSegment::At( Edge& e, Tetra& t ){
}
{\small   TriPosition T( 2, e.getSerialNumber(), t.getSerialNumber() );
}
{\small   if( Index == NULL ) Index = new DualAreaSegmentIndex();
}
{\small   DualAreaSegmentIndex::iterator iter = Index->find( T );
}
 
{\small   if( iter == Index->end() ){
}
{\small     DualAreaSegment* val = new DualAreaSegment( e, t );
}
{\small     Index->insert( make_pair( T, val ) );
}
{\small     return val;
}
{\small   } else {
}
{\small     return iter->second;
}
{\small   }
}
{\small }
}
 
{\small void DualAreaSegment::CleanUp(){
}
{\small   if( Index == NULL ) return;
}
{\small   DualAreaSegmentIndex::iterator iter;
}
{\small   for(iter = Index->begin(); iter != Index->end(); iter++)
}
{\small     delete iter->second;
}
{\small   delete Index;
}
{\small }
}
 
\end{verbatim}

\subsection*{Common Mistakes}

A panoply of debugging hints should go here.

\subsection*{Fancier Tricks}

Techniques for less common quantities go here.

\subsection*{Limitations, Areas to Improve}

Discussion of the current topological assumptions our code makes.

% html: End of file: `clean.html'

%%%%% END OF DOCUMENT BODY %%%%%
% In the future, we might want to put some additional data here, such
% as when the documentation was converted from wiki to TeX.
%
}}%
%BeginExpansion
%TCIDATA{Version=5.00.0.2606}
%TCIDATA{LaTeXparent=0,0,cplusplus.tex}
                      

%%%%% BEGINNING OF DOCUMENT BODY %%%%%
% html: Beginning of file: `clean.html'
% DOCTYPE HTML PUBLIC "-//W3C//DTD HTML 4.01//EN"
%  This is a (PRE) block.  Make sure it's left aligned or your toc title will be off. 

\section*{Working with a ``memoized-pipeline'' data structure (WORK IN
PROGRESS)}

\label{f0}

\subsection*{Key Words}

geometry, memoized-pipeline, extending, modifying, data structure, geoquant,
quantities, singleton, observer, observable

\subsection*{Authors}

\begin{itemize}
\item Alex Henniges

\item Joseph Thomas
\end{itemize}

\subsection*{Introduction}

The memoized-pipeline is a data structure we developed for investigating
geometries defined on triangulations. It is particularly suited to the
situation in which we need to specify the values of some geometric
quantities (independent variables) and then need to rapidly calculate the
values of some other quantities (the dependent variables). Basically, we
achieve this speedup by trading space for time. Usually, the definitions of
the dependent variables have many intermediate values in common. By saving
these values the first time we compute them, and then reusing them later, we
can avoid a lot of useless recalculation. This strategy of saving calculated
values, which can be found in most algorithms textbooks, is called
``memoization.''

In implementing various geometries, we have already developed code and
techniques for making memoization an automatic part of encoding a geometry.
In this tutorial, we describe how to take advantage of this existing code.

\subsection*{Implementation Details}

The underlying implementation of the pipeline is designed to solve two
problems in a fairly user-friendly way:

\begin{enumerate}
\item We would like to be able to identify geometric quantities with
positions on the triangulation. For example, we can speak of the dihedral
angle associated with a particular edge on a tetrahedron. We would like to
be able to write code in the same way.

\item We would like memoization to be nearly automatic. In other words, when
writing a particular quantity, the programmer shouldn't have to think much
about what happens to memoize that quantity's value.
\end{enumerate}

Taking the programmer's perspective, we can view quantities as being
specified by 3 pieces of information:

\begin{enumerate}
\item A position on the triangulation.

\item A definition of the other quantities (if any) needed to calculate the
value of the current quantity, and where those quantities can be found on
the triangulation.

\item A formula for calculating a quantity's value, given the values of the
other quantities it depends on.
\end{enumerate}

Usually, specifying just these 3 pieces of information is enough to create a
new type of quantity. To help speed the development of quantities, we have
developed a Ruby script, \texttt{makeQuantity.rb}, that generates much of
the source code. This can be invoked at the command line as follows: {\small 
}
\begin{verbatim}
{\small > ruby makeQuantity.rb [quantity]
}
\end{verbatim}

This produces two files, \texttt{\mbox{$[$}quantity\mbox{$]$}.h} and \texttt{%
\mbox{$[$}quantity\mbox{$]$}.cpp}.

\subsection{The ``anatomy'' of \texttt{quantity.h}}

In \texttt{C++}, header files serve several purposes. Among other uses, a
header file can:

\begin{itemize}
\item Specify dependencies on other parts of the project.

\item Define an interface for other parts of your project to use. This
includes:

\begin{itemize}
\item Definitions for new data-types (like classes).

\item Definitions for procedure calls (what arguments a procedure takes, and
what it returns).
\end{itemize}
\end{itemize}

By default, \texttt{makeQuantity.rb} gives you the following header file to
use (here, we chose \texttt{quantity/QUANTITY} as the quantity name, in
practice, this is filled out by the script). {\small }
\begin{verbatim}
{\small #ifndef QUANTITY_H_
}
{\small #define QUANTITY_H_
}
 
{\small #include "geoquant.h"
}
 
{\small /******************REGION 1*******************
}
{\small  * This is where you load the headers of the *
}
{\small  * quantities you require.                   *
}
{\small  *********************************************/
}
 
{\small class quantity : public virtual GeoQuant {
}
{\small protected:
}
{\small   quantity( SIMPLICES );
}
{\small   void recalculate();
}
{\small   /****************REGION 2*********************
}
{\small    * The quantity references you need go here. *
}
{\small    *********************************************/
}
 
{\small public:
}
{\small   ~quantity();
}
{\small   static quantity* At( SIMPLICES );
}
{\small   static void CleanUp();
}
{\small };
}
{\small #endif /* QUANTITY_H_ */
}
\end{verbatim}

The two important areas of the header are labeled \texttt{REGION 1} and 
\texttt{REGION 2}. In region 1, you specify the header files for the
quantities and utilities you use in the rest of your quantity. These \texttt{%
\#include} statements can be thought of as providing definitions for the
data and procedures you want to use in building your quantity. In region 2,
you specify the data associated with a given instance of the quantity;
typically this amounts to several references to other quantities, or a data
structure that manages references to other quantities. Lastly, you will need
to modify the region tagged \texttt{SIMPLICES} so that it reflects a
collection of simplices that describe your quantity's position on the
triangulation.

\subsection{The ``anatomy'' of \texttt{quantity.cpp}}

In general, a \texttt{.cpp} file provides the internal implementation to
support the operations described in the corresponding header file. Editing
this file will be a little more complicated. By default, \texttt{%
makeQuantity.rb} will produce the following \texttt{.cpp} file: {\small }
\begin{verbatim}
{\small #include "quantity.h"
}
 
{\small #include <map>
}
{\small #include <new>
}
{\small using namespace std;
}
 
{\small 
#define map<TriPosition, quantity*, TriPositionCompare> quantityIndex 
}
{\small static quantityIndex* Index = NULL;
}
 
{\small quantity::quantity( SIMPLICES ){
}
{\small   /* REGION 1 */
}
{\small }
}
 
{\small quantity::~quantity(){
}
{\small   /* REGION 2 */
}
{\small }
}
 
{\small void quantity::recalculate(){
}
{\small   /* REGION 3 */
}
{\small }
}
 
{\small quantity* quantity::At( SIMPLICES ){
}
{\small   TriPosition T( NUMSIMPLICES, SIMPLICES );
}
{\small   if( Index == NULL ) Index = new quantityIndex();
}
{\small   quantityIndex::iterator iter = Index->find( T );
}
 
{\small   if( iter == Index->end() ){
}
{\small     quantity* val = new quantity( SIMPLICES );
}
{\small     Index->insert( make_pair( T, val ) );
}
{\small     return val;
}
{\small   } else {
}
{\small     return iter->second;
}
{\small   }
}
{\small }
}
 
{\small void quantity::CleanUp(){
}
{\small   if( Index == NULL ) return;
}
{\small   quantityIndex::iterator iter;
}
{\small   for(iter = Index->begin(); iter != Index->end(); iter++)
}
{\small     delete iter->second;
}
{\small   delete Index;
}
{\small }
}
\end{verbatim}

There are a few smaller areas to fill out, but in general defining the
quantity requires the following three definitions:

\begin{itemize}
\item Region 1 specifies how to obtain references on the quantities your
quantity depends on. Typically, this will involve using the input simplex
information and some utilities for inspecting the triangulation to look up
the quantities needed for later calculations.

\item Region 2 specifies how to release any data structures built up using
dynamic memory. In many cases, this field will be left blank.

\item Region 3 specifies how to calculate the value of an instance of the
quantity. Typically, this will occur in two steps:

\begin{enumerate}
\item Using the quantity references obtained in region 1, we acquire the
current values of the quantities used in the calculation.

\item Using a formula and the values found in step 1, we calculate the value
of the current quantity.
\end{enumerate}
\end{itemize}

\subsection*{An Extended Example}

Perhaps the easiest way to understand the system is by examining a few
working quantities. In this example, we consider the code written to
represent the ``Dual Area Segment\"{} quantity discussed in \"{}Discrete
conformal variations and scalar curvature on piecewise flat two and three
dimensional manifolds''\footnote{%
See URL http://arxiv.org/abs/0906.1560} (in the paper, this quantity is also
called A$_{ij,kl}$).

PICTURES AND A PROSE DESCRIPTION GO HERE

{\small }
\begin{verbatim}
{\small #ifndef DUALAREASEGMENT_H_
}
{\small #define DUALAREASEGMENT_H_
}
 
{\small #include "geoquant.h"
}
{\small #include "triposition.h"
}
 
{\small #include "edge_height.h"
}
{\small #include "face_height.h"
}
 
{\small class DualAreaSegment : public virtual GeoQuant {
}
{\small private:
}
{\small   EdgeHeight* hij_k;
}
{\small   EdgeHeight* hij_l;
}
{\small   FaceHeight* hijk_l;
}
{\small   FaceHeight* hijl_k;
}
 
{\small protected:
}
{\small   DualAreaSegment( Edge& e, Tetra& t );
}
{\small   void recalculate();
}
 
{\small public:
}
{\small   ~DualAreaSegment();
}
{\small   static DualAreaSegment* At( Edge& e, Tetra& t );
}
{\small   static void CleanUp();
}
{\small   static void Record( char* filename );
}
{\small };
}
 
{\small #endif /* DUALAREASEGMENT_H_ */
}
\end{verbatim}

{\small }
\begin{verbatim}
{\small #include "dualareasegment.h"
}
{\small #include "math/miscmath.h"
}
 
{\small #include <stdio.h>
}
 
{\small 
typedef map<TriPosition, DualAreaSegment*, TriPositionCompare> DualAreaSegmentIndex;
}
{\small static DualAreaSegmentIndex* Index = NULL;
}
 
{\small DualAreaSegment::DualAreaSegment( Edge& e, Tetra& t ){
}
{\small 
  StdTetra st = labelTetra( t, e );  // We use the topological tools in miscmath to
}
{\small 
                                     // label the tetrahedron with respect to edge e.
}
 
{\small   Face& fa123 = Triangulation::faceTable[ st.f123 ];
}
{\small   Face& fa124 = Triangulation::faceTable[ st.f124 ];
}
 
{\small 
  hij_k = EdgeHeight::At( e, fa123 );  // Here we use the calculated topological values
}
{\small 
  hij_l = EdgeHeight::At( e, fa124 );  // to look up the edge and face heights required
}
{\small 
  hijk_l = FaceHeight::At( fa123, t ); // to calculate the dual area.
}
{\small   hijl_k = FaceHeight::At( fa124, t );
}
 
{\small 
  hij_k->addDependent(this);  // Here we notify the quantities we reference
}
{\small 
  hij_l->addDependent(this);  // that we wish to observe them (this is the            
}
{\small 
  hijk_l->addDependent(this); // where an important part of our observer-
}
{\small   hijl_k->addDependent(this); // observable design is implemented).
}
{\small }
}
 
{\small 
DualAreaSegment::~DualAreaSegment(){} // We didn't allocate any memory to store
}
{\small 
                                      // this quantity's data, so the destructor 
}
{\small                                       // can be left blank.
}
 
{\small void DualAreaSegment::recalculate(){
}
{\small 
  double Hij_k = hij_k->getValue();   // Step 1: We use the quantity references
}
{\small 
  double Hijk_l = hijk_l->getValue(); // to obtain correct values for the referenced
}
{\small   double Hij_l = hij_l->getValue();   // quantities.
}
{\small   double Hijl_k = hijl_k->getValue();
}
 
{\small 
  // Step 2: We use a formula to calculate the value of the dual-area segment.
}
{\small   value = 0.5*(Hij_k * Hijk_l + Hij_l * Hijl_k); 
}
{\small }
}
 
{\small DualAreaSegment* DualAreaSegment::At( Edge& e, Tetra& t ){
}
{\small   TriPosition T( 2, e.getSerialNumber(), t.getSerialNumber() );
}
{\small   if( Index == NULL ) Index = new DualAreaSegmentIndex();
}
{\small   DualAreaSegmentIndex::iterator iter = Index->find( T );
}
 
{\small   if( iter == Index->end() ){
}
{\small     DualAreaSegment* val = new DualAreaSegment( e, t );
}
{\small     Index->insert( make_pair( T, val ) );
}
{\small     return val;
}
{\small   } else {
}
{\small     return iter->second;
}
{\small   }
}
{\small }
}
 
{\small void DualAreaSegment::CleanUp(){
}
{\small   if( Index == NULL ) return;
}
{\small   DualAreaSegmentIndex::iterator iter;
}
{\small   for(iter = Index->begin(); iter != Index->end(); iter++)
}
{\small     delete iter->second;
}
{\small   delete Index;
}
{\small }
}
 
\end{verbatim}

\subsection*{Common Mistakes}

A panoply of debugging hints should go here.

\subsection*{Fancier Tricks}

Techniques for less common quantities go here.

\subsection*{Limitations, Areas to Improve}

Discussion of the current topological assumptions our code makes.

% html: End of file: `clean.html'

%%%%% END OF DOCUMENT BODY %%%%%
% In the future, we might want to put some additional data here, such
% as when the documentation was converted from wiki to TeX.
%
%
%EndExpansion

\bigskip

\bigskip
%
%EndExpansion

\part{System Documentation}

%TCIMACRO{\QSubDoc{Include system}{\input{system.tex}}}%
%BeginExpansion
\input{system.tex}%
%EndExpansion

%TCIMACRO{\QSubDoc{Include subsystems}{\input{subsystems.tex}}}%
%BeginExpansion
\input{subsystems.tex}%
%EndExpansion

%TCIMACRO{\QSubDoc{Include functions}{%TCIDATA{Version=5.00.0.2606}
%TCIDATA{LaTeXparent=0,0,geocam.tex}
                      

\chapter{Functions}

%TCIMACRO{\QSubDoc{Include Aij_kl}{%TCIDATA{Version=5.00.0.2606}
%TCIDATA{LaTeXparent=1,1,functions.tex}
                      

\section*{\texttt{DualAreaSegment::DualAreaSegment}}

\subsection*{Function Prototype}

\texttt{double DualAreaSegment( Vertex vi, Vertex vj, Vertex vk, Vertex vl)}

\subsection*{Key Words}

Dual area, curvature, partial derivative, Einstein-Hilbert-Regge, geoquant.

\subsection*{Authors}

Daniel Champion

\subsection*{Introduction}

\texttt{DualAreaSegment} calculates the dual area to an edge of a
tetrahedron. \ 

\subsection*{Subsidiaries}

\textbf{Functions:}

\qquad \texttt{EdgeHeight}

\qquad \qquad \texttt{PartialEdge}

\qquad\qquad\texttt{Geometry::angle}

\qquad \texttt{FaceHeight}

\qquad\qquad\texttt{Geometry::dihedralAngle}

\textbf{Global Variables: \ }radii, etas

\textbf{Local Variables:} \ none.

\subsection*{Description}

\texttt{DualAreaSegment} is calculated with the formula:%
\begin{equation*}
\text{\texttt{DualAreaSegment(vi, vj, vk, vl)}}=
\end{equation*}%
\begin{equation*}
\frac{1}{2}\left( 
\begin{array}{c}
\text{\texttt{EdgeHeight(vi,vj,vk)}}\cdot \text{\texttt{%
FaceHeight(vi,vj,vk,vl)}} \\ 
+\text{\texttt{EdgeHeight(vi,vj,vl)}}\cdot \text{\texttt{%
FaceHeight(vi,vj,vl,vk)}}%
\end{array}%
\right) 
\end{equation*}%
\texttt{EdgeHeight} and \texttt{FaceHeight} are calculated with the
following formulae:%
\begin{align*}
\text{\texttt{EdgeHeight(vi, vj, vk)}}& =\frac{\left( \text{\texttt{%
PartialEdge(vi,vk)}}-\text{\texttt{PartialEdge(vi,vj)}}\cos \left( \alpha
_{i,jk}\right) \right) }{\sin \left( \alpha _{i,jk}\right) } \\
\text{\texttt{FaceHeight(vi, vj, vk, vl)}}& =\frac{\left( \text{\texttt{%
EdgeHeight(vi,vj,vl)}}-\text{\texttt{EdgeHeight(vi,vj,vk)}}\cos (\beta
_{ij,kl})\right) }{\sin \left( \beta _{ij,kl}\right) }
\end{align*}%
where $\alpha _{i,jk}$ is the angle at vertex $vi$ of triangle $\left\{
vi,vj,vk\right\} $, and $\beta _{ij,kl}$ is the dihedral angle along edge $%
\left\{ vi,vj\right\} $ of tetrahedron $\left\{ vi,vj,vk,vl\right\} $
(implemented with the functions \texttt{Geometry::angle} and \texttt{%
Geometry::dihedralAngle} respectively).

\texttt{DualAreaSegment} was created for the calculation performed in the
function \texttt{DualArea}, which is used in the computation of the partial
derivatives of curvature. \ These partial derivatives of curvature are used
in the calculation of the second order partial derivatives of the
Einstein-Hilbert-Regge functional for use in the optimization of said
functional using Newton's method. \ 

\subsection*{Practicum}

As an example of the usage of this function, we will calculate the dual area
to the edge $eij=\left\{ vi,vj\right\} $ (see entry: \texttt{DualArea}). \
To do this, we will sum the dual areas to each tetrahedron containing the
edge $eij$. \ 

\bigskip

\texttt{vector\TEXTsymbol{<}int\TEXTsymbol{>} sum\_over =
*(eij.getLocalTetras());}

\texttt{double sum = 0.0;}

\texttt{vector\TEXTsymbol{<}int\TEXTsymbol{>} T\_vertices, e\_vertices;}

\texttt{Tetra T;}

\texttt{Vertex vi,vj,vk,vl;}

\texttt{for(i=0; i\TEXTsymbol{<}sum\_over.size(); ++i) \{}

\qquad\texttt{T = Triangulation::tetraTable[sum\_over[i]];}

\qquad\texttt{T\_vertices = *(T.getLocalVertices());}

\qquad\texttt{e\_vertices = *(eij.getLocalVertices());}

\qquad\texttt{vi = Triangulation::vertexTable[e\_vertices[0]];}

\qquad\texttt{vj = Triangulation::vertexTable[e\_vertices[1]];}

\qquad\texttt{vk = Triangulation::vertexTable[listDifference(\&T\_vertices,
\&e\_vertices)[0]];}

\qquad\texttt{vl = Triangulation::vertexTable[listDifference(\&T\_vertices,
\&e\_vertices)[1]];}

\qquad \texttt{sum += DualAreaSegment(vi, vj, vk, vl);}

\qquad\texttt{\}}

\texttt{return sum;}

\subsection*{Limitations}

\texttt{DualAreaSegment} if fully operational and has no known limitations.
\ The function will output appropriate values provided it receives as input
four distinct vertices that define a tetrahedron.

\subsection*{Revisions}

subversion 757, 7/8/09, \texttt{DualAreaSegment} created.

subversion 1055, 3/12/10, \texttt{DualAreaSegment}\ converted to a geoquant.

\subsection*{Testing}

Trials were run and the calculations returned were verified by hand.

\subsection*{Future Work}

No future work planned.
}}%
%BeginExpansion
%TCIDATA{Version=5.00.0.2606}
%TCIDATA{LaTeXparent=1,1,functions.tex}
                      

\section*{\texttt{DualAreaSegment::DualAreaSegment}}

\subsection*{Function Prototype}

\texttt{double DualAreaSegment( Vertex vi, Vertex vj, Vertex vk, Vertex vl)}

\subsection*{Key Words}

Dual area, curvature, partial derivative, Einstein-Hilbert-Regge, geoquant.

\subsection*{Authors}

Daniel Champion

\subsection*{Introduction}

\texttt{DualAreaSegment} calculates the dual area to an edge of a
tetrahedron. \ 

\subsection*{Subsidiaries}

\textbf{Functions:}

\qquad \texttt{EdgeHeight}

\qquad \qquad \texttt{PartialEdge}

\qquad\qquad\texttt{Geometry::angle}

\qquad \texttt{FaceHeight}

\qquad\qquad\texttt{Geometry::dihedralAngle}

\textbf{Global Variables: \ }radii, etas

\textbf{Local Variables:} \ none.

\subsection*{Description}

\texttt{DualAreaSegment} is calculated with the formula:%
\begin{equation*}
\text{\texttt{DualAreaSegment(vi, vj, vk, vl)}}=
\end{equation*}%
\begin{equation*}
\frac{1}{2}\left( 
\begin{array}{c}
\text{\texttt{EdgeHeight(vi,vj,vk)}}\cdot \text{\texttt{%
FaceHeight(vi,vj,vk,vl)}} \\ 
+\text{\texttt{EdgeHeight(vi,vj,vl)}}\cdot \text{\texttt{%
FaceHeight(vi,vj,vl,vk)}}%
\end{array}%
\right) 
\end{equation*}%
\texttt{EdgeHeight} and \texttt{FaceHeight} are calculated with the
following formulae:%
\begin{align*}
\text{\texttt{EdgeHeight(vi, vj, vk)}}& =\frac{\left( \text{\texttt{%
PartialEdge(vi,vk)}}-\text{\texttt{PartialEdge(vi,vj)}}\cos \left( \alpha
_{i,jk}\right) \right) }{\sin \left( \alpha _{i,jk}\right) } \\
\text{\texttt{FaceHeight(vi, vj, vk, vl)}}& =\frac{\left( \text{\texttt{%
EdgeHeight(vi,vj,vl)}}-\text{\texttt{EdgeHeight(vi,vj,vk)}}\cos (\beta
_{ij,kl})\right) }{\sin \left( \beta _{ij,kl}\right) }
\end{align*}%
where $\alpha _{i,jk}$ is the angle at vertex $vi$ of triangle $\left\{
vi,vj,vk\right\} $, and $\beta _{ij,kl}$ is the dihedral angle along edge $%
\left\{ vi,vj\right\} $ of tetrahedron $\left\{ vi,vj,vk,vl\right\} $
(implemented with the functions \texttt{Geometry::angle} and \texttt{%
Geometry::dihedralAngle} respectively).

\texttt{DualAreaSegment} was created for the calculation performed in the
function \texttt{DualArea}, which is used in the computation of the partial
derivatives of curvature. \ These partial derivatives of curvature are used
in the calculation of the second order partial derivatives of the
Einstein-Hilbert-Regge functional for use in the optimization of said
functional using Newton's method. \ 

\subsection*{Practicum}

As an example of the usage of this function, we will calculate the dual area
to the edge $eij=\left\{ vi,vj\right\} $ (see entry: \texttt{DualArea}). \
To do this, we will sum the dual areas to each tetrahedron containing the
edge $eij$. \ 

\bigskip

\texttt{vector\TEXTsymbol{<}int\TEXTsymbol{>} sum\_over =
*(eij.getLocalTetras());}

\texttt{double sum = 0.0;}

\texttt{vector\TEXTsymbol{<}int\TEXTsymbol{>} T\_vertices, e\_vertices;}

\texttt{Tetra T;}

\texttt{Vertex vi,vj,vk,vl;}

\texttt{for(i=0; i\TEXTsymbol{<}sum\_over.size(); ++i) \{}

\qquad\texttt{T = Triangulation::tetraTable[sum\_over[i]];}

\qquad\texttt{T\_vertices = *(T.getLocalVertices());}

\qquad\texttt{e\_vertices = *(eij.getLocalVertices());}

\qquad\texttt{vi = Triangulation::vertexTable[e\_vertices[0]];}

\qquad\texttt{vj = Triangulation::vertexTable[e\_vertices[1]];}

\qquad\texttt{vk = Triangulation::vertexTable[listDifference(\&T\_vertices,
\&e\_vertices)[0]];}

\qquad\texttt{vl = Triangulation::vertexTable[listDifference(\&T\_vertices,
\&e\_vertices)[1]];}

\qquad \texttt{sum += DualAreaSegment(vi, vj, vk, vl);}

\qquad\texttt{\}}

\texttt{return sum;}

\subsection*{Limitations}

\texttt{DualAreaSegment} if fully operational and has no known limitations.
\ The function will output appropriate values provided it receives as input
four distinct vertices that define a tetrahedron.

\subsection*{Revisions}

subversion 757, 7/8/09, \texttt{DualAreaSegment} created.

subversion 1055, 3/12/10, \texttt{DualAreaSegment}\ converted to a geoquant.

\subsection*{Testing}

Trials were run and the calculations returned were verified by hand.

\subsection*{Future Work}

No future work planned.
%
%EndExpansion

\bigskip

\bigskip

%TCIMACRO{\QSubDoc{Include ApproximatorRun}{\input{ApproximatorRun.tex}}}%
%BeginExpansion
\input{ApproximatorRun.tex}%
%EndExpansion

\bigskip

\bigskip

%TCIMACRO{%
%\QSubDoc{Include Curvature_Partial}{%TCIDATA{Version=5.00.0.2606}
%TCIDATA{LaTeXparent=1,1,functions.tex}
                      

\section*{\texttt{CurvaturePartial::CurvaturePartial}}

\subsection*{Function Prototype}

\texttt{double CurvaturePartial( Vertex v\_i, Vertex v\_l )}

\subsection*{Key Words}

Curvature, Einstein-Hilbert-Regge, functional, partial derivative, geoquant.

\subsection*{Authors}

Daniel Champion

\subsection*{Introduction}

CurvaturePartial calculates the partial derivative of the curvature at a
vertex with respect to the log radius of another (possibly the same) vertex.
\ 

\subsection*{Subsidiaries}

Functions:

\qquad\texttt{isAdjVertex}

\qquad \texttt{DualArea}

\qquad \qquad \texttt{DualAreaSegment}

\qquad \qquad \qquad \texttt{FaceHeight}

\qquad \qquad \qquad \qquad \texttt{EdgeHeight}

\qquad \qquad \qquad \qquad \qquad \texttt{PartialEdge}

\qquad\texttt{listDifference}

\qquad\texttt{listIntersection}

Global Variables: \ curvature, dihedralAngle, eta, length, radius

Local Variables: none.

\subsection*{Description}

\texttt{CurvaturePartial} receives as inputs two vertices $v_{i}$ and $v_{l}$%
. \ The first corresponds to the vertex of interest, the second corresponds
to the vertex of differentiation. \ That is,%
\begin{equation*}
\text{\texttt{CurvaturePartial (v\_i, v\_l)}}=\frac{\partial }{\partial \log
r_{l}}K_{i},
\end{equation*}%
where $r_{l}$ is the radius at vertex $v_{l}$, and $K_{i}$ is the curvature
at vertex $v_{i}$. \ 

The function begins implementation by determining the relationship between $i
$ and $l$ via the trichotomy $v_{i}=v_{l}$, $v_{i}$ is adjacent to $v_{l}$,
or $v_{i}$ and $v_{l}$ are not endpoints of any edge. \ Each of the three
cases are calculated differently. \ The general formula for the variation of
curvature w.r.t. log radii was calculated by Prof. David Glickenstein and is
available at arXiv:0906.1560v1:%
\begin{equation*}
\delta K_{i}=-\sum_{edges\text{ }\left\{ i,j\right\} }\left( 2\frac{%
l_{ij}^{\ast }}{l_{ij}}-\frac{r_{i}^{2}r_{j}^{2}\left( 1-\eta
_{ij}^{2}\right) }{l_{ij}^{2}}K_{ij}\right) \left( \delta f_{j}-\delta
f_{i}\right) +K_{i}\delta f_{i},
\end{equation*}%
where $f_{i}=\log r_{i}$, $l_{ij}$ is the length of the edge $\left\{
i,j\right\} $, and $l_{ij}^{\ast }$ is the dual area calculated with the
function \texttt{DualArea}.

When $i=l$, the formula for the partial derivative $\frac{\partial}{%
\partial\log r_{l}}K_{i}$ becomes:%
\begin{equation*}
\frac{\partial}{\partial\log r_{i}}K_{i}=\sum_{edges\text{ }\left\{
i,j\right\} }\left( 2\frac{l_{ij}^{\ast}}{l_{ij}}-\frac{r_{i}^{2}r_{j}^{2}%
\left( 1-\eta_{ij}^{2}\right) }{l_{ij}^{2}}K_{ij}\right) +K_{i}.
\end{equation*}

When $v_{i}$ is adjacent to $v_{l}$ only one term in the sum survives:%
\begin{equation*}
\frac{\partial}{\partial\log r_{l}}K_{i}=-\left( 2\frac{l_{il}^{\ast}}{l_{il}%
}-\frac{r_{i}^{2}r_{l}^{2}\left( 1-\eta_{il}^{2}\right) }{l_{il}^{2}}%
K_{il}\right) .
\end{equation*}

When $v_{i}$ and $v_{j}$ are not endpoints of any edge the partial
derivative is zero. \ 

This function was created to assist in the computation of the second
derivatives of the normalized Einstein-Hilbert-Regge functional:%
\begin{equation*}
EHR=\frac{\sum K_{i}}{\sqrt[3]{\sum\limits_{tetra\text{ }t}Vol(t)}}.
\end{equation*}

Surprisingly the first derivatives of the normalized $EHR$ functional do not
require the formula for the partial derivative of curvature since 
\begin{equation*}
\frac{\partial EHR}{\partial\log r_{i}}=K_{i}.
\end{equation*}

However, the second order partial derivatives of $EHR$ certainly require the
formulas for $\frac{\partial}{\partial\log r_{l}}K_{i}$ given above. \ These
second order partial derivatives are used to construct a Hessian matrix
which is then used the optimization of the $EHR$ functional using Newton's
method (implemented by \texttt{Newtons\_Method}).

\subsection*{Practicum}

When called, \texttt{CurvaturePartial(v\_i, v\_l)} returns the partial
derivative $\frac{\partial }{\partial \log r_{l}}K_{i}$. \ An example of its
usage is in the calculation of the second order partial derivatives of the
normalized $EHR$ functional. \ For this example let 
\begin{align*}
\text{\texttt{VolSumPartial\_i}}& =\sum_{tetra\text{ }t}\frac{\partial V_{t}%
}{\partial \log r_{i}}, \\
\text{\texttt{VolSumPartial\_j}}& =\sum_{tetra\text{ }t}\frac{\partial V_{t}%
}{\partial \log r_{j}}, \\
\text{\texttt{VolSumSecondPartial}}& =\sum_{tetra\text{ }t}\frac{\partial
^{2}V_{t}}{\partial \log r_{i}\partial \log r_{j}} \\
K& =\sum_{i}K_{i}, \\
V& =\sum_{tetra\text{ }t}V_{t}.
\end{align*}%
Then the second order partial derivative $\frac{\partial ^{2}EHR}{\partial
\log r_{i}\partial \log r_{j}}$ is calculated by:

\bigskip

\qquad \texttt{result = pow(V,
(-4.0/3.0))*(1.0/3.0)*(3*V*CurvaturePartial(i,j)}

\qquad\qquad\qquad \texttt{%
-Geometry::curvature(Triangulation::vertexTable[i])*VolSumPartial\_j}

\qquad\qquad\qquad \texttt{%
-Geometry::curvature(Triangulation::vertexTable[j])*VolSumPartial\_i}

\qquad\qquad\qquad\texttt{+(4.0/3.0)*K*pow(V,
-1.0)*VolSumPartial\_i*VolSumPartial\_j}

\qquad\qquad\qquad\texttt{-K*VolSumSecondPartial);}

\bigskip

\subsection*{Limitations}

The function \texttt{CurvaturePartial} is operational for all pairs of input
integers $i$ and $l$ that are in the vertex table. \ If one of the arguments
is not in the vertex table, the function will output zero. \ 

\subsection*{Revisions}

subversion 757, 7/6/09, \texttt{CurvaturePartial} created.

subversion 1055, 3/12/10, \texttt{CurvaturePartial}\ converted to a geoquant.

\subsection*{Testing}

This function has not been tested.

\subsection*{Future Work}

Using the calculation of the partial derivative of curvature in Mathematica,
it should be compared to the output from \texttt{CurvaturePartial}.
}}%
%BeginExpansion
%TCIDATA{Version=5.00.0.2606}
%TCIDATA{LaTeXparent=1,1,functions.tex}
                      

\section*{\texttt{CurvaturePartial::CurvaturePartial}}

\subsection*{Function Prototype}

\texttt{double CurvaturePartial( Vertex v\_i, Vertex v\_l )}

\subsection*{Key Words}

Curvature, Einstein-Hilbert-Regge, functional, partial derivative, geoquant.

\subsection*{Authors}

Daniel Champion

\subsection*{Introduction}

CurvaturePartial calculates the partial derivative of the curvature at a
vertex with respect to the log radius of another (possibly the same) vertex.
\ 

\subsection*{Subsidiaries}

Functions:

\qquad\texttt{isAdjVertex}

\qquad \texttt{DualArea}

\qquad \qquad \texttt{DualAreaSegment}

\qquad \qquad \qquad \texttt{FaceHeight}

\qquad \qquad \qquad \qquad \texttt{EdgeHeight}

\qquad \qquad \qquad \qquad \qquad \texttt{PartialEdge}

\qquad\texttt{listDifference}

\qquad\texttt{listIntersection}

Global Variables: \ curvature, dihedralAngle, eta, length, radius

Local Variables: none.

\subsection*{Description}

\texttt{CurvaturePartial} receives as inputs two vertices $v_{i}$ and $v_{l}$%
. \ The first corresponds to the vertex of interest, the second corresponds
to the vertex of differentiation. \ That is,%
\begin{equation*}
\text{\texttt{CurvaturePartial (v\_i, v\_l)}}=\frac{\partial }{\partial \log
r_{l}}K_{i},
\end{equation*}%
where $r_{l}$ is the radius at vertex $v_{l}$, and $K_{i}$ is the curvature
at vertex $v_{i}$. \ 

The function begins implementation by determining the relationship between $i
$ and $l$ via the trichotomy $v_{i}=v_{l}$, $v_{i}$ is adjacent to $v_{l}$,
or $v_{i}$ and $v_{l}$ are not endpoints of any edge. \ Each of the three
cases are calculated differently. \ The general formula for the variation of
curvature w.r.t. log radii was calculated by Prof. David Glickenstein and is
available at arXiv:0906.1560v1:%
\begin{equation*}
\delta K_{i}=-\sum_{edges\text{ }\left\{ i,j\right\} }\left( 2\frac{%
l_{ij}^{\ast }}{l_{ij}}-\frac{r_{i}^{2}r_{j}^{2}\left( 1-\eta
_{ij}^{2}\right) }{l_{ij}^{2}}K_{ij}\right) \left( \delta f_{j}-\delta
f_{i}\right) +K_{i}\delta f_{i},
\end{equation*}%
where $f_{i}=\log r_{i}$, $l_{ij}$ is the length of the edge $\left\{
i,j\right\} $, and $l_{ij}^{\ast }$ is the dual area calculated with the
function \texttt{DualArea}.

When $i=l$, the formula for the partial derivative $\frac{\partial}{%
\partial\log r_{l}}K_{i}$ becomes:%
\begin{equation*}
\frac{\partial}{\partial\log r_{i}}K_{i}=\sum_{edges\text{ }\left\{
i,j\right\} }\left( 2\frac{l_{ij}^{\ast}}{l_{ij}}-\frac{r_{i}^{2}r_{j}^{2}%
\left( 1-\eta_{ij}^{2}\right) }{l_{ij}^{2}}K_{ij}\right) +K_{i}.
\end{equation*}

When $v_{i}$ is adjacent to $v_{l}$ only one term in the sum survives:%
\begin{equation*}
\frac{\partial}{\partial\log r_{l}}K_{i}=-\left( 2\frac{l_{il}^{\ast}}{l_{il}%
}-\frac{r_{i}^{2}r_{l}^{2}\left( 1-\eta_{il}^{2}\right) }{l_{il}^{2}}%
K_{il}\right) .
\end{equation*}

When $v_{i}$ and $v_{j}$ are not endpoints of any edge the partial
derivative is zero. \ 

This function was created to assist in the computation of the second
derivatives of the normalized Einstein-Hilbert-Regge functional:%
\begin{equation*}
EHR=\frac{\sum K_{i}}{\sqrt[3]{\sum\limits_{tetra\text{ }t}Vol(t)}}.
\end{equation*}

Surprisingly the first derivatives of the normalized $EHR$ functional do not
require the formula for the partial derivative of curvature since 
\begin{equation*}
\frac{\partial EHR}{\partial\log r_{i}}=K_{i}.
\end{equation*}

However, the second order partial derivatives of $EHR$ certainly require the
formulas for $\frac{\partial}{\partial\log r_{l}}K_{i}$ given above. \ These
second order partial derivatives are used to construct a Hessian matrix
which is then used the optimization of the $EHR$ functional using Newton's
method (implemented by \texttt{Newtons\_Method}).

\subsection*{Practicum}

When called, \texttt{CurvaturePartial(v\_i, v\_l)} returns the partial
derivative $\frac{\partial }{\partial \log r_{l}}K_{i}$. \ An example of its
usage is in the calculation of the second order partial derivatives of the
normalized $EHR$ functional. \ For this example let 
\begin{align*}
\text{\texttt{VolSumPartial\_i}}& =\sum_{tetra\text{ }t}\frac{\partial V_{t}%
}{\partial \log r_{i}}, \\
\text{\texttt{VolSumPartial\_j}}& =\sum_{tetra\text{ }t}\frac{\partial V_{t}%
}{\partial \log r_{j}}, \\
\text{\texttt{VolSumSecondPartial}}& =\sum_{tetra\text{ }t}\frac{\partial
^{2}V_{t}}{\partial \log r_{i}\partial \log r_{j}} \\
K& =\sum_{i}K_{i}, \\
V& =\sum_{tetra\text{ }t}V_{t}.
\end{align*}%
Then the second order partial derivative $\frac{\partial ^{2}EHR}{\partial
\log r_{i}\partial \log r_{j}}$ is calculated by:

\bigskip

\qquad \texttt{result = pow(V,
(-4.0/3.0))*(1.0/3.0)*(3*V*CurvaturePartial(i,j)}

\qquad\qquad\qquad \texttt{%
-Geometry::curvature(Triangulation::vertexTable[i])*VolSumPartial\_j}

\qquad\qquad\qquad \texttt{%
-Geometry::curvature(Triangulation::vertexTable[j])*VolSumPartial\_i}

\qquad\qquad\qquad\texttt{+(4.0/3.0)*K*pow(V,
-1.0)*VolSumPartial\_i*VolSumPartial\_j}

\qquad\qquad\qquad\texttt{-K*VolSumSecondPartial);}

\bigskip

\subsection*{Limitations}

The function \texttt{CurvaturePartial} is operational for all pairs of input
integers $i$ and $l$ that are in the vertex table. \ If one of the arguments
is not in the vertex table, the function will output zero. \ 

\subsection*{Revisions}

subversion 757, 7/6/09, \texttt{CurvaturePartial} created.

subversion 1055, 3/12/10, \texttt{CurvaturePartial}\ converted to a geoquant.

\subsection*{Testing}

This function has not been tested.

\subsection*{Future Work}

Using the calculation of the partial derivative of curvature in Mathematica,
it should be compared to the output from \texttt{CurvaturePartial}.
%
%EndExpansion

\bigskip

\bigskip

%TCIMACRO{\QSubDoc{Include dij}{%TCIDATA{Version=5.00.0.2606}
%TCIDATA{LaTeXparent=1,1,functions.tex}
                      

\section*{\texttt{PartialEdge::PartialEdge}}

\subsection*{Function Prototype}

\texttt{double PartialEdge( Vertex vi, Vertex vj)}

\subsection*{Key Words}

Partial length, geoquant.

\subsection*{Authors}

Daniel Champion, ???

\subsection*{Introduction}

The function \texttt{PartialEdge} calculates the distance from a vertex to
the center of an edge (as determined by the center of a decorated triangle).

\subsection*{Subsidiaries}

\textbf{Functions:}

\qquad\texttt{Geometry::length}

\qquad\texttt{listIntersection}

\textbf{Global Variables:} \ radii, etas.

\textbf{Local Variables:} \ Vertex vi, vj.

\subsection*{Description}

The function \texttt{PartialEdge} is calculated with the simple formula:%
\begin{equation*}
\mathtt{PartialEdge}\text{\texttt{(vi,vj)}}=\frac{%
L_{ij}^{2}+r_{i}^{2}-r_{j}^{2}}{2L_{ij}},
\end{equation*}%
where $r_{i},r_{j}$ are the radii at vertices \texttt{vi}, and \texttt{vj}
respectively, and $L_{ij}$ is the length of the edge $\left\{ vi,vj\right\} $%
. \ Notice that this formula is not symmetric in $i$ and $j$. \ 

This function plays an important role in several areas of the project
including curvature, Dirichlet energy, and the optimization of the
Einstein-Hilbert-Regge functional. \ \texttt{PartialEdge} is used in the
calculation of several quantities used in the implementation of the \texttt{%
CurvaturePartial} function, which is used in the optimization of the
normalized Einstein-Hilbert-Regge functional.

\subsection*{Practicum}

An example of the use of this function is in the calculation of the edge
height function \texttt{EdgeHeight}:

\qquad \texttt{double EdgeHeight( Vertex vi, Vertex vj, Vertex vk) \{}

\qquad\qquad\texttt{Face fijk;}

\qquad\qquad\texttt{vector\TEXTsymbol{<}int\TEXTsymbol{>} temp\_ij = }

\qquad\qquad\qquad\texttt{listIntersection(vi.getLocalFaces(),
vj.getLocalFaces());}

\qquad\qquad\texttt{vector\TEXTsymbol{<}int\TEXTsymbol{>} temp = }

\qquad\qquad\qquad\texttt{listIntersection( \&temp\_ij, vk.getLocalFaces());}

\qquad\qquad\texttt{fijk = Triangulation::faceTable[temp[0]];}

\qquad \qquad \texttt{double result = (PartialEdge(vi, vk)-PartialEdge(vi,vj)%
}

\qquad\qquad\qquad\texttt{*cos(Geometry::angle(vi,
fijk)))/sin(Geometry::angle(vi, fijk));}

\qquad\qquad\texttt{return result;}

\qquad\qquad\texttt{\}}

\subsection*{Limitations}

\texttt{PartialEdge} must receive as input two vertices that define an edge
in the triangulation. \ \texttt{PartialEdge} returns distinct values for
each permutation of the input vertices. \ 

\subsection*{Revisions}

subversion 757, 7/13/09, \texttt{PartialEdge} created.

subversion 1055, 3/12/10, \texttt{PartialEdge}\ converted to a geoquant.

\subsection*{Testing}

\texttt{dij} was tested by working out several examples by hand.

\subsection*{Future Work}

This function has been added to the Geometry class geoquants, and thus this
entry needs to be updated.
}}%
%BeginExpansion
%TCIDATA{Version=5.00.0.2606}
%TCIDATA{LaTeXparent=1,1,functions.tex}
                      

\section*{\texttt{PartialEdge::PartialEdge}}

\subsection*{Function Prototype}

\texttt{double PartialEdge( Vertex vi, Vertex vj)}

\subsection*{Key Words}

Partial length, geoquant.

\subsection*{Authors}

Daniel Champion, ???

\subsection*{Introduction}

The function \texttt{PartialEdge} calculates the distance from a vertex to
the center of an edge (as determined by the center of a decorated triangle).

\subsection*{Subsidiaries}

\textbf{Functions:}

\qquad\texttt{Geometry::length}

\qquad\texttt{listIntersection}

\textbf{Global Variables:} \ radii, etas.

\textbf{Local Variables:} \ Vertex vi, vj.

\subsection*{Description}

The function \texttt{PartialEdge} is calculated with the simple formula:%
\begin{equation*}
\mathtt{PartialEdge}\text{\texttt{(vi,vj)}}=\frac{%
L_{ij}^{2}+r_{i}^{2}-r_{j}^{2}}{2L_{ij}},
\end{equation*}%
where $r_{i},r_{j}$ are the radii at vertices \texttt{vi}, and \texttt{vj}
respectively, and $L_{ij}$ is the length of the edge $\left\{ vi,vj\right\} $%
. \ Notice that this formula is not symmetric in $i$ and $j$. \ 

This function plays an important role in several areas of the project
including curvature, Dirichlet energy, and the optimization of the
Einstein-Hilbert-Regge functional. \ \texttt{PartialEdge} is used in the
calculation of several quantities used in the implementation of the \texttt{%
CurvaturePartial} function, which is used in the optimization of the
normalized Einstein-Hilbert-Regge functional.

\subsection*{Practicum}

An example of the use of this function is in the calculation of the edge
height function \texttt{EdgeHeight}:

\qquad \texttt{double EdgeHeight( Vertex vi, Vertex vj, Vertex vk) \{}

\qquad\qquad\texttt{Face fijk;}

\qquad\qquad\texttt{vector\TEXTsymbol{<}int\TEXTsymbol{>} temp\_ij = }

\qquad\qquad\qquad\texttt{listIntersection(vi.getLocalFaces(),
vj.getLocalFaces());}

\qquad\qquad\texttt{vector\TEXTsymbol{<}int\TEXTsymbol{>} temp = }

\qquad\qquad\qquad\texttt{listIntersection( \&temp\_ij, vk.getLocalFaces());}

\qquad\qquad\texttt{fijk = Triangulation::faceTable[temp[0]];}

\qquad \qquad \texttt{double result = (PartialEdge(vi, vk)-PartialEdge(vi,vj)%
}

\qquad\qquad\qquad\texttt{*cos(Geometry::angle(vi,
fijk)))/sin(Geometry::angle(vi, fijk));}

\qquad\qquad\texttt{return result;}

\qquad\qquad\texttt{\}}

\subsection*{Limitations}

\texttt{PartialEdge} must receive as input two vertices that define an edge
in the triangulation. \ \texttt{PartialEdge} returns distinct values for
each permutation of the input vertices. \ 

\subsection*{Revisions}

subversion 757, 7/13/09, \texttt{PartialEdge} created.

subversion 1055, 3/12/10, \texttt{PartialEdge}\ converted to a geoquant.

\subsection*{Testing}

\texttt{dij} was tested by working out several examples by hand.

\subsection*{Future Work}

This function has been added to the Geometry class geoquants, and thus this
entry needs to be updated.
%
%EndExpansion

\bigskip

\bigskip

%TCIMACRO{\QSubDoc{Include EHR_Partial}{%TCIDATA{Version=5.00.0.2606}
%TCIDATA{LaTeXparent=1,1,functions.tex}
                      

\section*{\texttt{EHRPartial::EHRPartial}}

\subsection*{Function Prototype}

\texttt{double EHRPartial(int i)}

\subsection*{Key Words}

Einstein-Hilbert-Regge, functional, Newton's method, partial derivative,
geoquant.

\subsection*{Authors}

Daniel Champion

\subsection*{Introduction}

\texttt{EHRPartial} calculates the partial derivative of the normalized
Einstein-Hilbert-Regge functional with respect to log radii. \ 

\subsection*{Subsidiaries}

\textbf{Functions:}

\qquad \texttt{TotalVolume}

\qquad \texttt{TotalCurvature}

\qquad \texttt{VolumePartial}

\qquad\qquad\texttt{listDifference}

\qquad\qquad\texttt{listIntersection}

\textbf{Global Variables:} radii, etas, curvature, volume

\textbf{Local Variables:} \ none

\subsection*{Description}

The normalized Einstein-Hilbert-Regge functional is given by the expression:%
\begin{equation*}
EHR=\frac{\sum\limits_{j}K_{j}}{\sqrt[3]{\sum\limits_{tetra\text{ }t}V_{t}}},
\end{equation*}%
where $K_{i}$ is the curvature at vertex $j$, and $V_{t}$ is the volume of
tetrahedron $t$. \ It can be shown (see arXiv:0906.1560v1) that 
\begin{equation*}
\frac{\partial }{\partial \log r_{i}}\left( \sum\limits_{j}K_{j}\right)
=K_{i},
\end{equation*}%
hence the partial derivative of the normalized EHR functional becomes:%
\begin{align*}
\frac{\partial }{\partial \log r_{i}}EHR& =\frac{K_{i}\sqrt[3]{%
\sum\limits_{tetra\text{ }t}V_{t}}-\frac{1}{3}\left( \sum\limits_{tetra\text{
}t}V_{t}\right) ^{-\frac{2}{3}}\sum\limits_{tetra\text{ }t}\frac{\partial
V_{t}}{\partial \log r_{i}}\sum\limits_{j}K_{j}}{\left( \sum\limits_{tetra%
\text{ }t}V_{t}\right) ^{\frac{2}{3}}} \\
& =V^{-\frac{4}{3}}\left( K_{i}V-\frac{1}{3}K\sum\limits_{tetra\text{ }t}%
\frac{\partial V_{t}}{\partial \log r_{i}}\right) ,
\end{align*}%
where $V$ is the total volume of all tetrahedra in the triangulation and $K$
is the sum of the curvatures over all vertices in the triangulation. \ 
\texttt{EHRPartial (v\_i)} calculates $\frac{\partial }{\partial \log r_{i}}%
EHR$. \ 

The primary use of this function is in the calculation of the negative
gradient of the EHR functional for use in optimization of the functional
using Newton's method. \ The formula for $\frac{\partial }{\partial \log
r_{i}}EHR$ given above was also used in the calculation of the second order
partial derivatives of the EHR functional, implemented in \texttt{%
EHRSecondPartial}.

\subsection*{Practicum}

As an example of the use of this function, the calculation of the gradient
of the EHR functional will be calculated. \ The negative gradient will be
outputted as the array \texttt{EHRneg\_gradient}.

\bigskip

\qquad\texttt{double EHRneg\_gradient[Triangulation::vertexTable.size()];}

\qquad \texttt{for(int i=0; i \TEXTsymbol{<}
Triangulation::vertexTable.size(); ++i) \{}

\qquad \qquad \texttt{Vertex v\_i = Triangulation::vertexTable[i+1];}

\qquad \qquad \texttt{EHRneg\_gradient[i] = -1.0*EHRPartial(v\_i);}

\qquad\qquad\texttt{\}}

\subsection*{Limitations}

The function \texttt{EHRPartial} is fully functional with no known
limitations. \ It will return appropriate values so long as it is called
with an integer in the vertex table. \ 

\subsection*{Revisions}

List the major revisions to the function with dates and a one sentence
comment. \ Example:

subversion 757, 7/7/09, \texttt{EHRPartial} created.

subversion 1055, 3/12/10, \texttt{EHRPartial}\ converted to a geoquant.

\subsection*{Testing}

This function has not been tested.

\subsection*{Future Work}

A testing regime should be instituted for this function. \ 
}}%
%BeginExpansion
%TCIDATA{Version=5.00.0.2606}
%TCIDATA{LaTeXparent=1,1,functions.tex}
                      

\section*{\texttt{EHRPartial::EHRPartial}}

\subsection*{Function Prototype}

\texttt{double EHRPartial(int i)}

\subsection*{Key Words}

Einstein-Hilbert-Regge, functional, Newton's method, partial derivative,
geoquant.

\subsection*{Authors}

Daniel Champion

\subsection*{Introduction}

\texttt{EHRPartial} calculates the partial derivative of the normalized
Einstein-Hilbert-Regge functional with respect to log radii. \ 

\subsection*{Subsidiaries}

\textbf{Functions:}

\qquad \texttt{TotalVolume}

\qquad \texttt{TotalCurvature}

\qquad \texttt{VolumePartial}

\qquad\qquad\texttt{listDifference}

\qquad\qquad\texttt{listIntersection}

\textbf{Global Variables:} radii, etas, curvature, volume

\textbf{Local Variables:} \ none

\subsection*{Description}

The normalized Einstein-Hilbert-Regge functional is given by the expression:%
\begin{equation*}
EHR=\frac{\sum\limits_{j}K_{j}}{\sqrt[3]{\sum\limits_{tetra\text{ }t}V_{t}}},
\end{equation*}%
where $K_{i}$ is the curvature at vertex $j$, and $V_{t}$ is the volume of
tetrahedron $t$. \ It can be shown (see arXiv:0906.1560v1) that 
\begin{equation*}
\frac{\partial }{\partial \log r_{i}}\left( \sum\limits_{j}K_{j}\right)
=K_{i},
\end{equation*}%
hence the partial derivative of the normalized EHR functional becomes:%
\begin{align*}
\frac{\partial }{\partial \log r_{i}}EHR& =\frac{K_{i}\sqrt[3]{%
\sum\limits_{tetra\text{ }t}V_{t}}-\frac{1}{3}\left( \sum\limits_{tetra\text{
}t}V_{t}\right) ^{-\frac{2}{3}}\sum\limits_{tetra\text{ }t}\frac{\partial
V_{t}}{\partial \log r_{i}}\sum\limits_{j}K_{j}}{\left( \sum\limits_{tetra%
\text{ }t}V_{t}\right) ^{\frac{2}{3}}} \\
& =V^{-\frac{4}{3}}\left( K_{i}V-\frac{1}{3}K\sum\limits_{tetra\text{ }t}%
\frac{\partial V_{t}}{\partial \log r_{i}}\right) ,
\end{align*}%
where $V$ is the total volume of all tetrahedra in the triangulation and $K$
is the sum of the curvatures over all vertices in the triangulation. \ 
\texttt{EHRPartial (v\_i)} calculates $\frac{\partial }{\partial \log r_{i}}%
EHR$. \ 

The primary use of this function is in the calculation of the negative
gradient of the EHR functional for use in optimization of the functional
using Newton's method. \ The formula for $\frac{\partial }{\partial \log
r_{i}}EHR$ given above was also used in the calculation of the second order
partial derivatives of the EHR functional, implemented in \texttt{%
EHRSecondPartial}.

\subsection*{Practicum}

As an example of the use of this function, the calculation of the gradient
of the EHR functional will be calculated. \ The negative gradient will be
outputted as the array \texttt{EHRneg\_gradient}.

\bigskip

\qquad\texttt{double EHRneg\_gradient[Triangulation::vertexTable.size()];}

\qquad \texttt{for(int i=0; i \TEXTsymbol{<}
Triangulation::vertexTable.size(); ++i) \{}

\qquad \qquad \texttt{Vertex v\_i = Triangulation::vertexTable[i+1];}

\qquad \qquad \texttt{EHRneg\_gradient[i] = -1.0*EHRPartial(v\_i);}

\qquad\qquad\texttt{\}}

\subsection*{Limitations}

The function \texttt{EHRPartial} is fully functional with no known
limitations. \ It will return appropriate values so long as it is called
with an integer in the vertex table. \ 

\subsection*{Revisions}

List the major revisions to the function with dates and a one sentence
comment. \ Example:

subversion 757, 7/7/09, \texttt{EHRPartial} created.

subversion 1055, 3/12/10, \texttt{EHRPartial}\ converted to a geoquant.

\subsection*{Testing}

This function has not been tested.

\subsection*{Future Work}

A testing regime should be instituted for this function. \ 
%
%EndExpansion

\bigskip

\bigskip

%TCIMACRO{%
%\QSubDoc{Include EHR_Second_Partial}{%TCIDATA{Version=5.00.0.2606}
%TCIDATA{LaTeXparent=1,1,functions.tex}
                      

\section*{\texttt{EHRSecondPartial::EHRSecondPartial}}

\subsection*{Function Prototype}

\texttt{double EHRSecondPartial (Vertex v\_i, Vertex v\_j)}

\subsection*{Key Words}

Einstein-Hilbert-Regge, functional, partial derivative, Hessian, geoquant.

\subsection*{Authors}

Daniel Champion

\subsection*{Introduction}

\texttt{EHRSecondPartial} calculates the second order partial derivatives of
the normalized Einstein-Hilbert-Regge functional with respect to log radii.
\ 

\subsection*{Subsidiaries}

\textbf{Functions:}

\qquad \texttt{CurvaturePartial}

\qquad\qquad\texttt{isAdjVertex}

\qquad \qquad \texttt{DualArea}

\qquad \qquad \qquad \texttt{DualAreaSegment}

\qquad \qquad \qquad \qquad \texttt{FaceHeight}

\qquad \qquad \qquad \qquad \qquad \texttt{EdgeHeight}

\qquad \qquad \qquad \qquad \qquad \qquad \texttt{PartialEdge}

\qquad\qquad\texttt{listDifference}

\qquad\qquad\texttt{listIntersection}

\qquad \texttt{TotalCurvature}

\qquad\qquad\texttt{Geometry::curvature}

\qquad \texttt{TotalVolume}

\qquad\qquad\texttt{Geometry::volume}

\qquad \texttt{VolumePartial}

\qquad\qquad\texttt{listDifference}

\qquad\qquad\texttt{listIntersection.}

\qquad \texttt{VolumeSecondPartial}

\textbf{Global Variables: }\ radii, etas.

\textbf{Local Variables:} \ none.

\subsection*{Description}

The normalized Einstein-Hilbert-Regge functional is given by the expression:%
\begin{equation*}
EHR=\frac{\sum\limits_{j}K_{j}}{\sqrt[3]{\sum\limits_{tetra\text{ }t}V_{t}}},
\end{equation*}
where $K_{i}$ is the curvature at vertex $j$, and $V_{t}$ is the volume of
tetrahedron $t$. \ It can be shown (see arXiv:0906.1560v1) that 
\begin{equation*}
\frac{\partial}{\partial\log r_{i}}\left( \sum\limits_{j}K_{j}\right) =K_{i},
\end{equation*}
hence the partial derivative of the normalized EHR functional simplifies to
become:%
\begin{equation*}
\frac{\partial}{\partial\log r_{i}}EHR=V^{-\frac{4}{3}}\left( K_{i}V-\frac {1%
}{3}K\sum\limits_{tetra\text{ }t}\frac{\partial V_{t}}{\partial\log r_{i}}%
\right) ,
\end{equation*}
where $V$ is the total volume of all tetrahedra in the triangulation and $K$
is the sum of the curvatures over all vertices in the triangulation. \
Differentiating this result with respect to $\log r_{j}$ yields:%
\begin{equation*}
\frac{\partial^{2}}{\partial\log r_{i}\partial\log r_{j}}EHR=V^{-\frac{4}{3}%
}\left( 
\begin{array}{c}
V\frac{\partial K_{i}}{\partial\log r_{j}}-\frac{1}{3}K_{i}\sum\limits_{t}%
\frac{\partial V_{t}}{\partial\log r_{j}}-\frac{1}{3}K_{j}\sum\limits_{t}%
\frac{\partial V_{t}}{\partial\log r_{i}} \\ 
+\frac{4}{9}KV^{-1}\sum\limits_{t}\frac{\partial V_{t}}{\partial\log r_{j}}%
\sum\limits_{t}\frac{\partial V_{t}}{\partial\log r_{i}}-\frac{1}{3}%
K\sum\limits_{t}\frac{\partial^{2}V_{t}}{\partial\log r_{i}\partial\log r_{j}%
}%
\end{array}
\right) .
\end{equation*}

When called, \texttt{EHRSecondPartial} calculates the formula above, that is:%
\begin{equation*}
\text{\texttt{EHR\_Second\_Partial (i,j)}}=\frac{\partial ^{2}}{\partial
\log r_{i}\partial \log r_{j}}EHR.
\end{equation*}

The use of this function is in the population of the Hessian matrix for the
normalized EHR functional. \ This Hessian matrix is used in the optimization
of the EHR functional using Newton's method.

\subsection*{Practicum}

As an example of the usage of \texttt{EHRSecondPartial}, the Hessian matrix
of the normalized EHR functional will be populated. \ In this example, the
Hessian matrix is the array EHRhessian. \ The example reduced computation
time by only calling \texttt{EHRSecondPartial} for the upper triangular
portion of the EHRhessian array, and symmetrically copies the entries above
the diagonal to the corresponding location below the diagonal. \
Furthermore, in C++ arrays begin indexing at zero, however the vertices of
the triangulations begin indexing at 1, requiring a shift of one in the
population step. \ Note that triangulations that do not label the vertices
consecutively will not be compatible with the following code. \ 

\bigskip

\qquad\texttt{double
EHRhessian[Triangulation::vertexTable.size()][Triangulation::vertexTable.size()];%
}

\qquad\texttt{for(int i = 0; i \TEXTsymbol{<}
Triangulation::vertexTable.size(); ++i) \{}

\qquad \qquad \texttt{for(int j = 0; j \TEXTsymbol{<}
Triangulation::vertexTable.size(); ++j) \{}

\qquad \qquad \qquad \texttt{Vertex vi = Triangulation::vertexYable[i+1];}

\qquad \qquad \qquad \texttt{Vertex vj = Triangulation::vertexTable[j+1];}

\qquad \qquad \qquad \texttt{if (i \TEXTsymbol{<}= j) \{}

\qquad \qquad \qquad \qquad \texttt{EHRhessian[i][j]=EHRSecondPartial( vi ,
vj );}

\qquad\qquad\qquad\qquad\texttt{EHRhessian[j][i]=EHRhessian[i][j];}

\qquad\qquad\qquad\qquad\texttt{\}}

\qquad\qquad\qquad\texttt{\}}

\qquad\qquad\texttt{\}}

\subsection*{Limitations}

\texttt{EHRSecondPartial} is fully operational with no known limitations. \
The function will output appropriate values provided it receives as inputs a
pair of integers in the vertex table. \ 

\subsection*{Revisions}

subversion 757, 7/7/09, \texttt{EHRSecondPartial} created.

subversion 1055, 3/12/10, \texttt{EHRSecondPartial}\ converted to a geoquant.

\subsection*{Testing}

This function has not been tested.

\subsection*{Future Work}

A testing regime should be instituted for this function. \ 
}}%
%BeginExpansion
%TCIDATA{Version=5.00.0.2606}
%TCIDATA{LaTeXparent=1,1,functions.tex}
                      

\section*{\texttt{EHRSecondPartial::EHRSecondPartial}}

\subsection*{Function Prototype}

\texttt{double EHRSecondPartial (Vertex v\_i, Vertex v\_j)}

\subsection*{Key Words}

Einstein-Hilbert-Regge, functional, partial derivative, Hessian, geoquant.

\subsection*{Authors}

Daniel Champion

\subsection*{Introduction}

\texttt{EHRSecondPartial} calculates the second order partial derivatives of
the normalized Einstein-Hilbert-Regge functional with respect to log radii.
\ 

\subsection*{Subsidiaries}

\textbf{Functions:}

\qquad \texttt{CurvaturePartial}

\qquad\qquad\texttt{isAdjVertex}

\qquad \qquad \texttt{DualArea}

\qquad \qquad \qquad \texttt{DualAreaSegment}

\qquad \qquad \qquad \qquad \texttt{FaceHeight}

\qquad \qquad \qquad \qquad \qquad \texttt{EdgeHeight}

\qquad \qquad \qquad \qquad \qquad \qquad \texttt{PartialEdge}

\qquad\qquad\texttt{listDifference}

\qquad\qquad\texttt{listIntersection}

\qquad \texttt{TotalCurvature}

\qquad\qquad\texttt{Geometry::curvature}

\qquad \texttt{TotalVolume}

\qquad\qquad\texttt{Geometry::volume}

\qquad \texttt{VolumePartial}

\qquad\qquad\texttt{listDifference}

\qquad\qquad\texttt{listIntersection.}

\qquad \texttt{VolumeSecondPartial}

\textbf{Global Variables: }\ radii, etas.

\textbf{Local Variables:} \ none.

\subsection*{Description}

The normalized Einstein-Hilbert-Regge functional is given by the expression:%
\begin{equation*}
EHR=\frac{\sum\limits_{j}K_{j}}{\sqrt[3]{\sum\limits_{tetra\text{ }t}V_{t}}},
\end{equation*}
where $K_{i}$ is the curvature at vertex $j$, and $V_{t}$ is the volume of
tetrahedron $t$. \ It can be shown (see arXiv:0906.1560v1) that 
\begin{equation*}
\frac{\partial}{\partial\log r_{i}}\left( \sum\limits_{j}K_{j}\right) =K_{i},
\end{equation*}
hence the partial derivative of the normalized EHR functional simplifies to
become:%
\begin{equation*}
\frac{\partial}{\partial\log r_{i}}EHR=V^{-\frac{4}{3}}\left( K_{i}V-\frac {1%
}{3}K\sum\limits_{tetra\text{ }t}\frac{\partial V_{t}}{\partial\log r_{i}}%
\right) ,
\end{equation*}
where $V$ is the total volume of all tetrahedra in the triangulation and $K$
is the sum of the curvatures over all vertices in the triangulation. \
Differentiating this result with respect to $\log r_{j}$ yields:%
\begin{equation*}
\frac{\partial^{2}}{\partial\log r_{i}\partial\log r_{j}}EHR=V^{-\frac{4}{3}%
}\left( 
\begin{array}{c}
V\frac{\partial K_{i}}{\partial\log r_{j}}-\frac{1}{3}K_{i}\sum\limits_{t}%
\frac{\partial V_{t}}{\partial\log r_{j}}-\frac{1}{3}K_{j}\sum\limits_{t}%
\frac{\partial V_{t}}{\partial\log r_{i}} \\ 
+\frac{4}{9}KV^{-1}\sum\limits_{t}\frac{\partial V_{t}}{\partial\log r_{j}}%
\sum\limits_{t}\frac{\partial V_{t}}{\partial\log r_{i}}-\frac{1}{3}%
K\sum\limits_{t}\frac{\partial^{2}V_{t}}{\partial\log r_{i}\partial\log r_{j}%
}%
\end{array}
\right) .
\end{equation*}

When called, \texttt{EHRSecondPartial} calculates the formula above, that is:%
\begin{equation*}
\text{\texttt{EHR\_Second\_Partial (i,j)}}=\frac{\partial ^{2}}{\partial
\log r_{i}\partial \log r_{j}}EHR.
\end{equation*}

The use of this function is in the population of the Hessian matrix for the
normalized EHR functional. \ This Hessian matrix is used in the optimization
of the EHR functional using Newton's method.

\subsection*{Practicum}

As an example of the usage of \texttt{EHRSecondPartial}, the Hessian matrix
of the normalized EHR functional will be populated. \ In this example, the
Hessian matrix is the array EHRhessian. \ The example reduced computation
time by only calling \texttt{EHRSecondPartial} for the upper triangular
portion of the EHRhessian array, and symmetrically copies the entries above
the diagonal to the corresponding location below the diagonal. \
Furthermore, in C++ arrays begin indexing at zero, however the vertices of
the triangulations begin indexing at 1, requiring a shift of one in the
population step. \ Note that triangulations that do not label the vertices
consecutively will not be compatible with the following code. \ 

\bigskip

\qquad\texttt{double
EHRhessian[Triangulation::vertexTable.size()][Triangulation::vertexTable.size()];%
}

\qquad\texttt{for(int i = 0; i \TEXTsymbol{<}
Triangulation::vertexTable.size(); ++i) \{}

\qquad \qquad \texttt{for(int j = 0; j \TEXTsymbol{<}
Triangulation::vertexTable.size(); ++j) \{}

\qquad \qquad \qquad \texttt{Vertex vi = Triangulation::vertexYable[i+1];}

\qquad \qquad \qquad \texttt{Vertex vj = Triangulation::vertexTable[j+1];}

\qquad \qquad \qquad \texttt{if (i \TEXTsymbol{<}= j) \{}

\qquad \qquad \qquad \qquad \texttt{EHRhessian[i][j]=EHRSecondPartial( vi ,
vj );}

\qquad\qquad\qquad\qquad\texttt{EHRhessian[j][i]=EHRhessian[i][j];}

\qquad\qquad\qquad\qquad\texttt{\}}

\qquad\qquad\qquad\texttt{\}}

\qquad\qquad\texttt{\}}

\subsection*{Limitations}

\texttt{EHRSecondPartial} is fully operational with no known limitations. \
The function will output appropriate values provided it receives as inputs a
pair of integers in the vertex table. \ 

\subsection*{Revisions}

subversion 757, 7/7/09, \texttt{EHRSecondPartial} created.

subversion 1055, 3/12/10, \texttt{EHRSecondPartial}\ converted to a geoquant.

\subsection*{Testing}

This function has not been tested.

\subsection*{Future Work}

A testing regime should be instituted for this function. \ 
%
%EndExpansion

\bigskip

\bigskip

%TCIMACRO{\QSubDoc{Include flip}{\input{flip.tex}}}%
%BeginExpansion
\input{flip.tex}%
%EndExpansion

\bigskip

\bigskip

%TCIMACRO{\QSubDoc{Include GeoquantAt}{\input{GeoquantAt.tex}}}%
%BeginExpansion
\input{GeoquantAt.tex}%
%EndExpansion

\bigskip

\bigskip

%TCIMACRO{\QSubDoc{Include hij_k}{%TCIDATA{Version=5.00.0.2606}
%TCIDATA{LaTeXparent=1,1,functions.tex}
                      

\section*{\texttt{EdgeHeight::EdgeHeight\label{Edge Height FUNCTION}}}

\subsection*{Function Prototype}

\texttt{double EdgeHeight( Vertex vi, Vertex vj, Vertex vk)}

\subsection*{Key Words}

Edge height, partial edge, geoquant.

\subsection*{Authors}

Daniel Champion

\subsection*{Introduction}

The function \texttt{EdgeHeight} calculates the edge height (to the center
of a triangular face) of an edge in the triangulation. \ 

\subsection*{Subsidiaries}

\textbf{Functions:}

\qquad \texttt{PartialEdge}

\qquad\texttt{Geometry::angle}

\qquad\texttt{listIntersection}

\textbf{Global Variables:} radii, etas.

\textbf{Local Variables:} Vertex vi, vj, vk.

\subsection*{Description}

The calculation of \texttt{EdgeHeight} involves the simple formula:%
\begin{equation*}
\text{\texttt{EdgeHeight(vi, vj, vk)}}=
\end{equation*}%
\begin{equation*}
\frac{\left( \text{\texttt{PartialEdge(vi,vk)}}-\text{\texttt{%
PartialEdge(vi,vj)}}\cos \left( \alpha _{i,jk}\right) \right) }{\sin \left(
\alpha _{i,jk}\right) }
\end{equation*}%
where $\alpha _{i,jk}$ is the angle at vertex $vi$ of triangle $\left\{
vi,vj,vk\right\} $. \ A geometric interpretation of this quantity is a
follows. \ Given a decorated triangle (triangle with radii and eta values
assigned to the vertices and edges respectively), the center of this
triangle can be calculated as the common power point of its embedding into
two-dimensional Euclidean space. \ The perpendicular distance from this
center point to the edge $\left\{ vi,vj\right\} $ is exactly \texttt{%
EdgeHeight(vi, vj, vk)}. \ Take note that the first two vertices in the
function call correspond to the preferred edge, and the third vertex in the
function call identifies the triangle. \ 

A primary use of this function is in the calculation of several quantities
needed for the \texttt{CurvaturePartial} function used in the optimization
of the normalized Einstein-Hilbert-Regge functional.

\subsection*{Practicum}

An example of the use of this function is in the calculation of the dual
areas, \texttt{DualAreaSegment}, to an edge of a three dimensional
triangulation.

\qquad \texttt{double DualAreaSegment( Vertex vi, Vertex vj, Vertex vk,
Vertex vl)}

\qquad\qquad\texttt{\{}

\qquad \qquad \texttt{double result =
0.5*(EdgeHeight(vi,vj,vk)*FaceHeight(vi,vj,vk,vl)}

\qquad \qquad \qquad \texttt{+EdgeHeight(vi,vj,vl)*FaceHeight(vi,vj,vl,vk));}

\qquad\qquad\texttt{return result;}

\qquad\qquad\texttt{\}}

\subsection*{Limitations}

\texttt{EdgeHeight} must receive as input three vertices of a face of the
triangulation. \ Moreover, the first two vertices in the function call
identify an edge, and can be in any order, however the third vertex in the
function call identifies the face and can not be permuted with the other two
vertices. \ 

\subsection*{Revisions}

subversion 757, 6/8/09, \texttt{EdgeHeight} created.

subversion 1055, 3/12/10, \texttt{EdgeHeight}\ converted to a geoquant.

\subsection*{Testing}

This function was not tested.

\subsection*{Future Work}

This function has been incorporated into the Geometry class geoquants, and
thus this entry needs to be updated. \ 
}}%
%BeginExpansion
%TCIDATA{Version=5.00.0.2606}
%TCIDATA{LaTeXparent=1,1,functions.tex}
                      

\section*{\texttt{EdgeHeight::EdgeHeight\label{Edge Height FUNCTION}}}

\subsection*{Function Prototype}

\texttt{double EdgeHeight( Vertex vi, Vertex vj, Vertex vk)}

\subsection*{Key Words}

Edge height, partial edge, geoquant.

\subsection*{Authors}

Daniel Champion

\subsection*{Introduction}

The function \texttt{EdgeHeight} calculates the edge height (to the center
of a triangular face) of an edge in the triangulation. \ 

\subsection*{Subsidiaries}

\textbf{Functions:}

\qquad \texttt{PartialEdge}

\qquad\texttt{Geometry::angle}

\qquad\texttt{listIntersection}

\textbf{Global Variables:} radii, etas.

\textbf{Local Variables:} Vertex vi, vj, vk.

\subsection*{Description}

The calculation of \texttt{EdgeHeight} involves the simple formula:%
\begin{equation*}
\text{\texttt{EdgeHeight(vi, vj, vk)}}=
\end{equation*}%
\begin{equation*}
\frac{\left( \text{\texttt{PartialEdge(vi,vk)}}-\text{\texttt{%
PartialEdge(vi,vj)}}\cos \left( \alpha _{i,jk}\right) \right) }{\sin \left(
\alpha _{i,jk}\right) }
\end{equation*}%
where $\alpha _{i,jk}$ is the angle at vertex $vi$ of triangle $\left\{
vi,vj,vk\right\} $. \ A geometric interpretation of this quantity is a
follows. \ Given a decorated triangle (triangle with radii and eta values
assigned to the vertices and edges respectively), the center of this
triangle can be calculated as the common power point of its embedding into
two-dimensional Euclidean space. \ The perpendicular distance from this
center point to the edge $\left\{ vi,vj\right\} $ is exactly \texttt{%
EdgeHeight(vi, vj, vk)}. \ Take note that the first two vertices in the
function call correspond to the preferred edge, and the third vertex in the
function call identifies the triangle. \ 

A primary use of this function is in the calculation of several quantities
needed for the \texttt{CurvaturePartial} function used in the optimization
of the normalized Einstein-Hilbert-Regge functional.

\subsection*{Practicum}

An example of the use of this function is in the calculation of the dual
areas, \texttt{DualAreaSegment}, to an edge of a three dimensional
triangulation.

\qquad \texttt{double DualAreaSegment( Vertex vi, Vertex vj, Vertex vk,
Vertex vl)}

\qquad\qquad\texttt{\{}

\qquad \qquad \texttt{double result =
0.5*(EdgeHeight(vi,vj,vk)*FaceHeight(vi,vj,vk,vl)}

\qquad \qquad \qquad \texttt{+EdgeHeight(vi,vj,vl)*FaceHeight(vi,vj,vl,vk));}

\qquad\qquad\texttt{return result;}

\qquad\qquad\texttt{\}}

\subsection*{Limitations}

\texttt{EdgeHeight} must receive as input three vertices of a face of the
triangulation. \ Moreover, the first two vertices in the function call
identify an edge, and can be in any order, however the third vertex in the
function call identifies the face and can not be permuted with the other two
vertices. \ 

\subsection*{Revisions}

subversion 757, 6/8/09, \texttt{EdgeHeight} created.

subversion 1055, 3/12/10, \texttt{EdgeHeight}\ converted to a geoquant.

\subsection*{Testing}

This function was not tested.

\subsection*{Future Work}

This function has been incorporated into the Geometry class geoquants, and
thus this entry needs to be updated. \ 
%
%EndExpansion

\bigskip

\bigskip

%TCIMACRO{\QSubDoc{Include hijk_l}{%TCIDATA{Version=5.00.0.2606}
%TCIDATA{LaTeXparent=1,1,functions.tex}
                      

\section*{\texttt{FaceHeight::FaceHeight\label{Face Height FUNCTION}}}

\subsection*{Function Prototype}

\texttt{double FaceHeight( Vertex vi, Vertex vj, Vertex vk, Vertex vl)}

\subsection*{Key Words}

Face height, edge height, geoquant.

\subsection*{Authors}

Daniel Champion

\subsection*{Introduction}

The function \texttt{FaceHeight}\ calculates the face height to the center
of a tetrahedron. \ 

\subsection*{Subsidiaries}

\textbf{Functions:}

\qquad\texttt{Geometry::dihedralAngle}

\qquad \texttt{EdgeHeight}

\qquad\texttt{listIntersection}

\textbf{Global Variables: }\ radii, etas.

\textbf{Local Variables:} \ Vertex vi, vj, vk, vl.

\subsection*{Description}

The calculation of \texttt{FaceHeight} involves the simple formula:%
\begin{equation*}
\text{\texttt{FaceHeight(vi, vj, vk, vl)}}=
\end{equation*}

\begin{equation*}
\frac{\left( \text{\texttt{EdgeHeight(vi,vj,vl)}}-\text{\texttt{%
EdgeHeight(vi,vj,vk)}}\cos (\beta _{ij,kl})\right) }{\sin \left( \beta
_{ij,kl}\right) }
\end{equation*}%
where $\beta _{ij,kl}$ is the dihedral angle along edge $\left\{
vi,vj\right\} $ of tetrahedron $\left\{ vi,vj,vk,vl\right\} $. \ A geometric
interpretation of this quantity is a follows. \ Given a decorated
tetrahedron (tetrahedron with radii and eta values assigned to the vertices
and edges respectively), the center of this tetrahedron can be calculated as
the common power point of its embedding into three-dimensional Euclidean
space. \ The perpendicular distance from this center point to the face $%
\left\{ vi,vj,vk\right\} $ is exactly \texttt{FaceHeight (vi, vj, vk, vl)}.
\ Take note that the first three vertices in the function call correspond to
the preferred face, and the fourth vertex in the function call identifies
the tetrahedron. \ 

A primary use of this function is in the calculation of several quantities
needed for the \texttt{CurvaturePartial} function used in the optimization
of the normalized Einstein-Hilbert-Regge functional.

\subsection*{Practicum}

An example of the use of this function is in the calculation of the dual
areas, \texttt{DualAreaSegment}, to an edge of a three dimensional
triangulation.

\qquad \texttt{double DualAreaSegment( Vertex vi, Vertex vj, Vertex vk,
Vertex vl)}

\qquad\qquad\texttt{\{}

\qquad \qquad \texttt{double result =
0.5*(EdgeHeight(vi,vj,vk)*FaceHeight(vi,vj,vk,vl)}

\qquad \qquad \qquad \texttt{+EdgeHeight(vi,vj,vl)*FaceHeight(vi,vj,vl,vk));}

\qquad\qquad\texttt{return result;}

\qquad\qquad\texttt{\}}

\subsection*{Limitations}

\texttt{faceHeight} must receive as input four vertices of a tetrahedron of
the triangulation. \ Moreover, the first three vertices in the function call
identify a face and can be in any order, however the fourth vertex in the
function call identifies the tetrahedron and can not be permuted with the
other three vertices. \ 

\subsection*{Revisions}

subversion 757, 6/8/09, \texttt{FaceHeight} created.

subversion 1055, 3/12/10, \texttt{FaceHeight}\ converted to a geoquant.

\subsection*{Testing}

This function was not tested.

\subsection*{Future Work}

This function has been incorporated into the Geometry class geoquants, and
thus this entry needs to be updated. \ 
}}%
%BeginExpansion
%TCIDATA{Version=5.00.0.2606}
%TCIDATA{LaTeXparent=1,1,functions.tex}
                      

\section*{\texttt{FaceHeight::FaceHeight\label{Face Height FUNCTION}}}

\subsection*{Function Prototype}

\texttt{double FaceHeight( Vertex vi, Vertex vj, Vertex vk, Vertex vl)}

\subsection*{Key Words}

Face height, edge height, geoquant.

\subsection*{Authors}

Daniel Champion

\subsection*{Introduction}

The function \texttt{FaceHeight}\ calculates the face height to the center
of a tetrahedron. \ 

\subsection*{Subsidiaries}

\textbf{Functions:}

\qquad\texttt{Geometry::dihedralAngle}

\qquad \texttt{EdgeHeight}

\qquad\texttt{listIntersection}

\textbf{Global Variables: }\ radii, etas.

\textbf{Local Variables:} \ Vertex vi, vj, vk, vl.

\subsection*{Description}

The calculation of \texttt{FaceHeight} involves the simple formula:%
\begin{equation*}
\text{\texttt{FaceHeight(vi, vj, vk, vl)}}=
\end{equation*}

\begin{equation*}
\frac{\left( \text{\texttt{EdgeHeight(vi,vj,vl)}}-\text{\texttt{%
EdgeHeight(vi,vj,vk)}}\cos (\beta _{ij,kl})\right) }{\sin \left( \beta
_{ij,kl}\right) }
\end{equation*}%
where $\beta _{ij,kl}$ is the dihedral angle along edge $\left\{
vi,vj\right\} $ of tetrahedron $\left\{ vi,vj,vk,vl\right\} $. \ A geometric
interpretation of this quantity is a follows. \ Given a decorated
tetrahedron (tetrahedron with radii and eta values assigned to the vertices
and edges respectively), the center of this tetrahedron can be calculated as
the common power point of its embedding into three-dimensional Euclidean
space. \ The perpendicular distance from this center point to the face $%
\left\{ vi,vj,vk\right\} $ is exactly \texttt{FaceHeight (vi, vj, vk, vl)}.
\ Take note that the first three vertices in the function call correspond to
the preferred face, and the fourth vertex in the function call identifies
the tetrahedron. \ 

A primary use of this function is in the calculation of several quantities
needed for the \texttt{CurvaturePartial} function used in the optimization
of the normalized Einstein-Hilbert-Regge functional.

\subsection*{Practicum}

An example of the use of this function is in the calculation of the dual
areas, \texttt{DualAreaSegment}, to an edge of a three dimensional
triangulation.

\qquad \texttt{double DualAreaSegment( Vertex vi, Vertex vj, Vertex vk,
Vertex vl)}

\qquad\qquad\texttt{\{}

\qquad \qquad \texttt{double result =
0.5*(EdgeHeight(vi,vj,vk)*FaceHeight(vi,vj,vk,vl)}

\qquad \qquad \qquad \texttt{+EdgeHeight(vi,vj,vl)*FaceHeight(vi,vj,vl,vk));}

\qquad\qquad\texttt{return result;}

\qquad\qquad\texttt{\}}

\subsection*{Limitations}

\texttt{faceHeight} must receive as input four vertices of a tetrahedron of
the triangulation. \ Moreover, the first three vertices in the function call
identify a face and can be in any order, however the fourth vertex in the
function call identifies the tetrahedron and can not be permuted with the
other three vertices. \ 

\subsection*{Revisions}

subversion 757, 6/8/09, \texttt{FaceHeight} created.

subversion 1055, 3/12/10, \texttt{FaceHeight}\ converted to a geoquant.

\subsection*{Testing}

This function was not tested.

\subsection*{Future Work}

This function has been incorporated into the Geometry class geoquants, and
thus this entry needs to be updated. \ 
%
%EndExpansion

\bigskip

\bigskip

%TCIMACRO{\QSubDoc{Include Lij_star}{%TCIDATA{Version=5.00.0.2606}
%TCIDATA{LaTeXparent=1,1,functions.tex}
                      

\section*{\texttt{DualArea::DualArea}}

\subsection*{Function Prototype}

\texttt{double DualArea(Edge e)}

\subsection*{Key Words}

Dual area, curvature, partial derivative, edge, geoquant.

\subsection*{Authors}

Daniel Champion

\subsection*{Introduction}

\texttt{DualArea} calculates the dual area of an edge.

\subsection*{Subsidiaries}

\subsubsection*{Functions: \ }

\qquad \texttt{DualAreaSegment}

\qquad \qquad \texttt{FaceHeight}

\qquad \qquad \qquad \texttt{EdgeHeight}

\qquad \qquad \qquad \qquad \texttt{PartialEdge}

\subsubsection*{Global Variables: \ }

\qquad Only radii and eta values are needed.

\subsubsection*{Local Variables: \ }

\qquad none

\subsection*{Description}

\texttt{DualArea} is defined as:%
\begin{equation*}
\text{\texttt{DualArea(edge e\_ij))}}=l_{ij}^{\ast }=\sum_{\substack{ \text{%
all tetrahedra }(i,j,k,l) \\ \text{containing edge }(i,j)\text{.}}}A_{ij,kl},
\end{equation*}%
where $A_{ij,kl}$ is computed with the function \texttt{DualAreaSegment}
applied to the vertices of the tetrahedron being summed over. \ Note that 
\texttt{DualAreaSegment} utilizes the functions \texttt{FaceHeight}, \texttt{%
EdgeHeight}, and \texttt{partialEdge}, however only the radii and eta values
are needed to calculate all of these quantities. \ 

This function was created for use in the \texttt{CurvaturePartial} function
which serves an essential role in calculating the second derivatives of the
Einstein-Hilbert-Regge functional (\texttt{EHRSecondPartial}). \ The second
order partial derivatives of the EHR functional are used in the optimization
of the EHR functional using Newton's method. \ \texttt{DualArea} will
eventually be used in the study of laplacians. \ 

\subsection*{Practicum}

Currently \texttt{DualArea} is only used to calculate the partial derivative
of curvature with respect to $\log $ radius. \ The following example
calculates the partial derivative of the curvature at vertex V with respect
to the $\log $ radius $r_{l}$ corresponding to vertex Vprime (adjacent to V).

\bigskip

\qquad\texttt{double sum = 0.0;}

\qquad\texttt{double dihedral\_sum = 0.0;}

\qquad\texttt{Vprime = Triangulation::vertexTable[l];}

\qquad\texttt{E =
Triangulation::edgeTable[listIntersection(V.getLocalEdges(),}

\qquad\qquad\texttt{Vprime.getLocalEdges())[0]];}

\qquad\texttt{// This assumes that there is a unique edge between two
vertices.}

\qquad\texttt{local\_tetra = E.getLocalTetras();}

\qquad\texttt{for (int m=0; m \TEXTsymbol{<} (*(local\_tetra)).size(); ++m)
\{}

\qquad\qquad\texttt{T = Triangulation::tetraTable[local\_tetra-\TEXTsymbol{>}%
at(m)];}

\qquad\qquad\texttt{dihedral\_sum += Geometry::dihedralAngle(E,T);}

\qquad\texttt{\}}

\qquad \texttt{result =
DualArea(E)/(Geometry::Length(E))-(2*PI-dihedral\_sum)}

\qquad \qquad \texttt{*(pow(Geometry::Radius(V),
2)*pow(Geometry::Radius(Vprime),2)}

\qquad \qquad \texttt{%
*(1-pow(Geometry::Eta(E),2)))/pow(Geometry::Length(E),3);}

\subsection*{Limitations}

\texttt{DualArea} can operate on any and all edges of a 3D triangulation
however it is only appropriate for triangulations where tetrahedra have
distinct edges. \ 

\subsection*{Revisions}

Subversion 676, 5/15/09, \texttt{DualArea} created within \texttt{%
Newtons\_Method}.

subversion 1055, 3/12/10, \texttt{DualArea}\ converted to a geoquant.

\subsection*{Testing}

\texttt{Lij\_star} has not been tested. \ 

\subsection*{Future Work}

\texttt{Lij\_star} should be moved to a more appropriate section of the
code. \ A more general volume function should be created that would take any
simplex object (including a boolean for dual simplices) and return the
appropriate volume. \ This general volume function would be an excellent
location for \texttt{Lij\_star}. \ It should be tested some time as well. \ 
}}%
%BeginExpansion
%TCIDATA{Version=5.00.0.2606}
%TCIDATA{LaTeXparent=1,1,functions.tex}
                      

\section*{\texttt{DualArea::DualArea}}

\subsection*{Function Prototype}

\texttt{double DualArea(Edge e)}

\subsection*{Key Words}

Dual area, curvature, partial derivative, edge, geoquant.

\subsection*{Authors}

Daniel Champion

\subsection*{Introduction}

\texttt{DualArea} calculates the dual area of an edge.

\subsection*{Subsidiaries}

\subsubsection*{Functions: \ }

\qquad \texttt{DualAreaSegment}

\qquad \qquad \texttt{FaceHeight}

\qquad \qquad \qquad \texttt{EdgeHeight}

\qquad \qquad \qquad \qquad \texttt{PartialEdge}

\subsubsection*{Global Variables: \ }

\qquad Only radii and eta values are needed.

\subsubsection*{Local Variables: \ }

\qquad none

\subsection*{Description}

\texttt{DualArea} is defined as:%
\begin{equation*}
\text{\texttt{DualArea(edge e\_ij))}}=l_{ij}^{\ast }=\sum_{\substack{ \text{%
all tetrahedra }(i,j,k,l) \\ \text{containing edge }(i,j)\text{.}}}A_{ij,kl},
\end{equation*}%
where $A_{ij,kl}$ is computed with the function \texttt{DualAreaSegment}
applied to the vertices of the tetrahedron being summed over. \ Note that 
\texttt{DualAreaSegment} utilizes the functions \texttt{FaceHeight}, \texttt{%
EdgeHeight}, and \texttt{partialEdge}, however only the radii and eta values
are needed to calculate all of these quantities. \ 

This function was created for use in the \texttt{CurvaturePartial} function
which serves an essential role in calculating the second derivatives of the
Einstein-Hilbert-Regge functional (\texttt{EHRSecondPartial}). \ The second
order partial derivatives of the EHR functional are used in the optimization
of the EHR functional using Newton's method. \ \texttt{DualArea} will
eventually be used in the study of laplacians. \ 

\subsection*{Practicum}

Currently \texttt{DualArea} is only used to calculate the partial derivative
of curvature with respect to $\log $ radius. \ The following example
calculates the partial derivative of the curvature at vertex V with respect
to the $\log $ radius $r_{l}$ corresponding to vertex Vprime (adjacent to V).

\bigskip

\qquad\texttt{double sum = 0.0;}

\qquad\texttt{double dihedral\_sum = 0.0;}

\qquad\texttt{Vprime = Triangulation::vertexTable[l];}

\qquad\texttt{E =
Triangulation::edgeTable[listIntersection(V.getLocalEdges(),}

\qquad\qquad\texttt{Vprime.getLocalEdges())[0]];}

\qquad\texttt{// This assumes that there is a unique edge between two
vertices.}

\qquad\texttt{local\_tetra = E.getLocalTetras();}

\qquad\texttt{for (int m=0; m \TEXTsymbol{<} (*(local\_tetra)).size(); ++m)
\{}

\qquad\qquad\texttt{T = Triangulation::tetraTable[local\_tetra-\TEXTsymbol{>}%
at(m)];}

\qquad\qquad\texttt{dihedral\_sum += Geometry::dihedralAngle(E,T);}

\qquad\texttt{\}}

\qquad \texttt{result =
DualArea(E)/(Geometry::Length(E))-(2*PI-dihedral\_sum)}

\qquad \qquad \texttt{*(pow(Geometry::Radius(V),
2)*pow(Geometry::Radius(Vprime),2)}

\qquad \qquad \texttt{%
*(1-pow(Geometry::Eta(E),2)))/pow(Geometry::Length(E),3);}

\subsection*{Limitations}

\texttt{DualArea} can operate on any and all edges of a 3D triangulation
however it is only appropriate for triangulations where tetrahedra have
distinct edges. \ 

\subsection*{Revisions}

Subversion 676, 5/15/09, \texttt{DualArea} created within \texttt{%
Newtons\_Method}.

subversion 1055, 3/12/10, \texttt{DualArea}\ converted to a geoquant.

\subsection*{Testing}

\texttt{Lij\_star} has not been tested. \ 

\subsection*{Future Work}

\texttt{Lij\_star} should be moved to a more appropriate section of the
code. \ A more general volume function should be created that would take any
simplex object (including a boolean for dual simplices) and return the
appropriate volume. \ This general volume function would be an excellent
location for \texttt{Lij\_star}. \ It should be tested some time as well. \ 
%
%EndExpansion

\bigskip

\bigskip

%TCIMACRO{%
%\QSubDoc{Include makeTriangulationFile}{\input{makeTriangulationFile.tex}}}%
%BeginExpansion
\input{makeTriangulationFile.tex}%
%EndExpansion

\bigskip

\bigskip

%TCIMACRO{%
%\QSubDoc{Include NewtonsMethodOptimize}{\input{NewtonsMethodOptimize.tex}}}%
%BeginExpansion
\input{NewtonsMethodOptimize.tex}%
%EndExpansion

\bigskip

\bigskip

%TCIMACRO{\QSubDoc{Include pause}{\input{pause.tex}}}%
%BeginExpansion
\input{pause.tex}%
%EndExpansion

\bigskip

\bigskip

%TCIMACRO{%
%\QSubDoc{Include print3DResultsStep}{\input{print3DResultsStep.tex}}}%
%BeginExpansion
\input{print3DResultsStep.tex}%
%EndExpansion

\bigskip

\bigskip

%TCIMACRO{\QSubDoc{Include printResultsNum}{\input{printResultsNum.tex}}}%
%BeginExpansion
\input{printResultsNum.tex}%
%EndExpansion

\bigskip

\bigskip

%TCIMACRO{%
%\QSubDoc{Include printResultsNumSteps}{\input{printResultsNumSteps.tex}}}%
%BeginExpansion
\input{printResultsNumSteps.tex}%
%EndExpansion

\bigskip

\bigskip

%TCIMACRO{\QSubDoc{Include printResultsStep}{\input{printResultsStep.tex}}}%
%BeginExpansion
\input{printResultsStep.tex}%
%EndExpansion

\bigskip

\bigskip

%TCIMACRO{%
%\QSubDoc{Include printResultsVertex}{\input{printResultsVertex.tex}}}%
%BeginExpansion
\input{printResultsVertex.tex}%
%EndExpansion

\bigskip

\bigskip

%TCIMACRO{%
%\QSubDoc{Include Total_Volume}{%TCIDATA{Version=5.00.0.2606}
%TCIDATA{LaTeXparent=1,1,functions.tex}
                      

\section*{\texttt{TotalVolume::TotalVolume}}

\subsection*{Function Prototype}

\texttt{double TotalVolume()}

\subsection*{Key Words}

Volume, tetrahedron, Cayley-Menger determinant, geoquant.

\subsection*{Authors}

Daniel Champion

\subsection*{Introduction}

The function \texttt{TotalVolume} calculates the total volume of a three
dimensional triangulated manifold.

\subsection*{Subsidiaries}

\textbf{Functions:}

\qquad Geometry::Volume

\textbf{Global Variables:} \ radii, etas.

\textbf{Local Variables:} none.

\subsection*{Description}

\texttt{TotalVolume} is calculated by summing the volumes of each
tetrahedron in a triangulation. \ The volume of a tetrahedron is calculated
with the Cayley-Menger determinant:%
\begin{equation*}
288V^{2}=\det \left[ 
\begin{array}{ccccc}
0 & 1 & 1 & 1 & 1 \\ 
1 & 0 & L_{12}^{2} & L_{13}^{2} & L_{14}^{2} \\ 
1 & L_{12}^{2} & 0 & L_{23}^{2} & L_{24}^{2} \\ 
1 & L_{13}^{2} & L_{23}^{2} & 0 & L_{34}^{2} \\ 
1 & L_{14}^{2} & L_{24}^{2} & L_{34}^{2} & 0%
\end{array}%
\right] 
\end{equation*}%
where the lengths were determined from the radii and eta values using the
formula%
\begin{equation*}
L_{ij}^{2}=r_{i}^{2}+r_{j}^{2}+2r_{i}r_{j}Eta_{ij}.
\end{equation*}

The formula was obtained using calculations within Mathematica and was
output into the C programming language using the function CForm. \ 

The total volume of a triangulation is used in multiple locations within the
project. \ One example of its use is in the calculation of the normalized
Einstein-Hilbert-Regge functional:%
\begin{equation*}
\widetilde{EHR}=\frac{\sum_{i}K_{i}}{\left( \text{\texttt{TotalVolume()}}%
\right) ^{\frac{1}{3}}}
\end{equation*}%
where $K_{i}$ is the curvature at vertex $i$.

\subsection*{Practicum}

An excellent example of the use of this function is in the calculation of
the normalized Einstein-Hilbert-Regge functional. \ 

\qquad\texttt{double EHR () \{}

\qquad\qquad\texttt{double result;}

\qquad \qquad \texttt{result = (TotalCurvature())/pow(TotalVolume (),
1.0/3.0);}

\qquad\qquad\texttt{return result;}

\qquad\qquad\texttt{\}}

\subsection*{Limitations}

Since \texttt{TotalVolume()} relies critically on the Geometry::volume
function, it thus has the same limitations. \ Specifically, if the edge
lengths of any tetrahedron do not satisfy the necessary conditions to
produce a positive volume tetrahedron, \texttt{TotalVolume()} will output an
undefined number. \ The Cayley-Menger determinant can be used to check this
condition on the edge lengths.

\subsection*{Revisions}

subversion 757, 7/11/09, \texttt{TotalVolume} created.

subversion 1055, 3/12/10, \texttt{TotalVolume}\ converted to a geoquant.

\subsection*{Testing}

Using known volumes of several tetrahedra, the total volume was calculated
by hand and compared with \texttt{TotalVolume}.

\subsection*{Future Work}

No planned future work.
}}%
%BeginExpansion
%TCIDATA{Version=5.00.0.2606}
%TCIDATA{LaTeXparent=1,1,functions.tex}
                      

\section*{\texttt{TotalVolume::TotalVolume}}

\subsection*{Function Prototype}

\texttt{double TotalVolume()}

\subsection*{Key Words}

Volume, tetrahedron, Cayley-Menger determinant, geoquant.

\subsection*{Authors}

Daniel Champion

\subsection*{Introduction}

The function \texttt{TotalVolume} calculates the total volume of a three
dimensional triangulated manifold.

\subsection*{Subsidiaries}

\textbf{Functions:}

\qquad Geometry::Volume

\textbf{Global Variables:} \ radii, etas.

\textbf{Local Variables:} none.

\subsection*{Description}

\texttt{TotalVolume} is calculated by summing the volumes of each
tetrahedron in a triangulation. \ The volume of a tetrahedron is calculated
with the Cayley-Menger determinant:%
\begin{equation*}
288V^{2}=\det \left[ 
\begin{array}{ccccc}
0 & 1 & 1 & 1 & 1 \\ 
1 & 0 & L_{12}^{2} & L_{13}^{2} & L_{14}^{2} \\ 
1 & L_{12}^{2} & 0 & L_{23}^{2} & L_{24}^{2} \\ 
1 & L_{13}^{2} & L_{23}^{2} & 0 & L_{34}^{2} \\ 
1 & L_{14}^{2} & L_{24}^{2} & L_{34}^{2} & 0%
\end{array}%
\right] 
\end{equation*}%
where the lengths were determined from the radii and eta values using the
formula%
\begin{equation*}
L_{ij}^{2}=r_{i}^{2}+r_{j}^{2}+2r_{i}r_{j}Eta_{ij}.
\end{equation*}

The formula was obtained using calculations within Mathematica and was
output into the C programming language using the function CForm. \ 

The total volume of a triangulation is used in multiple locations within the
project. \ One example of its use is in the calculation of the normalized
Einstein-Hilbert-Regge functional:%
\begin{equation*}
\widetilde{EHR}=\frac{\sum_{i}K_{i}}{\left( \text{\texttt{TotalVolume()}}%
\right) ^{\frac{1}{3}}}
\end{equation*}%
where $K_{i}$ is the curvature at vertex $i$.

\subsection*{Practicum}

An excellent example of the use of this function is in the calculation of
the normalized Einstein-Hilbert-Regge functional. \ 

\qquad\texttt{double EHR () \{}

\qquad\qquad\texttt{double result;}

\qquad \qquad \texttt{result = (TotalCurvature())/pow(TotalVolume (),
1.0/3.0);}

\qquad\qquad\texttt{return result;}

\qquad\qquad\texttt{\}}

\subsection*{Limitations}

Since \texttt{TotalVolume()} relies critically on the Geometry::volume
function, it thus has the same limitations. \ Specifically, if the edge
lengths of any tetrahedron do not satisfy the necessary conditions to
produce a positive volume tetrahedron, \texttt{TotalVolume()} will output an
undefined number. \ The Cayley-Menger determinant can be used to check this
condition on the edge lengths.

\subsection*{Revisions}

subversion 757, 7/11/09, \texttt{TotalVolume} created.

subversion 1055, 3/12/10, \texttt{TotalVolume}\ converted to a geoquant.

\subsection*{Testing}

Using known volumes of several tetrahedra, the total volume was calculated
by hand and compared with \texttt{TotalVolume}.

\subsection*{Future Work}

No planned future work.
%
%EndExpansion

\bigskip

\bigskip

%TCIMACRO{%
%\QSubDoc{Include Volume_Partial}{%TCIDATA{Version=5.00.0.2606}
%TCIDATA{LaTeXparent=1,1,functions.tex}
                      

\section*{\texttt{VolumePartial::VolumePartial}}

\subsection*{Function Prototype}

\texttt{double VolumePartial(Vertex v\_i, Tetra t)}

\subsection*{Key Words}

Volume, tetrahedron, vertex, radius, Cayley-Menger determinant, standard
form, geoquant.

\subsection*{Authors}

Daniel Champion

\subsection*{Introduction}

\texttt{VolumePartial} calculates the partial derivative of the volume of a
tetrahedron with respect to the logarithm of the radius of a vertex.

\subsection*{Subsidiaries}

\textbf{Functions:} \ 

\texttt{listDifference}

\texttt{listIntersection}

\texttt{Simplex::isAdjVertex}

\textbf{Global Variables:} \ radii, etas.

\textbf{Local Variables:}

\subsection*{Description}

The volume of a tetrahedron only depends on the lengths of its edges as
calculated from the Cayley-Menger determinant. \ Thus for a given
tetrahedron $t$, it's partial derivatives with respect to $\log $ radii will
vanish except for those radii corresponding to the vertices of $t$. \ The
function \texttt{isAdjVertex} of the simplex class determines this
condition. \ \texttt{VolumePartial} then proceeds with an initialization
procedure that labels the vertices and edges (radii and etas) in standard
form. \ Specifically, \texttt{Volume\_Partial} receives as inputs an integer 
\texttt{i} corresponding to a vertex (in the vertex table) which is labeled
vertex v1. \ The remaining vertices are labeled $v2,v3,v4$, and the edges $%
e12,e13,e14,e23,e24,e34$ are labeled preserving the structure implied by the
assignment of the vertices. \ The radii $r_{1}$, $r_{2}$,... and eta values $%
Eta_{12},Eta_{13},$... are assigned to the corresponding vertices and edges.
\ 

The formula for the partial derivative in terms of these standard form
variables was calculated in Mathematica using the Cayley-Menger determinant,
that is:%
\begin{equation*}
288V^{2}=\det\left[ 
\begin{array}{ccccc}
0 & 1 & 1 & 1 & 1 \\ 
1 & 0 & L_{12}^{2} & L_{13}^{2} & L_{14}^{2} \\ 
1 & L_{12}^{2} & 0 & L_{23}^{2} & L_{24}^{2} \\ 
1 & L_{13}^{2} & L_{23}^{2} & 0 & L_{34}^{2} \\ 
1 & L_{14}^{2} & L_{24}^{2} & L_{34}^{2} & 0%
\end{array}
\right] ,
\end{equation*}
where the lengths were determined from the radii and eta values using the
formula%
\begin{equation*}
L_{ij}^{2}=r_{i}^{2}+r_{j}^{2}+2r_{i}r_{j}Eta_{ij}.
\end{equation*}

The formula obtained from Mathematica was outputted into the C programming
language using the function CForm. \ 

This function was designed for use in the optimization of the
Einstein-Hilbert-Regge functional using Newton's method. \ In this procedure
the gradient of the EHR functional is needed which contains the partial
derivatives of the volume. \ 

\subsection*{Practicum}

Usage:

\texttt{VolumePartial(Vertex v\_i, Tetra t)}

The integer \texttt{i} corresponds to a vertex in the vertex table, that is%
\begin{equation*}
\text{\texttt{VolumePartial (v\_i, t)} }=\frac{\partial }{\partial \log r_{i}%
}Volume(t).
\end{equation*}

\subsection*{Limitations}

\texttt{VolumePartial} was designed to output the correct partial derivative
for any integer $i$ in the vertex table and any tetrahedron $t$ in the
triangulation. \ 

\subsection*{Revisions}

subversion 757, 7/7/09, \texttt{VolumePartial} created within
NewtonsMethod.cpp.

subversion 1055, 3/12/10, \texttt{VolumePartial} converted to a geoquant.

\subsection*{Testing}

The partial derivative of volume of several known tetrahedra were calculated
using \texttt{Volume\_Partial} and verified using Mathematica.

\subsection*{Future Work}

The procedure that initializes the tetrahedron into standard form should be
removed from this program and placed elsewhere. \ There are several
occurrences of this type of procedure that should be consolidated. \ 
}}%
%BeginExpansion
%TCIDATA{Version=5.00.0.2606}
%TCIDATA{LaTeXparent=1,1,functions.tex}
                      

\section*{\texttt{VolumePartial::VolumePartial}}

\subsection*{Function Prototype}

\texttt{double VolumePartial(Vertex v\_i, Tetra t)}

\subsection*{Key Words}

Volume, tetrahedron, vertex, radius, Cayley-Menger determinant, standard
form, geoquant.

\subsection*{Authors}

Daniel Champion

\subsection*{Introduction}

\texttt{VolumePartial} calculates the partial derivative of the volume of a
tetrahedron with respect to the logarithm of the radius of a vertex.

\subsection*{Subsidiaries}

\textbf{Functions:} \ 

\texttt{listDifference}

\texttt{listIntersection}

\texttt{Simplex::isAdjVertex}

\textbf{Global Variables:} \ radii, etas.

\textbf{Local Variables:}

\subsection*{Description}

The volume of a tetrahedron only depends on the lengths of its edges as
calculated from the Cayley-Menger determinant. \ Thus for a given
tetrahedron $t$, it's partial derivatives with respect to $\log $ radii will
vanish except for those radii corresponding to the vertices of $t$. \ The
function \texttt{isAdjVertex} of the simplex class determines this
condition. \ \texttt{VolumePartial} then proceeds with an initialization
procedure that labels the vertices and edges (radii and etas) in standard
form. \ Specifically, \texttt{Volume\_Partial} receives as inputs an integer 
\texttt{i} corresponding to a vertex (in the vertex table) which is labeled
vertex v1. \ The remaining vertices are labeled $v2,v3,v4$, and the edges $%
e12,e13,e14,e23,e24,e34$ are labeled preserving the structure implied by the
assignment of the vertices. \ The radii $r_{1}$, $r_{2}$,... and eta values $%
Eta_{12},Eta_{13},$... are assigned to the corresponding vertices and edges.
\ 

The formula for the partial derivative in terms of these standard form
variables was calculated in Mathematica using the Cayley-Menger determinant,
that is:%
\begin{equation*}
288V^{2}=\det\left[ 
\begin{array}{ccccc}
0 & 1 & 1 & 1 & 1 \\ 
1 & 0 & L_{12}^{2} & L_{13}^{2} & L_{14}^{2} \\ 
1 & L_{12}^{2} & 0 & L_{23}^{2} & L_{24}^{2} \\ 
1 & L_{13}^{2} & L_{23}^{2} & 0 & L_{34}^{2} \\ 
1 & L_{14}^{2} & L_{24}^{2} & L_{34}^{2} & 0%
\end{array}
\right] ,
\end{equation*}
where the lengths were determined from the radii and eta values using the
formula%
\begin{equation*}
L_{ij}^{2}=r_{i}^{2}+r_{j}^{2}+2r_{i}r_{j}Eta_{ij}.
\end{equation*}

The formula obtained from Mathematica was outputted into the C programming
language using the function CForm. \ 

This function was designed for use in the optimization of the
Einstein-Hilbert-Regge functional using Newton's method. \ In this procedure
the gradient of the EHR functional is needed which contains the partial
derivatives of the volume. \ 

\subsection*{Practicum}

Usage:

\texttt{VolumePartial(Vertex v\_i, Tetra t)}

The integer \texttt{i} corresponds to a vertex in the vertex table, that is%
\begin{equation*}
\text{\texttt{VolumePartial (v\_i, t)} }=\frac{\partial }{\partial \log r_{i}%
}Volume(t).
\end{equation*}

\subsection*{Limitations}

\texttt{VolumePartial} was designed to output the correct partial derivative
for any integer $i$ in the vertex table and any tetrahedron $t$ in the
triangulation. \ 

\subsection*{Revisions}

subversion 757, 7/7/09, \texttt{VolumePartial} created within
NewtonsMethod.cpp.

subversion 1055, 3/12/10, \texttt{VolumePartial} converted to a geoquant.

\subsection*{Testing}

The partial derivative of volume of several known tetrahedra were calculated
using \texttt{Volume\_Partial} and verified using Mathematica.

\subsection*{Future Work}

The procedure that initializes the tetrahedron into standard form should be
removed from this program and placed elsewhere. \ There are several
occurrences of this type of procedure that should be consolidated. \ 
%
%EndExpansion

\bigskip

\bigskip

%TCIMACRO{%
%\QSubDoc{Include Volume_Second_Partial}{%TCIDATA{Version=5.00.0.2606}
%TCIDATA{LaTeXparent=1,1,functions.tex}
                      

\section*{\texttt{VolumeSecondPartial::VolumeSecondPartial}}

\subsection*{Function Prototype}

\texttt{double VolumeSecondPartial(Vertex v\_i, Vertex v\_j, Tetra t)}

\subsection*{Key Words}

Volume, Hessian Matrix, Newton's Method, partial derivative,
Einstein-Hilbert-Regge functional, geoquant.

\subsection*{Authors}

Daniel Champion

\subsection*{Introduction}

\texttt{VolumeSecondPartial} calculates the second order partial derivatives
of the volume of a tetrahedron with respect to log radii for all pairs of
indices (not necessarily distinct) in the vertex table. \ 

\subsection*{Subsidiaries}

\textbf{Functions:}

\texttt{listDifference}

\texttt{listIntersection}

\texttt{Simplex::isAdjVertex}

\textbf{Global Variables:} \ radii, etas.

\textbf{Local Variables:}

\subsection*{Description}

The volume of a tetrahedron only depends on the lengths of its edges as
calculated from the Cayley-Menger determinant. \ Thus for a given
tetrahedron $t$, it's second order partial derivatives with respect to $\log 
$ radii will vanish except for pairs of radii (not necessarily distinct)
corresponding to the vertices of $t$. \ The first step in the implementation
of \texttt{VolumeSecondPartial} is the determination of the following
trichotomy for a pair $\left\{ i,j\right\} $ of indices in the vertex table:%
\begin{equation*}
\begin{array}{l}
\text{A. \ }i=j\text{ and }i\text{ is a vertex of tetrahedron }t \\ 
\text{B. \ }i\neq j\text{ and both }i\text{ and }j\text{ belong to }t \\ 
\text{C. \ at least one of }i\text{ or }j\text{ doesn't belong to }t.%
\end{array}%
\end{equation*}

Each condition of the trichotomy requires a distinct calculation to
determine the desired partial derivative. \ Nevertheless, the next step in
the implementation is to place the tetrahedron in "standard form" relative
to the indices $i$ and $j$ (for conditions A and B only). \ More
specifically, for condition A the radius for vertex $i$ is stored as $r_{1}$%
, and the remaining radii of the tetrahedron $t$ are assigned $r_{2},r_{3},$
and $r_{4}$ in no particular order. \ The eta values $Eta_{12},Eta_{13},...$
are then assigned preserving the preceding assignments. \ In the case of
condition B, the radii at vertices $i$ and $j$ are assigned to $r_{1}$ and $%
r_{2}$ respectively, and $r_{3}$, and $r_{4}$ the remaining radii of $t$. \
The eta values $Eta_{12},Eta_{13},...$ are again assigned preserving the
preceding assignments.

The formulas for the second order partial derivatives in terms of these
standard form variables was calculated in Mathematica using the
Cayley-Menger determinant, that is:%
\begin{equation*}
288V^{2}=\det\left[ 
\begin{array}{ccccc}
0 & 1 & 1 & 1 & 1 \\ 
1 & 0 & L_{12}^{2} & L_{13}^{2} & L_{14}^{2} \\ 
1 & L_{12}^{2} & 0 & L_{23}^{2} & L_{24}^{2} \\ 
1 & L_{13}^{2} & L_{23}^{2} & 0 & L_{34}^{2} \\ 
1 & L_{14}^{2} & L_{24}^{2} & L_{34}^{2} & 0%
\end{array}
\right] ,
\end{equation*}
where the lengths were determined from the radii and eta values using the
formula%
\begin{equation*}
L_{ij}^{2}=r_{i}^{2}+r_{j}^{2}+2r_{i}r_{j}Eta_{ij}.
\end{equation*}

The formula obtained from Mathematica was outputted into the C programming
language using the function CForm.

This function was designed for use in the optimization of the
Einstein-Hilbert-Regge functional using Newton's method. \ In this procedure
the Hessian matrix of the normalized EHR functional is needed, each entry of
which uses the second order partial derivatives of volume. \ See the entry
on \texttt{EHRSecondPartial}.

\subsection*{Practicum}

Usage:

\texttt{VolumeSecondPartial (Vertex v\_i, Vertex v\_j, Tetra t)}

The integers \texttt{i} and \texttt{j} correspond to vertices in the vertex
table and \texttt{t} is a tetrahedron in the triangulation. \ Specifically
the function returns:%
\begin{equation*}
\text{\texttt{VolumeSecondPartial(v\_i, v\_j, t)} }=\frac{\partial ^{2}}{%
\partial \log r_{i}\partial \log r_{j}}Volume(t).
\end{equation*}

\subsection*{Limitations}

\texttt{VolumeSecondPartial} is fully operational with no know limitations.
\ The function will output appropriate values when given indices $i,$ and $j$
in the vertex table, and a tetrahedron $t$. \ 

\subsection*{Revisions}

subversion 757, 7/9/09, \texttt{VolumeSecondPartial} created.

subversion 1055, 3/12/10, \texttt{VolumeSecondPartial} converted to a
geoquant.

\subsection*{Testing}

Several trials were run outputting the values of \texttt{VolumeSecondPartial}
for a variety of vertices and tetrahedra. \ These values were compared with
calculations performed on Mathematica. \ 

\subsection*{Future Work}

None planned.
}}%
%BeginExpansion
%TCIDATA{Version=5.00.0.2606}
%TCIDATA{LaTeXparent=1,1,functions.tex}
                      

\section*{\texttt{VolumeSecondPartial::VolumeSecondPartial}}

\subsection*{Function Prototype}

\texttt{double VolumeSecondPartial(Vertex v\_i, Vertex v\_j, Tetra t)}

\subsection*{Key Words}

Volume, Hessian Matrix, Newton's Method, partial derivative,
Einstein-Hilbert-Regge functional, geoquant.

\subsection*{Authors}

Daniel Champion

\subsection*{Introduction}

\texttt{VolumeSecondPartial} calculates the second order partial derivatives
of the volume of a tetrahedron with respect to log radii for all pairs of
indices (not necessarily distinct) in the vertex table. \ 

\subsection*{Subsidiaries}

\textbf{Functions:}

\texttt{listDifference}

\texttt{listIntersection}

\texttt{Simplex::isAdjVertex}

\textbf{Global Variables:} \ radii, etas.

\textbf{Local Variables:}

\subsection*{Description}

The volume of a tetrahedron only depends on the lengths of its edges as
calculated from the Cayley-Menger determinant. \ Thus for a given
tetrahedron $t$, it's second order partial derivatives with respect to $\log 
$ radii will vanish except for pairs of radii (not necessarily distinct)
corresponding to the vertices of $t$. \ The first step in the implementation
of \texttt{VolumeSecondPartial} is the determination of the following
trichotomy for a pair $\left\{ i,j\right\} $ of indices in the vertex table:%
\begin{equation*}
\begin{array}{l}
\text{A. \ }i=j\text{ and }i\text{ is a vertex of tetrahedron }t \\ 
\text{B. \ }i\neq j\text{ and both }i\text{ and }j\text{ belong to }t \\ 
\text{C. \ at least one of }i\text{ or }j\text{ doesn't belong to }t.%
\end{array}%
\end{equation*}

Each condition of the trichotomy requires a distinct calculation to
determine the desired partial derivative. \ Nevertheless, the next step in
the implementation is to place the tetrahedron in "standard form" relative
to the indices $i$ and $j$ (for conditions A and B only). \ More
specifically, for condition A the radius for vertex $i$ is stored as $r_{1}$%
, and the remaining radii of the tetrahedron $t$ are assigned $r_{2},r_{3},$
and $r_{4}$ in no particular order. \ The eta values $Eta_{12},Eta_{13},...$
are then assigned preserving the preceding assignments. \ In the case of
condition B, the radii at vertices $i$ and $j$ are assigned to $r_{1}$ and $%
r_{2}$ respectively, and $r_{3}$, and $r_{4}$ the remaining radii of $t$. \
The eta values $Eta_{12},Eta_{13},...$ are again assigned preserving the
preceding assignments.

The formulas for the second order partial derivatives in terms of these
standard form variables was calculated in Mathematica using the
Cayley-Menger determinant, that is:%
\begin{equation*}
288V^{2}=\det\left[ 
\begin{array}{ccccc}
0 & 1 & 1 & 1 & 1 \\ 
1 & 0 & L_{12}^{2} & L_{13}^{2} & L_{14}^{2} \\ 
1 & L_{12}^{2} & 0 & L_{23}^{2} & L_{24}^{2} \\ 
1 & L_{13}^{2} & L_{23}^{2} & 0 & L_{34}^{2} \\ 
1 & L_{14}^{2} & L_{24}^{2} & L_{34}^{2} & 0%
\end{array}
\right] ,
\end{equation*}
where the lengths were determined from the radii and eta values using the
formula%
\begin{equation*}
L_{ij}^{2}=r_{i}^{2}+r_{j}^{2}+2r_{i}r_{j}Eta_{ij}.
\end{equation*}

The formula obtained from Mathematica was outputted into the C programming
language using the function CForm.

This function was designed for use in the optimization of the
Einstein-Hilbert-Regge functional using Newton's method. \ In this procedure
the Hessian matrix of the normalized EHR functional is needed, each entry of
which uses the second order partial derivatives of volume. \ See the entry
on \texttt{EHRSecondPartial}.

\subsection*{Practicum}

Usage:

\texttt{VolumeSecondPartial (Vertex v\_i, Vertex v\_j, Tetra t)}

The integers \texttt{i} and \texttt{j} correspond to vertices in the vertex
table and \texttt{t} is a tetrahedron in the triangulation. \ Specifically
the function returns:%
\begin{equation*}
\text{\texttt{VolumeSecondPartial(v\_i, v\_j, t)} }=\frac{\partial ^{2}}{%
\partial \log r_{i}\partial \log r_{j}}Volume(t).
\end{equation*}

\subsection*{Limitations}

\texttt{VolumeSecondPartial} is fully operational with no know limitations.
\ The function will output appropriate values when given indices $i,$ and $j$
in the vertex table, and a tetrahedron $t$. \ 

\subsection*{Revisions}

subversion 757, 7/9/09, \texttt{VolumeSecondPartial} created.

subversion 1055, 3/12/10, \texttt{VolumeSecondPartial} converted to a
geoquant.

\subsection*{Testing}

Several trials were run outputting the values of \texttt{VolumeSecondPartial}
for a variety of vertices and tetrahedra. \ These values were compared with
calculations performed on Mathematica. \ 

\subsection*{Future Work}

None planned.
%
%EndExpansion
}}%
%BeginExpansion
%TCIDATA{Version=5.00.0.2606}
%TCIDATA{LaTeXparent=0,0,geocam.tex}
                      

\chapter{Functions}

%TCIMACRO{\QSubDoc{Include Aij_kl}{%TCIDATA{Version=5.00.0.2606}
%TCIDATA{LaTeXparent=1,1,functions.tex}
                      

\section*{\texttt{DualAreaSegment::DualAreaSegment}}

\subsection*{Function Prototype}

\texttt{double DualAreaSegment( Vertex vi, Vertex vj, Vertex vk, Vertex vl)}

\subsection*{Key Words}

Dual area, curvature, partial derivative, Einstein-Hilbert-Regge, geoquant.

\subsection*{Authors}

Daniel Champion

\subsection*{Introduction}

\texttt{DualAreaSegment} calculates the dual area to an edge of a
tetrahedron. \ 

\subsection*{Subsidiaries}

\textbf{Functions:}

\qquad \texttt{EdgeHeight}

\qquad \qquad \texttt{PartialEdge}

\qquad\qquad\texttt{Geometry::angle}

\qquad \texttt{FaceHeight}

\qquad\qquad\texttt{Geometry::dihedralAngle}

\textbf{Global Variables: \ }radii, etas

\textbf{Local Variables:} \ none.

\subsection*{Description}

\texttt{DualAreaSegment} is calculated with the formula:%
\begin{equation*}
\text{\texttt{DualAreaSegment(vi, vj, vk, vl)}}=
\end{equation*}%
\begin{equation*}
\frac{1}{2}\left( 
\begin{array}{c}
\text{\texttt{EdgeHeight(vi,vj,vk)}}\cdot \text{\texttt{%
FaceHeight(vi,vj,vk,vl)}} \\ 
+\text{\texttt{EdgeHeight(vi,vj,vl)}}\cdot \text{\texttt{%
FaceHeight(vi,vj,vl,vk)}}%
\end{array}%
\right) 
\end{equation*}%
\texttt{EdgeHeight} and \texttt{FaceHeight} are calculated with the
following formulae:%
\begin{align*}
\text{\texttt{EdgeHeight(vi, vj, vk)}}& =\frac{\left( \text{\texttt{%
PartialEdge(vi,vk)}}-\text{\texttt{PartialEdge(vi,vj)}}\cos \left( \alpha
_{i,jk}\right) \right) }{\sin \left( \alpha _{i,jk}\right) } \\
\text{\texttt{FaceHeight(vi, vj, vk, vl)}}& =\frac{\left( \text{\texttt{%
EdgeHeight(vi,vj,vl)}}-\text{\texttt{EdgeHeight(vi,vj,vk)}}\cos (\beta
_{ij,kl})\right) }{\sin \left( \beta _{ij,kl}\right) }
\end{align*}%
where $\alpha _{i,jk}$ is the angle at vertex $vi$ of triangle $\left\{
vi,vj,vk\right\} $, and $\beta _{ij,kl}$ is the dihedral angle along edge $%
\left\{ vi,vj\right\} $ of tetrahedron $\left\{ vi,vj,vk,vl\right\} $
(implemented with the functions \texttt{Geometry::angle} and \texttt{%
Geometry::dihedralAngle} respectively).

\texttt{DualAreaSegment} was created for the calculation performed in the
function \texttt{DualArea}, which is used in the computation of the partial
derivatives of curvature. \ These partial derivatives of curvature are used
in the calculation of the second order partial derivatives of the
Einstein-Hilbert-Regge functional for use in the optimization of said
functional using Newton's method. \ 

\subsection*{Practicum}

As an example of the usage of this function, we will calculate the dual area
to the edge $eij=\left\{ vi,vj\right\} $ (see entry: \texttt{DualArea}). \
To do this, we will sum the dual areas to each tetrahedron containing the
edge $eij$. \ 

\bigskip

\texttt{vector\TEXTsymbol{<}int\TEXTsymbol{>} sum\_over =
*(eij.getLocalTetras());}

\texttt{double sum = 0.0;}

\texttt{vector\TEXTsymbol{<}int\TEXTsymbol{>} T\_vertices, e\_vertices;}

\texttt{Tetra T;}

\texttt{Vertex vi,vj,vk,vl;}

\texttt{for(i=0; i\TEXTsymbol{<}sum\_over.size(); ++i) \{}

\qquad\texttt{T = Triangulation::tetraTable[sum\_over[i]];}

\qquad\texttt{T\_vertices = *(T.getLocalVertices());}

\qquad\texttt{e\_vertices = *(eij.getLocalVertices());}

\qquad\texttt{vi = Triangulation::vertexTable[e\_vertices[0]];}

\qquad\texttt{vj = Triangulation::vertexTable[e\_vertices[1]];}

\qquad\texttt{vk = Triangulation::vertexTable[listDifference(\&T\_vertices,
\&e\_vertices)[0]];}

\qquad\texttt{vl = Triangulation::vertexTable[listDifference(\&T\_vertices,
\&e\_vertices)[1]];}

\qquad \texttt{sum += DualAreaSegment(vi, vj, vk, vl);}

\qquad\texttt{\}}

\texttt{return sum;}

\subsection*{Limitations}

\texttt{DualAreaSegment} if fully operational and has no known limitations.
\ The function will output appropriate values provided it receives as input
four distinct vertices that define a tetrahedron.

\subsection*{Revisions}

subversion 757, 7/8/09, \texttt{DualAreaSegment} created.

subversion 1055, 3/12/10, \texttt{DualAreaSegment}\ converted to a geoquant.

\subsection*{Testing}

Trials were run and the calculations returned were verified by hand.

\subsection*{Future Work}

No future work planned.
}}%
%BeginExpansion
%TCIDATA{Version=5.00.0.2606}
%TCIDATA{LaTeXparent=1,1,functions.tex}
                      

\section*{\texttt{DualAreaSegment::DualAreaSegment}}

\subsection*{Function Prototype}

\texttt{double DualAreaSegment( Vertex vi, Vertex vj, Vertex vk, Vertex vl)}

\subsection*{Key Words}

Dual area, curvature, partial derivative, Einstein-Hilbert-Regge, geoquant.

\subsection*{Authors}

Daniel Champion

\subsection*{Introduction}

\texttt{DualAreaSegment} calculates the dual area to an edge of a
tetrahedron. \ 

\subsection*{Subsidiaries}

\textbf{Functions:}

\qquad \texttt{EdgeHeight}

\qquad \qquad \texttt{PartialEdge}

\qquad\qquad\texttt{Geometry::angle}

\qquad \texttt{FaceHeight}

\qquad\qquad\texttt{Geometry::dihedralAngle}

\textbf{Global Variables: \ }radii, etas

\textbf{Local Variables:} \ none.

\subsection*{Description}

\texttt{DualAreaSegment} is calculated with the formula:%
\begin{equation*}
\text{\texttt{DualAreaSegment(vi, vj, vk, vl)}}=
\end{equation*}%
\begin{equation*}
\frac{1}{2}\left( 
\begin{array}{c}
\text{\texttt{EdgeHeight(vi,vj,vk)}}\cdot \text{\texttt{%
FaceHeight(vi,vj,vk,vl)}} \\ 
+\text{\texttt{EdgeHeight(vi,vj,vl)}}\cdot \text{\texttt{%
FaceHeight(vi,vj,vl,vk)}}%
\end{array}%
\right) 
\end{equation*}%
\texttt{EdgeHeight} and \texttt{FaceHeight} are calculated with the
following formulae:%
\begin{align*}
\text{\texttt{EdgeHeight(vi, vj, vk)}}& =\frac{\left( \text{\texttt{%
PartialEdge(vi,vk)}}-\text{\texttt{PartialEdge(vi,vj)}}\cos \left( \alpha
_{i,jk}\right) \right) }{\sin \left( \alpha _{i,jk}\right) } \\
\text{\texttt{FaceHeight(vi, vj, vk, vl)}}& =\frac{\left( \text{\texttt{%
EdgeHeight(vi,vj,vl)}}-\text{\texttt{EdgeHeight(vi,vj,vk)}}\cos (\beta
_{ij,kl})\right) }{\sin \left( \beta _{ij,kl}\right) }
\end{align*}%
where $\alpha _{i,jk}$ is the angle at vertex $vi$ of triangle $\left\{
vi,vj,vk\right\} $, and $\beta _{ij,kl}$ is the dihedral angle along edge $%
\left\{ vi,vj\right\} $ of tetrahedron $\left\{ vi,vj,vk,vl\right\} $
(implemented with the functions \texttt{Geometry::angle} and \texttt{%
Geometry::dihedralAngle} respectively).

\texttt{DualAreaSegment} was created for the calculation performed in the
function \texttt{DualArea}, which is used in the computation of the partial
derivatives of curvature. \ These partial derivatives of curvature are used
in the calculation of the second order partial derivatives of the
Einstein-Hilbert-Regge functional for use in the optimization of said
functional using Newton's method. \ 

\subsection*{Practicum}

As an example of the usage of this function, we will calculate the dual area
to the edge $eij=\left\{ vi,vj\right\} $ (see entry: \texttt{DualArea}). \
To do this, we will sum the dual areas to each tetrahedron containing the
edge $eij$. \ 

\bigskip

\texttt{vector\TEXTsymbol{<}int\TEXTsymbol{>} sum\_over =
*(eij.getLocalTetras());}

\texttt{double sum = 0.0;}

\texttt{vector\TEXTsymbol{<}int\TEXTsymbol{>} T\_vertices, e\_vertices;}

\texttt{Tetra T;}

\texttt{Vertex vi,vj,vk,vl;}

\texttt{for(i=0; i\TEXTsymbol{<}sum\_over.size(); ++i) \{}

\qquad\texttt{T = Triangulation::tetraTable[sum\_over[i]];}

\qquad\texttt{T\_vertices = *(T.getLocalVertices());}

\qquad\texttt{e\_vertices = *(eij.getLocalVertices());}

\qquad\texttt{vi = Triangulation::vertexTable[e\_vertices[0]];}

\qquad\texttt{vj = Triangulation::vertexTable[e\_vertices[1]];}

\qquad\texttt{vk = Triangulation::vertexTable[listDifference(\&T\_vertices,
\&e\_vertices)[0]];}

\qquad\texttt{vl = Triangulation::vertexTable[listDifference(\&T\_vertices,
\&e\_vertices)[1]];}

\qquad \texttt{sum += DualAreaSegment(vi, vj, vk, vl);}

\qquad\texttt{\}}

\texttt{return sum;}

\subsection*{Limitations}

\texttt{DualAreaSegment} if fully operational and has no known limitations.
\ The function will output appropriate values provided it receives as input
four distinct vertices that define a tetrahedron.

\subsection*{Revisions}

subversion 757, 7/8/09, \texttt{DualAreaSegment} created.

subversion 1055, 3/12/10, \texttt{DualAreaSegment}\ converted to a geoquant.

\subsection*{Testing}

Trials were run and the calculations returned were verified by hand.

\subsection*{Future Work}

No future work planned.
%
%EndExpansion

\bigskip

\bigskip

%TCIMACRO{\QSubDoc{Include ApproximatorRun}{\input{ApproximatorRun.tex}}}%
%BeginExpansion
\input{ApproximatorRun.tex}%
%EndExpansion

\bigskip

\bigskip

%TCIMACRO{%
%\QSubDoc{Include Curvature_Partial}{%TCIDATA{Version=5.00.0.2606}
%TCIDATA{LaTeXparent=1,1,functions.tex}
                      

\section*{\texttt{CurvaturePartial::CurvaturePartial}}

\subsection*{Function Prototype}

\texttt{double CurvaturePartial( Vertex v\_i, Vertex v\_l )}

\subsection*{Key Words}

Curvature, Einstein-Hilbert-Regge, functional, partial derivative, geoquant.

\subsection*{Authors}

Daniel Champion

\subsection*{Introduction}

CurvaturePartial calculates the partial derivative of the curvature at a
vertex with respect to the log radius of another (possibly the same) vertex.
\ 

\subsection*{Subsidiaries}

Functions:

\qquad\texttt{isAdjVertex}

\qquad \texttt{DualArea}

\qquad \qquad \texttt{DualAreaSegment}

\qquad \qquad \qquad \texttt{FaceHeight}

\qquad \qquad \qquad \qquad \texttt{EdgeHeight}

\qquad \qquad \qquad \qquad \qquad \texttt{PartialEdge}

\qquad\texttt{listDifference}

\qquad\texttt{listIntersection}

Global Variables: \ curvature, dihedralAngle, eta, length, radius

Local Variables: none.

\subsection*{Description}

\texttt{CurvaturePartial} receives as inputs two vertices $v_{i}$ and $v_{l}$%
. \ The first corresponds to the vertex of interest, the second corresponds
to the vertex of differentiation. \ That is,%
\begin{equation*}
\text{\texttt{CurvaturePartial (v\_i, v\_l)}}=\frac{\partial }{\partial \log
r_{l}}K_{i},
\end{equation*}%
where $r_{l}$ is the radius at vertex $v_{l}$, and $K_{i}$ is the curvature
at vertex $v_{i}$. \ 

The function begins implementation by determining the relationship between $i
$ and $l$ via the trichotomy $v_{i}=v_{l}$, $v_{i}$ is adjacent to $v_{l}$,
or $v_{i}$ and $v_{l}$ are not endpoints of any edge. \ Each of the three
cases are calculated differently. \ The general formula for the variation of
curvature w.r.t. log radii was calculated by Prof. David Glickenstein and is
available at arXiv:0906.1560v1:%
\begin{equation*}
\delta K_{i}=-\sum_{edges\text{ }\left\{ i,j\right\} }\left( 2\frac{%
l_{ij}^{\ast }}{l_{ij}}-\frac{r_{i}^{2}r_{j}^{2}\left( 1-\eta
_{ij}^{2}\right) }{l_{ij}^{2}}K_{ij}\right) \left( \delta f_{j}-\delta
f_{i}\right) +K_{i}\delta f_{i},
\end{equation*}%
where $f_{i}=\log r_{i}$, $l_{ij}$ is the length of the edge $\left\{
i,j\right\} $, and $l_{ij}^{\ast }$ is the dual area calculated with the
function \texttt{DualArea}.

When $i=l$, the formula for the partial derivative $\frac{\partial}{%
\partial\log r_{l}}K_{i}$ becomes:%
\begin{equation*}
\frac{\partial}{\partial\log r_{i}}K_{i}=\sum_{edges\text{ }\left\{
i,j\right\} }\left( 2\frac{l_{ij}^{\ast}}{l_{ij}}-\frac{r_{i}^{2}r_{j}^{2}%
\left( 1-\eta_{ij}^{2}\right) }{l_{ij}^{2}}K_{ij}\right) +K_{i}.
\end{equation*}

When $v_{i}$ is adjacent to $v_{l}$ only one term in the sum survives:%
\begin{equation*}
\frac{\partial}{\partial\log r_{l}}K_{i}=-\left( 2\frac{l_{il}^{\ast}}{l_{il}%
}-\frac{r_{i}^{2}r_{l}^{2}\left( 1-\eta_{il}^{2}\right) }{l_{il}^{2}}%
K_{il}\right) .
\end{equation*}

When $v_{i}$ and $v_{j}$ are not endpoints of any edge the partial
derivative is zero. \ 

This function was created to assist in the computation of the second
derivatives of the normalized Einstein-Hilbert-Regge functional:%
\begin{equation*}
EHR=\frac{\sum K_{i}}{\sqrt[3]{\sum\limits_{tetra\text{ }t}Vol(t)}}.
\end{equation*}

Surprisingly the first derivatives of the normalized $EHR$ functional do not
require the formula for the partial derivative of curvature since 
\begin{equation*}
\frac{\partial EHR}{\partial\log r_{i}}=K_{i}.
\end{equation*}

However, the second order partial derivatives of $EHR$ certainly require the
formulas for $\frac{\partial}{\partial\log r_{l}}K_{i}$ given above. \ These
second order partial derivatives are used to construct a Hessian matrix
which is then used the optimization of the $EHR$ functional using Newton's
method (implemented by \texttt{Newtons\_Method}).

\subsection*{Practicum}

When called, \texttt{CurvaturePartial(v\_i, v\_l)} returns the partial
derivative $\frac{\partial }{\partial \log r_{l}}K_{i}$. \ An example of its
usage is in the calculation of the second order partial derivatives of the
normalized $EHR$ functional. \ For this example let 
\begin{align*}
\text{\texttt{VolSumPartial\_i}}& =\sum_{tetra\text{ }t}\frac{\partial V_{t}%
}{\partial \log r_{i}}, \\
\text{\texttt{VolSumPartial\_j}}& =\sum_{tetra\text{ }t}\frac{\partial V_{t}%
}{\partial \log r_{j}}, \\
\text{\texttt{VolSumSecondPartial}}& =\sum_{tetra\text{ }t}\frac{\partial
^{2}V_{t}}{\partial \log r_{i}\partial \log r_{j}} \\
K& =\sum_{i}K_{i}, \\
V& =\sum_{tetra\text{ }t}V_{t}.
\end{align*}%
Then the second order partial derivative $\frac{\partial ^{2}EHR}{\partial
\log r_{i}\partial \log r_{j}}$ is calculated by:

\bigskip

\qquad \texttt{result = pow(V,
(-4.0/3.0))*(1.0/3.0)*(3*V*CurvaturePartial(i,j)}

\qquad\qquad\qquad \texttt{%
-Geometry::curvature(Triangulation::vertexTable[i])*VolSumPartial\_j}

\qquad\qquad\qquad \texttt{%
-Geometry::curvature(Triangulation::vertexTable[j])*VolSumPartial\_i}

\qquad\qquad\qquad\texttt{+(4.0/3.0)*K*pow(V,
-1.0)*VolSumPartial\_i*VolSumPartial\_j}

\qquad\qquad\qquad\texttt{-K*VolSumSecondPartial);}

\bigskip

\subsection*{Limitations}

The function \texttt{CurvaturePartial} is operational for all pairs of input
integers $i$ and $l$ that are in the vertex table. \ If one of the arguments
is not in the vertex table, the function will output zero. \ 

\subsection*{Revisions}

subversion 757, 7/6/09, \texttt{CurvaturePartial} created.

subversion 1055, 3/12/10, \texttt{CurvaturePartial}\ converted to a geoquant.

\subsection*{Testing}

This function has not been tested.

\subsection*{Future Work}

Using the calculation of the partial derivative of curvature in Mathematica,
it should be compared to the output from \texttt{CurvaturePartial}.
}}%
%BeginExpansion
%TCIDATA{Version=5.00.0.2606}
%TCIDATA{LaTeXparent=1,1,functions.tex}
                      

\section*{\texttt{CurvaturePartial::CurvaturePartial}}

\subsection*{Function Prototype}

\texttt{double CurvaturePartial( Vertex v\_i, Vertex v\_l )}

\subsection*{Key Words}

Curvature, Einstein-Hilbert-Regge, functional, partial derivative, geoquant.

\subsection*{Authors}

Daniel Champion

\subsection*{Introduction}

CurvaturePartial calculates the partial derivative of the curvature at a
vertex with respect to the log radius of another (possibly the same) vertex.
\ 

\subsection*{Subsidiaries}

Functions:

\qquad\texttt{isAdjVertex}

\qquad \texttt{DualArea}

\qquad \qquad \texttt{DualAreaSegment}

\qquad \qquad \qquad \texttt{FaceHeight}

\qquad \qquad \qquad \qquad \texttt{EdgeHeight}

\qquad \qquad \qquad \qquad \qquad \texttt{PartialEdge}

\qquad\texttt{listDifference}

\qquad\texttt{listIntersection}

Global Variables: \ curvature, dihedralAngle, eta, length, radius

Local Variables: none.

\subsection*{Description}

\texttt{CurvaturePartial} receives as inputs two vertices $v_{i}$ and $v_{l}$%
. \ The first corresponds to the vertex of interest, the second corresponds
to the vertex of differentiation. \ That is,%
\begin{equation*}
\text{\texttt{CurvaturePartial (v\_i, v\_l)}}=\frac{\partial }{\partial \log
r_{l}}K_{i},
\end{equation*}%
where $r_{l}$ is the radius at vertex $v_{l}$, and $K_{i}$ is the curvature
at vertex $v_{i}$. \ 

The function begins implementation by determining the relationship between $i
$ and $l$ via the trichotomy $v_{i}=v_{l}$, $v_{i}$ is adjacent to $v_{l}$,
or $v_{i}$ and $v_{l}$ are not endpoints of any edge. \ Each of the three
cases are calculated differently. \ The general formula for the variation of
curvature w.r.t. log radii was calculated by Prof. David Glickenstein and is
available at arXiv:0906.1560v1:%
\begin{equation*}
\delta K_{i}=-\sum_{edges\text{ }\left\{ i,j\right\} }\left( 2\frac{%
l_{ij}^{\ast }}{l_{ij}}-\frac{r_{i}^{2}r_{j}^{2}\left( 1-\eta
_{ij}^{2}\right) }{l_{ij}^{2}}K_{ij}\right) \left( \delta f_{j}-\delta
f_{i}\right) +K_{i}\delta f_{i},
\end{equation*}%
where $f_{i}=\log r_{i}$, $l_{ij}$ is the length of the edge $\left\{
i,j\right\} $, and $l_{ij}^{\ast }$ is the dual area calculated with the
function \texttt{DualArea}.

When $i=l$, the formula for the partial derivative $\frac{\partial}{%
\partial\log r_{l}}K_{i}$ becomes:%
\begin{equation*}
\frac{\partial}{\partial\log r_{i}}K_{i}=\sum_{edges\text{ }\left\{
i,j\right\} }\left( 2\frac{l_{ij}^{\ast}}{l_{ij}}-\frac{r_{i}^{2}r_{j}^{2}%
\left( 1-\eta_{ij}^{2}\right) }{l_{ij}^{2}}K_{ij}\right) +K_{i}.
\end{equation*}

When $v_{i}$ is adjacent to $v_{l}$ only one term in the sum survives:%
\begin{equation*}
\frac{\partial}{\partial\log r_{l}}K_{i}=-\left( 2\frac{l_{il}^{\ast}}{l_{il}%
}-\frac{r_{i}^{2}r_{l}^{2}\left( 1-\eta_{il}^{2}\right) }{l_{il}^{2}}%
K_{il}\right) .
\end{equation*}

When $v_{i}$ and $v_{j}$ are not endpoints of any edge the partial
derivative is zero. \ 

This function was created to assist in the computation of the second
derivatives of the normalized Einstein-Hilbert-Regge functional:%
\begin{equation*}
EHR=\frac{\sum K_{i}}{\sqrt[3]{\sum\limits_{tetra\text{ }t}Vol(t)}}.
\end{equation*}

Surprisingly the first derivatives of the normalized $EHR$ functional do not
require the formula for the partial derivative of curvature since 
\begin{equation*}
\frac{\partial EHR}{\partial\log r_{i}}=K_{i}.
\end{equation*}

However, the second order partial derivatives of $EHR$ certainly require the
formulas for $\frac{\partial}{\partial\log r_{l}}K_{i}$ given above. \ These
second order partial derivatives are used to construct a Hessian matrix
which is then used the optimization of the $EHR$ functional using Newton's
method (implemented by \texttt{Newtons\_Method}).

\subsection*{Practicum}

When called, \texttt{CurvaturePartial(v\_i, v\_l)} returns the partial
derivative $\frac{\partial }{\partial \log r_{l}}K_{i}$. \ An example of its
usage is in the calculation of the second order partial derivatives of the
normalized $EHR$ functional. \ For this example let 
\begin{align*}
\text{\texttt{VolSumPartial\_i}}& =\sum_{tetra\text{ }t}\frac{\partial V_{t}%
}{\partial \log r_{i}}, \\
\text{\texttt{VolSumPartial\_j}}& =\sum_{tetra\text{ }t}\frac{\partial V_{t}%
}{\partial \log r_{j}}, \\
\text{\texttt{VolSumSecondPartial}}& =\sum_{tetra\text{ }t}\frac{\partial
^{2}V_{t}}{\partial \log r_{i}\partial \log r_{j}} \\
K& =\sum_{i}K_{i}, \\
V& =\sum_{tetra\text{ }t}V_{t}.
\end{align*}%
Then the second order partial derivative $\frac{\partial ^{2}EHR}{\partial
\log r_{i}\partial \log r_{j}}$ is calculated by:

\bigskip

\qquad \texttt{result = pow(V,
(-4.0/3.0))*(1.0/3.0)*(3*V*CurvaturePartial(i,j)}

\qquad\qquad\qquad \texttt{%
-Geometry::curvature(Triangulation::vertexTable[i])*VolSumPartial\_j}

\qquad\qquad\qquad \texttt{%
-Geometry::curvature(Triangulation::vertexTable[j])*VolSumPartial\_i}

\qquad\qquad\qquad\texttt{+(4.0/3.0)*K*pow(V,
-1.0)*VolSumPartial\_i*VolSumPartial\_j}

\qquad\qquad\qquad\texttt{-K*VolSumSecondPartial);}

\bigskip

\subsection*{Limitations}

The function \texttt{CurvaturePartial} is operational for all pairs of input
integers $i$ and $l$ that are in the vertex table. \ If one of the arguments
is not in the vertex table, the function will output zero. \ 

\subsection*{Revisions}

subversion 757, 7/6/09, \texttt{CurvaturePartial} created.

subversion 1055, 3/12/10, \texttt{CurvaturePartial}\ converted to a geoquant.

\subsection*{Testing}

This function has not been tested.

\subsection*{Future Work}

Using the calculation of the partial derivative of curvature in Mathematica,
it should be compared to the output from \texttt{CurvaturePartial}.
%
%EndExpansion

\bigskip

\bigskip

%TCIMACRO{\QSubDoc{Include dij}{%TCIDATA{Version=5.00.0.2606}
%TCIDATA{LaTeXparent=1,1,functions.tex}
                      

\section*{\texttt{PartialEdge::PartialEdge}}

\subsection*{Function Prototype}

\texttt{double PartialEdge( Vertex vi, Vertex vj)}

\subsection*{Key Words}

Partial length, geoquant.

\subsection*{Authors}

Daniel Champion, ???

\subsection*{Introduction}

The function \texttt{PartialEdge} calculates the distance from a vertex to
the center of an edge (as determined by the center of a decorated triangle).

\subsection*{Subsidiaries}

\textbf{Functions:}

\qquad\texttt{Geometry::length}

\qquad\texttt{listIntersection}

\textbf{Global Variables:} \ radii, etas.

\textbf{Local Variables:} \ Vertex vi, vj.

\subsection*{Description}

The function \texttt{PartialEdge} is calculated with the simple formula:%
\begin{equation*}
\mathtt{PartialEdge}\text{\texttt{(vi,vj)}}=\frac{%
L_{ij}^{2}+r_{i}^{2}-r_{j}^{2}}{2L_{ij}},
\end{equation*}%
where $r_{i},r_{j}$ are the radii at vertices \texttt{vi}, and \texttt{vj}
respectively, and $L_{ij}$ is the length of the edge $\left\{ vi,vj\right\} $%
. \ Notice that this formula is not symmetric in $i$ and $j$. \ 

This function plays an important role in several areas of the project
including curvature, Dirichlet energy, and the optimization of the
Einstein-Hilbert-Regge functional. \ \texttt{PartialEdge} is used in the
calculation of several quantities used in the implementation of the \texttt{%
CurvaturePartial} function, which is used in the optimization of the
normalized Einstein-Hilbert-Regge functional.

\subsection*{Practicum}

An example of the use of this function is in the calculation of the edge
height function \texttt{EdgeHeight}:

\qquad \texttt{double EdgeHeight( Vertex vi, Vertex vj, Vertex vk) \{}

\qquad\qquad\texttt{Face fijk;}

\qquad\qquad\texttt{vector\TEXTsymbol{<}int\TEXTsymbol{>} temp\_ij = }

\qquad\qquad\qquad\texttt{listIntersection(vi.getLocalFaces(),
vj.getLocalFaces());}

\qquad\qquad\texttt{vector\TEXTsymbol{<}int\TEXTsymbol{>} temp = }

\qquad\qquad\qquad\texttt{listIntersection( \&temp\_ij, vk.getLocalFaces());}

\qquad\qquad\texttt{fijk = Triangulation::faceTable[temp[0]];}

\qquad \qquad \texttt{double result = (PartialEdge(vi, vk)-PartialEdge(vi,vj)%
}

\qquad\qquad\qquad\texttt{*cos(Geometry::angle(vi,
fijk)))/sin(Geometry::angle(vi, fijk));}

\qquad\qquad\texttt{return result;}

\qquad\qquad\texttt{\}}

\subsection*{Limitations}

\texttt{PartialEdge} must receive as input two vertices that define an edge
in the triangulation. \ \texttt{PartialEdge} returns distinct values for
each permutation of the input vertices. \ 

\subsection*{Revisions}

subversion 757, 7/13/09, \texttt{PartialEdge} created.

subversion 1055, 3/12/10, \texttt{PartialEdge}\ converted to a geoquant.

\subsection*{Testing}

\texttt{dij} was tested by working out several examples by hand.

\subsection*{Future Work}

This function has been added to the Geometry class geoquants, and thus this
entry needs to be updated.
}}%
%BeginExpansion
%TCIDATA{Version=5.00.0.2606}
%TCIDATA{LaTeXparent=1,1,functions.tex}
                      

\section*{\texttt{PartialEdge::PartialEdge}}

\subsection*{Function Prototype}

\texttt{double PartialEdge( Vertex vi, Vertex vj)}

\subsection*{Key Words}

Partial length, geoquant.

\subsection*{Authors}

Daniel Champion, ???

\subsection*{Introduction}

The function \texttt{PartialEdge} calculates the distance from a vertex to
the center of an edge (as determined by the center of a decorated triangle).

\subsection*{Subsidiaries}

\textbf{Functions:}

\qquad\texttt{Geometry::length}

\qquad\texttt{listIntersection}

\textbf{Global Variables:} \ radii, etas.

\textbf{Local Variables:} \ Vertex vi, vj.

\subsection*{Description}

The function \texttt{PartialEdge} is calculated with the simple formula:%
\begin{equation*}
\mathtt{PartialEdge}\text{\texttt{(vi,vj)}}=\frac{%
L_{ij}^{2}+r_{i}^{2}-r_{j}^{2}}{2L_{ij}},
\end{equation*}%
where $r_{i},r_{j}$ are the radii at vertices \texttt{vi}, and \texttt{vj}
respectively, and $L_{ij}$ is the length of the edge $\left\{ vi,vj\right\} $%
. \ Notice that this formula is not symmetric in $i$ and $j$. \ 

This function plays an important role in several areas of the project
including curvature, Dirichlet energy, and the optimization of the
Einstein-Hilbert-Regge functional. \ \texttt{PartialEdge} is used in the
calculation of several quantities used in the implementation of the \texttt{%
CurvaturePartial} function, which is used in the optimization of the
normalized Einstein-Hilbert-Regge functional.

\subsection*{Practicum}

An example of the use of this function is in the calculation of the edge
height function \texttt{EdgeHeight}:

\qquad \texttt{double EdgeHeight( Vertex vi, Vertex vj, Vertex vk) \{}

\qquad\qquad\texttt{Face fijk;}

\qquad\qquad\texttt{vector\TEXTsymbol{<}int\TEXTsymbol{>} temp\_ij = }

\qquad\qquad\qquad\texttt{listIntersection(vi.getLocalFaces(),
vj.getLocalFaces());}

\qquad\qquad\texttt{vector\TEXTsymbol{<}int\TEXTsymbol{>} temp = }

\qquad\qquad\qquad\texttt{listIntersection( \&temp\_ij, vk.getLocalFaces());}

\qquad\qquad\texttt{fijk = Triangulation::faceTable[temp[0]];}

\qquad \qquad \texttt{double result = (PartialEdge(vi, vk)-PartialEdge(vi,vj)%
}

\qquad\qquad\qquad\texttt{*cos(Geometry::angle(vi,
fijk)))/sin(Geometry::angle(vi, fijk));}

\qquad\qquad\texttt{return result;}

\qquad\qquad\texttt{\}}

\subsection*{Limitations}

\texttt{PartialEdge} must receive as input two vertices that define an edge
in the triangulation. \ \texttt{PartialEdge} returns distinct values for
each permutation of the input vertices. \ 

\subsection*{Revisions}

subversion 757, 7/13/09, \texttt{PartialEdge} created.

subversion 1055, 3/12/10, \texttt{PartialEdge}\ converted to a geoquant.

\subsection*{Testing}

\texttt{dij} was tested by working out several examples by hand.

\subsection*{Future Work}

This function has been added to the Geometry class geoquants, and thus this
entry needs to be updated.
%
%EndExpansion

\bigskip

\bigskip

%TCIMACRO{\QSubDoc{Include EHR_Partial}{%TCIDATA{Version=5.00.0.2606}
%TCIDATA{LaTeXparent=1,1,functions.tex}
                      

\section*{\texttt{EHRPartial::EHRPartial}}

\subsection*{Function Prototype}

\texttt{double EHRPartial(int i)}

\subsection*{Key Words}

Einstein-Hilbert-Regge, functional, Newton's method, partial derivative,
geoquant.

\subsection*{Authors}

Daniel Champion

\subsection*{Introduction}

\texttt{EHRPartial} calculates the partial derivative of the normalized
Einstein-Hilbert-Regge functional with respect to log radii. \ 

\subsection*{Subsidiaries}

\textbf{Functions:}

\qquad \texttt{TotalVolume}

\qquad \texttt{TotalCurvature}

\qquad \texttt{VolumePartial}

\qquad\qquad\texttt{listDifference}

\qquad\qquad\texttt{listIntersection}

\textbf{Global Variables:} radii, etas, curvature, volume

\textbf{Local Variables:} \ none

\subsection*{Description}

The normalized Einstein-Hilbert-Regge functional is given by the expression:%
\begin{equation*}
EHR=\frac{\sum\limits_{j}K_{j}}{\sqrt[3]{\sum\limits_{tetra\text{ }t}V_{t}}},
\end{equation*}%
where $K_{i}$ is the curvature at vertex $j$, and $V_{t}$ is the volume of
tetrahedron $t$. \ It can be shown (see arXiv:0906.1560v1) that 
\begin{equation*}
\frac{\partial }{\partial \log r_{i}}\left( \sum\limits_{j}K_{j}\right)
=K_{i},
\end{equation*}%
hence the partial derivative of the normalized EHR functional becomes:%
\begin{align*}
\frac{\partial }{\partial \log r_{i}}EHR& =\frac{K_{i}\sqrt[3]{%
\sum\limits_{tetra\text{ }t}V_{t}}-\frac{1}{3}\left( \sum\limits_{tetra\text{
}t}V_{t}\right) ^{-\frac{2}{3}}\sum\limits_{tetra\text{ }t}\frac{\partial
V_{t}}{\partial \log r_{i}}\sum\limits_{j}K_{j}}{\left( \sum\limits_{tetra%
\text{ }t}V_{t}\right) ^{\frac{2}{3}}} \\
& =V^{-\frac{4}{3}}\left( K_{i}V-\frac{1}{3}K\sum\limits_{tetra\text{ }t}%
\frac{\partial V_{t}}{\partial \log r_{i}}\right) ,
\end{align*}%
where $V$ is the total volume of all tetrahedra in the triangulation and $K$
is the sum of the curvatures over all vertices in the triangulation. \ 
\texttt{EHRPartial (v\_i)} calculates $\frac{\partial }{\partial \log r_{i}}%
EHR$. \ 

The primary use of this function is in the calculation of the negative
gradient of the EHR functional for use in optimization of the functional
using Newton's method. \ The formula for $\frac{\partial }{\partial \log
r_{i}}EHR$ given above was also used in the calculation of the second order
partial derivatives of the EHR functional, implemented in \texttt{%
EHRSecondPartial}.

\subsection*{Practicum}

As an example of the use of this function, the calculation of the gradient
of the EHR functional will be calculated. \ The negative gradient will be
outputted as the array \texttt{EHRneg\_gradient}.

\bigskip

\qquad\texttt{double EHRneg\_gradient[Triangulation::vertexTable.size()];}

\qquad \texttt{for(int i=0; i \TEXTsymbol{<}
Triangulation::vertexTable.size(); ++i) \{}

\qquad \qquad \texttt{Vertex v\_i = Triangulation::vertexTable[i+1];}

\qquad \qquad \texttt{EHRneg\_gradient[i] = -1.0*EHRPartial(v\_i);}

\qquad\qquad\texttt{\}}

\subsection*{Limitations}

The function \texttt{EHRPartial} is fully functional with no known
limitations. \ It will return appropriate values so long as it is called
with an integer in the vertex table. \ 

\subsection*{Revisions}

List the major revisions to the function with dates and a one sentence
comment. \ Example:

subversion 757, 7/7/09, \texttt{EHRPartial} created.

subversion 1055, 3/12/10, \texttt{EHRPartial}\ converted to a geoquant.

\subsection*{Testing}

This function has not been tested.

\subsection*{Future Work}

A testing regime should be instituted for this function. \ 
}}%
%BeginExpansion
%TCIDATA{Version=5.00.0.2606}
%TCIDATA{LaTeXparent=1,1,functions.tex}
                      

\section*{\texttt{EHRPartial::EHRPartial}}

\subsection*{Function Prototype}

\texttt{double EHRPartial(int i)}

\subsection*{Key Words}

Einstein-Hilbert-Regge, functional, Newton's method, partial derivative,
geoquant.

\subsection*{Authors}

Daniel Champion

\subsection*{Introduction}

\texttt{EHRPartial} calculates the partial derivative of the normalized
Einstein-Hilbert-Regge functional with respect to log radii. \ 

\subsection*{Subsidiaries}

\textbf{Functions:}

\qquad \texttt{TotalVolume}

\qquad \texttt{TotalCurvature}

\qquad \texttt{VolumePartial}

\qquad\qquad\texttt{listDifference}

\qquad\qquad\texttt{listIntersection}

\textbf{Global Variables:} radii, etas, curvature, volume

\textbf{Local Variables:} \ none

\subsection*{Description}

The normalized Einstein-Hilbert-Regge functional is given by the expression:%
\begin{equation*}
EHR=\frac{\sum\limits_{j}K_{j}}{\sqrt[3]{\sum\limits_{tetra\text{ }t}V_{t}}},
\end{equation*}%
where $K_{i}$ is the curvature at vertex $j$, and $V_{t}$ is the volume of
tetrahedron $t$. \ It can be shown (see arXiv:0906.1560v1) that 
\begin{equation*}
\frac{\partial }{\partial \log r_{i}}\left( \sum\limits_{j}K_{j}\right)
=K_{i},
\end{equation*}%
hence the partial derivative of the normalized EHR functional becomes:%
\begin{align*}
\frac{\partial }{\partial \log r_{i}}EHR& =\frac{K_{i}\sqrt[3]{%
\sum\limits_{tetra\text{ }t}V_{t}}-\frac{1}{3}\left( \sum\limits_{tetra\text{
}t}V_{t}\right) ^{-\frac{2}{3}}\sum\limits_{tetra\text{ }t}\frac{\partial
V_{t}}{\partial \log r_{i}}\sum\limits_{j}K_{j}}{\left( \sum\limits_{tetra%
\text{ }t}V_{t}\right) ^{\frac{2}{3}}} \\
& =V^{-\frac{4}{3}}\left( K_{i}V-\frac{1}{3}K\sum\limits_{tetra\text{ }t}%
\frac{\partial V_{t}}{\partial \log r_{i}}\right) ,
\end{align*}%
where $V$ is the total volume of all tetrahedra in the triangulation and $K$
is the sum of the curvatures over all vertices in the triangulation. \ 
\texttt{EHRPartial (v\_i)} calculates $\frac{\partial }{\partial \log r_{i}}%
EHR$. \ 

The primary use of this function is in the calculation of the negative
gradient of the EHR functional for use in optimization of the functional
using Newton's method. \ The formula for $\frac{\partial }{\partial \log
r_{i}}EHR$ given above was also used in the calculation of the second order
partial derivatives of the EHR functional, implemented in \texttt{%
EHRSecondPartial}.

\subsection*{Practicum}

As an example of the use of this function, the calculation of the gradient
of the EHR functional will be calculated. \ The negative gradient will be
outputted as the array \texttt{EHRneg\_gradient}.

\bigskip

\qquad\texttt{double EHRneg\_gradient[Triangulation::vertexTable.size()];}

\qquad \texttt{for(int i=0; i \TEXTsymbol{<}
Triangulation::vertexTable.size(); ++i) \{}

\qquad \qquad \texttt{Vertex v\_i = Triangulation::vertexTable[i+1];}

\qquad \qquad \texttt{EHRneg\_gradient[i] = -1.0*EHRPartial(v\_i);}

\qquad\qquad\texttt{\}}

\subsection*{Limitations}

The function \texttt{EHRPartial} is fully functional with no known
limitations. \ It will return appropriate values so long as it is called
with an integer in the vertex table. \ 

\subsection*{Revisions}

List the major revisions to the function with dates and a one sentence
comment. \ Example:

subversion 757, 7/7/09, \texttt{EHRPartial} created.

subversion 1055, 3/12/10, \texttt{EHRPartial}\ converted to a geoquant.

\subsection*{Testing}

This function has not been tested.

\subsection*{Future Work}

A testing regime should be instituted for this function. \ 
%
%EndExpansion

\bigskip

\bigskip

%TCIMACRO{%
%\QSubDoc{Include EHR_Second_Partial}{%TCIDATA{Version=5.00.0.2606}
%TCIDATA{LaTeXparent=1,1,functions.tex}
                      

\section*{\texttt{EHRSecondPartial::EHRSecondPartial}}

\subsection*{Function Prototype}

\texttt{double EHRSecondPartial (Vertex v\_i, Vertex v\_j)}

\subsection*{Key Words}

Einstein-Hilbert-Regge, functional, partial derivative, Hessian, geoquant.

\subsection*{Authors}

Daniel Champion

\subsection*{Introduction}

\texttt{EHRSecondPartial} calculates the second order partial derivatives of
the normalized Einstein-Hilbert-Regge functional with respect to log radii.
\ 

\subsection*{Subsidiaries}

\textbf{Functions:}

\qquad \texttt{CurvaturePartial}

\qquad\qquad\texttt{isAdjVertex}

\qquad \qquad \texttt{DualArea}

\qquad \qquad \qquad \texttt{DualAreaSegment}

\qquad \qquad \qquad \qquad \texttt{FaceHeight}

\qquad \qquad \qquad \qquad \qquad \texttt{EdgeHeight}

\qquad \qquad \qquad \qquad \qquad \qquad \texttt{PartialEdge}

\qquad\qquad\texttt{listDifference}

\qquad\qquad\texttt{listIntersection}

\qquad \texttt{TotalCurvature}

\qquad\qquad\texttt{Geometry::curvature}

\qquad \texttt{TotalVolume}

\qquad\qquad\texttt{Geometry::volume}

\qquad \texttt{VolumePartial}

\qquad\qquad\texttt{listDifference}

\qquad\qquad\texttt{listIntersection.}

\qquad \texttt{VolumeSecondPartial}

\textbf{Global Variables: }\ radii, etas.

\textbf{Local Variables:} \ none.

\subsection*{Description}

The normalized Einstein-Hilbert-Regge functional is given by the expression:%
\begin{equation*}
EHR=\frac{\sum\limits_{j}K_{j}}{\sqrt[3]{\sum\limits_{tetra\text{ }t}V_{t}}},
\end{equation*}
where $K_{i}$ is the curvature at vertex $j$, and $V_{t}$ is the volume of
tetrahedron $t$. \ It can be shown (see arXiv:0906.1560v1) that 
\begin{equation*}
\frac{\partial}{\partial\log r_{i}}\left( \sum\limits_{j}K_{j}\right) =K_{i},
\end{equation*}
hence the partial derivative of the normalized EHR functional simplifies to
become:%
\begin{equation*}
\frac{\partial}{\partial\log r_{i}}EHR=V^{-\frac{4}{3}}\left( K_{i}V-\frac {1%
}{3}K\sum\limits_{tetra\text{ }t}\frac{\partial V_{t}}{\partial\log r_{i}}%
\right) ,
\end{equation*}
where $V$ is the total volume of all tetrahedra in the triangulation and $K$
is the sum of the curvatures over all vertices in the triangulation. \
Differentiating this result with respect to $\log r_{j}$ yields:%
\begin{equation*}
\frac{\partial^{2}}{\partial\log r_{i}\partial\log r_{j}}EHR=V^{-\frac{4}{3}%
}\left( 
\begin{array}{c}
V\frac{\partial K_{i}}{\partial\log r_{j}}-\frac{1}{3}K_{i}\sum\limits_{t}%
\frac{\partial V_{t}}{\partial\log r_{j}}-\frac{1}{3}K_{j}\sum\limits_{t}%
\frac{\partial V_{t}}{\partial\log r_{i}} \\ 
+\frac{4}{9}KV^{-1}\sum\limits_{t}\frac{\partial V_{t}}{\partial\log r_{j}}%
\sum\limits_{t}\frac{\partial V_{t}}{\partial\log r_{i}}-\frac{1}{3}%
K\sum\limits_{t}\frac{\partial^{2}V_{t}}{\partial\log r_{i}\partial\log r_{j}%
}%
\end{array}
\right) .
\end{equation*}

When called, \texttt{EHRSecondPartial} calculates the formula above, that is:%
\begin{equation*}
\text{\texttt{EHR\_Second\_Partial (i,j)}}=\frac{\partial ^{2}}{\partial
\log r_{i}\partial \log r_{j}}EHR.
\end{equation*}

The use of this function is in the population of the Hessian matrix for the
normalized EHR functional. \ This Hessian matrix is used in the optimization
of the EHR functional using Newton's method.

\subsection*{Practicum}

As an example of the usage of \texttt{EHRSecondPartial}, the Hessian matrix
of the normalized EHR functional will be populated. \ In this example, the
Hessian matrix is the array EHRhessian. \ The example reduced computation
time by only calling \texttt{EHRSecondPartial} for the upper triangular
portion of the EHRhessian array, and symmetrically copies the entries above
the diagonal to the corresponding location below the diagonal. \
Furthermore, in C++ arrays begin indexing at zero, however the vertices of
the triangulations begin indexing at 1, requiring a shift of one in the
population step. \ Note that triangulations that do not label the vertices
consecutively will not be compatible with the following code. \ 

\bigskip

\qquad\texttt{double
EHRhessian[Triangulation::vertexTable.size()][Triangulation::vertexTable.size()];%
}

\qquad\texttt{for(int i = 0; i \TEXTsymbol{<}
Triangulation::vertexTable.size(); ++i) \{}

\qquad \qquad \texttt{for(int j = 0; j \TEXTsymbol{<}
Triangulation::vertexTable.size(); ++j) \{}

\qquad \qquad \qquad \texttt{Vertex vi = Triangulation::vertexYable[i+1];}

\qquad \qquad \qquad \texttt{Vertex vj = Triangulation::vertexTable[j+1];}

\qquad \qquad \qquad \texttt{if (i \TEXTsymbol{<}= j) \{}

\qquad \qquad \qquad \qquad \texttt{EHRhessian[i][j]=EHRSecondPartial( vi ,
vj );}

\qquad\qquad\qquad\qquad\texttt{EHRhessian[j][i]=EHRhessian[i][j];}

\qquad\qquad\qquad\qquad\texttt{\}}

\qquad\qquad\qquad\texttt{\}}

\qquad\qquad\texttt{\}}

\subsection*{Limitations}

\texttt{EHRSecondPartial} is fully operational with no known limitations. \
The function will output appropriate values provided it receives as inputs a
pair of integers in the vertex table. \ 

\subsection*{Revisions}

subversion 757, 7/7/09, \texttt{EHRSecondPartial} created.

subversion 1055, 3/12/10, \texttt{EHRSecondPartial}\ converted to a geoquant.

\subsection*{Testing}

This function has not been tested.

\subsection*{Future Work}

A testing regime should be instituted for this function. \ 
}}%
%BeginExpansion
%TCIDATA{Version=5.00.0.2606}
%TCIDATA{LaTeXparent=1,1,functions.tex}
                      

\section*{\texttt{EHRSecondPartial::EHRSecondPartial}}

\subsection*{Function Prototype}

\texttt{double EHRSecondPartial (Vertex v\_i, Vertex v\_j)}

\subsection*{Key Words}

Einstein-Hilbert-Regge, functional, partial derivative, Hessian, geoquant.

\subsection*{Authors}

Daniel Champion

\subsection*{Introduction}

\texttt{EHRSecondPartial} calculates the second order partial derivatives of
the normalized Einstein-Hilbert-Regge functional with respect to log radii.
\ 

\subsection*{Subsidiaries}

\textbf{Functions:}

\qquad \texttt{CurvaturePartial}

\qquad\qquad\texttt{isAdjVertex}

\qquad \qquad \texttt{DualArea}

\qquad \qquad \qquad \texttt{DualAreaSegment}

\qquad \qquad \qquad \qquad \texttt{FaceHeight}

\qquad \qquad \qquad \qquad \qquad \texttt{EdgeHeight}

\qquad \qquad \qquad \qquad \qquad \qquad \texttt{PartialEdge}

\qquad\qquad\texttt{listDifference}

\qquad\qquad\texttt{listIntersection}

\qquad \texttt{TotalCurvature}

\qquad\qquad\texttt{Geometry::curvature}

\qquad \texttt{TotalVolume}

\qquad\qquad\texttt{Geometry::volume}

\qquad \texttt{VolumePartial}

\qquad\qquad\texttt{listDifference}

\qquad\qquad\texttt{listIntersection.}

\qquad \texttt{VolumeSecondPartial}

\textbf{Global Variables: }\ radii, etas.

\textbf{Local Variables:} \ none.

\subsection*{Description}

The normalized Einstein-Hilbert-Regge functional is given by the expression:%
\begin{equation*}
EHR=\frac{\sum\limits_{j}K_{j}}{\sqrt[3]{\sum\limits_{tetra\text{ }t}V_{t}}},
\end{equation*}
where $K_{i}$ is the curvature at vertex $j$, and $V_{t}$ is the volume of
tetrahedron $t$. \ It can be shown (see arXiv:0906.1560v1) that 
\begin{equation*}
\frac{\partial}{\partial\log r_{i}}\left( \sum\limits_{j}K_{j}\right) =K_{i},
\end{equation*}
hence the partial derivative of the normalized EHR functional simplifies to
become:%
\begin{equation*}
\frac{\partial}{\partial\log r_{i}}EHR=V^{-\frac{4}{3}}\left( K_{i}V-\frac {1%
}{3}K\sum\limits_{tetra\text{ }t}\frac{\partial V_{t}}{\partial\log r_{i}}%
\right) ,
\end{equation*}
where $V$ is the total volume of all tetrahedra in the triangulation and $K$
is the sum of the curvatures over all vertices in the triangulation. \
Differentiating this result with respect to $\log r_{j}$ yields:%
\begin{equation*}
\frac{\partial^{2}}{\partial\log r_{i}\partial\log r_{j}}EHR=V^{-\frac{4}{3}%
}\left( 
\begin{array}{c}
V\frac{\partial K_{i}}{\partial\log r_{j}}-\frac{1}{3}K_{i}\sum\limits_{t}%
\frac{\partial V_{t}}{\partial\log r_{j}}-\frac{1}{3}K_{j}\sum\limits_{t}%
\frac{\partial V_{t}}{\partial\log r_{i}} \\ 
+\frac{4}{9}KV^{-1}\sum\limits_{t}\frac{\partial V_{t}}{\partial\log r_{j}}%
\sum\limits_{t}\frac{\partial V_{t}}{\partial\log r_{i}}-\frac{1}{3}%
K\sum\limits_{t}\frac{\partial^{2}V_{t}}{\partial\log r_{i}\partial\log r_{j}%
}%
\end{array}
\right) .
\end{equation*}

When called, \texttt{EHRSecondPartial} calculates the formula above, that is:%
\begin{equation*}
\text{\texttt{EHR\_Second\_Partial (i,j)}}=\frac{\partial ^{2}}{\partial
\log r_{i}\partial \log r_{j}}EHR.
\end{equation*}

The use of this function is in the population of the Hessian matrix for the
normalized EHR functional. \ This Hessian matrix is used in the optimization
of the EHR functional using Newton's method.

\subsection*{Practicum}

As an example of the usage of \texttt{EHRSecondPartial}, the Hessian matrix
of the normalized EHR functional will be populated. \ In this example, the
Hessian matrix is the array EHRhessian. \ The example reduced computation
time by only calling \texttt{EHRSecondPartial} for the upper triangular
portion of the EHRhessian array, and symmetrically copies the entries above
the diagonal to the corresponding location below the diagonal. \
Furthermore, in C++ arrays begin indexing at zero, however the vertices of
the triangulations begin indexing at 1, requiring a shift of one in the
population step. \ Note that triangulations that do not label the vertices
consecutively will not be compatible with the following code. \ 

\bigskip

\qquad\texttt{double
EHRhessian[Triangulation::vertexTable.size()][Triangulation::vertexTable.size()];%
}

\qquad\texttt{for(int i = 0; i \TEXTsymbol{<}
Triangulation::vertexTable.size(); ++i) \{}

\qquad \qquad \texttt{for(int j = 0; j \TEXTsymbol{<}
Triangulation::vertexTable.size(); ++j) \{}

\qquad \qquad \qquad \texttt{Vertex vi = Triangulation::vertexYable[i+1];}

\qquad \qquad \qquad \texttt{Vertex vj = Triangulation::vertexTable[j+1];}

\qquad \qquad \qquad \texttt{if (i \TEXTsymbol{<}= j) \{}

\qquad \qquad \qquad \qquad \texttt{EHRhessian[i][j]=EHRSecondPartial( vi ,
vj );}

\qquad\qquad\qquad\qquad\texttt{EHRhessian[j][i]=EHRhessian[i][j];}

\qquad\qquad\qquad\qquad\texttt{\}}

\qquad\qquad\qquad\texttt{\}}

\qquad\qquad\texttt{\}}

\subsection*{Limitations}

\texttt{EHRSecondPartial} is fully operational with no known limitations. \
The function will output appropriate values provided it receives as inputs a
pair of integers in the vertex table. \ 

\subsection*{Revisions}

subversion 757, 7/7/09, \texttt{EHRSecondPartial} created.

subversion 1055, 3/12/10, \texttt{EHRSecondPartial}\ converted to a geoquant.

\subsection*{Testing}

This function has not been tested.

\subsection*{Future Work}

A testing regime should be instituted for this function. \ 
%
%EndExpansion

\bigskip

\bigskip

%TCIMACRO{\QSubDoc{Include flip}{\input{flip.tex}}}%
%BeginExpansion
\input{flip.tex}%
%EndExpansion

\bigskip

\bigskip

%TCIMACRO{\QSubDoc{Include GeoquantAt}{\input{GeoquantAt.tex}}}%
%BeginExpansion
\input{GeoquantAt.tex}%
%EndExpansion

\bigskip

\bigskip

%TCIMACRO{\QSubDoc{Include hij_k}{%TCIDATA{Version=5.00.0.2606}
%TCIDATA{LaTeXparent=1,1,functions.tex}
                      

\section*{\texttt{EdgeHeight::EdgeHeight\label{Edge Height FUNCTION}}}

\subsection*{Function Prototype}

\texttt{double EdgeHeight( Vertex vi, Vertex vj, Vertex vk)}

\subsection*{Key Words}

Edge height, partial edge, geoquant.

\subsection*{Authors}

Daniel Champion

\subsection*{Introduction}

The function \texttt{EdgeHeight} calculates the edge height (to the center
of a triangular face) of an edge in the triangulation. \ 

\subsection*{Subsidiaries}

\textbf{Functions:}

\qquad \texttt{PartialEdge}

\qquad\texttt{Geometry::angle}

\qquad\texttt{listIntersection}

\textbf{Global Variables:} radii, etas.

\textbf{Local Variables:} Vertex vi, vj, vk.

\subsection*{Description}

The calculation of \texttt{EdgeHeight} involves the simple formula:%
\begin{equation*}
\text{\texttt{EdgeHeight(vi, vj, vk)}}=
\end{equation*}%
\begin{equation*}
\frac{\left( \text{\texttt{PartialEdge(vi,vk)}}-\text{\texttt{%
PartialEdge(vi,vj)}}\cos \left( \alpha _{i,jk}\right) \right) }{\sin \left(
\alpha _{i,jk}\right) }
\end{equation*}%
where $\alpha _{i,jk}$ is the angle at vertex $vi$ of triangle $\left\{
vi,vj,vk\right\} $. \ A geometric interpretation of this quantity is a
follows. \ Given a decorated triangle (triangle with radii and eta values
assigned to the vertices and edges respectively), the center of this
triangle can be calculated as the common power point of its embedding into
two-dimensional Euclidean space. \ The perpendicular distance from this
center point to the edge $\left\{ vi,vj\right\} $ is exactly \texttt{%
EdgeHeight(vi, vj, vk)}. \ Take note that the first two vertices in the
function call correspond to the preferred edge, and the third vertex in the
function call identifies the triangle. \ 

A primary use of this function is in the calculation of several quantities
needed for the \texttt{CurvaturePartial} function used in the optimization
of the normalized Einstein-Hilbert-Regge functional.

\subsection*{Practicum}

An example of the use of this function is in the calculation of the dual
areas, \texttt{DualAreaSegment}, to an edge of a three dimensional
triangulation.

\qquad \texttt{double DualAreaSegment( Vertex vi, Vertex vj, Vertex vk,
Vertex vl)}

\qquad\qquad\texttt{\{}

\qquad \qquad \texttt{double result =
0.5*(EdgeHeight(vi,vj,vk)*FaceHeight(vi,vj,vk,vl)}

\qquad \qquad \qquad \texttt{+EdgeHeight(vi,vj,vl)*FaceHeight(vi,vj,vl,vk));}

\qquad\qquad\texttt{return result;}

\qquad\qquad\texttt{\}}

\subsection*{Limitations}

\texttt{EdgeHeight} must receive as input three vertices of a face of the
triangulation. \ Moreover, the first two vertices in the function call
identify an edge, and can be in any order, however the third vertex in the
function call identifies the face and can not be permuted with the other two
vertices. \ 

\subsection*{Revisions}

subversion 757, 6/8/09, \texttt{EdgeHeight} created.

subversion 1055, 3/12/10, \texttt{EdgeHeight}\ converted to a geoquant.

\subsection*{Testing}

This function was not tested.

\subsection*{Future Work}

This function has been incorporated into the Geometry class geoquants, and
thus this entry needs to be updated. \ 
}}%
%BeginExpansion
%TCIDATA{Version=5.00.0.2606}
%TCIDATA{LaTeXparent=1,1,functions.tex}
                      

\section*{\texttt{EdgeHeight::EdgeHeight\label{Edge Height FUNCTION}}}

\subsection*{Function Prototype}

\texttt{double EdgeHeight( Vertex vi, Vertex vj, Vertex vk)}

\subsection*{Key Words}

Edge height, partial edge, geoquant.

\subsection*{Authors}

Daniel Champion

\subsection*{Introduction}

The function \texttt{EdgeHeight} calculates the edge height (to the center
of a triangular face) of an edge in the triangulation. \ 

\subsection*{Subsidiaries}

\textbf{Functions:}

\qquad \texttt{PartialEdge}

\qquad\texttt{Geometry::angle}

\qquad\texttt{listIntersection}

\textbf{Global Variables:} radii, etas.

\textbf{Local Variables:} Vertex vi, vj, vk.

\subsection*{Description}

The calculation of \texttt{EdgeHeight} involves the simple formula:%
\begin{equation*}
\text{\texttt{EdgeHeight(vi, vj, vk)}}=
\end{equation*}%
\begin{equation*}
\frac{\left( \text{\texttt{PartialEdge(vi,vk)}}-\text{\texttt{%
PartialEdge(vi,vj)}}\cos \left( \alpha _{i,jk}\right) \right) }{\sin \left(
\alpha _{i,jk}\right) }
\end{equation*}%
where $\alpha _{i,jk}$ is the angle at vertex $vi$ of triangle $\left\{
vi,vj,vk\right\} $. \ A geometric interpretation of this quantity is a
follows. \ Given a decorated triangle (triangle with radii and eta values
assigned to the vertices and edges respectively), the center of this
triangle can be calculated as the common power point of its embedding into
two-dimensional Euclidean space. \ The perpendicular distance from this
center point to the edge $\left\{ vi,vj\right\} $ is exactly \texttt{%
EdgeHeight(vi, vj, vk)}. \ Take note that the first two vertices in the
function call correspond to the preferred edge, and the third vertex in the
function call identifies the triangle. \ 

A primary use of this function is in the calculation of several quantities
needed for the \texttt{CurvaturePartial} function used in the optimization
of the normalized Einstein-Hilbert-Regge functional.

\subsection*{Practicum}

An example of the use of this function is in the calculation of the dual
areas, \texttt{DualAreaSegment}, to an edge of a three dimensional
triangulation.

\qquad \texttt{double DualAreaSegment( Vertex vi, Vertex vj, Vertex vk,
Vertex vl)}

\qquad\qquad\texttt{\{}

\qquad \qquad \texttt{double result =
0.5*(EdgeHeight(vi,vj,vk)*FaceHeight(vi,vj,vk,vl)}

\qquad \qquad \qquad \texttt{+EdgeHeight(vi,vj,vl)*FaceHeight(vi,vj,vl,vk));}

\qquad\qquad\texttt{return result;}

\qquad\qquad\texttt{\}}

\subsection*{Limitations}

\texttt{EdgeHeight} must receive as input three vertices of a face of the
triangulation. \ Moreover, the first two vertices in the function call
identify an edge, and can be in any order, however the third vertex in the
function call identifies the face and can not be permuted with the other two
vertices. \ 

\subsection*{Revisions}

subversion 757, 6/8/09, \texttt{EdgeHeight} created.

subversion 1055, 3/12/10, \texttt{EdgeHeight}\ converted to a geoquant.

\subsection*{Testing}

This function was not tested.

\subsection*{Future Work}

This function has been incorporated into the Geometry class geoquants, and
thus this entry needs to be updated. \ 
%
%EndExpansion

\bigskip

\bigskip

%TCIMACRO{\QSubDoc{Include hijk_l}{%TCIDATA{Version=5.00.0.2606}
%TCIDATA{LaTeXparent=1,1,functions.tex}
                      

\section*{\texttt{FaceHeight::FaceHeight\label{Face Height FUNCTION}}}

\subsection*{Function Prototype}

\texttt{double FaceHeight( Vertex vi, Vertex vj, Vertex vk, Vertex vl)}

\subsection*{Key Words}

Face height, edge height, geoquant.

\subsection*{Authors}

Daniel Champion

\subsection*{Introduction}

The function \texttt{FaceHeight}\ calculates the face height to the center
of a tetrahedron. \ 

\subsection*{Subsidiaries}

\textbf{Functions:}

\qquad\texttt{Geometry::dihedralAngle}

\qquad \texttt{EdgeHeight}

\qquad\texttt{listIntersection}

\textbf{Global Variables: }\ radii, etas.

\textbf{Local Variables:} \ Vertex vi, vj, vk, vl.

\subsection*{Description}

The calculation of \texttt{FaceHeight} involves the simple formula:%
\begin{equation*}
\text{\texttt{FaceHeight(vi, vj, vk, vl)}}=
\end{equation*}

\begin{equation*}
\frac{\left( \text{\texttt{EdgeHeight(vi,vj,vl)}}-\text{\texttt{%
EdgeHeight(vi,vj,vk)}}\cos (\beta _{ij,kl})\right) }{\sin \left( \beta
_{ij,kl}\right) }
\end{equation*}%
where $\beta _{ij,kl}$ is the dihedral angle along edge $\left\{
vi,vj\right\} $ of tetrahedron $\left\{ vi,vj,vk,vl\right\} $. \ A geometric
interpretation of this quantity is a follows. \ Given a decorated
tetrahedron (tetrahedron with radii and eta values assigned to the vertices
and edges respectively), the center of this tetrahedron can be calculated as
the common power point of its embedding into three-dimensional Euclidean
space. \ The perpendicular distance from this center point to the face $%
\left\{ vi,vj,vk\right\} $ is exactly \texttt{FaceHeight (vi, vj, vk, vl)}.
\ Take note that the first three vertices in the function call correspond to
the preferred face, and the fourth vertex in the function call identifies
the tetrahedron. \ 

A primary use of this function is in the calculation of several quantities
needed for the \texttt{CurvaturePartial} function used in the optimization
of the normalized Einstein-Hilbert-Regge functional.

\subsection*{Practicum}

An example of the use of this function is in the calculation of the dual
areas, \texttt{DualAreaSegment}, to an edge of a three dimensional
triangulation.

\qquad \texttt{double DualAreaSegment( Vertex vi, Vertex vj, Vertex vk,
Vertex vl)}

\qquad\qquad\texttt{\{}

\qquad \qquad \texttt{double result =
0.5*(EdgeHeight(vi,vj,vk)*FaceHeight(vi,vj,vk,vl)}

\qquad \qquad \qquad \texttt{+EdgeHeight(vi,vj,vl)*FaceHeight(vi,vj,vl,vk));}

\qquad\qquad\texttt{return result;}

\qquad\qquad\texttt{\}}

\subsection*{Limitations}

\texttt{faceHeight} must receive as input four vertices of a tetrahedron of
the triangulation. \ Moreover, the first three vertices in the function call
identify a face and can be in any order, however the fourth vertex in the
function call identifies the tetrahedron and can not be permuted with the
other three vertices. \ 

\subsection*{Revisions}

subversion 757, 6/8/09, \texttt{FaceHeight} created.

subversion 1055, 3/12/10, \texttt{FaceHeight}\ converted to a geoquant.

\subsection*{Testing}

This function was not tested.

\subsection*{Future Work}

This function has been incorporated into the Geometry class geoquants, and
thus this entry needs to be updated. \ 
}}%
%BeginExpansion
%TCIDATA{Version=5.00.0.2606}
%TCIDATA{LaTeXparent=1,1,functions.tex}
                      

\section*{\texttt{FaceHeight::FaceHeight\label{Face Height FUNCTION}}}

\subsection*{Function Prototype}

\texttt{double FaceHeight( Vertex vi, Vertex vj, Vertex vk, Vertex vl)}

\subsection*{Key Words}

Face height, edge height, geoquant.

\subsection*{Authors}

Daniel Champion

\subsection*{Introduction}

The function \texttt{FaceHeight}\ calculates the face height to the center
of a tetrahedron. \ 

\subsection*{Subsidiaries}

\textbf{Functions:}

\qquad\texttt{Geometry::dihedralAngle}

\qquad \texttt{EdgeHeight}

\qquad\texttt{listIntersection}

\textbf{Global Variables: }\ radii, etas.

\textbf{Local Variables:} \ Vertex vi, vj, vk, vl.

\subsection*{Description}

The calculation of \texttt{FaceHeight} involves the simple formula:%
\begin{equation*}
\text{\texttt{FaceHeight(vi, vj, vk, vl)}}=
\end{equation*}

\begin{equation*}
\frac{\left( \text{\texttt{EdgeHeight(vi,vj,vl)}}-\text{\texttt{%
EdgeHeight(vi,vj,vk)}}\cos (\beta _{ij,kl})\right) }{\sin \left( \beta
_{ij,kl}\right) }
\end{equation*}%
where $\beta _{ij,kl}$ is the dihedral angle along edge $\left\{
vi,vj\right\} $ of tetrahedron $\left\{ vi,vj,vk,vl\right\} $. \ A geometric
interpretation of this quantity is a follows. \ Given a decorated
tetrahedron (tetrahedron with radii and eta values assigned to the vertices
and edges respectively), the center of this tetrahedron can be calculated as
the common power point of its embedding into three-dimensional Euclidean
space. \ The perpendicular distance from this center point to the face $%
\left\{ vi,vj,vk\right\} $ is exactly \texttt{FaceHeight (vi, vj, vk, vl)}.
\ Take note that the first three vertices in the function call correspond to
the preferred face, and the fourth vertex in the function call identifies
the tetrahedron. \ 

A primary use of this function is in the calculation of several quantities
needed for the \texttt{CurvaturePartial} function used in the optimization
of the normalized Einstein-Hilbert-Regge functional.

\subsection*{Practicum}

An example of the use of this function is in the calculation of the dual
areas, \texttt{DualAreaSegment}, to an edge of a three dimensional
triangulation.

\qquad \texttt{double DualAreaSegment( Vertex vi, Vertex vj, Vertex vk,
Vertex vl)}

\qquad\qquad\texttt{\{}

\qquad \qquad \texttt{double result =
0.5*(EdgeHeight(vi,vj,vk)*FaceHeight(vi,vj,vk,vl)}

\qquad \qquad \qquad \texttt{+EdgeHeight(vi,vj,vl)*FaceHeight(vi,vj,vl,vk));}

\qquad\qquad\texttt{return result;}

\qquad\qquad\texttt{\}}

\subsection*{Limitations}

\texttt{faceHeight} must receive as input four vertices of a tetrahedron of
the triangulation. \ Moreover, the first three vertices in the function call
identify a face and can be in any order, however the fourth vertex in the
function call identifies the tetrahedron and can not be permuted with the
other three vertices. \ 

\subsection*{Revisions}

subversion 757, 6/8/09, \texttt{FaceHeight} created.

subversion 1055, 3/12/10, \texttt{FaceHeight}\ converted to a geoquant.

\subsection*{Testing}

This function was not tested.

\subsection*{Future Work}

This function has been incorporated into the Geometry class geoquants, and
thus this entry needs to be updated. \ 
%
%EndExpansion

\bigskip

\bigskip

%TCIMACRO{\QSubDoc{Include Lij_star}{%TCIDATA{Version=5.00.0.2606}
%TCIDATA{LaTeXparent=1,1,functions.tex}
                      

\section*{\texttt{DualArea::DualArea}}

\subsection*{Function Prototype}

\texttt{double DualArea(Edge e)}

\subsection*{Key Words}

Dual area, curvature, partial derivative, edge, geoquant.

\subsection*{Authors}

Daniel Champion

\subsection*{Introduction}

\texttt{DualArea} calculates the dual area of an edge.

\subsection*{Subsidiaries}

\subsubsection*{Functions: \ }

\qquad \texttt{DualAreaSegment}

\qquad \qquad \texttt{FaceHeight}

\qquad \qquad \qquad \texttt{EdgeHeight}

\qquad \qquad \qquad \qquad \texttt{PartialEdge}

\subsubsection*{Global Variables: \ }

\qquad Only radii and eta values are needed.

\subsubsection*{Local Variables: \ }

\qquad none

\subsection*{Description}

\texttt{DualArea} is defined as:%
\begin{equation*}
\text{\texttt{DualArea(edge e\_ij))}}=l_{ij}^{\ast }=\sum_{\substack{ \text{%
all tetrahedra }(i,j,k,l) \\ \text{containing edge }(i,j)\text{.}}}A_{ij,kl},
\end{equation*}%
where $A_{ij,kl}$ is computed with the function \texttt{DualAreaSegment}
applied to the vertices of the tetrahedron being summed over. \ Note that 
\texttt{DualAreaSegment} utilizes the functions \texttt{FaceHeight}, \texttt{%
EdgeHeight}, and \texttt{partialEdge}, however only the radii and eta values
are needed to calculate all of these quantities. \ 

This function was created for use in the \texttt{CurvaturePartial} function
which serves an essential role in calculating the second derivatives of the
Einstein-Hilbert-Regge functional (\texttt{EHRSecondPartial}). \ The second
order partial derivatives of the EHR functional are used in the optimization
of the EHR functional using Newton's method. \ \texttt{DualArea} will
eventually be used in the study of laplacians. \ 

\subsection*{Practicum}

Currently \texttt{DualArea} is only used to calculate the partial derivative
of curvature with respect to $\log $ radius. \ The following example
calculates the partial derivative of the curvature at vertex V with respect
to the $\log $ radius $r_{l}$ corresponding to vertex Vprime (adjacent to V).

\bigskip

\qquad\texttt{double sum = 0.0;}

\qquad\texttt{double dihedral\_sum = 0.0;}

\qquad\texttt{Vprime = Triangulation::vertexTable[l];}

\qquad\texttt{E =
Triangulation::edgeTable[listIntersection(V.getLocalEdges(),}

\qquad\qquad\texttt{Vprime.getLocalEdges())[0]];}

\qquad\texttt{// This assumes that there is a unique edge between two
vertices.}

\qquad\texttt{local\_tetra = E.getLocalTetras();}

\qquad\texttt{for (int m=0; m \TEXTsymbol{<} (*(local\_tetra)).size(); ++m)
\{}

\qquad\qquad\texttt{T = Triangulation::tetraTable[local\_tetra-\TEXTsymbol{>}%
at(m)];}

\qquad\qquad\texttt{dihedral\_sum += Geometry::dihedralAngle(E,T);}

\qquad\texttt{\}}

\qquad \texttt{result =
DualArea(E)/(Geometry::Length(E))-(2*PI-dihedral\_sum)}

\qquad \qquad \texttt{*(pow(Geometry::Radius(V),
2)*pow(Geometry::Radius(Vprime),2)}

\qquad \qquad \texttt{%
*(1-pow(Geometry::Eta(E),2)))/pow(Geometry::Length(E),3);}

\subsection*{Limitations}

\texttt{DualArea} can operate on any and all edges of a 3D triangulation
however it is only appropriate for triangulations where tetrahedra have
distinct edges. \ 

\subsection*{Revisions}

Subversion 676, 5/15/09, \texttt{DualArea} created within \texttt{%
Newtons\_Method}.

subversion 1055, 3/12/10, \texttt{DualArea}\ converted to a geoquant.

\subsection*{Testing}

\texttt{Lij\_star} has not been tested. \ 

\subsection*{Future Work}

\texttt{Lij\_star} should be moved to a more appropriate section of the
code. \ A more general volume function should be created that would take any
simplex object (including a boolean for dual simplices) and return the
appropriate volume. \ This general volume function would be an excellent
location for \texttt{Lij\_star}. \ It should be tested some time as well. \ 
}}%
%BeginExpansion
%TCIDATA{Version=5.00.0.2606}
%TCIDATA{LaTeXparent=1,1,functions.tex}
                      

\section*{\texttt{DualArea::DualArea}}

\subsection*{Function Prototype}

\texttt{double DualArea(Edge e)}

\subsection*{Key Words}

Dual area, curvature, partial derivative, edge, geoquant.

\subsection*{Authors}

Daniel Champion

\subsection*{Introduction}

\texttt{DualArea} calculates the dual area of an edge.

\subsection*{Subsidiaries}

\subsubsection*{Functions: \ }

\qquad \texttt{DualAreaSegment}

\qquad \qquad \texttt{FaceHeight}

\qquad \qquad \qquad \texttt{EdgeHeight}

\qquad \qquad \qquad \qquad \texttt{PartialEdge}

\subsubsection*{Global Variables: \ }

\qquad Only radii and eta values are needed.

\subsubsection*{Local Variables: \ }

\qquad none

\subsection*{Description}

\texttt{DualArea} is defined as:%
\begin{equation*}
\text{\texttt{DualArea(edge e\_ij))}}=l_{ij}^{\ast }=\sum_{\substack{ \text{%
all tetrahedra }(i,j,k,l) \\ \text{containing edge }(i,j)\text{.}}}A_{ij,kl},
\end{equation*}%
where $A_{ij,kl}$ is computed with the function \texttt{DualAreaSegment}
applied to the vertices of the tetrahedron being summed over. \ Note that 
\texttt{DualAreaSegment} utilizes the functions \texttt{FaceHeight}, \texttt{%
EdgeHeight}, and \texttt{partialEdge}, however only the radii and eta values
are needed to calculate all of these quantities. \ 

This function was created for use in the \texttt{CurvaturePartial} function
which serves an essential role in calculating the second derivatives of the
Einstein-Hilbert-Regge functional (\texttt{EHRSecondPartial}). \ The second
order partial derivatives of the EHR functional are used in the optimization
of the EHR functional using Newton's method. \ \texttt{DualArea} will
eventually be used in the study of laplacians. \ 

\subsection*{Practicum}

Currently \texttt{DualArea} is only used to calculate the partial derivative
of curvature with respect to $\log $ radius. \ The following example
calculates the partial derivative of the curvature at vertex V with respect
to the $\log $ radius $r_{l}$ corresponding to vertex Vprime (adjacent to V).

\bigskip

\qquad\texttt{double sum = 0.0;}

\qquad\texttt{double dihedral\_sum = 0.0;}

\qquad\texttt{Vprime = Triangulation::vertexTable[l];}

\qquad\texttt{E =
Triangulation::edgeTable[listIntersection(V.getLocalEdges(),}

\qquad\qquad\texttt{Vprime.getLocalEdges())[0]];}

\qquad\texttt{// This assumes that there is a unique edge between two
vertices.}

\qquad\texttt{local\_tetra = E.getLocalTetras();}

\qquad\texttt{for (int m=0; m \TEXTsymbol{<} (*(local\_tetra)).size(); ++m)
\{}

\qquad\qquad\texttt{T = Triangulation::tetraTable[local\_tetra-\TEXTsymbol{>}%
at(m)];}

\qquad\qquad\texttt{dihedral\_sum += Geometry::dihedralAngle(E,T);}

\qquad\texttt{\}}

\qquad \texttt{result =
DualArea(E)/(Geometry::Length(E))-(2*PI-dihedral\_sum)}

\qquad \qquad \texttt{*(pow(Geometry::Radius(V),
2)*pow(Geometry::Radius(Vprime),2)}

\qquad \qquad \texttt{%
*(1-pow(Geometry::Eta(E),2)))/pow(Geometry::Length(E),3);}

\subsection*{Limitations}

\texttt{DualArea} can operate on any and all edges of a 3D triangulation
however it is only appropriate for triangulations where tetrahedra have
distinct edges. \ 

\subsection*{Revisions}

Subversion 676, 5/15/09, \texttt{DualArea} created within \texttt{%
Newtons\_Method}.

subversion 1055, 3/12/10, \texttt{DualArea}\ converted to a geoquant.

\subsection*{Testing}

\texttt{Lij\_star} has not been tested. \ 

\subsection*{Future Work}

\texttt{Lij\_star} should be moved to a more appropriate section of the
code. \ A more general volume function should be created that would take any
simplex object (including a boolean for dual simplices) and return the
appropriate volume. \ This general volume function would be an excellent
location for \texttt{Lij\_star}. \ It should be tested some time as well. \ 
%
%EndExpansion

\bigskip

\bigskip

%TCIMACRO{%
%\QSubDoc{Include makeTriangulationFile}{\input{makeTriangulationFile.tex}}}%
%BeginExpansion
\input{makeTriangulationFile.tex}%
%EndExpansion

\bigskip

\bigskip

%TCIMACRO{%
%\QSubDoc{Include NewtonsMethodOptimize}{\input{NewtonsMethodOptimize.tex}}}%
%BeginExpansion
\input{NewtonsMethodOptimize.tex}%
%EndExpansion

\bigskip

\bigskip

%TCIMACRO{\QSubDoc{Include pause}{\input{pause.tex}}}%
%BeginExpansion
\input{pause.tex}%
%EndExpansion

\bigskip

\bigskip

%TCIMACRO{%
%\QSubDoc{Include print3DResultsStep}{\input{print3DResultsStep.tex}}}%
%BeginExpansion
\input{print3DResultsStep.tex}%
%EndExpansion

\bigskip

\bigskip

%TCIMACRO{\QSubDoc{Include printResultsNum}{\input{printResultsNum.tex}}}%
%BeginExpansion
\input{printResultsNum.tex}%
%EndExpansion

\bigskip

\bigskip

%TCIMACRO{%
%\QSubDoc{Include printResultsNumSteps}{\input{printResultsNumSteps.tex}}}%
%BeginExpansion
\input{printResultsNumSteps.tex}%
%EndExpansion

\bigskip

\bigskip

%TCIMACRO{\QSubDoc{Include printResultsStep}{\input{printResultsStep.tex}}}%
%BeginExpansion
\input{printResultsStep.tex}%
%EndExpansion

\bigskip

\bigskip

%TCIMACRO{%
%\QSubDoc{Include printResultsVertex}{\input{printResultsVertex.tex}}}%
%BeginExpansion
\input{printResultsVertex.tex}%
%EndExpansion

\bigskip

\bigskip

%TCIMACRO{%
%\QSubDoc{Include Total_Volume}{%TCIDATA{Version=5.00.0.2606}
%TCIDATA{LaTeXparent=1,1,functions.tex}
                      

\section*{\texttt{TotalVolume::TotalVolume}}

\subsection*{Function Prototype}

\texttt{double TotalVolume()}

\subsection*{Key Words}

Volume, tetrahedron, Cayley-Menger determinant, geoquant.

\subsection*{Authors}

Daniel Champion

\subsection*{Introduction}

The function \texttt{TotalVolume} calculates the total volume of a three
dimensional triangulated manifold.

\subsection*{Subsidiaries}

\textbf{Functions:}

\qquad Geometry::Volume

\textbf{Global Variables:} \ radii, etas.

\textbf{Local Variables:} none.

\subsection*{Description}

\texttt{TotalVolume} is calculated by summing the volumes of each
tetrahedron in a triangulation. \ The volume of a tetrahedron is calculated
with the Cayley-Menger determinant:%
\begin{equation*}
288V^{2}=\det \left[ 
\begin{array}{ccccc}
0 & 1 & 1 & 1 & 1 \\ 
1 & 0 & L_{12}^{2} & L_{13}^{2} & L_{14}^{2} \\ 
1 & L_{12}^{2} & 0 & L_{23}^{2} & L_{24}^{2} \\ 
1 & L_{13}^{2} & L_{23}^{2} & 0 & L_{34}^{2} \\ 
1 & L_{14}^{2} & L_{24}^{2} & L_{34}^{2} & 0%
\end{array}%
\right] 
\end{equation*}%
where the lengths were determined from the radii and eta values using the
formula%
\begin{equation*}
L_{ij}^{2}=r_{i}^{2}+r_{j}^{2}+2r_{i}r_{j}Eta_{ij}.
\end{equation*}

The formula was obtained using calculations within Mathematica and was
output into the C programming language using the function CForm. \ 

The total volume of a triangulation is used in multiple locations within the
project. \ One example of its use is in the calculation of the normalized
Einstein-Hilbert-Regge functional:%
\begin{equation*}
\widetilde{EHR}=\frac{\sum_{i}K_{i}}{\left( \text{\texttt{TotalVolume()}}%
\right) ^{\frac{1}{3}}}
\end{equation*}%
where $K_{i}$ is the curvature at vertex $i$.

\subsection*{Practicum}

An excellent example of the use of this function is in the calculation of
the normalized Einstein-Hilbert-Regge functional. \ 

\qquad\texttt{double EHR () \{}

\qquad\qquad\texttt{double result;}

\qquad \qquad \texttt{result = (TotalCurvature())/pow(TotalVolume (),
1.0/3.0);}

\qquad\qquad\texttt{return result;}

\qquad\qquad\texttt{\}}

\subsection*{Limitations}

Since \texttt{TotalVolume()} relies critically on the Geometry::volume
function, it thus has the same limitations. \ Specifically, if the edge
lengths of any tetrahedron do not satisfy the necessary conditions to
produce a positive volume tetrahedron, \texttt{TotalVolume()} will output an
undefined number. \ The Cayley-Menger determinant can be used to check this
condition on the edge lengths.

\subsection*{Revisions}

subversion 757, 7/11/09, \texttt{TotalVolume} created.

subversion 1055, 3/12/10, \texttt{TotalVolume}\ converted to a geoquant.

\subsection*{Testing}

Using known volumes of several tetrahedra, the total volume was calculated
by hand and compared with \texttt{TotalVolume}.

\subsection*{Future Work}

No planned future work.
}}%
%BeginExpansion
%TCIDATA{Version=5.00.0.2606}
%TCIDATA{LaTeXparent=1,1,functions.tex}
                      

\section*{\texttt{TotalVolume::TotalVolume}}

\subsection*{Function Prototype}

\texttt{double TotalVolume()}

\subsection*{Key Words}

Volume, tetrahedron, Cayley-Menger determinant, geoquant.

\subsection*{Authors}

Daniel Champion

\subsection*{Introduction}

The function \texttt{TotalVolume} calculates the total volume of a three
dimensional triangulated manifold.

\subsection*{Subsidiaries}

\textbf{Functions:}

\qquad Geometry::Volume

\textbf{Global Variables:} \ radii, etas.

\textbf{Local Variables:} none.

\subsection*{Description}

\texttt{TotalVolume} is calculated by summing the volumes of each
tetrahedron in a triangulation. \ The volume of a tetrahedron is calculated
with the Cayley-Menger determinant:%
\begin{equation*}
288V^{2}=\det \left[ 
\begin{array}{ccccc}
0 & 1 & 1 & 1 & 1 \\ 
1 & 0 & L_{12}^{2} & L_{13}^{2} & L_{14}^{2} \\ 
1 & L_{12}^{2} & 0 & L_{23}^{2} & L_{24}^{2} \\ 
1 & L_{13}^{2} & L_{23}^{2} & 0 & L_{34}^{2} \\ 
1 & L_{14}^{2} & L_{24}^{2} & L_{34}^{2} & 0%
\end{array}%
\right] 
\end{equation*}%
where the lengths were determined from the radii and eta values using the
formula%
\begin{equation*}
L_{ij}^{2}=r_{i}^{2}+r_{j}^{2}+2r_{i}r_{j}Eta_{ij}.
\end{equation*}

The formula was obtained using calculations within Mathematica and was
output into the C programming language using the function CForm. \ 

The total volume of a triangulation is used in multiple locations within the
project. \ One example of its use is in the calculation of the normalized
Einstein-Hilbert-Regge functional:%
\begin{equation*}
\widetilde{EHR}=\frac{\sum_{i}K_{i}}{\left( \text{\texttt{TotalVolume()}}%
\right) ^{\frac{1}{3}}}
\end{equation*}%
where $K_{i}$ is the curvature at vertex $i$.

\subsection*{Practicum}

An excellent example of the use of this function is in the calculation of
the normalized Einstein-Hilbert-Regge functional. \ 

\qquad\texttt{double EHR () \{}

\qquad\qquad\texttt{double result;}

\qquad \qquad \texttt{result = (TotalCurvature())/pow(TotalVolume (),
1.0/3.0);}

\qquad\qquad\texttt{return result;}

\qquad\qquad\texttt{\}}

\subsection*{Limitations}

Since \texttt{TotalVolume()} relies critically on the Geometry::volume
function, it thus has the same limitations. \ Specifically, if the edge
lengths of any tetrahedron do not satisfy the necessary conditions to
produce a positive volume tetrahedron, \texttt{TotalVolume()} will output an
undefined number. \ The Cayley-Menger determinant can be used to check this
condition on the edge lengths.

\subsection*{Revisions}

subversion 757, 7/11/09, \texttt{TotalVolume} created.

subversion 1055, 3/12/10, \texttt{TotalVolume}\ converted to a geoquant.

\subsection*{Testing}

Using known volumes of several tetrahedra, the total volume was calculated
by hand and compared with \texttt{TotalVolume}.

\subsection*{Future Work}

No planned future work.
%
%EndExpansion

\bigskip

\bigskip

%TCIMACRO{%
%\QSubDoc{Include Volume_Partial}{%TCIDATA{Version=5.00.0.2606}
%TCIDATA{LaTeXparent=1,1,functions.tex}
                      

\section*{\texttt{VolumePartial::VolumePartial}}

\subsection*{Function Prototype}

\texttt{double VolumePartial(Vertex v\_i, Tetra t)}

\subsection*{Key Words}

Volume, tetrahedron, vertex, radius, Cayley-Menger determinant, standard
form, geoquant.

\subsection*{Authors}

Daniel Champion

\subsection*{Introduction}

\texttt{VolumePartial} calculates the partial derivative of the volume of a
tetrahedron with respect to the logarithm of the radius of a vertex.

\subsection*{Subsidiaries}

\textbf{Functions:} \ 

\texttt{listDifference}

\texttt{listIntersection}

\texttt{Simplex::isAdjVertex}

\textbf{Global Variables:} \ radii, etas.

\textbf{Local Variables:}

\subsection*{Description}

The volume of a tetrahedron only depends on the lengths of its edges as
calculated from the Cayley-Menger determinant. \ Thus for a given
tetrahedron $t$, it's partial derivatives with respect to $\log $ radii will
vanish except for those radii corresponding to the vertices of $t$. \ The
function \texttt{isAdjVertex} of the simplex class determines this
condition. \ \texttt{VolumePartial} then proceeds with an initialization
procedure that labels the vertices and edges (radii and etas) in standard
form. \ Specifically, \texttt{Volume\_Partial} receives as inputs an integer 
\texttt{i} corresponding to a vertex (in the vertex table) which is labeled
vertex v1. \ The remaining vertices are labeled $v2,v3,v4$, and the edges $%
e12,e13,e14,e23,e24,e34$ are labeled preserving the structure implied by the
assignment of the vertices. \ The radii $r_{1}$, $r_{2}$,... and eta values $%
Eta_{12},Eta_{13},$... are assigned to the corresponding vertices and edges.
\ 

The formula for the partial derivative in terms of these standard form
variables was calculated in Mathematica using the Cayley-Menger determinant,
that is:%
\begin{equation*}
288V^{2}=\det\left[ 
\begin{array}{ccccc}
0 & 1 & 1 & 1 & 1 \\ 
1 & 0 & L_{12}^{2} & L_{13}^{2} & L_{14}^{2} \\ 
1 & L_{12}^{2} & 0 & L_{23}^{2} & L_{24}^{2} \\ 
1 & L_{13}^{2} & L_{23}^{2} & 0 & L_{34}^{2} \\ 
1 & L_{14}^{2} & L_{24}^{2} & L_{34}^{2} & 0%
\end{array}
\right] ,
\end{equation*}
where the lengths were determined from the radii and eta values using the
formula%
\begin{equation*}
L_{ij}^{2}=r_{i}^{2}+r_{j}^{2}+2r_{i}r_{j}Eta_{ij}.
\end{equation*}

The formula obtained from Mathematica was outputted into the C programming
language using the function CForm. \ 

This function was designed for use in the optimization of the
Einstein-Hilbert-Regge functional using Newton's method. \ In this procedure
the gradient of the EHR functional is needed which contains the partial
derivatives of the volume. \ 

\subsection*{Practicum}

Usage:

\texttt{VolumePartial(Vertex v\_i, Tetra t)}

The integer \texttt{i} corresponds to a vertex in the vertex table, that is%
\begin{equation*}
\text{\texttt{VolumePartial (v\_i, t)} }=\frac{\partial }{\partial \log r_{i}%
}Volume(t).
\end{equation*}

\subsection*{Limitations}

\texttt{VolumePartial} was designed to output the correct partial derivative
for any integer $i$ in the vertex table and any tetrahedron $t$ in the
triangulation. \ 

\subsection*{Revisions}

subversion 757, 7/7/09, \texttt{VolumePartial} created within
NewtonsMethod.cpp.

subversion 1055, 3/12/10, \texttt{VolumePartial} converted to a geoquant.

\subsection*{Testing}

The partial derivative of volume of several known tetrahedra were calculated
using \texttt{Volume\_Partial} and verified using Mathematica.

\subsection*{Future Work}

The procedure that initializes the tetrahedron into standard form should be
removed from this program and placed elsewhere. \ There are several
occurrences of this type of procedure that should be consolidated. \ 
}}%
%BeginExpansion
%TCIDATA{Version=5.00.0.2606}
%TCIDATA{LaTeXparent=1,1,functions.tex}
                      

\section*{\texttt{VolumePartial::VolumePartial}}

\subsection*{Function Prototype}

\texttt{double VolumePartial(Vertex v\_i, Tetra t)}

\subsection*{Key Words}

Volume, tetrahedron, vertex, radius, Cayley-Menger determinant, standard
form, geoquant.

\subsection*{Authors}

Daniel Champion

\subsection*{Introduction}

\texttt{VolumePartial} calculates the partial derivative of the volume of a
tetrahedron with respect to the logarithm of the radius of a vertex.

\subsection*{Subsidiaries}

\textbf{Functions:} \ 

\texttt{listDifference}

\texttt{listIntersection}

\texttt{Simplex::isAdjVertex}

\textbf{Global Variables:} \ radii, etas.

\textbf{Local Variables:}

\subsection*{Description}

The volume of a tetrahedron only depends on the lengths of its edges as
calculated from the Cayley-Menger determinant. \ Thus for a given
tetrahedron $t$, it's partial derivatives with respect to $\log $ radii will
vanish except for those radii corresponding to the vertices of $t$. \ The
function \texttt{isAdjVertex} of the simplex class determines this
condition. \ \texttt{VolumePartial} then proceeds with an initialization
procedure that labels the vertices and edges (radii and etas) in standard
form. \ Specifically, \texttt{Volume\_Partial} receives as inputs an integer 
\texttt{i} corresponding to a vertex (in the vertex table) which is labeled
vertex v1. \ The remaining vertices are labeled $v2,v3,v4$, and the edges $%
e12,e13,e14,e23,e24,e34$ are labeled preserving the structure implied by the
assignment of the vertices. \ The radii $r_{1}$, $r_{2}$,... and eta values $%
Eta_{12},Eta_{13},$... are assigned to the corresponding vertices and edges.
\ 

The formula for the partial derivative in terms of these standard form
variables was calculated in Mathematica using the Cayley-Menger determinant,
that is:%
\begin{equation*}
288V^{2}=\det\left[ 
\begin{array}{ccccc}
0 & 1 & 1 & 1 & 1 \\ 
1 & 0 & L_{12}^{2} & L_{13}^{2} & L_{14}^{2} \\ 
1 & L_{12}^{2} & 0 & L_{23}^{2} & L_{24}^{2} \\ 
1 & L_{13}^{2} & L_{23}^{2} & 0 & L_{34}^{2} \\ 
1 & L_{14}^{2} & L_{24}^{2} & L_{34}^{2} & 0%
\end{array}
\right] ,
\end{equation*}
where the lengths were determined from the radii and eta values using the
formula%
\begin{equation*}
L_{ij}^{2}=r_{i}^{2}+r_{j}^{2}+2r_{i}r_{j}Eta_{ij}.
\end{equation*}

The formula obtained from Mathematica was outputted into the C programming
language using the function CForm. \ 

This function was designed for use in the optimization of the
Einstein-Hilbert-Regge functional using Newton's method. \ In this procedure
the gradient of the EHR functional is needed which contains the partial
derivatives of the volume. \ 

\subsection*{Practicum}

Usage:

\texttt{VolumePartial(Vertex v\_i, Tetra t)}

The integer \texttt{i} corresponds to a vertex in the vertex table, that is%
\begin{equation*}
\text{\texttt{VolumePartial (v\_i, t)} }=\frac{\partial }{\partial \log r_{i}%
}Volume(t).
\end{equation*}

\subsection*{Limitations}

\texttt{VolumePartial} was designed to output the correct partial derivative
for any integer $i$ in the vertex table and any tetrahedron $t$ in the
triangulation. \ 

\subsection*{Revisions}

subversion 757, 7/7/09, \texttt{VolumePartial} created within
NewtonsMethod.cpp.

subversion 1055, 3/12/10, \texttt{VolumePartial} converted to a geoquant.

\subsection*{Testing}

The partial derivative of volume of several known tetrahedra were calculated
using \texttt{Volume\_Partial} and verified using Mathematica.

\subsection*{Future Work}

The procedure that initializes the tetrahedron into standard form should be
removed from this program and placed elsewhere. \ There are several
occurrences of this type of procedure that should be consolidated. \ 
%
%EndExpansion

\bigskip

\bigskip

%TCIMACRO{%
%\QSubDoc{Include Volume_Second_Partial}{%TCIDATA{Version=5.00.0.2606}
%TCIDATA{LaTeXparent=1,1,functions.tex}
                      

\section*{\texttt{VolumeSecondPartial::VolumeSecondPartial}}

\subsection*{Function Prototype}

\texttt{double VolumeSecondPartial(Vertex v\_i, Vertex v\_j, Tetra t)}

\subsection*{Key Words}

Volume, Hessian Matrix, Newton's Method, partial derivative,
Einstein-Hilbert-Regge functional, geoquant.

\subsection*{Authors}

Daniel Champion

\subsection*{Introduction}

\texttt{VolumeSecondPartial} calculates the second order partial derivatives
of the volume of a tetrahedron with respect to log radii for all pairs of
indices (not necessarily distinct) in the vertex table. \ 

\subsection*{Subsidiaries}

\textbf{Functions:}

\texttt{listDifference}

\texttt{listIntersection}

\texttt{Simplex::isAdjVertex}

\textbf{Global Variables:} \ radii, etas.

\textbf{Local Variables:}

\subsection*{Description}

The volume of a tetrahedron only depends on the lengths of its edges as
calculated from the Cayley-Menger determinant. \ Thus for a given
tetrahedron $t$, it's second order partial derivatives with respect to $\log 
$ radii will vanish except for pairs of radii (not necessarily distinct)
corresponding to the vertices of $t$. \ The first step in the implementation
of \texttt{VolumeSecondPartial} is the determination of the following
trichotomy for a pair $\left\{ i,j\right\} $ of indices in the vertex table:%
\begin{equation*}
\begin{array}{l}
\text{A. \ }i=j\text{ and }i\text{ is a vertex of tetrahedron }t \\ 
\text{B. \ }i\neq j\text{ and both }i\text{ and }j\text{ belong to }t \\ 
\text{C. \ at least one of }i\text{ or }j\text{ doesn't belong to }t.%
\end{array}%
\end{equation*}

Each condition of the trichotomy requires a distinct calculation to
determine the desired partial derivative. \ Nevertheless, the next step in
the implementation is to place the tetrahedron in "standard form" relative
to the indices $i$ and $j$ (for conditions A and B only). \ More
specifically, for condition A the radius for vertex $i$ is stored as $r_{1}$%
, and the remaining radii of the tetrahedron $t$ are assigned $r_{2},r_{3},$
and $r_{4}$ in no particular order. \ The eta values $Eta_{12},Eta_{13},...$
are then assigned preserving the preceding assignments. \ In the case of
condition B, the radii at vertices $i$ and $j$ are assigned to $r_{1}$ and $%
r_{2}$ respectively, and $r_{3}$, and $r_{4}$ the remaining radii of $t$. \
The eta values $Eta_{12},Eta_{13},...$ are again assigned preserving the
preceding assignments.

The formulas for the second order partial derivatives in terms of these
standard form variables was calculated in Mathematica using the
Cayley-Menger determinant, that is:%
\begin{equation*}
288V^{2}=\det\left[ 
\begin{array}{ccccc}
0 & 1 & 1 & 1 & 1 \\ 
1 & 0 & L_{12}^{2} & L_{13}^{2} & L_{14}^{2} \\ 
1 & L_{12}^{2} & 0 & L_{23}^{2} & L_{24}^{2} \\ 
1 & L_{13}^{2} & L_{23}^{2} & 0 & L_{34}^{2} \\ 
1 & L_{14}^{2} & L_{24}^{2} & L_{34}^{2} & 0%
\end{array}
\right] ,
\end{equation*}
where the lengths were determined from the radii and eta values using the
formula%
\begin{equation*}
L_{ij}^{2}=r_{i}^{2}+r_{j}^{2}+2r_{i}r_{j}Eta_{ij}.
\end{equation*}

The formula obtained from Mathematica was outputted into the C programming
language using the function CForm.

This function was designed for use in the optimization of the
Einstein-Hilbert-Regge functional using Newton's method. \ In this procedure
the Hessian matrix of the normalized EHR functional is needed, each entry of
which uses the second order partial derivatives of volume. \ See the entry
on \texttt{EHRSecondPartial}.

\subsection*{Practicum}

Usage:

\texttt{VolumeSecondPartial (Vertex v\_i, Vertex v\_j, Tetra t)}

The integers \texttt{i} and \texttt{j} correspond to vertices in the vertex
table and \texttt{t} is a tetrahedron in the triangulation. \ Specifically
the function returns:%
\begin{equation*}
\text{\texttt{VolumeSecondPartial(v\_i, v\_j, t)} }=\frac{\partial ^{2}}{%
\partial \log r_{i}\partial \log r_{j}}Volume(t).
\end{equation*}

\subsection*{Limitations}

\texttt{VolumeSecondPartial} is fully operational with no know limitations.
\ The function will output appropriate values when given indices $i,$ and $j$
in the vertex table, and a tetrahedron $t$. \ 

\subsection*{Revisions}

subversion 757, 7/9/09, \texttt{VolumeSecondPartial} created.

subversion 1055, 3/12/10, \texttt{VolumeSecondPartial} converted to a
geoquant.

\subsection*{Testing}

Several trials were run outputting the values of \texttt{VolumeSecondPartial}
for a variety of vertices and tetrahedra. \ These values were compared with
calculations performed on Mathematica. \ 

\subsection*{Future Work}

None planned.
}}%
%BeginExpansion
%TCIDATA{Version=5.00.0.2606}
%TCIDATA{LaTeXparent=1,1,functions.tex}
                      

\section*{\texttt{VolumeSecondPartial::VolumeSecondPartial}}

\subsection*{Function Prototype}

\texttt{double VolumeSecondPartial(Vertex v\_i, Vertex v\_j, Tetra t)}

\subsection*{Key Words}

Volume, Hessian Matrix, Newton's Method, partial derivative,
Einstein-Hilbert-Regge functional, geoquant.

\subsection*{Authors}

Daniel Champion

\subsection*{Introduction}

\texttt{VolumeSecondPartial} calculates the second order partial derivatives
of the volume of a tetrahedron with respect to log radii for all pairs of
indices (not necessarily distinct) in the vertex table. \ 

\subsection*{Subsidiaries}

\textbf{Functions:}

\texttt{listDifference}

\texttt{listIntersection}

\texttt{Simplex::isAdjVertex}

\textbf{Global Variables:} \ radii, etas.

\textbf{Local Variables:}

\subsection*{Description}

The volume of a tetrahedron only depends on the lengths of its edges as
calculated from the Cayley-Menger determinant. \ Thus for a given
tetrahedron $t$, it's second order partial derivatives with respect to $\log 
$ radii will vanish except for pairs of radii (not necessarily distinct)
corresponding to the vertices of $t$. \ The first step in the implementation
of \texttt{VolumeSecondPartial} is the determination of the following
trichotomy for a pair $\left\{ i,j\right\} $ of indices in the vertex table:%
\begin{equation*}
\begin{array}{l}
\text{A. \ }i=j\text{ and }i\text{ is a vertex of tetrahedron }t \\ 
\text{B. \ }i\neq j\text{ and both }i\text{ and }j\text{ belong to }t \\ 
\text{C. \ at least one of }i\text{ or }j\text{ doesn't belong to }t.%
\end{array}%
\end{equation*}

Each condition of the trichotomy requires a distinct calculation to
determine the desired partial derivative. \ Nevertheless, the next step in
the implementation is to place the tetrahedron in "standard form" relative
to the indices $i$ and $j$ (for conditions A and B only). \ More
specifically, for condition A the radius for vertex $i$ is stored as $r_{1}$%
, and the remaining radii of the tetrahedron $t$ are assigned $r_{2},r_{3},$
and $r_{4}$ in no particular order. \ The eta values $Eta_{12},Eta_{13},...$
are then assigned preserving the preceding assignments. \ In the case of
condition B, the radii at vertices $i$ and $j$ are assigned to $r_{1}$ and $%
r_{2}$ respectively, and $r_{3}$, and $r_{4}$ the remaining radii of $t$. \
The eta values $Eta_{12},Eta_{13},...$ are again assigned preserving the
preceding assignments.

The formulas for the second order partial derivatives in terms of these
standard form variables was calculated in Mathematica using the
Cayley-Menger determinant, that is:%
\begin{equation*}
288V^{2}=\det\left[ 
\begin{array}{ccccc}
0 & 1 & 1 & 1 & 1 \\ 
1 & 0 & L_{12}^{2} & L_{13}^{2} & L_{14}^{2} \\ 
1 & L_{12}^{2} & 0 & L_{23}^{2} & L_{24}^{2} \\ 
1 & L_{13}^{2} & L_{23}^{2} & 0 & L_{34}^{2} \\ 
1 & L_{14}^{2} & L_{24}^{2} & L_{34}^{2} & 0%
\end{array}
\right] ,
\end{equation*}
where the lengths were determined from the radii and eta values using the
formula%
\begin{equation*}
L_{ij}^{2}=r_{i}^{2}+r_{j}^{2}+2r_{i}r_{j}Eta_{ij}.
\end{equation*}

The formula obtained from Mathematica was outputted into the C programming
language using the function CForm.

This function was designed for use in the optimization of the
Einstein-Hilbert-Regge functional using Newton's method. \ In this procedure
the Hessian matrix of the normalized EHR functional is needed, each entry of
which uses the second order partial derivatives of volume. \ See the entry
on \texttt{EHRSecondPartial}.

\subsection*{Practicum}

Usage:

\texttt{VolumeSecondPartial (Vertex v\_i, Vertex v\_j, Tetra t)}

The integers \texttt{i} and \texttt{j} correspond to vertices in the vertex
table and \texttt{t} is a tetrahedron in the triangulation. \ Specifically
the function returns:%
\begin{equation*}
\text{\texttt{VolumeSecondPartial(v\_i, v\_j, t)} }=\frac{\partial ^{2}}{%
\partial \log r_{i}\partial \log r_{j}}Volume(t).
\end{equation*}

\subsection*{Limitations}

\texttt{VolumeSecondPartial} is fully operational with no know limitations.
\ The function will output appropriate values when given indices $i,$ and $j$
in the vertex table, and a tetrahedron $t$. \ 

\subsection*{Revisions}

subversion 757, 7/9/09, \texttt{VolumeSecondPartial} created.

subversion 1055, 3/12/10, \texttt{VolumeSecondPartial} converted to a
geoquant.

\subsection*{Testing}

Several trials were run outputting the values of \texttt{VolumeSecondPartial}
for a variety of vertices and tetrahedra. \ These values were compared with
calculations performed on Mathematica. \ 

\subsection*{Future Work}

None planned.
%
%EndExpansion
%
%EndExpansion

%TCIMACRO{\QSubDoc{Include classes}{\input{classes.tex}}}%
%BeginExpansion
\input{classes.tex}%
%EndExpansion

%TCIMACRO{\QSubDoc{Include Geoquants}{%TCIDATA{Version=5.00.0.2552}
%TCIDATA{LaTeXparent=0,0,geocam.tex}
                      
%TCIDATA{ChildDefaults=chapter:1,page:1}


\chapter{Geoquants}

\section{Introduction}

Geoquants were developed by Alex Henniges, Joeseph Thomas, and Kurtis
Norwood during the spring of 2009 as the primary component in streamlining
the execution of the code during extended optimization routines. The purpose
is to provide easy access to calculations of geometric quantities without
requiring extraneous calculations of those quantities while they remain the
same.\ Geoquants achieve this through the use of an invalidation tree to
prevent uneccesary calcuations.\bigskip

\section{General Properties}

Each geoquant contains a calculation of a quantity, invalidation
information, and input/output functionality. \ Some of these components are
specific to each geoquant while some are common to all geoquants.

\subsubsection{Calculation}

Each geoquant has a unique calculation yielding a geometric quantity. \ In
general there are no commonalities among the calculations except that all
geoquants output a quantity of geometric nature. \ 

\subsubsection{Input/Output}

Geoquants recieve as input topological data (vertex, face, tetrahedron) \
which serves as a unique identifier among all such quantities of the same
type across the triangulation. The topological data is composed of zero or
more of each dimension of simplex. The following functions are available to
all geoquants:

\begin{itemize}
\item ExampleGeoQuant::At(input)

At returns an instance of the ExampleGeoQuant identified by input.

\item ExampleGeoquant::valueAt(input)

valueAt returns the value of the geoquant identified by input as a
double-precision number.

\item getValue()

getValue returns the value of an already obtained geoquant.

\item setValue(value)

\ setValue will set the value of this geoquant to the specified value and
subsequently invalidate all dependent geoquants.
\end{itemize}

The functions getValue and setValue are called as:

\texttt{Q=ExampleGeoquant::At(input)}

\texttt{Q-\TEXTsymbol{>}getValue}

\texttt{Q-\TEXTsymbol{>}setValue}

\subsubsection{Invalidation}

The value of a geoquant is obtained in one of two ways: it is set by the
user or it is calculated from other geoquants. In the situation where a
geoquant B is calculated in part from the value of geoquant A, we refer to A
as the parent and B as the dependent. When the value of a parent changes,
the parent notifies all of its dependents that their own value is no longer
valid. Once a request for the value of an invalid geoquant is made, that
geoquant requests the value of each of its parents in order to perform the
recalculation of its own value.

\section{Geoquant Catalog}

\subsection{Alpha}

\paragraph{Input}

Alpha(Vertex v)

\paragraph{Calculation}

This geoquant is never calculated, only set by the user. Its value is
generally 0 or 1.

\paragraph{Invalidation}

Parents: None

\subsection{Area}

\paragraph{Input}

Area( Face f )

\paragraph{Calculation}

Area returns the area of a triangular face (in a triangulation). \ The
calculation is done using Heron's formula which takes as inputs the lengths
of the sides of the triangle.%
\begin{equation*}
A=\sqrt{s\left( s-a\right) \left( s-b\right) \left( s-c\right) }
\end{equation*}%
where%
\begin{equation*}
s=\frac{a+b+c}{2}.
\end{equation*}

\paragraph{Invalidation}

Parents: \ all lengths of edges in the input triangle.

\subsection{CurvaturePartial}

\paragraph{Input}

\begin{itemize}
\item CurvaturePartial( Vertex v, Vertex w )

\item CurvaturePartial( Vertex v, Edge e )
\end{itemize}

\paragraph{Calculation}

\begin{itemize}
\item ( Vertex v, Vertex w ): Returns the following partial derivative:%
\begin{equation*}
CurvaturePartial(v,w)=\frac{\partial K_{v}}{\partial \log r_{w}}.
\end{equation*}%
The calculation of $CurvaturePartial(v,w)$ differs depending on the
combintorial relationship between $v$ and $w$. \ See \textbf{Functions: }%
\texttt{CurvaturePartial::CurvaturePartial} for more information.

\item ( Vertex v, Edge e ): Returns the following partial derivative: 
\begin{equation*}
CurvaturePartial(v,e)=\frac{\partial K_{v}}{\partial \eta _{e}}
\end{equation*}
\end{itemize}

\paragraph{Invalidation}

Parents: Radius of v and all radii of vertices adjacent to both v and w, the
curvature at v, and the dual areas, dihedral angles, and eta of every edge
adjacent to both v and w.

\subsection{Curvature2D}

\paragraph{Input}

Curvature2D( Vertex v )

\paragraph{Calculation}

Curvature2D calculates the discrete curvature at a vertex in a two
dimensional manifold. \ The calculation is done by:%
\begin{equation*}
Curvature2D\left( v\right) =2\pi -\sum\limits_{\substack{ all\text{ }%
triangles\text{ }t  \\ incident\text{ }to\text{ }v}}\theta _{v,t}
\end{equation*}%
where $\theta _{v,t}$ is the angle at vertex $v$ in triangle $t$. \ 

\paragraph{Invalidation}

\bigskip Parents: Every angle incident on vertex v.

\subsection{Curvature3D}

\paragraph{Input}

Curvature3D( Vertex v )

\paragraph{Calculation}

Curvature3D calculates the curvature at a vertex in a three dimensional
manifold. \ If we let $K_{ij}$ be the secional curvature at an edge $\left\{
i,j\right\} $ and let vertex $v$ correspond to $i$, then%
\begin{equation*}
Curvature3D\left( v\right) =\sum\limits_{\substack{ j\text{ s.t. }\left\{
i,j\right\}  \\ \text{is an edge}}}K_{ij}d_{ij}.
\end{equation*}

\paragraph{Invalidation}

Parents: For every edge of v, the sectional curvature of that edge and its
partial edge with respect to v.\bigskip

\subsection{DihedralAngle}

\paragraph{Input}

DihedralAngle( Edge e, Tetra t )

\paragraph{Calculation}

The dihedral angle of an edge in a tetrahedron is calculated using the
spherical law of cosines. \ Given a geometrically determined tetrahedron,
the face angles can be computed. \ At a vertex of the tetrahedron, the face
angles correspond to the sides lengths of a spherical triangle and the
dihedral angles correspond to the angles of the spherical triangle. \ If we
denote the dihedral angle at edge $e=\left\{ i,j\right\} $ in tetrahedron $%
t=\left\{ i,j,k,l\right\} $ by $\beta _{ij,kl}$, and denote the face angle
at vertex $v=i$ of triangle $\left\{ i,j,k\right\} $ by $\gamma _{i,jk}$ we
have that:%
\begin{equation*}
\beta _{ij,kl}=\arccos \left( \frac{\cos \left( \gamma _{i,kl}\right) -\cos
\left( \gamma _{i,jk}\right) \cos \left( \gamma _{i,jl}\right) }{\sin \left(
\gamma _{i,jk}\right) \sin \left( \gamma _{i,jl}\right) }\right)
\end{equation*}

\paragraph{Invalidation}

\bigskip Parents: The face angles incident on vertex $v=i$ and with face $%
f\in t$.

\section{DihedralAngleSum}

\paragraph{Input}

DihedralAngleSum( Edge e )

\paragraph{Calculation}

This geoquant sums every dihedral angle incident on the given edge.

\paragraph{Invalidation}

\bigskip Parents: For each tetrahedron containing e, the dihedral angle of
the tetrahedron incident on e.

\subsection{DualArea}

\paragraph{Input}

DualArea( Edge e )

\paragraph{Calculation}

See \textbf{Functions: }\texttt{DualArea::DualArea} for more information.

\paragraph{Invalidation}

Parents: The DualAreaSegments involving this edge across every tetrahedron
local to e.

\bigskip

\subsection{DualAreaSegment}

\paragraph{Input}

DualAreaSegment( Edge e, Tetra t )

\paragraph{Calculation}

See \textbf{Functions: }\texttt{DualAreaSegment::DualAreaSegment} for more
information.

\paragraph{Invalidation}

Parents: For both faces in t that contain e, the EdgeHeights between those
faces and e, and the FaceHeights between those faces and t.

\bigskip

\subsection{EdgeHeight}

\paragraph{Input}

EdgeHeight( Edge e, Face f )

\paragraph{Calculation}

See \textbf{Functions: }\texttt{EdgeHeight::EdgeHeight} for more information.

\paragraph{Invalidation}

Parents: For one of the two vertices, v, of e, the EuclideanAngle of f
incident on v, and the PartialEdges of the edges in f containing v, incident
on v.

\bigskip

\subsection{NEHRPartial}

\paragraph{Input}

\begin{itemize}
\item NEHRPartial( Vertex v )

\item NEHRPartial( Edge e )
\end{itemize}

\paragraph{Calculation}

\begin{itemize}
\item ( Vertex v ): Returns the following partial derivative: 
\begin{equation*}
NEHRPartial(v)=\frac{\partial NEHR}{\partial r_{v}}
\end{equation*}%
See \textbf{Functions: }\texttt{EHRPartial::EHRPartial} for more information.

\item ( Edge e ): Returns the following partial derivative: 
\begin{equation*}
NEHRPartial(e)=\frac{\partial NEHR}{\partial \eta _{e}}
\end{equation*}
\end{itemize}

\paragraph{Invalidation}

Parents: Both versions use TotalVolume, TotalCurvature, and the
TotalVolumePartial of vertex v (edge e). The vertex version also depends on
the Curvature3D at v. The edge version depends on all CurvaturePartials of e.

\bigskip

\subsection{NEHRSecondPartial}

\paragraph{Input}

\begin{itemize}
\item NEHRSecondPartial( Vertex v, Vertex w )

\item NEHRSecondPartial( Vertex v, Edge e )
\end{itemize}

\paragraph{Calculation}

\begin{itemize}
\item ( Vertex v, Vertex w ): Returns the following second-order partial
derivative: 
\begin{equation*}
NEHRSecPartial(v,w)=\frac{\partial NEHR}{\partial r_{w}\partial r_{v}}
\end{equation*}%
See \textbf{Functions: }\texttt{EHRSecondPartial::EHRSecondPartial} for more
information.

\item ( Vertex v, Edge e ) Returns the following second-order partial
derivative: 
\begin{equation*}
NEHRSecPartial(v,w)=\frac{\partial ^{2}NEHR}{\partial \eta _{e}\partial r_{v}%
}
\end{equation*}
\end{itemize}

\paragraph{Invalidation}

\bigskip Parents: Both versions use TotalVolume, TotalCurvature, and the
Radius, TotalVolumePartial and Curvature3D at v. 

\begin{itemize}
\item (v, w): The Curvature3D and TotalVolumePartial of w, the
CurvaturePartial of v with respect to w, and the TotalVolumeSecondPartial
with respect to v and w.

\item (v, e): The CurvaturePartials and TotalVolumePartial at e, the
CurvaturePartial of v with repsect to e, and the TotalVolumeSecondPartial
with respect to v and e.
\end{itemize}

\subsection{Eta}

\paragraph{Input}

Eta( Edge e )

\paragraph{Calculation}

This geoquant is never calculated, only set by the user.

\paragraph{Invalidation}

Parents: None

\bigskip

\subsection{EuclideanAngle}

\paragraph{Input}

EuclideanAngle( Vertex v, Face f )

\paragraph{Calculation}

The Euclidean angle is calculated usign the Euclidean law of cosines,
namely, let $v$ be indexed by $i$ and $f=\{i,j,k\}$. Then the angle at
vertex $i$ is%
\begin{equation*}
\cos (\alpha _{i,jk})=\frac{l_{ij}^{2}+l_{ik}^{2}-l_{jk}^{2}}{2l_{ij}l_{jk}}
\end{equation*}

\paragraph{Invalidation}

\bigskip Parents: The Lengths corresponding to the three edges of face f.

\subsection{FaceHeight}

\paragraph{Input}

FaceHeight( Face f, Tetra t )

\paragraph{Calculation}

See \textbf{Functions: }\texttt{FaceHeight::FaceHeight} for more information.

\paragraph{Invalidation}

\bigskip Parents: For one of the two edges, e, of f, the DihedralAngle of t
incident on e, and the EdgeHeights of the faces in t containing e, incident
on e.

\subsection{Length}

\paragraph{Input}

Length( Edge e )

\paragraph{Calculation}

Let $\alpha _{i}$, $\alpha _{j}$, $r_{i}$, and $r_{j}$ be the alphas and
radii at the vertices of edge $e=\{i,j\}$. Let $\eta _{ij}$ be the eta value
of edge e. Then the length is 
\begin{equation*}
l_{ij}^{2}=\alpha _{i}r_{i}^{2}+\alpha _{j}r_{j}^{2}+2r_{i}r_{j}\eta _{ij}
\end{equation*}

\paragraph{Invalidation}

\bigskip Parents: The Alpha and Radius values at both vertices of edge e and
the Eta value at edge e.

\subsection{PartialEdge}

\paragraph{Input}

PartialEdge( Vertex v, Edge e )

\paragraph{Calculation}

See \textbf{Functions: }\texttt{PartialEdge::PartialEdge} for more
information.

\paragraph{Invalidation}

\bigskip Parents: The Alpha and Radius values at both vertices of edge e and
the Length at edge e.

\subsection{Radius}

\paragraph{Input}

Radius( Vertex v )

\paragraph{Calculation}

This geoquant is never calcualted, only set by the user.

\paragraph{Invalidation}

Parents: None

\bigskip 

\subsection{RadiusPartial}

\paragraph{Input}

RadiusPartial(Vertex v, Edge e)

\paragraph{Calculation}

This returns the following partial derivative: 
\begin{equation*}
RadiusPartial(v,e)=\frac{\partial r_{v}}{\partial \eta _{e}}
\end{equation*}%
This quantity is calculated for all radii with $e$ fixed as follows: 
\begin{equation*}
\overrightarrow{v}+A\overrightarrow{x}=0
\end{equation*}%
where $A$ is a matrix with entries $a_{ij}=\frac{\partial ^{2}NEHR}{\partial
r_{j}\partial r_{i}}$ and $v_{i}=\frac{\partial ^{2}NEHR}{\partial \eta
_{e}\partial r_{i}}$, $x_{i}=\frac{\partial r_{i}}{\partial \eta _{e}}$. The
vector $\overrightarrow{x}$ can then be solved for using Gaussian
Elimination or other methods.

\paragraph{Invalidation}

Parents: All NEHRPartial and NEHRSecondPartial quantities.

\subsection{SectionalCurvature}

\paragraph{Input}

SectionalCurvature( Edge e )

\paragraph{Calculation}

SectionalCurvature calculates the discrete curvature at an edge in a three
dimensional manifold. \ The calculation is done by: 
\begin{equation*}
SectionalCurvature(e)=2\pi -\sum_{\substack{ All~tetra~t~  \\ incident~to~e}}%
\beta _{e,t}
\end{equation*}%
Where $\beta _{e,t}$ is the DihedralAngle at edge $e$ in tetra $t$.

\paragraph{Invalidation}

\bigskip Parents: The DihedralAngles of all tetrahedron containing e,
incident on e.

\subsection{TotalCurvature}

\paragraph{Input}

TotalCurvature( )

\paragraph{Calculation}

This is simply the sum of the Curvature3Ds of every vertex of the
triangulation.

\paragraph{Invalidation}

Parents: The Curvature3Ds of every vertex of the triangulation.

\bigskip

\subsection{TotalVolume}

\paragraph{Input}

TotalVolume( )

\paragraph{Calculation}

See \textbf{Functions: }\texttt{TotalVolume::TotalVolume} for more
information.

\paragraph{Invalidation}

\bigskip Parents: The Volume of every tetrahedron in the triangulation.

\subsection{Volume}

\paragraph{Input}

Volume(Tetra t)

\paragraph{Calculation}

The volume of a tetrahedron is done by calculating the Cayley Menger
Determinant, the determinant of a matrix that calculates the volume of a
simplex of any dimension. For the 3-simplex (tetrahedron) with $t=\{i,j,k,l\}
$, the volume is given by the formula 
\begin{equation*}
288V^{2}=\left\vert 
\begin{array}{ccccc}
0 & 1 & 1 & 1 & 1 \\ 
1 & 0 & l_{ij} & l_{ik} & l_{il} \\ 
1 & l_{ji} & 0 & l_{jk} & l_{jl} \\ 
1 & l_{ki} & l_{kj} & 0 & l_{kl} \\ 
1 & l_{li} & l_{lj} & l_{lk} & 0%
\end{array}%
\right\vert 
\end{equation*}%
where $l_{ij}$ is the length of edge $\{i,j\}$.

\paragraph{Invalidation}

\bigskip Parents: The Lengths of the six edges of $t$.

\subsection{TotalVolumePartial}

\paragraph{Input}

\begin{itemize}
\item TotalVolumePartial( Vertex v )

\item TotalVolumePartial( Edge e )
\end{itemize}

\paragraph{Calculation}

\begin{itemize}
\item ( Vertex v ): This sums up every VolumePartial(v, t) over all
tetrahedrons t containing v.

\item ( Edge e ): This sums up every VolumePartial(e, t) over all
tetrahedrons t containing e.
\end{itemize}

\paragraph{Invalidation}

Parents: All VolumePartials involving vertex v (edge e).

\subsection{TotalVolumeSecondPartial}

\paragraph{Input}

\begin{itemize}
\item TotalVolumePartial( Vertex v , Vertex w )

\item TotalVolumePartial( Vertex v, Edge e )
\end{itemize}

\paragraph{Calculation}

\begin{itemize}
\item ( Vertex v , Vertex w ): This sums up every VolumeSecondPartial(v, w,
t) over all tetrahedrons t containing v and w.

\item ( Vertex v, Edge e ): This sums up every VolumePartial(v, e, t) over
all tetrahedrons t containing v and e.
\end{itemize}

\paragraph{Invalidation}

Parents: All VolumeSecondPartials involving vertex v and vertex w (edge e).

\subsection{VolumeLengthPartial}

\paragraph{Input}

THIS GEOQUANT DOES NOT EXIST YET (EVER???)

\paragraph{Calculation}

\paragraph{Invalidation}

\bigskip

\subsection{VolumeLengthTetraPartial}

\paragraph{Input}

THIS GEOQUANT EXISTS, BUT WAS\ INTENDED\ FOR\ USE\ WITH\
VolumeLengthPartial. 

\paragraph{Calculation}

\paragraph{Invalidation}

\bigskip

\subsection{VolumePartial}

\paragraph{Input}

\begin{itemize}
\item VolumePartial( Vertex v, Tetra t )

\item VolumePartial( Edge e, Tetra t )
\end{itemize}

\paragraph{Calculation}

\begin{itemize}
\item (Vertex v, Tetra t): Returns the following partial derivative: 
\begin{equation*}
VolPartial(v,t)=\frac{\partial V_{t}}{\partial r_{v}}
\end{equation*}%
See \textbf{Functions: }\texttt{VolumePartial::VolumePartial} for more
information.

\item (Edge e, Tetra t): Returns the following partial derivative: 
\begin{equation*}
VolPartial(e,t)=\frac{\partial V_{t}}{\partial \eta _{e}}
\end{equation*}
\end{itemize}

\paragraph{Invalidation}

Parents: The Alphas, Radii, and Etas that are associated with $t$.

\bigskip

\subsection{VolumePartialSum}

\paragraph{Input}

EQUIVALENT\ TO\ TotalVolumePartial... TotalVolumePartial IS UP-TO-DATE.

\paragraph{Calculation}

\paragraph{Invalidation}

\bigskip

\subsection{VolumeSecondPartial}

\paragraph{Input}

\begin{itemize}
\item VolumeSecondPartial( Vertex v, Vertex w, Tetra t )

\item VolumeSecondPartial( Vertex v, Edge e, Tetra t )

\item VolumeSecondPartial( Edge e, Edge f, Tetra t )
\end{itemize}

\paragraph{Calculation}

\begin{itemize}
\item ( Vertex v, Vertex w, Tetra t ): Returns the following second-order
partial derivative: 
\begin{equation*}
VolSecPartial(v,w,t)=\frac{\partial ^{2}V_{t}}{\partial r_{w}\partial r_{v}}
\end{equation*}%
See \textbf{Functions: }\texttt{VolumeSecondPartial::VolumeSecondPartial}
for more information.

\item ( Vertex v, Edge e, Tetra t ): Returns the following second-order
partial derivative: 
\begin{equation*}
VolSecPartial(v,e,t)=\frac{\partial ^{2}V_{t}}{\partial \eta _{e}\partial
r_{v}}
\end{equation*}

\item ( Edge e, Edge f, Tetra t ): Returns the following second-order
partial derivative: 
\begin{equation*}
VolSecPartial(e,f,t)=\frac{\partial ^{2}V_{t}}{\partial \eta _{f}\partial
\eta _{e}}
\end{equation*}
\end{itemize}

\paragraph{Invalidation}

Parents: The Alphas, Radii, and Etas that are associated with $t$.

\bigskip
}}%
%BeginExpansion
%TCIDATA{Version=5.00.0.2552}
%TCIDATA{LaTeXparent=0,0,geocam.tex}
                      
%TCIDATA{ChildDefaults=chapter:1,page:1}


\chapter{Geoquants}

\section{Introduction}

Geoquants were developed by Alex Henniges, Joeseph Thomas, and Kurtis
Norwood during the spring of 2009 as the primary component in streamlining
the execution of the code during extended optimization routines. The purpose
is to provide easy access to calculations of geometric quantities without
requiring extraneous calculations of those quantities while they remain the
same.\ Geoquants achieve this through the use of an invalidation tree to
prevent uneccesary calcuations.\bigskip

\section{General Properties}

Each geoquant contains a calculation of a quantity, invalidation
information, and input/output functionality. \ Some of these components are
specific to each geoquant while some are common to all geoquants.

\subsubsection{Calculation}

Each geoquant has a unique calculation yielding a geometric quantity. \ In
general there are no commonalities among the calculations except that all
geoquants output a quantity of geometric nature. \ 

\subsubsection{Input/Output}

Geoquants recieve as input topological data (vertex, face, tetrahedron) \
which serves as a unique identifier among all such quantities of the same
type across the triangulation. The topological data is composed of zero or
more of each dimension of simplex. The following functions are available to
all geoquants:

\begin{itemize}
\item ExampleGeoQuant::At(input)

At returns an instance of the ExampleGeoQuant identified by input.

\item ExampleGeoquant::valueAt(input)

valueAt returns the value of the geoquant identified by input as a
double-precision number.

\item getValue()

getValue returns the value of an already obtained geoquant.

\item setValue(value)

\ setValue will set the value of this geoquant to the specified value and
subsequently invalidate all dependent geoquants.
\end{itemize}

The functions getValue and setValue are called as:

\texttt{Q=ExampleGeoquant::At(input)}

\texttt{Q-\TEXTsymbol{>}getValue}

\texttt{Q-\TEXTsymbol{>}setValue}

\subsubsection{Invalidation}

The value of a geoquant is obtained in one of two ways: it is set by the
user or it is calculated from other geoquants. In the situation where a
geoquant B is calculated in part from the value of geoquant A, we refer to A
as the parent and B as the dependent. When the value of a parent changes,
the parent notifies all of its dependents that their own value is no longer
valid. Once a request for the value of an invalid geoquant is made, that
geoquant requests the value of each of its parents in order to perform the
recalculation of its own value.

\section{Geoquant Catalog}

\subsection{Alpha}

\paragraph{Input}

Alpha(Vertex v)

\paragraph{Calculation}

This geoquant is never calculated, only set by the user. Its value is
generally 0 or 1.

\paragraph{Invalidation}

Parents: None

\subsection{Area}

\paragraph{Input}

Area( Face f )

\paragraph{Calculation}

Area returns the area of a triangular face (in a triangulation). \ The
calculation is done using Heron's formula which takes as inputs the lengths
of the sides of the triangle.%
\begin{equation*}
A=\sqrt{s\left( s-a\right) \left( s-b\right) \left( s-c\right) }
\end{equation*}%
where%
\begin{equation*}
s=\frac{a+b+c}{2}.
\end{equation*}

\paragraph{Invalidation}

Parents: \ all lengths of edges in the input triangle.

\subsection{CurvaturePartial}

\paragraph{Input}

\begin{itemize}
\item CurvaturePartial( Vertex v, Vertex w )

\item CurvaturePartial( Vertex v, Edge e )
\end{itemize}

\paragraph{Calculation}

\begin{itemize}
\item ( Vertex v, Vertex w ): Returns the following partial derivative:%
\begin{equation*}
CurvaturePartial(v,w)=\frac{\partial K_{v}}{\partial \log r_{w}}.
\end{equation*}%
The calculation of $CurvaturePartial(v,w)$ differs depending on the
combintorial relationship between $v$ and $w$. \ See \textbf{Functions: }%
\texttt{CurvaturePartial::CurvaturePartial} for more information.

\item ( Vertex v, Edge e ): Returns the following partial derivative: 
\begin{equation*}
CurvaturePartial(v,e)=\frac{\partial K_{v}}{\partial \eta _{e}}
\end{equation*}
\end{itemize}

\paragraph{Invalidation}

Parents: Radius of v and all radii of vertices adjacent to both v and w, the
curvature at v, and the dual areas, dihedral angles, and eta of every edge
adjacent to both v and w.

\subsection{Curvature2D}

\paragraph{Input}

Curvature2D( Vertex v )

\paragraph{Calculation}

Curvature2D calculates the discrete curvature at a vertex in a two
dimensional manifold. \ The calculation is done by:%
\begin{equation*}
Curvature2D\left( v\right) =2\pi -\sum\limits_{\substack{ all\text{ }%
triangles\text{ }t  \\ incident\text{ }to\text{ }v}}\theta _{v,t}
\end{equation*}%
where $\theta _{v,t}$ is the angle at vertex $v$ in triangle $t$. \ 

\paragraph{Invalidation}

\bigskip Parents: Every angle incident on vertex v.

\subsection{Curvature3D}

\paragraph{Input}

Curvature3D( Vertex v )

\paragraph{Calculation}

Curvature3D calculates the curvature at a vertex in a three dimensional
manifold. \ If we let $K_{ij}$ be the secional curvature at an edge $\left\{
i,j\right\} $ and let vertex $v$ correspond to $i$, then%
\begin{equation*}
Curvature3D\left( v\right) =\sum\limits_{\substack{ j\text{ s.t. }\left\{
i,j\right\}  \\ \text{is an edge}}}K_{ij}d_{ij}.
\end{equation*}

\paragraph{Invalidation}

Parents: For every edge of v, the sectional curvature of that edge and its
partial edge with respect to v.\bigskip

\subsection{DihedralAngle}

\paragraph{Input}

DihedralAngle( Edge e, Tetra t )

\paragraph{Calculation}

The dihedral angle of an edge in a tetrahedron is calculated using the
spherical law of cosines. \ Given a geometrically determined tetrahedron,
the face angles can be computed. \ At a vertex of the tetrahedron, the face
angles correspond to the sides lengths of a spherical triangle and the
dihedral angles correspond to the angles of the spherical triangle. \ If we
denote the dihedral angle at edge $e=\left\{ i,j\right\} $ in tetrahedron $%
t=\left\{ i,j,k,l\right\} $ by $\beta _{ij,kl}$, and denote the face angle
at vertex $v=i$ of triangle $\left\{ i,j,k\right\} $ by $\gamma _{i,jk}$ we
have that:%
\begin{equation*}
\beta _{ij,kl}=\arccos \left( \frac{\cos \left( \gamma _{i,kl}\right) -\cos
\left( \gamma _{i,jk}\right) \cos \left( \gamma _{i,jl}\right) }{\sin \left(
\gamma _{i,jk}\right) \sin \left( \gamma _{i,jl}\right) }\right)
\end{equation*}

\paragraph{Invalidation}

\bigskip Parents: The face angles incident on vertex $v=i$ and with face $%
f\in t$.

\section{DihedralAngleSum}

\paragraph{Input}

DihedralAngleSum( Edge e )

\paragraph{Calculation}

This geoquant sums every dihedral angle incident on the given edge.

\paragraph{Invalidation}

\bigskip Parents: For each tetrahedron containing e, the dihedral angle of
the tetrahedron incident on e.

\subsection{DualArea}

\paragraph{Input}

DualArea( Edge e )

\paragraph{Calculation}

See \textbf{Functions: }\texttt{DualArea::DualArea} for more information.

\paragraph{Invalidation}

Parents: The DualAreaSegments involving this edge across every tetrahedron
local to e.

\bigskip

\subsection{DualAreaSegment}

\paragraph{Input}

DualAreaSegment( Edge e, Tetra t )

\paragraph{Calculation}

See \textbf{Functions: }\texttt{DualAreaSegment::DualAreaSegment} for more
information.

\paragraph{Invalidation}

Parents: For both faces in t that contain e, the EdgeHeights between those
faces and e, and the FaceHeights between those faces and t.

\bigskip

\subsection{EdgeHeight}

\paragraph{Input}

EdgeHeight( Edge e, Face f )

\paragraph{Calculation}

See \textbf{Functions: }\texttt{EdgeHeight::EdgeHeight} for more information.

\paragraph{Invalidation}

Parents: For one of the two vertices, v, of e, the EuclideanAngle of f
incident on v, and the PartialEdges of the edges in f containing v, incident
on v.

\bigskip

\subsection{NEHRPartial}

\paragraph{Input}

\begin{itemize}
\item NEHRPartial( Vertex v )

\item NEHRPartial( Edge e )
\end{itemize}

\paragraph{Calculation}

\begin{itemize}
\item ( Vertex v ): Returns the following partial derivative: 
\begin{equation*}
NEHRPartial(v)=\frac{\partial NEHR}{\partial r_{v}}
\end{equation*}%
See \textbf{Functions: }\texttt{EHRPartial::EHRPartial} for more information.

\item ( Edge e ): Returns the following partial derivative: 
\begin{equation*}
NEHRPartial(e)=\frac{\partial NEHR}{\partial \eta _{e}}
\end{equation*}
\end{itemize}

\paragraph{Invalidation}

Parents: Both versions use TotalVolume, TotalCurvature, and the
TotalVolumePartial of vertex v (edge e). The vertex version also depends on
the Curvature3D at v. The edge version depends on all CurvaturePartials of e.

\bigskip

\subsection{NEHRSecondPartial}

\paragraph{Input}

\begin{itemize}
\item NEHRSecondPartial( Vertex v, Vertex w )

\item NEHRSecondPartial( Vertex v, Edge e )
\end{itemize}

\paragraph{Calculation}

\begin{itemize}
\item ( Vertex v, Vertex w ): Returns the following second-order partial
derivative: 
\begin{equation*}
NEHRSecPartial(v,w)=\frac{\partial NEHR}{\partial r_{w}\partial r_{v}}
\end{equation*}%
See \textbf{Functions: }\texttt{EHRSecondPartial::EHRSecondPartial} for more
information.

\item ( Vertex v, Edge e ) Returns the following second-order partial
derivative: 
\begin{equation*}
NEHRSecPartial(v,w)=\frac{\partial ^{2}NEHR}{\partial \eta _{e}\partial r_{v}%
}
\end{equation*}
\end{itemize}

\paragraph{Invalidation}

\bigskip Parents: Both versions use TotalVolume, TotalCurvature, and the
Radius, TotalVolumePartial and Curvature3D at v. 

\begin{itemize}
\item (v, w): The Curvature3D and TotalVolumePartial of w, the
CurvaturePartial of v with respect to w, and the TotalVolumeSecondPartial
with respect to v and w.

\item (v, e): The CurvaturePartials and TotalVolumePartial at e, the
CurvaturePartial of v with repsect to e, and the TotalVolumeSecondPartial
with respect to v and e.
\end{itemize}

\subsection{Eta}

\paragraph{Input}

Eta( Edge e )

\paragraph{Calculation}

This geoquant is never calculated, only set by the user.

\paragraph{Invalidation}

Parents: None

\bigskip

\subsection{EuclideanAngle}

\paragraph{Input}

EuclideanAngle( Vertex v, Face f )

\paragraph{Calculation}

The Euclidean angle is calculated usign the Euclidean law of cosines,
namely, let $v$ be indexed by $i$ and $f=\{i,j,k\}$. Then the angle at
vertex $i$ is%
\begin{equation*}
\cos (\alpha _{i,jk})=\frac{l_{ij}^{2}+l_{ik}^{2}-l_{jk}^{2}}{2l_{ij}l_{jk}}
\end{equation*}

\paragraph{Invalidation}

\bigskip Parents: The Lengths corresponding to the three edges of face f.

\subsection{FaceHeight}

\paragraph{Input}

FaceHeight( Face f, Tetra t )

\paragraph{Calculation}

See \textbf{Functions: }\texttt{FaceHeight::FaceHeight} for more information.

\paragraph{Invalidation}

\bigskip Parents: For one of the two edges, e, of f, the DihedralAngle of t
incident on e, and the EdgeHeights of the faces in t containing e, incident
on e.

\subsection{Length}

\paragraph{Input}

Length( Edge e )

\paragraph{Calculation}

Let $\alpha _{i}$, $\alpha _{j}$, $r_{i}$, and $r_{j}$ be the alphas and
radii at the vertices of edge $e=\{i,j\}$. Let $\eta _{ij}$ be the eta value
of edge e. Then the length is 
\begin{equation*}
l_{ij}^{2}=\alpha _{i}r_{i}^{2}+\alpha _{j}r_{j}^{2}+2r_{i}r_{j}\eta _{ij}
\end{equation*}

\paragraph{Invalidation}

\bigskip Parents: The Alpha and Radius values at both vertices of edge e and
the Eta value at edge e.

\subsection{PartialEdge}

\paragraph{Input}

PartialEdge( Vertex v, Edge e )

\paragraph{Calculation}

See \textbf{Functions: }\texttt{PartialEdge::PartialEdge} for more
information.

\paragraph{Invalidation}

\bigskip Parents: The Alpha and Radius values at both vertices of edge e and
the Length at edge e.

\subsection{Radius}

\paragraph{Input}

Radius( Vertex v )

\paragraph{Calculation}

This geoquant is never calcualted, only set by the user.

\paragraph{Invalidation}

Parents: None

\bigskip 

\subsection{RadiusPartial}

\paragraph{Input}

RadiusPartial(Vertex v, Edge e)

\paragraph{Calculation}

This returns the following partial derivative: 
\begin{equation*}
RadiusPartial(v,e)=\frac{\partial r_{v}}{\partial \eta _{e}}
\end{equation*}%
This quantity is calculated for all radii with $e$ fixed as follows: 
\begin{equation*}
\overrightarrow{v}+A\overrightarrow{x}=0
\end{equation*}%
where $A$ is a matrix with entries $a_{ij}=\frac{\partial ^{2}NEHR}{\partial
r_{j}\partial r_{i}}$ and $v_{i}=\frac{\partial ^{2}NEHR}{\partial \eta
_{e}\partial r_{i}}$, $x_{i}=\frac{\partial r_{i}}{\partial \eta _{e}}$. The
vector $\overrightarrow{x}$ can then be solved for using Gaussian
Elimination or other methods.

\paragraph{Invalidation}

Parents: All NEHRPartial and NEHRSecondPartial quantities.

\subsection{SectionalCurvature}

\paragraph{Input}

SectionalCurvature( Edge e )

\paragraph{Calculation}

SectionalCurvature calculates the discrete curvature at an edge in a three
dimensional manifold. \ The calculation is done by: 
\begin{equation*}
SectionalCurvature(e)=2\pi -\sum_{\substack{ All~tetra~t~  \\ incident~to~e}}%
\beta _{e,t}
\end{equation*}%
Where $\beta _{e,t}$ is the DihedralAngle at edge $e$ in tetra $t$.

\paragraph{Invalidation}

\bigskip Parents: The DihedralAngles of all tetrahedron containing e,
incident on e.

\subsection{TotalCurvature}

\paragraph{Input}

TotalCurvature( )

\paragraph{Calculation}

This is simply the sum of the Curvature3Ds of every vertex of the
triangulation.

\paragraph{Invalidation}

Parents: The Curvature3Ds of every vertex of the triangulation.

\bigskip

\subsection{TotalVolume}

\paragraph{Input}

TotalVolume( )

\paragraph{Calculation}

See \textbf{Functions: }\texttt{TotalVolume::TotalVolume} for more
information.

\paragraph{Invalidation}

\bigskip Parents: The Volume of every tetrahedron in the triangulation.

\subsection{Volume}

\paragraph{Input}

Volume(Tetra t)

\paragraph{Calculation}

The volume of a tetrahedron is done by calculating the Cayley Menger
Determinant, the determinant of a matrix that calculates the volume of a
simplex of any dimension. For the 3-simplex (tetrahedron) with $t=\{i,j,k,l\}
$, the volume is given by the formula 
\begin{equation*}
288V^{2}=\left\vert 
\begin{array}{ccccc}
0 & 1 & 1 & 1 & 1 \\ 
1 & 0 & l_{ij} & l_{ik} & l_{il} \\ 
1 & l_{ji} & 0 & l_{jk} & l_{jl} \\ 
1 & l_{ki} & l_{kj} & 0 & l_{kl} \\ 
1 & l_{li} & l_{lj} & l_{lk} & 0%
\end{array}%
\right\vert 
\end{equation*}%
where $l_{ij}$ is the length of edge $\{i,j\}$.

\paragraph{Invalidation}

\bigskip Parents: The Lengths of the six edges of $t$.

\subsection{TotalVolumePartial}

\paragraph{Input}

\begin{itemize}
\item TotalVolumePartial( Vertex v )

\item TotalVolumePartial( Edge e )
\end{itemize}

\paragraph{Calculation}

\begin{itemize}
\item ( Vertex v ): This sums up every VolumePartial(v, t) over all
tetrahedrons t containing v.

\item ( Edge e ): This sums up every VolumePartial(e, t) over all
tetrahedrons t containing e.
\end{itemize}

\paragraph{Invalidation}

Parents: All VolumePartials involving vertex v (edge e).

\subsection{TotalVolumeSecondPartial}

\paragraph{Input}

\begin{itemize}
\item TotalVolumePartial( Vertex v , Vertex w )

\item TotalVolumePartial( Vertex v, Edge e )
\end{itemize}

\paragraph{Calculation}

\begin{itemize}
\item ( Vertex v , Vertex w ): This sums up every VolumeSecondPartial(v, w,
t) over all tetrahedrons t containing v and w.

\item ( Vertex v, Edge e ): This sums up every VolumePartial(v, e, t) over
all tetrahedrons t containing v and e.
\end{itemize}

\paragraph{Invalidation}

Parents: All VolumeSecondPartials involving vertex v and vertex w (edge e).

\subsection{VolumeLengthPartial}

\paragraph{Input}

THIS GEOQUANT DOES NOT EXIST YET (EVER???)

\paragraph{Calculation}

\paragraph{Invalidation}

\bigskip

\subsection{VolumeLengthTetraPartial}

\paragraph{Input}

THIS GEOQUANT EXISTS, BUT WAS\ INTENDED\ FOR\ USE\ WITH\
VolumeLengthPartial. 

\paragraph{Calculation}

\paragraph{Invalidation}

\bigskip

\subsection{VolumePartial}

\paragraph{Input}

\begin{itemize}
\item VolumePartial( Vertex v, Tetra t )

\item VolumePartial( Edge e, Tetra t )
\end{itemize}

\paragraph{Calculation}

\begin{itemize}
\item (Vertex v, Tetra t): Returns the following partial derivative: 
\begin{equation*}
VolPartial(v,t)=\frac{\partial V_{t}}{\partial r_{v}}
\end{equation*}%
See \textbf{Functions: }\texttt{VolumePartial::VolumePartial} for more
information.

\item (Edge e, Tetra t): Returns the following partial derivative: 
\begin{equation*}
VolPartial(e,t)=\frac{\partial V_{t}}{\partial \eta _{e}}
\end{equation*}
\end{itemize}

\paragraph{Invalidation}

Parents: The Alphas, Radii, and Etas that are associated with $t$.

\bigskip

\subsection{VolumePartialSum}

\paragraph{Input}

EQUIVALENT\ TO\ TotalVolumePartial... TotalVolumePartial IS UP-TO-DATE.

\paragraph{Calculation}

\paragraph{Invalidation}

\bigskip

\subsection{VolumeSecondPartial}

\paragraph{Input}

\begin{itemize}
\item VolumeSecondPartial( Vertex v, Vertex w, Tetra t )

\item VolumeSecondPartial( Vertex v, Edge e, Tetra t )

\item VolumeSecondPartial( Edge e, Edge f, Tetra t )
\end{itemize}

\paragraph{Calculation}

\begin{itemize}
\item ( Vertex v, Vertex w, Tetra t ): Returns the following second-order
partial derivative: 
\begin{equation*}
VolSecPartial(v,w,t)=\frac{\partial ^{2}V_{t}}{\partial r_{w}\partial r_{v}}
\end{equation*}%
See \textbf{Functions: }\texttt{VolumeSecondPartial::VolumeSecondPartial}
for more information.

\item ( Vertex v, Edge e, Tetra t ): Returns the following second-order
partial derivative: 
\begin{equation*}
VolSecPartial(v,e,t)=\frac{\partial ^{2}V_{t}}{\partial \eta _{e}\partial
r_{v}}
\end{equation*}

\item ( Edge e, Edge f, Tetra t ): Returns the following second-order
partial derivative: 
\begin{equation*}
VolSecPartial(e,f,t)=\frac{\partial ^{2}V_{t}}{\partial \eta _{f}\partial
\eta _{e}}
\end{equation*}
\end{itemize}

\paragraph{Invalidation}

Parents: The Alphas, Radii, and Etas that are associated with $t$.

\bigskip
%
%EndExpansion

%TCIMACRO{\QSubDoc{Include Combinatorics}{%TCIDATA{Version=5.00.0.2606}
%TCIDATA{LaTeXparent=0,0,geocam.tex}
                      

\chapter{Combinatorics Code}

\begin{description}
\item[simplex] 
\end{description}

\bigskip 

\begin{description}
\item[vertex] 
\end{description}

\bigskip 

\begin{description}
\item[edge] 
\end{description}

\bigskip 

\begin{description}
\item[face] 
\end{description}

\bigskip 

\begin{description}
\item[tetra] 
\end{description}

\bigskip 

\begin{description}
\item[triangulation] 
\end{description}
}}%
%BeginExpansion
%TCIDATA{Version=5.00.0.2606}
%TCIDATA{LaTeXparent=0,0,geocam.tex}
                      

\chapter{Combinatorics Code}

\begin{description}
\item[simplex] 
\end{description}

\bigskip 

\begin{description}
\item[vertex] 
\end{description}

\bigskip 

\begin{description}
\item[edge] 
\end{description}

\bigskip 

\begin{description}
\item[face] 
\end{description}

\bigskip 

\begin{description}
\item[tetra] 
\end{description}

\bigskip 

\begin{description}
\item[triangulation] 
\end{description}
%
%EndExpansion

%TCIMACRO{\QSubDoc{Include Miscellaneous}{%TCIDATA{Version=5.00.0.2606}
%TCIDATA{LaTeXparent=0,0,geocam.tex}
                      

\chapter{Miscellaneous Code (organized by file)}

\section{3DInputOutput}

\begin{description}
\item[make3DTriangulationFile] 

\item[print3DResultsStep] 

\item[printResultsVolumes] 

\item[read3DTriangulationFile] 

\item[readEtas] 

\item[write3DTriangulationFile] 
\end{description}

\bigskip

\section{miscmath}

\begin{description}
\item[circleIntersection] 

\item[distancePoint] 

\item[findPoint] 

\item[labelEdge] 

\item[labelFace] 

\item[labelTetra] 

\item[LinearEquationsSolver] 

\item[listDifference] 

\item[listIntersection] 

\item[multiplicityDifference] 

\item[multiplicityIntersection] 

\item[multiplicityUnion] 

\item[quadratic] 

\item[rotateVector] 
\end{description}

\bigskip

\section{triangulationInputOutput}

\begin{description}
\item[makeTriangulationFile] 

\item[printResultsNum] 

\item[printResultsNumSteps] 

\item[printResultsStep] 

\item[printResultsVertex] 

\item[readTriangulationFile] 

\item[writeTriangulationFile] 

\item[writeTriangulationFileWithData] 
\end{description}

\bigskip

\section{utilties}

\begin{description}
\item[pause] 

\item[printGradient] 

\item[printHessian] 
\end{description}

\bigskip
}}%
%BeginExpansion
%TCIDATA{Version=5.00.0.2606}
%TCIDATA{LaTeXparent=0,0,geocam.tex}
                      

\chapter{Miscellaneous Code (organized by file)}

\section{3DInputOutput}

\begin{description}
\item[make3DTriangulationFile] 

\item[print3DResultsStep] 

\item[printResultsVolumes] 

\item[read3DTriangulationFile] 

\item[readEtas] 

\item[write3DTriangulationFile] 
\end{description}

\bigskip

\section{miscmath}

\begin{description}
\item[circleIntersection] 

\item[distancePoint] 

\item[findPoint] 

\item[labelEdge] 

\item[labelFace] 

\item[labelTetra] 

\item[LinearEquationsSolver] 

\item[listDifference] 

\item[listIntersection] 

\item[multiplicityDifference] 

\item[multiplicityIntersection] 

\item[multiplicityUnion] 

\item[quadratic] 

\item[rotateVector] 
\end{description}

\bigskip

\section{triangulationInputOutput}

\begin{description}
\item[makeTriangulationFile] 

\item[printResultsNum] 

\item[printResultsNumSteps] 

\item[printResultsStep] 

\item[printResultsVertex] 

\item[readTriangulationFile] 

\item[writeTriangulationFile] 

\item[writeTriangulationFileWithData] 
\end{description}

\bigskip

\section{utilties}

\begin{description}
\item[pause] 

\item[printGradient] 

\item[printHessian] 
\end{description}

\bigskip
%
%EndExpansion

\part{Theory}

%TCIMACRO{\QSubDoc{Include glossary}{%TCIDATA{Version=5.00.0.2606}
%TCIDATA{LaTeXparent=0,0,geocam.tex}
                      
%TCIDATA{ChildDefaults=chapter:1,page:1}


\chapter{Glossary}

\begin{description}
\item[center (of a simplex)] 

\item[circle power] Given a circle $C$ and a point $P$, let $L$ be the line
through the point $P$ that passes through the center of the circle. \ Let $A$
and $B$ be the intersection points of $L$ with the circle.\ Define a signed
distance $\left\Vert \overline{PX}\right\Vert $ for a line segment $%
\overline{PX}$ to be negative if the line segment lies entirely within the
circle, and positive otherwise. \ The \textit{circle power} of $P$ relative
to $C$, denoted by $pow_{C}\left( P\right) $, is given by:%
\[
pow_{C}\left( P\right) =\left\Vert \overline{PA}\right\Vert \left\Vert 
\overline{PB}\right\Vert . 
\]%
Alternatively, if $C$ is defined implicitly by $\left( x-x_{c}\right)
^{2}+\left( y-y_{c}\right) ^{2}=r_{c}^{2}$, then the circle power can be
expressed as:%
\[
pow_{C}\left( P\right) =\left( x-x_{c}\right) ^{2}+\left( y-y_{c}\right)
^{2}-r_{c}^{2}. 
\]

\item[common power point] The point of the plane (or $%
%TCIMACRO{\U{211d} }%
%BeginExpansion
\mathbb{R}
%EndExpansion
^{3}$) containing a decorated triangle (or tetrahedron) that has the same
circle power with respect to each of the weight circles. \ 

\item[decorated simplex (edge, triangle, tetrahedron,...)] A simplex is
called \textit{decorated} when weights are assigned to the vertices of the
simplex and then actualized by embedding the simplex into Euclidean space
together with spheres centered at the vertices with radii determined by the
weights. \ The orthocircle (if one exists) is sometimes considered part of
the decorated simplex when appropriate.

\item[edge curvature] Given a three-dimensional piecewise flat manifold $%
\left( M,\mathcal{T},d\right) $, the \textit{edge curvature} along an edge $%
\left\{ i,j\right\} $, measures how much that edge differs from Euclidean
space. \ Specifically, the edge curvature $K_{ij}$ is given by 
\[
K_{ij}=\left( 2\pi -\sum\limits_{\substack{ k,l\text{, such that}  \\ %
\left\{ i,j,k,l\right\} \in \mathcal{T}}}\beta _{ij,kl}\right) l_{ij}, 
\]%
where $l_{ij}$ is the edge length, and $\beta _{ij,kl}$ is the dihedral
angle of the edge $\left\{ i,j\right\} $ of the tetrahedron $\left\{
i,j,k,l\right\} $. \ In a triangulation of three-dimensional Euclidean space 
$K_{ij}=0$ for all edges. \ 

\item[Einstein constant] For a 3-dimensional piecewise flat manifold $\left(
M,\mathcal{T},d\right) $, the \textit{Einstein constant} $\lambda $ is given
by%
\[
\lambda =\frac{\mathcal{EHR}\left( M,\mathcal{T},d\right) }{3\mathcal{V}}, 
\]%
where $\mathcal{EHR}\left( M,\mathcal{T},d\right) $ is the
Einstein-Hilbert-Regge functional and $\mathcal{V}$ is the total volume.

\item[Einstein metric] Given a 3-dimensional piecewise flat manifold $\left(
M,\mathcal{T},d\right) $, we say that $d$ is an \textit{Einstein metric}
provided there exists $\lambda \in \mathbb{R}$ such that for all edges $%
\left\{ i,j\right\} $ in the triangulation we have:%
\[
K_{ij}=\lambda l_{ij}\frac{\partial \mathcal{V}}{\partial l_{ij}}, 
\]%
where $K_{ij}$ is the edge curvature, $l_{ij}$ is the edge length, and $%
\mathcal{V}$ is the total volume. By summing both sides, we see that $%
\lambda $ is the Einstein constant.

\item[hinge] A hinge consists of a pair of triangles that share a single
edge. \ Often, two adjacent triangles in a simplicial surface are identified
as a hinge while studying the edge they share. \ Any hinge can be
isometrically embedding $%
%TCIMACRO{\U{211d} }%
%BeginExpansion
\mathbb{R}
%EndExpansion
^{2}$. \ There is a natural generalization to three (and higher) dimensions
for two tetrahedra sharing a face. \ Similarly, this generalized hinge can
be isometrically embedded in $%
%TCIMACRO{\U{211d} }%
%BeginExpansion
\mathbb{R}
%EndExpansion
^{3}$. \ 

\item[inversive distance] Start with a decorated edge $e_{ij}$, that is, an
edge of length $l_{ij}$ with weight circles of radius $r_{i},r_{j}$ centered
on its vertices. \ The inversive distance $\eta _{ij}$ of the edge $e_{ij}$
can be calculated with the formula:%
\[
\eta _{ij}=\frac{l_{ij}^{2}-r_{i}^{2}-r_{j}^{2}}{2r_{i}r_{j}}. 
\]%
When the two weight circles intersect with angle $\theta _{ij}$, we have the
simple formula:%
\[
\eta _{ij}=-\cos \left( \theta _{ij}\right) . 
\]%
The former formula was obtained by using the law of cosines and solving for
the $\cos \left( \theta _{ij}\right) $ term. \ When the weight circles do
not intersect and do not contain one or the other $\eta _{ij}>1$. \ When the
weight circles intersect at some angle then $-1\leq \eta _{ij}\leq 1$. \ If
one weight circle contains the other we have $\eta _{ij}<1$. \ 

\item[manifold] A second countable, Hausdorff topological space $M$ is a 
\textit{manifold} provided there is an integer $n>0$ such that for each $%
x\in M$ there is an open set $U_{x}$ containing $x$ and a homeomorphism $%
h_{x}:U_{x}\rightarrow B\left( 1,0\right) \subset 
%TCIMACRO{\U{211d} }%
%BeginExpansion
\mathbb{R}
%EndExpansion
^{n}$. \ A discrete (or piecewise flat) space is a manifold provided the
sub-simplices satisfy the following conditions:

\item Dimension 2:

\begin{itemize}
\item All edges have exactly two adjacent faces.

\item For a vertex $v$, the faces incident upon $v$ can be arranged
cyclically as $f_{1},f_{2},...,f_{N},f_{1},...$ so that there is an edge
(containing $v$ as an endpoint) between each pair of consecutive faces $%
f_{i},f_{i+1}$, where $N+1$ is understood to be $1$. \ 
\end{itemize}

\item Dimension 3:

\begin{itemize}
\item All faces have exactly two adjacent tetrahedra.

\item For each edge $e$, the tetrahedra incident upon $e$ can be arranged
cyclically as $\sigma _{1},\sigma _{2},...,\sigma _{M},\sigma _{1},...$ so
that there is a face (containing $e$ as an edge) between each pair of
consecutive tetrahedra $\sigma _{i},\sigma _{i+1}$ where $M+1$ is understood
to be $1$. \ 

\item For each vertex $v$, the number of incident edges, faces and
tetrahedra, $E,F,T$, respectively (including multiple occurrences) satisfy:%
\[
E-F+T=2. 
\]
\end{itemize}

\item More generally, given a simplicial manifold $M$ of dimension $n$, and
a sub-simplex $\sigma $ of dimension $m<n$, the sub-simplices of $M$ of
dimension greater than $m$ have the structure of $S^{n-m-1}$.

\item[orthocircle] Given a decorated triangle, provided the common power
point is outside all of the weight circles, there exists a circle that is
orthogonal to each of the weight circles. \ That is, the \textit{orthocircle}
is the circle that intersects each of the weight circles orthogonally. \ The
orthocircle does not exist when the common power point is on or inside all
three circles, however if the common power point is at infinity, there is a
line that is orthogonal to the three weight circles which will also be
identified as the orthocircle.

\item[piecewise flat manifold] A triple $\left( M,\mathcal{T},\ell \right) $
where $\left( M,\mathcal{T}\right) $ is a triangulated manifold with
triangulation $\mathcal{T}$ and $\ell :E\rightarrow \mathbb{R}_{+}$ is a
function of the edges such that endowing the edges with lengths $\ell $
gives each simplex the structure of a nondegenerate Euclidean simplex (this
is equivalent to positivity of all relevant Cayley-Menger determinants). We
sometimes call two piecewise flat manifolds $\left( M,\mathcal{T},\ell
\right) $ and $\left( M,\mathcal{T}^{\prime },\ell ^{\prime }\right) $
isometric if they induce the same metric space structure, or if one can be
reached from the other by a sequence of metric Pachner moves (flips).\ 

\item[piecewise flat, metrized manifold] A triple $\left( M,\mathcal{T}%
,d\right) $ where $\left( M,\mathcal{T}\right) $ is a triangulated manifold
and $d:E^{+}\rightarrow \mathbb{R}$ is a function of oriented edges such
that if one defines $\ell _{ij}=d_{ij}+d_{ji}$ then $\left( M,\mathcal{T}%
,\ell \right) $ is a piecewise flat manifold and such that each simplex has
a uniquely determined center.

\item[triangulation] A collection of $n$-dimensional simplices $\mathcal{T}$
together with pairwise identifications for the $\left( n-1\right) $%
-dimensional faces of the simplices. \ More restrictions are needed to
ensure that the resultant space is a manifold. \ Alternatively, given a
space $M$ of dimension $n$, a \textit{triangulation} of $M$ is a subdivision
of $M$ into components $\left\{ \sigma _{i}\right\} $ by (hyper)surfaces (of
dimension $n-1$) so that each component is homeomorphic to an $n$%
-dimensional ball, and each component is combinatorially (as determined by
the subdivisions) equivalent to an $n$-simplex. \ 

\item[pseudo manifold] A discrete \textit{pseudo manifold} is a relaxation
of the manifold conditions for a discrete space. \ For dimensions two and
three, the bullet conditions given in the entry on manifold may no longer
hold. \ However, the manifold condition is still (trivially) satisfied in
the interior of all top dimensional simplices.

\item[weighted triangulation (or hinge)] A triangulation together with a map 
$w:V\rightarrow 
%TCIMACRO{\U{211d} }%
%BeginExpansion
\mathbb{R}
%EndExpansion
$, where $V$ is the set of vertices of the triangulation. \ Each simplex
becomes a decorated simplex. \ 

\item[dimension] A manifold has the property that every point has a
neighborhood homeomorphic to an open subset of $\mathbb{R}^{n}.$ By the
Invariance of Dimension theorem, $n$ must be locally constant (i.e., every
point in a given connected component must have the same dimension). We say $n
$ is the dimension of the manifold. For a triangulated manifold, the
dimension of the manifold is the dimension of the highest-dimensional
simplices in the triangulation.

\item[metric] Metric can have a number of meanings: (1) a metric can mean a
distance function for a metric space, (2)\ a Riemannian metric is an inner
product at each tangent space which varies smoothly with the basepoint. It
can also be described as a symmetric, positive definite 2-tensor field, (3)
a Lorenzian metric is a symmetric bilinear form at each tangent space with
signature +,-,-,...,- or -,+,...+ which varies smoothly with the basepoint.

\item[curvature] 

\item[scalar curvature] For a Riemannian manifold $\left( M,g\right) $,
scalar curvature is a function $R:M\rightarrow \mathbb{R}$ gotten from the
full trace of the Riemannian curvature tensor. For a piecewise flat,
metrized 3-manifold $\left( M,\mathcal{T},d\right) $, the scalar curvature
measure $K_{i}$ is defined as 
\[
K_{i}=\sum_{j}\frac{K_{ij}}{\ell _{ij}}d_{ij},
\]%
where $K_{ij}$ is the edge curvature of edge $\left\{ i,j\right\} .$
Sometimes we refer to the function $i\rightarrow K_{i}/V_{i}$ as the scalar
curvature function (see entry on constant scalar curvature).

\item[constant scalar curvature] For a Riemannian manifold $\left(
M,g\right) $, this means that the scalar curvature function $R:M\rightarrow 
\mathbb{R}$ is a constant function. For a piecewise flat, metrized
3-manifold, this means that the scalar curvature measure is a constant
multiple of the volume measure, i.e., there exists $\lambda \in \mathbb{R}$
such that 
\[
K_{i}=\lambda V_{i},
\]%
for all vertices $i,$ where $K_{i}=\sum_{j}\frac{K_{ij}}{\ell _{ij}}d_{ij}$
and $V_{i}=\ell _{ij}\frac{\partial \mathcal{V}}{\partial \ell _{ij}}=\frac{1%
}{3}\sum_{j,k,\ell }h_{ijk,\ell }A_{ijk},$ where $\mathcal{V}$ is the total
volume, $h_{ijk,\ell }$ is the (signed) height of the center to face $%
\left\{ i,j,k\right\} $ and $A_{ijk}$ is the area of face $\left\{
i,j,k\right\} .$

\item[geometric flow] A geometric flow is a differential equation on a
geometry (loosely defined) which depends only on geometric quantities
(usually curvatures). Usual examples are Ricci flow, mean curvature flow,
and Yamabe flow on Riemannian manifolds. On discrete geometries, there are
discrete Ricci and Yamabe flows.

\item[curvature flow] 

\item[Yamabe flow] 

\item[normalized total scalar curvature functional] 

\item[Einstein-Hilbert functional] Given a Riemannian manifold $\left(
M,g\right) ,$ the Einstein-Hilbert functional $\mathcal{EH}\left( M,g\right) 
$ is defined by 
\[
\mathcal{EH}\left( M,g\right) =\int_{M}RdV 
\]%
where $R$ is the scalar curvature of $g$ and $dV$ is the volume form of $g.$
The functional is constant on two-dimensional manifolds, giving $4\pi $
times the Euler characteristic. In higher dimensions, its critical points
are Ricci flat (i.e., the Ricci tensor is zero) and critical points
restricted to metrics of a fixed volume are Einstein manifolds.

\item[Einstein-Hilbert-Regge functional] Given a piecewise flat manifold $%
\left( M^{n},\ell \right) $, the Einstein-Hilbert-Regge functional $\mathcal{%
EHR}\left( M,\ell \right) $ is defined by 
\[
\mathcal{EHR}\left( M,\ell \right) =\sum_{\sigma ^{n-2}}K\left( \sigma
^{n-2}\right) , 
\]%
where $K\left( \sigma ^{n-2}\right) $ is $2\pi $ minus the sum of the angles
at $\sigma ^{n-2}$ times the volume $V\left( \sigma ^{n-2}\right) $ of the
simplex $\sigma ^{n-2}.$ In particular, for $n=3,$ we have 
\[
\mathcal{EHR}\left( M^{3},\ell \right) =\sum_{\left\{ i,j\right\}
}K_{ij}=\sum_{\left\{ i,j\right\} }\left( 2\pi -\sum_{k,\ell }\beta
_{ij,k\ell }\right) \ell _{ij}, 
\]%
where $\beta _{ij,k\ell }$ is the dihedral angle at edge $\left\{
i,j\right\} $ in $\left\{ i,j,k,\ell \right\} .$

\item[conformal class] On a smooth Riemannian manifold $\left( M,g\right) ,$
the conformal class $\left[ g\right] $ is the equivalence class of all
Riemannian manifolds $\left( M,e^{f}g\right) ,$ where $f$ is a smooth
function. On a metrized piecewise flat manifold $\left( M,T,d\right) $, the
conformal class $\left[ d\right] $ is the equivalence class .....

\item[conformal deformation] 

\item[min-max procedure] 

\item[Yamabe constant] On a closed manifold $M,$ the Yamabe constant (also
called the $\sigma $-constant) is the number 
\[
\sigma \left( M^{n}\right) =\sup_{g}\inf_{g^{\prime }\in \left[ g\right] }%
\frac{\mathcal{EH}\left( M,g^{\prime }\right) }{V\left( M,g^{\prime }\right)
^{2/n}}, 
\]%
where the $\sup $ is over all metrics and the $\inf $ is over all metrics in
the same conformal class $\left[ g\right] $ as $g.$

\item[flip] 

\item[Pachner move] 

\item[flip algorithm] 

\item[convex hinge] 

\item[nonconvex hinge] 

\item[bone] 

\item[Delaunay triangulation (or hinge)] 

\item[weighted Delaunay triangulation] 

\item[Voronoi diagram (or cell)] 

\item[weighted Voronoi diagram (or cell)] 

\item[power diagram (or cell)] 

\item[negative triangles] 

\item[dual, Poincare dual] 

\item[dual length] 

\item[dual volume] 
\end{description}
}}%
%BeginExpansion
%TCIDATA{Version=5.00.0.2606}
%TCIDATA{LaTeXparent=0,0,geocam.tex}
                      
%TCIDATA{ChildDefaults=chapter:1,page:1}


\chapter{Glossary}

\begin{description}
\item[center (of a simplex)] 

\item[circle power] Given a circle $C$ and a point $P$, let $L$ be the line
through the point $P$ that passes through the center of the circle. \ Let $A$
and $B$ be the intersection points of $L$ with the circle.\ Define a signed
distance $\left\Vert \overline{PX}\right\Vert $ for a line segment $%
\overline{PX}$ to be negative if the line segment lies entirely within the
circle, and positive otherwise. \ The \textit{circle power} of $P$ relative
to $C$, denoted by $pow_{C}\left( P\right) $, is given by:%
\[
pow_{C}\left( P\right) =\left\Vert \overline{PA}\right\Vert \left\Vert 
\overline{PB}\right\Vert . 
\]%
Alternatively, if $C$ is defined implicitly by $\left( x-x_{c}\right)
^{2}+\left( y-y_{c}\right) ^{2}=r_{c}^{2}$, then the circle power can be
expressed as:%
\[
pow_{C}\left( P\right) =\left( x-x_{c}\right) ^{2}+\left( y-y_{c}\right)
^{2}-r_{c}^{2}. 
\]

\item[common power point] The point of the plane (or $%
%TCIMACRO{\U{211d} }%
%BeginExpansion
\mathbb{R}
%EndExpansion
^{3}$) containing a decorated triangle (or tetrahedron) that has the same
circle power with respect to each of the weight circles. \ 

\item[decorated simplex (edge, triangle, tetrahedron,...)] A simplex is
called \textit{decorated} when weights are assigned to the vertices of the
simplex and then actualized by embedding the simplex into Euclidean space
together with spheres centered at the vertices with radii determined by the
weights. \ The orthocircle (if one exists) is sometimes considered part of
the decorated simplex when appropriate.

\item[edge curvature] Given a three-dimensional piecewise flat manifold $%
\left( M,\mathcal{T},d\right) $, the \textit{edge curvature} along an edge $%
\left\{ i,j\right\} $, measures how much that edge differs from Euclidean
space. \ Specifically, the edge curvature $K_{ij}$ is given by 
\[
K_{ij}=\left( 2\pi -\sum\limits_{\substack{ k,l\text{, such that}  \\ %
\left\{ i,j,k,l\right\} \in \mathcal{T}}}\beta _{ij,kl}\right) l_{ij}, 
\]%
where $l_{ij}$ is the edge length, and $\beta _{ij,kl}$ is the dihedral
angle of the edge $\left\{ i,j\right\} $ of the tetrahedron $\left\{
i,j,k,l\right\} $. \ In a triangulation of three-dimensional Euclidean space 
$K_{ij}=0$ for all edges. \ 

\item[Einstein constant] For a 3-dimensional piecewise flat manifold $\left(
M,\mathcal{T},d\right) $, the \textit{Einstein constant} $\lambda $ is given
by%
\[
\lambda =\frac{\mathcal{EHR}\left( M,\mathcal{T},d\right) }{3\mathcal{V}}, 
\]%
where $\mathcal{EHR}\left( M,\mathcal{T},d\right) $ is the
Einstein-Hilbert-Regge functional and $\mathcal{V}$ is the total volume.

\item[Einstein metric] Given a 3-dimensional piecewise flat manifold $\left(
M,\mathcal{T},d\right) $, we say that $d$ is an \textit{Einstein metric}
provided there exists $\lambda \in \mathbb{R}$ such that for all edges $%
\left\{ i,j\right\} $ in the triangulation we have:%
\[
K_{ij}=\lambda l_{ij}\frac{\partial \mathcal{V}}{\partial l_{ij}}, 
\]%
where $K_{ij}$ is the edge curvature, $l_{ij}$ is the edge length, and $%
\mathcal{V}$ is the total volume. By summing both sides, we see that $%
\lambda $ is the Einstein constant.

\item[hinge] A hinge consists of a pair of triangles that share a single
edge. \ Often, two adjacent triangles in a simplicial surface are identified
as a hinge while studying the edge they share. \ Any hinge can be
isometrically embedding $%
%TCIMACRO{\U{211d} }%
%BeginExpansion
\mathbb{R}
%EndExpansion
^{2}$. \ There is a natural generalization to three (and higher) dimensions
for two tetrahedra sharing a face. \ Similarly, this generalized hinge can
be isometrically embedded in $%
%TCIMACRO{\U{211d} }%
%BeginExpansion
\mathbb{R}
%EndExpansion
^{3}$. \ 

\item[inversive distance] Start with a decorated edge $e_{ij}$, that is, an
edge of length $l_{ij}$ with weight circles of radius $r_{i},r_{j}$ centered
on its vertices. \ The inversive distance $\eta _{ij}$ of the edge $e_{ij}$
can be calculated with the formula:%
\[
\eta _{ij}=\frac{l_{ij}^{2}-r_{i}^{2}-r_{j}^{2}}{2r_{i}r_{j}}. 
\]%
When the two weight circles intersect with angle $\theta _{ij}$, we have the
simple formula:%
\[
\eta _{ij}=-\cos \left( \theta _{ij}\right) . 
\]%
The former formula was obtained by using the law of cosines and solving for
the $\cos \left( \theta _{ij}\right) $ term. \ When the weight circles do
not intersect and do not contain one or the other $\eta _{ij}>1$. \ When the
weight circles intersect at some angle then $-1\leq \eta _{ij}\leq 1$. \ If
one weight circle contains the other we have $\eta _{ij}<1$. \ 

\item[manifold] A second countable, Hausdorff topological space $M$ is a 
\textit{manifold} provided there is an integer $n>0$ such that for each $%
x\in M$ there is an open set $U_{x}$ containing $x$ and a homeomorphism $%
h_{x}:U_{x}\rightarrow B\left( 1,0\right) \subset 
%TCIMACRO{\U{211d} }%
%BeginExpansion
\mathbb{R}
%EndExpansion
^{n}$. \ A discrete (or piecewise flat) space is a manifold provided the
sub-simplices satisfy the following conditions:

\item Dimension 2:

\begin{itemize}
\item All edges have exactly two adjacent faces.

\item For a vertex $v$, the faces incident upon $v$ can be arranged
cyclically as $f_{1},f_{2},...,f_{N},f_{1},...$ so that there is an edge
(containing $v$ as an endpoint) between each pair of consecutive faces $%
f_{i},f_{i+1}$, where $N+1$ is understood to be $1$. \ 
\end{itemize}

\item Dimension 3:

\begin{itemize}
\item All faces have exactly two adjacent tetrahedra.

\item For each edge $e$, the tetrahedra incident upon $e$ can be arranged
cyclically as $\sigma _{1},\sigma _{2},...,\sigma _{M},\sigma _{1},...$ so
that there is a face (containing $e$ as an edge) between each pair of
consecutive tetrahedra $\sigma _{i},\sigma _{i+1}$ where $M+1$ is understood
to be $1$. \ 

\item For each vertex $v$, the number of incident edges, faces and
tetrahedra, $E,F,T$, respectively (including multiple occurrences) satisfy:%
\[
E-F+T=2. 
\]
\end{itemize}

\item More generally, given a simplicial manifold $M$ of dimension $n$, and
a sub-simplex $\sigma $ of dimension $m<n$, the sub-simplices of $M$ of
dimension greater than $m$ have the structure of $S^{n-m-1}$.

\item[orthocircle] Given a decorated triangle, provided the common power
point is outside all of the weight circles, there exists a circle that is
orthogonal to each of the weight circles. \ That is, the \textit{orthocircle}
is the circle that intersects each of the weight circles orthogonally. \ The
orthocircle does not exist when the common power point is on or inside all
three circles, however if the common power point is at infinity, there is a
line that is orthogonal to the three weight circles which will also be
identified as the orthocircle.

\item[piecewise flat manifold] A triple $\left( M,\mathcal{T},\ell \right) $
where $\left( M,\mathcal{T}\right) $ is a triangulated manifold with
triangulation $\mathcal{T}$ and $\ell :E\rightarrow \mathbb{R}_{+}$ is a
function of the edges such that endowing the edges with lengths $\ell $
gives each simplex the structure of a nondegenerate Euclidean simplex (this
is equivalent to positivity of all relevant Cayley-Menger determinants). We
sometimes call two piecewise flat manifolds $\left( M,\mathcal{T},\ell
\right) $ and $\left( M,\mathcal{T}^{\prime },\ell ^{\prime }\right) $
isometric if they induce the same metric space structure, or if one can be
reached from the other by a sequence of metric Pachner moves (flips).\ 

\item[piecewise flat, metrized manifold] A triple $\left( M,\mathcal{T}%
,d\right) $ where $\left( M,\mathcal{T}\right) $ is a triangulated manifold
and $d:E^{+}\rightarrow \mathbb{R}$ is a function of oriented edges such
that if one defines $\ell _{ij}=d_{ij}+d_{ji}$ then $\left( M,\mathcal{T}%
,\ell \right) $ is a piecewise flat manifold and such that each simplex has
a uniquely determined center.

\item[triangulation] A collection of $n$-dimensional simplices $\mathcal{T}$
together with pairwise identifications for the $\left( n-1\right) $%
-dimensional faces of the simplices. \ More restrictions are needed to
ensure that the resultant space is a manifold. \ Alternatively, given a
space $M$ of dimension $n$, a \textit{triangulation} of $M$ is a subdivision
of $M$ into components $\left\{ \sigma _{i}\right\} $ by (hyper)surfaces (of
dimension $n-1$) so that each component is homeomorphic to an $n$%
-dimensional ball, and each component is combinatorially (as determined by
the subdivisions) equivalent to an $n$-simplex. \ 

\item[pseudo manifold] A discrete \textit{pseudo manifold} is a relaxation
of the manifold conditions for a discrete space. \ For dimensions two and
three, the bullet conditions given in the entry on manifold may no longer
hold. \ However, the manifold condition is still (trivially) satisfied in
the interior of all top dimensional simplices.

\item[weighted triangulation (or hinge)] A triangulation together with a map 
$w:V\rightarrow 
%TCIMACRO{\U{211d} }%
%BeginExpansion
\mathbb{R}
%EndExpansion
$, where $V$ is the set of vertices of the triangulation. \ Each simplex
becomes a decorated simplex. \ 

\item[dimension] A manifold has the property that every point has a
neighborhood homeomorphic to an open subset of $\mathbb{R}^{n}.$ By the
Invariance of Dimension theorem, $n$ must be locally constant (i.e., every
point in a given connected component must have the same dimension). We say $n
$ is the dimension of the manifold. For a triangulated manifold, the
dimension of the manifold is the dimension of the highest-dimensional
simplices in the triangulation.

\item[metric] Metric can have a number of meanings: (1) a metric can mean a
distance function for a metric space, (2)\ a Riemannian metric is an inner
product at each tangent space which varies smoothly with the basepoint. It
can also be described as a symmetric, positive definite 2-tensor field, (3)
a Lorenzian metric is a symmetric bilinear form at each tangent space with
signature +,-,-,...,- or -,+,...+ which varies smoothly with the basepoint.

\item[curvature] 

\item[scalar curvature] For a Riemannian manifold $\left( M,g\right) $,
scalar curvature is a function $R:M\rightarrow \mathbb{R}$ gotten from the
full trace of the Riemannian curvature tensor. For a piecewise flat,
metrized 3-manifold $\left( M,\mathcal{T},d\right) $, the scalar curvature
measure $K_{i}$ is defined as 
\[
K_{i}=\sum_{j}\frac{K_{ij}}{\ell _{ij}}d_{ij},
\]%
where $K_{ij}$ is the edge curvature of edge $\left\{ i,j\right\} .$
Sometimes we refer to the function $i\rightarrow K_{i}/V_{i}$ as the scalar
curvature function (see entry on constant scalar curvature).

\item[constant scalar curvature] For a Riemannian manifold $\left(
M,g\right) $, this means that the scalar curvature function $R:M\rightarrow 
\mathbb{R}$ is a constant function. For a piecewise flat, metrized
3-manifold, this means that the scalar curvature measure is a constant
multiple of the volume measure, i.e., there exists $\lambda \in \mathbb{R}$
such that 
\[
K_{i}=\lambda V_{i},
\]%
for all vertices $i,$ where $K_{i}=\sum_{j}\frac{K_{ij}}{\ell _{ij}}d_{ij}$
and $V_{i}=\ell _{ij}\frac{\partial \mathcal{V}}{\partial \ell _{ij}}=\frac{1%
}{3}\sum_{j,k,\ell }h_{ijk,\ell }A_{ijk},$ where $\mathcal{V}$ is the total
volume, $h_{ijk,\ell }$ is the (signed) height of the center to face $%
\left\{ i,j,k\right\} $ and $A_{ijk}$ is the area of face $\left\{
i,j,k\right\} .$

\item[geometric flow] A geometric flow is a differential equation on a
geometry (loosely defined) which depends only on geometric quantities
(usually curvatures). Usual examples are Ricci flow, mean curvature flow,
and Yamabe flow on Riemannian manifolds. On discrete geometries, there are
discrete Ricci and Yamabe flows.

\item[curvature flow] 

\item[Yamabe flow] 

\item[normalized total scalar curvature functional] 

\item[Einstein-Hilbert functional] Given a Riemannian manifold $\left(
M,g\right) ,$ the Einstein-Hilbert functional $\mathcal{EH}\left( M,g\right) 
$ is defined by 
\[
\mathcal{EH}\left( M,g\right) =\int_{M}RdV 
\]%
where $R$ is the scalar curvature of $g$ and $dV$ is the volume form of $g.$
The functional is constant on two-dimensional manifolds, giving $4\pi $
times the Euler characteristic. In higher dimensions, its critical points
are Ricci flat (i.e., the Ricci tensor is zero) and critical points
restricted to metrics of a fixed volume are Einstein manifolds.

\item[Einstein-Hilbert-Regge functional] Given a piecewise flat manifold $%
\left( M^{n},\ell \right) $, the Einstein-Hilbert-Regge functional $\mathcal{%
EHR}\left( M,\ell \right) $ is defined by 
\[
\mathcal{EHR}\left( M,\ell \right) =\sum_{\sigma ^{n-2}}K\left( \sigma
^{n-2}\right) , 
\]%
where $K\left( \sigma ^{n-2}\right) $ is $2\pi $ minus the sum of the angles
at $\sigma ^{n-2}$ times the volume $V\left( \sigma ^{n-2}\right) $ of the
simplex $\sigma ^{n-2}.$ In particular, for $n=3,$ we have 
\[
\mathcal{EHR}\left( M^{3},\ell \right) =\sum_{\left\{ i,j\right\}
}K_{ij}=\sum_{\left\{ i,j\right\} }\left( 2\pi -\sum_{k,\ell }\beta
_{ij,k\ell }\right) \ell _{ij}, 
\]%
where $\beta _{ij,k\ell }$ is the dihedral angle at edge $\left\{
i,j\right\} $ in $\left\{ i,j,k,\ell \right\} .$

\item[conformal class] On a smooth Riemannian manifold $\left( M,g\right) ,$
the conformal class $\left[ g\right] $ is the equivalence class of all
Riemannian manifolds $\left( M,e^{f}g\right) ,$ where $f$ is a smooth
function. On a metrized piecewise flat manifold $\left( M,T,d\right) $, the
conformal class $\left[ d\right] $ is the equivalence class .....

\item[conformal deformation] 

\item[min-max procedure] 

\item[Yamabe constant] On a closed manifold $M,$ the Yamabe constant (also
called the $\sigma $-constant) is the number 
\[
\sigma \left( M^{n}\right) =\sup_{g}\inf_{g^{\prime }\in \left[ g\right] }%
\frac{\mathcal{EH}\left( M,g^{\prime }\right) }{V\left( M,g^{\prime }\right)
^{2/n}}, 
\]%
where the $\sup $ is over all metrics and the $\inf $ is over all metrics in
the same conformal class $\left[ g\right] $ as $g.$

\item[flip] 

\item[Pachner move] 

\item[flip algorithm] 

\item[convex hinge] 

\item[nonconvex hinge] 

\item[bone] 

\item[Delaunay triangulation (or hinge)] 

\item[weighted Delaunay triangulation] 

\item[Voronoi diagram (or cell)] 

\item[weighted Voronoi diagram (or cell)] 

\item[power diagram (or cell)] 

\item[negative triangles] 

\item[dual, Poincare dual] 

\item[dual length] 

\item[dual volume] 
\end{description}
%
%EndExpansion

\bigskip

\appendix

\chapter{Appendix}

The appendix fragment is used only once. Subsequent appendices can be
created using the Chapter Section/Body Tag.

\backmatter

\chapter{Afterword}

The back matter often includes one or more of an index, an afterword,
acknowledgements, a bibliography, a colophon, or any other similar item. In
the back matter, chapters do not produce a chapter number, but they are
entered in the table of contents. If you are not using anything in the back
matter, you can delete the back matter TeX field and everything that follows
it.

\end{document}
