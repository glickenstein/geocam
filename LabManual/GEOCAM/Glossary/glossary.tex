%TCIDATA{Version=5.00.0.2606}
%TCIDATA{LaTeXparent=0,0,geocam.tex}
                      
%TCIDATA{ChildDefaults=chapter:1,page:1}


\chapter{Glossary}

\bigskip

\begin{description}
\item[curvature flow] !!!

\item[Yamabe flow] !!!

\item[normalized total scalar curvature functional] !!!

\item[curvature] !!!

\item[min-max procedure] !!!

\item[dual volume] !!!

\item[power diagram (or cell)] !!!

\item[dual, Poincare dual] !!!

\item[dual length] !!!

\item[Pachner move] !!!

\item[flip algorithm] !!!

\item \bigskip
\end{description}

\bigskip

\begin{description}
\item[center (of a decorated simplex)] The weigth spheres of a deocrated
simplex determine a common power point of an embedding of the simplex. \
This common power point is referred to as the center of the (decorated)
simplex.

\item[circle power] Given a circle $C$ and a point $P$, let $L$ be the line
through the point $P$ that passes through the center of the circle. \ Let $A$
and $B$ be the intersection points of $L$ with the circle.\ Define a signed
distance $\left\Vert \overline{PX}\right\Vert $ for a line segment $%
\overline{PX}$ to be negative if the line segment lies entirely within the
circle, and positive otherwise. \ The \textit{circle power} of $P$ relative
to $C$, denoted by $pow_{C}\left( P\right) $, is given by:%
\[
pow_{C}\left( P\right) =\left\Vert \overline{PA}\right\Vert \left\Vert 
\overline{PB}\right\Vert . 
\]%
Alternatively, if $C$ is defined implicitly by $\left( x-x_{c}\right)
^{2}+\left( y-y_{c}\right) ^{2}=r_{c}^{2}$, then the circle power can be
expressed as:%
\[
pow_{C}\left( P\right) =\left( x-x_{c}\right) ^{2}+\left( y-y_{c}\right)
^{2}-r_{c}^{2}. 
\]

\item[common power point] The point of the plane (or $%
%TCIMACRO{\U{211d} }%
%BeginExpansion
\mathbb{R}
%EndExpansion
^{3}$) containing a decorated triangle (or tetrahedron) that has the same
circle power with respect to each of the weight circles. \ 

\item[decorated simplex (edge, triangle, tetrahedron,...)] A simplex is
called \textit{decorated} when weights are assigned to the vertices of the
simplex and then actualized by embedding the simplex into Euclidean space
together with spheres centered at the vertices with radii determined by the
weights. \ The orthocircle (if one exists) is sometimes considered part of
the decorated simplex when appropriate.

\item[edge curvature $\left( K_{ij}\right) $] Given a three-dimensional
piecewise flat manifold $\left( M,\mathcal{T},d\right) $, the \textit{edge
curvature} along an edge $\left\{ i,j\right\} $, measures how much that edge
differs from Euclidean space. \ Specifically, the edge curvature $K_{ij}$ is
given by 
\[
K_{ij}=\left( 2\pi -\sum\limits_{\substack{ k,l\text{, such that}  \\ %
\left\{ i,j,k,l\right\} \in \mathcal{T}}}\beta _{ij,kl}\right) l_{ij}, 
\]%
where $l_{ij}$ is the edge length, and $\beta _{ij,kl}$ is the dihedral
angle of the edge $\left\{ i,j\right\} $ of the tetrahedron $\left\{
i,j,k,l\right\} $. \ In a triangulation of three-dimensional Euclidean space 
$K_{ij}=0$ for all edges. \ 

\item[Einstein constant] For a 3-dimensional piecewise flat manifold $\left(
M,\mathcal{T},d\right) $, the \textit{Einstein constant} $\lambda $ is given
by%
\[
\lambda =\frac{\mathcal{EHR}\left( M,\mathcal{T},d\right) }{3\mathcal{V}}, 
\]%
where $\mathcal{EHR}\left( M,\mathcal{T},d\right) $ is the
Einstein-Hilbert-Regge functional and $\mathcal{V}$ is the total volume.

\item[Einstein metric] Given a 3-dimensional piecewise flat manifold $\left(
M,\mathcal{T},d\right) $, we say that $d$ is an \textit{Einstein metric}
provided there exists $\lambda \in \mathbb{R}$ such that for all edges $%
\left\{ i,j\right\} $ in the triangulation we have:%
\[
K_{ij}=\lambda l_{ij}\frac{\partial \mathcal{V}}{\partial l_{ij}}, 
\]%
where $K_{ij}$ is the edge curvature, $l_{ij}$ is the edge length, and $%
\mathcal{V}$ is the total volume. By summing both sides, we see that $%
\lambda $ is the Einstein constant.

\item[hinge] A hinge is a pair of triangles that share an edge. We often
denote a hinge using the vertices of the triangles, for example $\{h,i,j\}$
and $\{i,j,k\}$ are faces that share the edge $\{i,j\}$ and so together they
form a hinge. In a two-dimensional manifold, a (non-boundary) edge
determines a unique hinge, for example $\{i,j\}$ determines the hinge formed
by $\{h,i,j\}$ and $\{i,j,k\}$. An important point is that a triangle may
have duplicate edges or vertices, and so a hinge being constructed of
triangles can also have duplicate edges or vertices. In this case, we would
need to denote a hinge by $\left\{ f_{1},f_{2},e\right\} ,$ where $f_{1}$
and $f_{2}$ are triangles, and $e$ is an edge local to both $f_{1}$ and $%
f_{2}.$

\item[inversive distance] Start with a decorated edge $e_{ij}$, that is, an
edge of length $l_{ij}$ with weight circles of radius $r_{i},r_{j}$ centered
on its vertices. \ The inversive distance $\eta _{ij}$ of the edge $e_{ij}$
can be calculated with the formula:%
\[
\eta _{ij}=\frac{l_{ij}^{2}-r_{i}^{2}-r_{j}^{2}}{2r_{i}r_{j}}. 
\]%
When the two weight circles intersect with angle $\theta _{ij}$, we have the
simple formula:%
\[
\eta _{ij}=-\cos \left( \theta _{ij}\right) . 
\]%
The former formula was obtained by using the law of cosines and solving for
the $\cos \left( \theta _{ij}\right) $ term. \ When the weight circles do
not intersect and do not contain one or the other $\eta _{ij}>1$. \ When the
weight circles intersect at some angle then $-1\leq \eta _{ij}\leq 1$. \ If
one weight circle contains the other we have $\eta _{ij}<1$. \ 

\item[manifold] A second countable, Hausdorff topological space $M$ is a 
\textit{manifold} provided there is an integer $n>0$ such that for each $%
x\in M$ there is an open set $U_{x}$ containing $x$ and a homeomorphism $%
h_{x}:U_{x}\rightarrow B\left( 1,0\right) \subset 
%TCIMACRO{\U{211d} }%
%BeginExpansion
\mathbb{R}
%EndExpansion
^{n}$. \ A discrete (or piecewise flat) space is a manifold provided the
sub-simplices satisfy the following conditions:

\item Dimension 2:

\begin{itemize}
\item All edges have exactly two adjacent faces.

\item For a vertex $v$, the faces incident upon $v$ can be arranged
cyclically as $f_{1},f_{2},...,f_{N},f_{1},...$ so that there is an edge
(containing $v$ as an endpoint) between each pair of consecutive faces $%
f_{i},f_{i+1}$, where $N+1$ is understood to be $1$. \ 
\end{itemize}

\item Dimension 3:

\begin{itemize}
\item All faces have exactly two adjacent tetrahedra.

\item For each edge $e$, the tetrahedra incident upon $e$ can be arranged
cyclically as $\sigma _{1},\sigma _{2},...,\sigma _{M},\sigma _{1},...$ so
that there is a face (containing $e$ as an edge) between each pair of
consecutive tetrahedra $\sigma _{i},\sigma _{i+1}$ where $M+1$ is understood
to be $1$. \ 

\item For each vertex $v$, the number of incident edges, faces and
tetrahedra, $E,F,T$, respectively (including multiple occurrences) satisfy:%
\[
E-F+T=2. 
\]
\end{itemize}

\item More generally, given a simplicial manifold $M$ of dimension $n$, and
a sub-simplex $\sigma $ of dimension $m<n$, the sub-simplices of $M$ of
dimension greater than $m$ have the structure of $S^{n-m-1}$.

\item[orthocircle] Given a decorated triangle, provided the common power
point is outside all of the weight circles, there exists a circle that is
orthogonal to each of the weight circles. \ That is, the \textit{orthocircle}
is the circle that intersects each of the weight circles orthogonally. \ The
orthocircle does not exist when the common power point is on or inside all
three circles, however if the common power point is at infinity, there is a
line that is orthogonal to the three weight circles which will also be
identified as the orthocircle.

\item[piecewise flat manifold] A triple $\left( M,\mathcal{T},\ell \right) $
where $\left( M,\mathcal{T}\right) $ is a triangulated manifold with
triangulation $\mathcal{T}$ and $\ell :E\rightarrow \mathbb{R}_{+}$ is a
function of the edges such that endowing the edges with lengths $\ell $
gives each simplex the structure of a nondegenerate Euclidean simplex (this
is equivalent to positivity of all relevant Cayley-Menger determinants). We
sometimes call two piecewise flat manifolds $\left( M,\mathcal{T},\ell
\right) $ and $\left( M,\mathcal{T}^{\prime },\ell ^{\prime }\right) $
isometric if they induce the same metric space structure, or if one can be
reached from the other by a sequence of metric Pachner moves (flips).\ 

\item[piecewise flat, metrized manifold] A triple $\left( M,\mathcal{T}%
,d\right) $ where $\left( M,\mathcal{T}\right) $ is a triangulated manifold
and $d:E^{+}\rightarrow \mathbb{R}$ is a function of oriented edges such
that if one defines $\ell _{ij}=d_{ij}+d_{ji}$ then $\left( M,\mathcal{T}%
,\ell \right) $ is a piecewise flat manifold and such that each simplex has
a uniquely determined center.

\item[triangulation] A collection of $n$-dimensional simplices $\mathcal{T}$
together with pairwise identifications for the $\left( n-1\right) $%
-dimensional faces of the simplices. \ More restrictions are needed to
ensure that the resultant space is a manifold. \ Alternatively, given a
space $M$ of dimension $n$, a \textit{triangulation} of $M$ is a subdivision
of $M$ into components $\left\{ \sigma _{i}\right\} $ by (hyper)surfaces (of
dimension $n-1$) so that each component is homeomorphic to an $n$%
-dimensional ball, and each component is combinatorially (as determined by
the subdivisions) equivalent to an $n$-simplex. \ 

\item[pseudo manifold] A discrete \textit{pseudo manifold} is a relaxation
of the manifold conditions for a discrete space. \ For dimensions two and
three, the bullet conditions given in the entry on manifold may no longer
hold. \ However, the manifold condition is still (trivially) satisfied in
the interior of all top dimensional simplices.

\item[weighted triangulation (or hinge)] A triangulation together with a map 
$w:V\rightarrow 
%TCIMACRO{\U{211d} }%
%BeginExpansion
\mathbb{R}
%EndExpansion
$, where $V$ is the set of vertices of the triangulation. \ Each simplex
becomes a decorated simplex. \ 

\item[dimension] A manifold has the property that every point has a
neighborhood homeomorphic to an open subset of $\mathbb{R}^{n}.$ By the
Invariance of Dimension theorem, $n$ must be locally constant (i.e., every
point in a given connected component must have the same dimension). We say $%
n $ is the dimension of the manifold. For a triangulated manifold, the
dimension of the manifold is the dimension of the highest-dimensional
simplices in the triangulation.

\item[metric] Metric can have a number of meanings: (1) a metric can mean a
distance function for a metric space, (2)\ a Riemannian metric is an inner
product at each tangent space which varies smoothly with the basepoint. It
can also be described as a symmetric, positive definite 2-tensor field, (3)
a Lorenzian metric is a symmetric bilinear form at each tangent space with
signature +,-,-,...,- or -,+,...+ which varies smoothly with the basepoint.

\item[scalar curvature] For a Riemannian manifold $\left( M,g\right) $,
scalar curvature is a function $R:M\rightarrow \mathbb{R}$ gotten from the
full trace of the Riemannian curvature tensor. For a piecewise flat,
metrized 3-manifold $\left( M,\mathcal{T},d\right) $, the scalar curvature
measure $K_{i}$ is defined as 
\[
K_{i}=\sum_{j}\frac{K_{ij}}{\ell _{ij}}d_{ij}, 
\]%
where $K_{ij}$ is the edge curvature of edge $\left\{ i,j\right\} .$
Sometimes we refer to the function $i\rightarrow K_{i}/V_{i}$ as the scalar
curvature function (see entry on constant scalar curvature).

\item[constant scalar curvature] For a Riemannian manifold $\left(
M,g\right) $, this means that the scalar curvature function $R:M\rightarrow 
\mathbb{R}$ is a constant function. For a piecewise flat, metrized
3-manifold, this means that the scalar curvature measure is a constant
multiple of the volume measure, i.e., there exists $\lambda \in \mathbb{R}$
such that 
\[
K_{i}=\lambda V_{i}, 
\]%
for all vertices $i,$ where $K_{i}=\sum_{j}\frac{K_{ij}}{\ell _{ij}}d_{ij}$
and $V_{i}=\ell _{ij}\frac{\partial \mathcal{V}}{\partial \ell _{ij}}=\frac{1%
}{3}\sum_{j,k,\ell }h_{ijk,\ell }A_{ijk},$ where $\mathcal{V}$ is the total
volume, $h_{ijk,\ell }$ is the (signed) height of the center to face $%
\left\{ i,j,k\right\} $ and $A_{ijk}$ is the area of face $\left\{
i,j,k\right\} .$

\item[geometric flow] A geometric flow is a differential equation on a
geometry (loosely defined) which depends only on geometric quantities
(usually curvatures). Usual examples are Ricci flow, mean curvature flow,
and Yamabe flow. On discrete geometries, there are discrete Ricci and Yamabe
flows.

\item[Einstein-Hilbert functional] Given a Riemannian manifold $\left(
M,g\right) ,$ the Einstein-Hilbert functional $\mathcal{EH}\left( M,g\right) 
$ is defined by 
\[
\mathcal{EH}\left( M,g\right) =\int_{M}RdV 
\]%
where $R$ is the scalar curvature of $g$ and $dV$ is the volume form of $g.$
The functional is constant on two-dimensional manifolds, giving $4\pi $
times the Euler characteristic. In higher dimensions, its critical points
are Ricci flat (i.e., the Ricci tensor is zero) and critical points
restricted to metrics of a fixed volume are Einstein manifolds.

\item[Einstein-Hilbert-Regge functional] Given a piecewise flat manifold $%
\left( M^{n},\ell \right) $, the Einstein-Hilbert-Regge functional $\mathcal{%
EHR}\left( M,\ell \right) $ is defined by 
\[
\mathcal{EHR}\left( M,\ell \right) =\sum_{\sigma ^{n-2}}K\left( \sigma
^{n-2}\right) , 
\]%
where $K\left( \sigma ^{n-2}\right) $ is $2\pi $ minus the sum of the angles
at $\sigma ^{n-2}$ times the volume $V\left( \sigma ^{n-2}\right) $ of the
simplex $\sigma ^{n-2}.$ In particular, for $n=3,$ we have 
\[
\mathcal{EHR}\left( M^{3},\ell \right) =\sum_{\left\{ i,j\right\}
}K_{ij}=\sum_{\left\{ i,j\right\} }\left( 2\pi -\sum_{k,\ell }\beta
_{ij,k\ell }\right) \ell _{ij}, 
\]%
where $\beta _{ij,k\ell }$ is the dihedral angle at edge $\left\{
i,j\right\} $ in $\left\{ i,j,k,\ell \right\} .$

\item[conformal class] On a smooth Riemannian manifold $\left( M,g\right) ,$
the conformal class $\left[ g\right] $ is the equivalence class of all
Riemannian manifolds $\left( M,e^{f}g\right) ,$ where $f$ is a smooth
function. On a metrized piecewise flat manifold $\left( M,T,d\right) $, the
conformal class $\left[ d\right] $ is the class of $d_{ij}=\frac{\alpha
_{i}r_{i}^{2}+\eta _{ij}r_{i}r_{j}}{\sqrt{\alpha _{i}r_{i}^{2}+\alpha
_{j}r_{j}^{2}+2r_{i}r_{j}\eta _{ij}}}$ for fixed $\alpha _{i}$ and $\eta
_{ij}$ and all $r_{i}$???

\item[Yamabe constant] On a closed manifold $M,$ the Yamabe constant (also
called the $\sigma $-constant) is the number 
\[
\sigma \left( M^{n}\right) =\sup_{g}\inf_{g^{\prime }\in \left[ g\right] }%
\frac{\mathcal{EH}\left( M,g^{\prime }\right) }{V\left( M,g^{\prime }\right)
^{2/n}}, 
\]%
where the $\sup $ is over all metrics and the $\inf $ is over all metrics in
the same conformal class $\left[ g\right] $ as $g.$

\item[flip] A flip of an edge (or hinge) is an action that changes triangles
in the triangulation. The $2-2$ flip in (dimension $2$) switches between two
decompositions of a quadrilateral into two triangles. \ Specifically, the
triangles $\left\{ i,j,k\right\} ,\left\{ i,j,l\right\} $ flip to become $%
\left\{ i,k,l\right\} $ and $\left\{ j,k,l\right\} $. We say that the edge $%
\{i,j\}$ flips to become $\left\{ k,l\right\} .$ \ In general, filps can be
performed in hinges of any dimension, and are called Pachner moves.

\item[Schl\"{a}fli formula] Let $M_{K}^{n+1}$ be the space of constant
curvature $K$ and dimension $n+1\geq 2$. \ Consider a smooth one parameter
family $P\left( t\right) $ of polyhedra in $M^{n+1}$ bounding compact
regions. \ NOT FINISHED

\item[convex hinge] Given a (2D) hinge, the boundary of the hinge forms a
quadrilateral. If all the interior angles of this quadrilateral are less
than or equal to $\pi $ (radians) then the hinge is called a convex hinge.

\item[nonconvex hinge] A hinge which is not convex is a nonconvex hinge. \
The hinge forms a nonconvex set in the plane. These hinges cannot flip
(geometrically) in the usual way.

\item[bone] In a simplicial $n$-manifold, a bone generally refers to a $%
\left( n-2\right) $-dimensional simplex, so in a $3$-manifold, bones consist
of (usually interior) edges. Sometimes bone refers to not only the simplex,
but also all of the $n$-simplices incident on the simplex. This terminology
is mainly used in the Regge calculus literature.

\item[Delaunay triangulation (or hinge)] If all the hinges of a
triangulation are Delaunay hinges, then the triangulation is called a
Delaunay triangulation.

\item[weighted Delaunay triangulation] If all the hinges in a weighted
triangulation are weighted Delaunay hinges, then the weighted triangulation
is called a weighted Delaunay triangulation.

\item[weighted triangulation] Given a piecewise flat triangulation we can
assign a weight to each vertex of the triangulation. Each weight can be
thought of as a radius of a circle around the vertex it is assigned to. A
triangulation with weights assigned to the vertices is called a weighted
triangulation. \ Note that each simplex in the triangulation is a decorated
simplex. \ The weights together with the piecewise-flat structure on a
manifold determine a piecewise-flat, metrized manifold.

\item[weighted Voronoi diagram (or cell)] A weighted Voronoi diagram is all
of the dual cells of a weighted Delaunay triangulation.

\item[combinatorial Ricci flow] A partial differential equation on radii of
a piecewise flat ????decorated two-dimensional manifold, originally due to
Chow-Luo. In the Euclidean (background) case, the system is $\frac{dr_{i}}{dt%
}=-K_{i}\ast r_{i}$. Often, we use a normalized version, $\frac{dr_{i}}{dt}=(%
\bar{K}-K_{i})\ast r_{i}$ where $\bar{K}$ is the average curvature (which is
a topological constant related to the Euler characteristic).

\item[Newton's method] A method for finding the zeroes of a function. For
real-valued functions on $\mathbb{R}$, the n-th step of Newton's Method
approximates a zero from the previous step as $x_{n}=x_{n-1}-\frac{f(x_{n-1})%
}{f^{\prime }(x_{n-1})}$. That is, we determine the equation of the tangent
line at $x_{n-1}$, then set $x_{n}$ to be the x-intercept of this line.%
\newline

\item We often want to optimize real-valued functions on $\mathbb{R}^{m}$,
so we want to use Newton's Method to find the points at which $Df(\mathbf{x}%
) $, the gradient of a function $f$, is the zero-vector. We therefore need
the second derivative, $D^{2}f(\mathbf{x})$, called the Hessian which is an $%
m$ by $m$ matrix. So a step of the method yields $\mathbf{x_{n}}=\mathbf{%
x_{n-1}}-[D^{2}f(\mathbf{x_{n-1}})]^{-1}\ast Df(\mathbf{x_{n-1}})$. Note: we
generally do not compute the inverse matrix to solve, but use another method
such as Gaussian elimination.

\item[Runge-Kutta method] A numerical approximation method for a system of
ordinary differential equations. It is considered superior to Euler's Method
as the size of its error term is orders of magnitude smaller. \ In
particular, this method utilizes a fourth order approximation. \ Given a
differential equation $x^{\prime }=f(t,x)$, a point $(t_{n},x_{n})$, and a
stepsize $h$ the next step $(t_{n+1},x_{n+1})$ is calculated as follows:
\end{description}

\begin{enumerate}
\item Set $k_{1}=f(t_{n},x_{n})$

\item Set $k_{2}=f(t_{n}+\frac{h}{2},x_{n}+\frac{h\ast k_{1}}{2})$

\item Set $k_{3}=f(t_{n}+\frac{h}{2},x_{n}+\frac{h\ast k_{2}}{2})$

\item Set $k_{4}=f(t_{n}+h,x_{n}+h\ast k_{3})$
\end{enumerate}

\noindent Then $t_{n+1}=t_{n}+h$ and $x_{n+1}=x_{n}+\frac{h}{6}(k_{1}+2\ast
k_{2}+2\ast k_{3}+k_{4}).$

\begin{description}
\item[dual edge] Given a triangle $\{i,j,k\}$ in a triangulation with no
weights, we consider the perpendicular bisector of each edge. These
bisectors intersect at a single point. We call the line segment that extends
from an edge to this point of intersection, the dual of that edge.

\item[conformal deformation] In Riemannian geometry, a conformal deformation
of a Riemannian manifold $\left( M,g\right) $ is a family of metrics $%
g\left( t\right) ,$ where $t\in \left( -\varepsilon ,\varepsilon \right) $
for some $\varepsilon >0,$ such that $g\left( 0\right) =0$ and $\frac{%
\partial }{\partial t}g=fg,$ where $f:M\rightarrow \mathbb{R}_{+}$ is a
(smooth) positive real-valued function. In piecewise Euclidean geometry,
there are a number of different definitions of conformal deformation; we
take the viewpoint that a conformal deformation is a family of functions $%
f\left( t\right) :V\rightarrow \mathbb{R}$ and a description of the lengths
in terms of variables $f_{i}$ at the vertices such that $\ell _{ij}=\ell
_{ij}\left( f_{i},f_{j}\right) ,$ $\frac{\partial }{\partial f_{i}}\ell
_{ij}=d_{ij},$ and for any simplex $\left\{ i,j,k\right\} ,$ $%
d_{ij}^{2}+d_{jk}^{2}+d_{ki}^{2}=d_{ji}^{2}+d_{ik}^{2}+d_{kj}^{2}$ (i.e.,
every simplex has a center, and $d_{ij}$ is the length of the segment from
vertex $i$ to the projection of the center onto the edge $\left\{
i,j\right\} $). One example of a conformal deformation is the Yamabe flow
(and discrete Yamabe flow).

\item[negative triangle] When one tries to flip a nonconvex hinge, the
nonconvex quadrilateral is decomposed into one large triangle and one
triangle which should be cut out. The triangle to be cut out is referred to
as a \textquotedblleft negative triangle.\textquotedblright\ Negative
triangles have a number of properties which contrast with usual
(\textquotedblleft positive\textquotedblright ) triangles. When computing
area, the negative triangle's area should be subtracted. Also, when a
manifold is developed (unfolded in the plane), when switching from a
positive to negative (or vice versa) triangle, the triangle should be folded
backward along the edge adjacent to the two triangles.

\item[circle and sphere packing] Circle packing is an arrangement of circles
such that no two circles overlap and all circles are mutually tangent to one
another. \ In this instance circle packing can be used to assign lengths to
triangle edges. \ Each vertex is the center of a circle, providing each
vertex with a corresponding radius. \ The lengths of the edges are
determined by 
\[
l_{ij}=r_{i}+r_{j} 
\]%
for mutually tangent vertices $i,j$. \ Circle packing guarantees that the
lengths satisfy the triangle inequality. \ Similarly, a sphere packing is an
arrangement of spheres such that adjacent spheres are tangent and the
tangency graph forms a triangulation. Each vertex functions as the origin of
a sphere, providing each vertex with a corresponding radius as determined by
sphere packing. \ The radii are used to determine edge lengths. \ Sphere
packing does not guarantee that the tetrahedra formed in the tangency graph
have positive volume.\ \ The astute reader will notice that circle and
sphere packings correspond to conformal structures with $\alpha _{i}=\eta
_{ij}=1$ for all $i$.

\item[Euclidean law of cosines] A triangle with angles $A$, $B$, $C$ has
edges directly opposite these angles with lengths $a$, $b$, $c$
respectively. The Euclidean Law of Cosines states that 
\[
c^{2}=a^{2}+b^{2}-2ab\cos (C) 
\]%
which, rearranged is 
\[
\cos (C)=(a^{2}+b^{2})/2ab. 
\]%
Since cosine is a bounded function the triangle inequality is violated when 
\[
(a^{2}+b^{2})/2ab>1 
\]%
or when 
\[
(a^{2}+b^{2})/2ab<-1. 
\]

\item[spherical law of cosines] Given a unit sphere the edges and vertices
of a triangle are formed by great circles intersecting on the surface. In
this instance the angle $\gamma $ is directly opposite the edge of length $C$%
. The other edges are of lengths $A$ and $B$ respectively. The spherical law
of cosines states that 
\[
\cos (C)=\cos (A)\cos (B)+\sin (A)\sin (B)\cos (\gamma ). 
\]
\end{description}

\bigskip

\chapter{Quantities}

\bigskip

$r_{i}$ (radii, vertex weights)

$\eta _{ij}$ (etas, inversive distance, edge weights)

$\alpha _{i}$ (alphas, vertex weight coefficient)

$d_{ij}$ partial edge

$l_{ij}$ length%
\begin{eqnarray*}
l_{ij} &=&\sqrt{\alpha _{i}r_{i}^{2}+\alpha _{j}r_{j}^{2}+2r_{i}r_{j}\eta
_{ij}} \\
&=&d_{ij}+d_{ji}
\end{eqnarray*}

$A_{ijk}$ face area

$V_{ijkl}$ tetrahedron volume

$\gamma _{i,jk}$ face angle

$\beta _{ij,kl}$ dihedral angle (at an edge in a tetrahedron)

$\alpha _{i,jkl}$ solid angle

$h_{ij,k}$ edge height%
\[
h_{ij,k}=\frac{d_{ik}-d_{ij}\cos \gamma _{i,jk}}{\sin \gamma _{i,jk}} 
\]

$h_{ijk,l}$ face height%
\[
h_{ijk,l}=\frac{h_{ij,l}-h_{ij,k}\cos \beta _{ij,kl}}{\sin \beta _{ij,kl}} 
\]

$A_{ij,kl}$ dual area (of an edge in a tetrahedron)%
\[
A_{ij,kl}=\frac{1}{2}\left( h_{ij,k}h_{ijk,l}+h_{ij,l}h_{ijl,k}\right) 
\]

$l_{ij}^{\ast }$ dual length (2D) or dual area (3D) 
\begin{eqnarray*}
2D &:&l_{ij}^{\ast }=h_{ij,k}+h_{ij,l} \\
3D &:&l_{ij}^{\ast }=\sum_{\substack{ k,l\text{ s.t.}  \\ \left\{
i,j,k,l\right\} \in \mathcal{T}}}A_{ij,kl}
\end{eqnarray*}

$K_{i}$ vertex curvature (scalar curvature)%
\begin{eqnarray*}
2D &:&K_{i}=2\pi -\sum_{\substack{ j,k\text{ s.t.}  \\ \left\{ i,j,k\right\}
\in \mathcal{T}}}\gamma _{i,jk} \\
3D &:&K_{i}=\sum_{\substack{ j\text{ s.t.}  \\ \left\{ i,j\right\} \in E}}%
\left( 2\pi -\sum_{\substack{ k,l\text{ s.t.}  \\ \left\{ i,j,k,l\right\}
\in \mathcal{T}}}\beta _{ij,kl}\right) d_{ij}
\end{eqnarray*}

$K_{ij}$ edge curvature (3D)%
\[
K_{ij}=\left( 2\pi -\sum_{\substack{ k,l\text{ s.t.}  \\ \left\{
i,j,k,l\right\} \in \mathcal{T}}}\beta _{ij,kl}\right) l_{ij} 
\]

$\lambda $ Einstein Constant%
\[
\lambda =\frac{EHR\left( M,\mathcal{T},l\right) }{3\mathcal{V}\left( M,%
\mathcal{T},l\right) } 
\]

$EHR\left( M,\mathcal{T},l\right) $ (3D)%
\begin{eqnarray*}
EHR\left( M,\mathcal{T},l\right) &=&\sum_{i}K_{i} \\
&=&\sum_{\substack{ i,j\text{ s.t.}  \\ \left\{ i,j\right\} \in E}}K_{ij}
\end{eqnarray*}

$V_{i}$ dual volume of a vertex%
\[
V_{i}=\sum_{\substack{ j,k,l\text{ s.t.}  \\ \left\{ i,j,k,l\right\} \in 
\mathcal{T}}}h_{ijk,l}A_{ijk} 
\]

$\mathcal{V}$ total volume

$c_{ijk}$ center of a weighted triangle (embedded)

$c_{ijkl}$ center of a weighted tetrahedron (embedded)
