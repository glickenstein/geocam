%TCIDATA{Version=5.00.0.2606}
%TCIDATA{LaTeXparent=1,1,functions.tex}
                      

\section*{\texttt{CurvaturePartial::CurvaturePartial}}

\subsection*{Function Prototype}

\texttt{double CurvaturePartial( Vertex v\_i, Vertex v\_l )}

\subsection*{Key Words}

Curvature, Einstein-Hilbert-Regge, functional, partial derivative, geoquant.

\subsection*{Authors}

Daniel Champion

\subsection*{Introduction}

CurvaturePartial calculates the partial derivative of the curvature at a
vertex with respect to the log radius of another (possibly the same) vertex.
\ 

\subsection*{Subsidiaries}

Functions:

\qquad\texttt{isAdjVertex}

\qquad \texttt{DualArea}

\qquad \qquad \texttt{DualAreaSegment}

\qquad \qquad \qquad \texttt{FaceHeight}

\qquad \qquad \qquad \qquad \texttt{EdgeHeight}

\qquad \qquad \qquad \qquad \qquad \texttt{PartialEdge}

\qquad\texttt{listDifference}

\qquad\texttt{listIntersection}

Global Variables: \ curvature, dihedralAngle, eta, length, radius

Local Variables: none.

\subsection*{Description}

\texttt{CurvaturePartial} receives as inputs two vertices $v_{i}$ and $v_{l}$%
. \ The first corresponds to the vertex of interest, the second corresponds
to the vertex of differentiation. \ That is,%
\begin{equation*}
\text{\texttt{CurvaturePartial (v\_i, v\_l)}}=\frac{\partial }{\partial \log
r_{l}}K_{i},
\end{equation*}%
where $r_{l}$ is the radius at vertex $v_{l}$, and $K_{i}$ is the curvature
at vertex $v_{i}$. \ 

The function begins implementation by determining the relationship between $i
$ and $l$ via the trichotomy $v_{i}=v_{l}$, $v_{i}$ is adjacent to $v_{l}$,
or $v_{i}$ and $v_{l}$ are not endpoints of any edge. \ Each of the three
cases are calculated differently. \ The general formula for the variation of
curvature w.r.t. log radii was calculated by Prof. David Glickenstein and is
available at arXiv:0906.1560v1:%
\begin{equation*}
\delta K_{i}=-\sum_{edges\text{ }\left\{ i,j\right\} }\left( 2\frac{%
l_{ij}^{\ast }}{l_{ij}}-\frac{r_{i}^{2}r_{j}^{2}\left( 1-\eta
_{ij}^{2}\right) }{l_{ij}^{2}}K_{ij}\right) \left( \delta f_{j}-\delta
f_{i}\right) +K_{i}\delta f_{i},
\end{equation*}%
where $f_{i}=\log r_{i}$, $l_{ij}$ is the length of the edge $\left\{
i,j\right\} $, and $l_{ij}^{\ast }$ is the dual area calculated with the
function \texttt{DualArea}.

When $i=l$, the formula for the partial derivative $\frac{\partial}{%
\partial\log r_{l}}K_{i}$ becomes:%
\begin{equation*}
\frac{\partial}{\partial\log r_{i}}K_{i}=\sum_{edges\text{ }\left\{
i,j\right\} }\left( 2\frac{l_{ij}^{\ast}}{l_{ij}}-\frac{r_{i}^{2}r_{j}^{2}%
\left( 1-\eta_{ij}^{2}\right) }{l_{ij}^{2}}K_{ij}\right) +K_{i}.
\end{equation*}

When $v_{i}$ is adjacent to $v_{l}$ only one term in the sum survives:%
\begin{equation*}
\frac{\partial}{\partial\log r_{l}}K_{i}=-\left( 2\frac{l_{il}^{\ast}}{l_{il}%
}-\frac{r_{i}^{2}r_{l}^{2}\left( 1-\eta_{il}^{2}\right) }{l_{il}^{2}}%
K_{il}\right) .
\end{equation*}

When $v_{i}$ and $v_{j}$ are not endpoints of any edge the partial
derivative is zero. \ 

This function was created to assist in the computation of the second
derivatives of the normalized Einstein-Hilbert-Regge functional:%
\begin{equation*}
EHR=\frac{\sum K_{i}}{\sqrt[3]{\sum\limits_{tetra\text{ }t}Vol(t)}}.
\end{equation*}

Surprisingly the first derivatives of the normalized $EHR$ functional do not
require the formula for the partial derivative of curvature since 
\begin{equation*}
\frac{\partial EHR}{\partial\log r_{i}}=K_{i}.
\end{equation*}

However, the second order partial derivatives of $EHR$ certainly require the
formulas for $\frac{\partial}{\partial\log r_{l}}K_{i}$ given above. \ These
second order partial derivatives are used to construct a Hessian matrix
which is then used the optimization of the $EHR$ functional using Newton's
method (implemented by \texttt{Newtons\_Method}).

\subsection*{Practicum}

When called, \texttt{CurvaturePartial(v\_i, v\_l)} returns the partial
derivative $\frac{\partial }{\partial \log r_{l}}K_{i}$. \ An example of its
usage is in the calculation of the second order partial derivatives of the
normalized $EHR$ functional. \ For this example let 
\begin{align*}
\text{\texttt{VolSumPartial\_i}}& =\sum_{tetra\text{ }t}\frac{\partial V_{t}%
}{\partial \log r_{i}}, \\
\text{\texttt{VolSumPartial\_j}}& =\sum_{tetra\text{ }t}\frac{\partial V_{t}%
}{\partial \log r_{j}}, \\
\text{\texttt{VolSumSecondPartial}}& =\sum_{tetra\text{ }t}\frac{\partial
^{2}V_{t}}{\partial \log r_{i}\partial \log r_{j}} \\
K& =\sum_{i}K_{i}, \\
V& =\sum_{tetra\text{ }t}V_{t}.
\end{align*}%
Then the second order partial derivative $\frac{\partial ^{2}EHR}{\partial
\log r_{i}\partial \log r_{j}}$ is calculated by:

\bigskip

\qquad \texttt{result = pow(V,
(-4.0/3.0))*(1.0/3.0)*(3*V*CurvaturePartial(i,j)}

\qquad\qquad\qquad \texttt{%
-Geometry::curvature(Triangulation::vertexTable[i])*VolSumPartial\_j}

\qquad\qquad\qquad \texttt{%
-Geometry::curvature(Triangulation::vertexTable[j])*VolSumPartial\_i}

\qquad\qquad\qquad\texttt{+(4.0/3.0)*K*pow(V,
-1.0)*VolSumPartial\_i*VolSumPartial\_j}

\qquad\qquad\qquad\texttt{-K*VolSumSecondPartial);}

\bigskip

\subsection*{Limitations}

The function \texttt{CurvaturePartial} is operational for all pairs of input
integers $i$ and $l$ that are in the vertex table. \ If one of the arguments
is not in the vertex table, the function will output zero. \ 

\subsection*{Revisions}

subversion 757, 7/6/09, \texttt{CurvaturePartial} created.

subversion 1055, 3/12/10, \texttt{CurvaturePartial}\ converted to a geoquant.

\subsection*{Testing}

This function has not been tested.

\subsection*{Future Work}

Using the calculation of the partial derivative of curvature in Mathematica,
it should be compared to the output from \texttt{CurvaturePartial}.
