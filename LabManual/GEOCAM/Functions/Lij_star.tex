%TCIDATA{Version=5.00.0.2606}
%TCIDATA{LaTeXparent=1,1,functions.tex}
                      

\section*{\texttt{DualArea::DualArea}}

\subsection*{Function Prototype}

\texttt{double DualArea(Edge e)}

\subsection*{Key Words}

Dual area, curvature, partial derivative, edge, geoquant.

\subsection*{Authors}

Daniel Champion

\subsection*{Introduction}

\texttt{DualArea} calculates the dual area of an edge.

\subsection*{Subsidiaries}

\subsubsection*{Functions: \ }

\qquad \texttt{DualAreaSegment}

\qquad \qquad \texttt{FaceHeight}

\qquad \qquad \qquad \texttt{EdgeHeight}

\qquad \qquad \qquad \qquad \texttt{PartialEdge}

\subsubsection*{Global Variables: \ }

\qquad Only radii and eta values are needed.

\subsubsection*{Local Variables: \ }

\qquad none

\subsection*{Description}

\texttt{DualArea} is defined as:%
\begin{equation*}
\text{\texttt{DualArea(edge e\_ij))}}=l_{ij}^{\ast }=\sum_{\substack{ \text{%
all tetrahedra }(i,j,k,l) \\ \text{containing edge }(i,j)\text{.}}}A_{ij,kl},
\end{equation*}%
where $A_{ij,kl}$ is computed with the function \texttt{DualAreaSegment}
applied to the vertices of the tetrahedron being summed over. \ Note that 
\texttt{DualAreaSegment} utilizes the functions \texttt{FaceHeight}, \texttt{%
EdgeHeight}, and \texttt{partialEdge}, however only the radii and eta values
are needed to calculate all of these quantities. \ 

This function was created for use in the \texttt{CurvaturePartial} function
which serves an essential role in calculating the second derivatives of the
Einstein-Hilbert-Regge functional (\texttt{EHRSecondPartial}). \ The second
order partial derivatives of the EHR functional are used in the optimization
of the EHR functional using Newton's method. \ \texttt{DualArea} will
eventually be used in the study of laplacians. \ 

\subsection*{Practicum}

Currently \texttt{DualArea} is only used to calculate the partial derivative
of curvature with respect to $\log $ radius. \ The following example
calculates the partial derivative of the curvature at vertex V with respect
to the $\log $ radius $r_{l}$ corresponding to vertex Vprime (adjacent to V).

\bigskip

\qquad\texttt{double sum = 0.0;}

\qquad\texttt{double dihedral\_sum = 0.0;}

\qquad\texttt{Vprime = Triangulation::vertexTable[l];}

\qquad\texttt{E =
Triangulation::edgeTable[listIntersection(V.getLocalEdges(),}

\qquad\qquad\texttt{Vprime.getLocalEdges())[0]];}

\qquad\texttt{// This assumes that there is a unique edge between two
vertices.}

\qquad\texttt{local\_tetra = E.getLocalTetras();}

\qquad\texttt{for (int m=0; m \TEXTsymbol{<} (*(local\_tetra)).size(); ++m)
\{}

\qquad\qquad\texttt{T = Triangulation::tetraTable[local\_tetra-\TEXTsymbol{>}%
at(m)];}

\qquad\qquad\texttt{dihedral\_sum += Geometry::dihedralAngle(E,T);}

\qquad\texttt{\}}

\qquad \texttt{result =
DualArea(E)/(Geometry::Length(E))-(2*PI-dihedral\_sum)}

\qquad \qquad \texttt{*(pow(Geometry::Radius(V),
2)*pow(Geometry::Radius(Vprime),2)}

\qquad \qquad \texttt{%
*(1-pow(Geometry::Eta(E),2)))/pow(Geometry::Length(E),3);}

\subsection*{Limitations}

\texttt{DualArea} can operate on any and all edges of a 3D triangulation
however it is only appropriate for triangulations where tetrahedra have
distinct edges. \ 

\subsection*{Revisions}

Subversion 676, 5/15/09, \texttt{DualArea} created within \texttt{%
Newtons\_Method}.

subversion 1055, 3/12/10, \texttt{DualArea}\ converted to a geoquant.

\subsection*{Testing}

\texttt{Lij\_star} has not been tested. \ 

\subsection*{Future Work}

\texttt{Lij\_star} should be moved to a more appropriate section of the
code. \ A more general volume function should be created that would take any
simplex object (including a boolean for dual simplices) and return the
appropriate volume. \ This general volume function would be an excellent
location for \texttt{Lij\_star}. \ It should be tested some time as well. \ 
