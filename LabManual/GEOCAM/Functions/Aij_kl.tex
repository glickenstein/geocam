%TCIDATA{Version=5.00.0.2606}
%TCIDATA{LaTeXparent=1,1,functions.tex}
                      

\section*{\texttt{DualAreaSegment::DualAreaSegment}}

\subsection*{Function Prototype}

\texttt{double DualAreaSegment( Vertex vi, Vertex vj, Vertex vk, Vertex vl)}

\subsection*{Key Words}

Dual area, curvature, partial derivative, Einstein-Hilbert-Regge, geoquant.

\subsection*{Authors}

Daniel Champion

\subsection*{Introduction}

\texttt{DualAreaSegment} calculates the dual area to an edge of a
tetrahedron. \ 

\subsection*{Subsidiaries}

\textbf{Functions:}

\qquad \texttt{EdgeHeight}

\qquad \qquad \texttt{PartialEdge}

\qquad\qquad\texttt{Geometry::angle}

\qquad \texttt{FaceHeight}

\qquad\qquad\texttt{Geometry::dihedralAngle}

\textbf{Global Variables: \ }radii, etas

\textbf{Local Variables:} \ none.

\subsection*{Description}

\texttt{DualAreaSegment} is calculated with the formula:%
\begin{equation*}
\text{\texttt{DualAreaSegment(vi, vj, vk, vl)}}=
\end{equation*}%
\begin{equation*}
\frac{1}{2}\left( 
\begin{array}{c}
\text{\texttt{EdgeHeight(vi,vj,vk)}}\cdot \text{\texttt{%
FaceHeight(vi,vj,vk,vl)}} \\ 
+\text{\texttt{EdgeHeight(vi,vj,vl)}}\cdot \text{\texttt{%
FaceHeight(vi,vj,vl,vk)}}%
\end{array}%
\right) 
\end{equation*}%
\texttt{EdgeHeight} and \texttt{FaceHeight} are calculated with the
following formulae:%
\begin{align*}
\text{\texttt{EdgeHeight(vi, vj, vk)}}& =\frac{\left( \text{\texttt{%
PartialEdge(vi,vk)}}-\text{\texttt{PartialEdge(vi,vj)}}\cos \left( \alpha
_{i,jk}\right) \right) }{\sin \left( \alpha _{i,jk}\right) } \\
\text{\texttt{FaceHeight(vi, vj, vk, vl)}}& =\frac{\left( \text{\texttt{%
EdgeHeight(vi,vj,vl)}}-\text{\texttt{EdgeHeight(vi,vj,vk)}}\cos (\beta
_{ij,kl})\right) }{\sin \left( \beta _{ij,kl}\right) }
\end{align*}%
where $\alpha _{i,jk}$ is the angle at vertex $vi$ of triangle $\left\{
vi,vj,vk\right\} $, and $\beta _{ij,kl}$ is the dihedral angle along edge $%
\left\{ vi,vj\right\} $ of tetrahedron $\left\{ vi,vj,vk,vl\right\} $
(implemented with the functions \texttt{Geometry::angle} and \texttt{%
Geometry::dihedralAngle} respectively).

\texttt{DualAreaSegment} was created for the calculation performed in the
function \texttt{DualArea}, which is used in the computation of the partial
derivatives of curvature. \ These partial derivatives of curvature are used
in the calculation of the second order partial derivatives of the
Einstein-Hilbert-Regge functional for use in the optimization of said
functional using Newton's method. \ 

\subsection*{Practicum}

As an example of the usage of this function, we will calculate the dual area
to the edge $eij=\left\{ vi,vj\right\} $ (see entry: \texttt{DualArea}). \
To do this, we will sum the dual areas to each tetrahedron containing the
edge $eij$. \ 

\bigskip

\texttt{vector\TEXTsymbol{<}int\TEXTsymbol{>} sum\_over =
*(eij.getLocalTetras());}

\texttt{double sum = 0.0;}

\texttt{vector\TEXTsymbol{<}int\TEXTsymbol{>} T\_vertices, e\_vertices;}

\texttt{Tetra T;}

\texttt{Vertex vi,vj,vk,vl;}

\texttt{for(i=0; i\TEXTsymbol{<}sum\_over.size(); ++i) \{}

\qquad\texttt{T = Triangulation::tetraTable[sum\_over[i]];}

\qquad\texttt{T\_vertices = *(T.getLocalVertices());}

\qquad\texttt{e\_vertices = *(eij.getLocalVertices());}

\qquad\texttt{vi = Triangulation::vertexTable[e\_vertices[0]];}

\qquad\texttt{vj = Triangulation::vertexTable[e\_vertices[1]];}

\qquad\texttt{vk = Triangulation::vertexTable[listDifference(\&T\_vertices,
\&e\_vertices)[0]];}

\qquad\texttt{vl = Triangulation::vertexTable[listDifference(\&T\_vertices,
\&e\_vertices)[1]];}

\qquad \texttt{sum += DualAreaSegment(vi, vj, vk, vl);}

\qquad\texttt{\}}

\texttt{return sum;}

\subsection*{Limitations}

\texttt{DualAreaSegment} if fully operational and has no known limitations.
\ The function will output appropriate values provided it receives as input
four distinct vertices that define a tetrahedron.

\subsection*{Revisions}

subversion 757, 7/8/09, \texttt{DualAreaSegment} created.

subversion 1055, 3/12/10, \texttt{DualAreaSegment}\ converted to a geoquant.

\subsection*{Testing}

Trials were run and the calculations returned were verified by hand.

\subsection*{Future Work}

No future work planned.
