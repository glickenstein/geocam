%TCIDATA{Version=5.00.0.2552}
%TCIDATA{LaTeXparent=0,0,geocam.tex}
                      
%TCIDATA{ChildDefaults=chapter:1,page:1}


\chapter{Geoquants}

\section{Introduction}

Geoquants were developed by Alex Henniges, Joeseph Thomas, and Kurtis
Norwood during the spring of 2009 as the primary component in streamlining
the execution of the code during extended optimization routines. The purpose
is to provide easy access to calculations of geometric quantities without
requiring extraneous calculations of those quantities while they remain the
same.\ Geoquants achieve this through the use of an invalidation tree to
prevent uneccesary calcuations.\bigskip

\section{General Properties}

Each geoquant contains a calculation of a quantity, invalidation
information, and input/output functionality. \ Some of these components are
specific to each geoquant while some are common to all geoquants.

\subsubsection{Calculation}

Each geoquant has a unique calculation yielding a geometric quantity. \ In
general there are no commonalities among the calculations except that all
geoquants output a quantity of geometric nature. \ 

\subsubsection{Input/Output}

Geoquants recieve as input topological data (vertex, face, tetrahedron) \
which serves as a unique identifier among all such quantities of the same
type across the triangulation. The topological data is composed of zero or
more of each dimension of simplex. The following functions are available to
all geoquants:

\begin{itemize}
\item ExampleGeoQuant::At(input)

At returns an instance of the ExampleGeoQuant identified by input.

\item ExampleGeoquant::valueAt(input)

valueAt returns the value of the geoquant identified by input as a
double-precision number.

\item getValue()

getValue returns the value of an already obtained geoquant.

\item setValue(value)

\ setValue will set the value of this geoquant to the specified value and
subsequently invalidate all dependent geoquants.
\end{itemize}

The functions getValue and setValue are called as:

\texttt{Q=ExampleGeoquant::At(input)}

\texttt{Q-\TEXTsymbol{>}getValue}

\texttt{Q-\TEXTsymbol{>}setValue}

\subsubsection{Invalidation}

The value of a geoquant is obtained in one of two ways: it is set by the
user or it is calculated from other geoquants. In the situation where a
geoquant B is calculated in part from the value of geoquant A, we refer to A
as the parent and B as the dependent. When the value of a parent changes,
the parent notifies all of its dependents that their own value is no longer
valid. Once a request for the value of an invalid geoquant is made, that
geoquant requests the value of each of its parents in order to perform the
recalculation of its own value.

\section{Geoquant Catalog}

\subsection{Alpha}

\paragraph{Input}

Alpha(Vertex v)

\paragraph{Calculation}

This geoquant is never calculated, only set by the user. Its value is
generally 0 or 1.

\paragraph{Invalidation}

Parents: None

\subsection{Area}

\paragraph{Input}

Area( Face f )

\paragraph{Calculation}

Area returns the area of a triangular face (in a triangulation). \ The
calculation is done using Heron's formula which takes as inputs the lengths
of the sides of the triangle.%
\begin{equation*}
A=\sqrt{s\left( s-a\right) \left( s-b\right) \left( s-c\right) }
\end{equation*}%
where%
\begin{equation*}
s=\frac{a+b+c}{2}.
\end{equation*}

\paragraph{Invalidation}

Parents: \ all lengths of edges in the input triangle.

\subsection{CurvaturePartial}

\paragraph{Input}

\begin{itemize}
\item CurvaturePartial( Vertex v, Vertex w )

\item CurvaturePartial( Vertex v, Edge e )
\end{itemize}

\paragraph{Calculation}

\begin{itemize}
\item ( Vertex v, Vertex w ): Returns the following partial derivative:%
\begin{equation*}
CurvaturePartial(v,w)=\frac{\partial K_{v}}{\partial \log r_{w}}.
\end{equation*}%
The calculation of $CurvaturePartial(v,w)$ differs depending on the
combintorial relationship between $v$ and $w$. \ See \textbf{Functions: }%
\texttt{CurvaturePartial::CurvaturePartial} for more information.

\item ( Vertex v, Edge e ): Returns the following partial derivative: 
\begin{equation*}
CurvaturePartial(v,e)=\frac{\partial K_{v}}{\partial \eta _{e}}
\end{equation*}
\end{itemize}

\paragraph{Invalidation}

Parents: Radius of v and all radii of vertices adjacent to both v and w, the
curvature at v, and the dual areas, dihedral angles, and eta of every edge
adjacent to both v and w.

\subsection{Curvature2D}

\paragraph{Input}

Curvature2D( Vertex v )

\paragraph{Calculation}

Curvature2D calculates the discrete curvature at a vertex in a two
dimensional manifold. \ The calculation is done by:%
\begin{equation*}
Curvature2D\left( v\right) =2\pi -\sum\limits_{\substack{ all\text{ }%
triangles\text{ }t  \\ incident\text{ }to\text{ }v}}\theta _{v,t}
\end{equation*}%
where $\theta _{v,t}$ is the angle at vertex $v$ in triangle $t$. \ 

\paragraph{Invalidation}

\bigskip Parents: Every angle incident on vertex v.

\subsection{Curvature3D}

\paragraph{Input}

Curvature3D( Vertex v )

\paragraph{Calculation}

Curvature3D calculates the curvature at a vertex in a three dimensional
manifold. \ If we let $K_{ij}$ be the secional curvature at an edge $\left\{
i,j\right\} $ and let vertex $v$ correspond to $i$, then%
\begin{equation*}
Curvature3D\left( v\right) =\sum\limits_{\substack{ j\text{ s.t. }\left\{
i,j\right\}  \\ \text{is an edge}}}K_{ij}d_{ij}.
\end{equation*}

\paragraph{Invalidation}

Parents: For every edge of v, the sectional curvature of that edge and its
partial edge with respect to v.\bigskip

\subsection{DihedralAngle}

\paragraph{Input}

DihedralAngle( Edge e, Tetra t )

\paragraph{Calculation}

The dihedral angle of an edge in a tetrahedron is calculated using the
spherical law of cosines. \ Given a geometrically determined tetrahedron,
the face angles can be computed. \ At a vertex of the tetrahedron, the face
angles correspond to the sides lengths of a spherical triangle and the
dihedral angles correspond to the angles of the spherical triangle. \ If we
denote the dihedral angle at edge $e=\left\{ i,j\right\} $ in tetrahedron $%
t=\left\{ i,j,k,l\right\} $ by $\beta _{ij,kl}$, and denote the face angle
at vertex $v=i$ of triangle $\left\{ i,j,k\right\} $ by $\gamma _{i,jk}$ we
have that:%
\begin{equation*}
\beta _{ij,kl}=\arccos \left( \frac{\cos \left( \gamma _{i,kl}\right) -\cos
\left( \gamma _{i,jk}\right) \cos \left( \gamma _{i,jl}\right) }{\sin \left(
\gamma _{i,jk}\right) \sin \left( \gamma _{i,jl}\right) }\right)
\end{equation*}

\paragraph{Invalidation}

\bigskip Parents: The face angles incident on vertex $v=i$ and with face $%
f\in t$.

\section{DihedralAngleSum}

\paragraph{Input}

DihedralAngleSum( Edge e )

\paragraph{Calculation}

This geoquant sums every dihedral angle incident on the given edge.

\paragraph{Invalidation}

\bigskip Parents: For each tetrahedron containing e, the dihedral angle of
the tetrahedron incident on e.

\subsection{DualArea}

\paragraph{Input}

DualArea( Edge e )

\paragraph{Calculation}

See \textbf{Functions: }\texttt{DualArea::DualArea} for more information.

\paragraph{Invalidation}

Parents: The DualAreaSegments involving this edge across every tetrahedron
local to e.

\bigskip

\subsection{DualAreaSegment}

\paragraph{Input}

DualAreaSegment( Edge e, Tetra t )

\paragraph{Calculation}

See \textbf{Functions: }\texttt{DualAreaSegment::DualAreaSegment} for more
information.

\paragraph{Invalidation}

Parents: For both faces in t that contain e, the EdgeHeights between those
faces and e, and the FaceHeights between those faces and t.

\bigskip

\subsection{EdgeHeight}

\paragraph{Input}

EdgeHeight( Edge e, Face f )

\paragraph{Calculation}

See \textbf{Functions: }\texttt{EdgeHeight::EdgeHeight} for more information.

\paragraph{Invalidation}

Parents: For one of the two vertices, v, of e, the EuclideanAngle of f
incident on v, and the PartialEdges of the edges in f containing v, incident
on v.

\bigskip

\subsection{NEHRPartial}

\paragraph{Input}

\begin{itemize}
\item NEHRPartial( Vertex v )

\item NEHRPartial( Edge e )
\end{itemize}

\paragraph{Calculation}

\begin{itemize}
\item ( Vertex v ): Returns the following partial derivative: 
\begin{equation*}
NEHRPartial(v)=\frac{\partial NEHR}{\partial r_{v}}
\end{equation*}%
See \textbf{Functions: }\texttt{EHRPartial::EHRPartial} for more information.

\item ( Edge e ): Returns the following partial derivative: 
\begin{equation*}
NEHRPartial(e)=\frac{\partial NEHR}{\partial \eta _{e}}
\end{equation*}
\end{itemize}

\paragraph{Invalidation}

Parents: Both versions use TotalVolume, TotalCurvature, and the
TotalVolumePartial of vertex v (edge e). The vertex version also depends on
the Curvature3D at v. The edge version depends on all CurvaturePartials of e.

\bigskip

\subsection{NEHRSecondPartial}

\paragraph{Input}

\begin{itemize}
\item NEHRSecondPartial( Vertex v, Vertex w )

\item NEHRSecondPartial( Vertex v, Edge e )
\end{itemize}

\paragraph{Calculation}

\begin{itemize}
\item ( Vertex v, Vertex w ): Returns the following second-order partial
derivative: 
\begin{equation*}
NEHRSecPartial(v,w)=\frac{\partial NEHR}{\partial r_{w}\partial r_{v}}
\end{equation*}%
See \textbf{Functions: }\texttt{EHRSecondPartial::EHRSecondPartial} for more
information.

\item ( Vertex v, Edge e ) Returns the following second-order partial
derivative: 
\begin{equation*}
NEHRSecPartial(v,w)=\frac{\partial ^{2}NEHR}{\partial \eta _{e}\partial r_{v}%
}
\end{equation*}
\end{itemize}

\paragraph{Invalidation}

\bigskip Parents: Both versions use TotalVolume, TotalCurvature, and the
Radius, TotalVolumePartial and Curvature3D at v. 

\begin{itemize}
\item (v, w): The Curvature3D and TotalVolumePartial of w, the
CurvaturePartial of v with respect to w, and the TotalVolumeSecondPartial
with respect to v and w.

\item (v, e): The CurvaturePartials and TotalVolumePartial at e, the
CurvaturePartial of v with repsect to e, and the TotalVolumeSecondPartial
with respect to v and e.
\end{itemize}

\subsection{Eta}

\paragraph{Input}

Eta( Edge e )

\paragraph{Calculation}

This geoquant is never calculated, only set by the user.

\paragraph{Invalidation}

Parents: None

\bigskip

\subsection{EuclideanAngle}

\paragraph{Input}

EuclideanAngle( Vertex v, Face f )

\paragraph{Calculation}

The Euclidean angle is calculated usign the Euclidean law of cosines,
namely, let $v$ be indexed by $i$ and $f=\{i,j,k\}$. Then the angle at
vertex $i$ is%
\begin{equation*}
\cos (\alpha _{i,jk})=\frac{l_{ij}^{2}+l_{ik}^{2}-l_{jk}^{2}}{2l_{ij}l_{jk}}
\end{equation*}

\paragraph{Invalidation}

\bigskip Parents: The Lengths corresponding to the three edges of face f.

\subsection{FaceHeight}

\paragraph{Input}

FaceHeight( Face f, Tetra t )

\paragraph{Calculation}

See \textbf{Functions: }\texttt{FaceHeight::FaceHeight} for more information.

\paragraph{Invalidation}

\bigskip Parents: For one of the two edges, e, of f, the DihedralAngle of t
incident on e, and the EdgeHeights of the faces in t containing e, incident
on e.

\subsection{Length}

\paragraph{Input}

Length( Edge e )

\paragraph{Calculation}

Let $\alpha _{i}$, $\alpha _{j}$, $r_{i}$, and $r_{j}$ be the alphas and
radii at the vertices of edge $e=\{i,j\}$. Let $\eta _{ij}$ be the eta value
of edge e. Then the length is 
\begin{equation*}
l_{ij}^{2}=\alpha _{i}r_{i}^{2}+\alpha _{j}r_{j}^{2}+2r_{i}r_{j}\eta _{ij}
\end{equation*}

\paragraph{Invalidation}

\bigskip Parents: The Alpha and Radius values at both vertices of edge e and
the Eta value at edge e.

\subsection{PartialEdge}

\paragraph{Input}

PartialEdge( Vertex v, Edge e )

\paragraph{Calculation}

See \textbf{Functions: }\texttt{PartialEdge::PartialEdge} for more
information.

\paragraph{Invalidation}

\bigskip Parents: The Alpha and Radius values at both vertices of edge e and
the Length at edge e.

\subsection{Radius}

\paragraph{Input}

Radius( Vertex v )

\paragraph{Calculation}

This geoquant is never calcualted, only set by the user.

\paragraph{Invalidation}

Parents: None

\bigskip 

\subsection{RadiusPartial}

\paragraph{Input}

RadiusPartial(Vertex v, Edge e)

\paragraph{Calculation}

This returns the following partial derivative: 
\begin{equation*}
RadiusPartial(v,e)=\frac{\partial r_{v}}{\partial \eta _{e}}
\end{equation*}%
This quantity is calculated for all radii with $e$ fixed as follows: 
\begin{equation*}
\overrightarrow{v}+A\overrightarrow{x}=0
\end{equation*}%
where $A$ is a matrix with entries $a_{ij}=\frac{\partial ^{2}NEHR}{\partial
r_{j}\partial r_{i}}$ and $v_{i}=\frac{\partial ^{2}NEHR}{\partial \eta
_{e}\partial r_{i}}$, $x_{i}=\frac{\partial r_{i}}{\partial \eta _{e}}$. The
vector $\overrightarrow{x}$ can then be solved for using Gaussian
Elimination or other methods.

\paragraph{Invalidation}

Parents: All NEHRPartial and NEHRSecondPartial quantities.

\subsection{SectionalCurvature}

\paragraph{Input}

SectionalCurvature( Edge e )

\paragraph{Calculation}

SectionalCurvature calculates the discrete curvature at an edge in a three
dimensional manifold. \ The calculation is done by: 
\begin{equation*}
SectionalCurvature(e)=2\pi -\sum_{\substack{ All~tetra~t~  \\ incident~to~e}}%
\beta _{e,t}
\end{equation*}%
Where $\beta _{e,t}$ is the DihedralAngle at edge $e$ in tetra $t$.

\paragraph{Invalidation}

\bigskip Parents: The DihedralAngles of all tetrahedron containing e,
incident on e.

\subsection{TotalCurvature}

\paragraph{Input}

TotalCurvature( )

\paragraph{Calculation}

This is simply the sum of the Curvature3Ds of every vertex of the
triangulation.

\paragraph{Invalidation}

Parents: The Curvature3Ds of every vertex of the triangulation.

\bigskip

\subsection{TotalVolume}

\paragraph{Input}

TotalVolume( )

\paragraph{Calculation}

See \textbf{Functions: }\texttt{TotalVolume::TotalVolume} for more
information.

\paragraph{Invalidation}

\bigskip Parents: The Volume of every tetrahedron in the triangulation.

\subsection{Volume}

\paragraph{Input}

Volume(Tetra t)

\paragraph{Calculation}

The volume of a tetrahedron is done by calculating the Cayley Menger
Determinant, the determinant of a matrix that calculates the volume of a
simplex of any dimension. For the 3-simplex (tetrahedron) with $t=\{i,j,k,l\}
$, the volume is given by the formula 
\begin{equation*}
288V^{2}=\left\vert 
\begin{array}{ccccc}
0 & 1 & 1 & 1 & 1 \\ 
1 & 0 & l_{ij} & l_{ik} & l_{il} \\ 
1 & l_{ji} & 0 & l_{jk} & l_{jl} \\ 
1 & l_{ki} & l_{kj} & 0 & l_{kl} \\ 
1 & l_{li} & l_{lj} & l_{lk} & 0%
\end{array}%
\right\vert 
\end{equation*}%
where $l_{ij}$ is the length of edge $\{i,j\}$.

\paragraph{Invalidation}

\bigskip Parents: The Lengths of the six edges of $t$.

\subsection{TotalVolumePartial}

\paragraph{Input}

\begin{itemize}
\item TotalVolumePartial( Vertex v )

\item TotalVolumePartial( Edge e )
\end{itemize}

\paragraph{Calculation}

\begin{itemize}
\item ( Vertex v ): This sums up every VolumePartial(v, t) over all
tetrahedrons t containing v.

\item ( Edge e ): This sums up every VolumePartial(e, t) over all
tetrahedrons t containing e.
\end{itemize}

\paragraph{Invalidation}

Parents: All VolumePartials involving vertex v (edge e).

\subsection{TotalVolumeSecondPartial}

\paragraph{Input}

\begin{itemize}
\item TotalVolumePartial( Vertex v , Vertex w )

\item TotalVolumePartial( Vertex v, Edge e )
\end{itemize}

\paragraph{Calculation}

\begin{itemize}
\item ( Vertex v , Vertex w ): This sums up every VolumeSecondPartial(v, w,
t) over all tetrahedrons t containing v and w.

\item ( Vertex v, Edge e ): This sums up every VolumeSecondPartial(v, e, t)
over all tetrahedrons t containing v and e.
\end{itemize}

\paragraph{Invalidation}

Parents: All VolumeSecondPartials involving vertex v and vertex w (edge e).

\subsection{VolumeLengthPartial}

\paragraph{Input}

THIS GEOQUANT DOES NOT EXIST YET (EVER???)

\paragraph{Calculation}

\paragraph{Invalidation}

\bigskip

\subsection{VolumeLengthTetraPartial}

\paragraph{Input}

THIS GEOQUANT EXISTS, BUT WAS\ INTENDED\ FOR\ USE\ WITH\
VolumeLengthPartial. 

\paragraph{Calculation}

\paragraph{Invalidation}

\bigskip

\subsection{VolumePartial}

\paragraph{Input}

\begin{itemize}
\item VolumePartial( Vertex v, Tetra t )

\item VolumePartial( Edge e, Tetra t )
\end{itemize}

\paragraph{Calculation}

\begin{itemize}
\item (Vertex v, Tetra t): Returns the following partial derivative: 
\begin{equation*}
VolPartial(v,t)=\frac{\partial V_{t}}{\partial r_{v}}
\end{equation*}%
See \textbf{Functions: }\texttt{VolumePartial::VolumePartial} for more
information.

\item (Edge e, Tetra t): Returns the following partial derivative: 
\begin{equation*}
VolPartial(e,t)=\frac{\partial V_{t}}{\partial \eta _{e}}
\end{equation*}
\end{itemize}

\paragraph{Invalidation}

Parents: The Alphas, Radii, and Etas that are associated with $t$.

\bigskip

\subsection{VolumePartialSum}

\paragraph{Input}

EQUIVALENT\ TO\ TotalVolumePartial... TotalVolumePartial IS UP-TO-DATE.

\paragraph{Calculation}

\paragraph{Invalidation}

\bigskip

\subsection{VolumeSecondPartial}

\paragraph{Input}

\begin{itemize}
\item VolumeSecondPartial( Vertex v, Vertex w, Tetra t )

\item VolumeSecondPartial( Vertex v, Edge e, Tetra t )

\item VolumeSecondPartial( Edge e, Edge f, Tetra t )
\end{itemize}

\paragraph{Calculation}

\begin{itemize}
\item ( Vertex v, Vertex w, Tetra t ): Returns the following second-order
partial derivative: 
\begin{equation*}
VolSecPartial(v,w,t)=\frac{\partial ^{2}V_{t}}{\partial r_{w}\partial r_{v}}
\end{equation*}%
See \textbf{Functions: }\texttt{VolumeSecondPartial::VolumeSecondPartial}
for more information.

\item ( Vertex v, Edge e, Tetra t ): Returns the following second-order
partial derivative: 
\begin{equation*}
VolSecPartial(v,e,t)=\frac{\partial ^{2}V_{t}}{\partial \eta _{e}\partial
r_{v}}
\end{equation*}

\item ( Edge e, Edge f, Tetra t ): Returns the following second-order
partial derivative: 
\begin{equation*}
VolSecPartial(e,f,t)=\frac{\partial ^{2}V_{t}}{\partial \eta _{f}\partial
\eta _{e}}
\end{equation*}
\end{itemize}

\paragraph{Invalidation}

Parents: The Alphas, Radii, and Etas that are associated with $t$.

\bigskip
